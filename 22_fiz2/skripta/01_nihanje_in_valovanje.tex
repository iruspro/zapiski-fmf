\section{Mehansko nihanje in valovanje}
\subsection{Enostavna nihala. Enačba dušenega nihanja.}

\begin{enumerate}
    \item Utež na vijačni vzmeti
    
    \begin{itemize}
        \item Določi odmik vzmeti, če jo obesimo z utežjo.
        \item Zapiši enačbo dušenega nihanja. Koeficient dušenja. Lastna frekvenca.
        \item Z nastavkom \(y = Ae^{\lambda t}\) reši enačbo:
        \begin{itemize}
            \item Podkritično dušenje. Fazni zamik. Nihajni čas. Frekvenca.  Določi parametri iz začetnih pogojev.
            \item Kritično dušenje. 
            \item Nadkritično dušenje.
        \end{itemize}
        \item Energija podkritičnega dušenja.
    \end{itemize}

    \item Nitno (matematično) nihalo
    \begin{itemize}
        \item Zapiši enačbo dušenega nihanja in njeno rešitev.
    \end{itemize}

    \item Fizično nihalo
    \begin{itemize}
        \item Zapiši enačbo dušenega nihanja in njeno rešitev.
    \end{itemize}

    \item Vsiljeno nihanje
    \begin{itemize}
        \item Zapiši enačbo vsiljenega nihanja ter obliko splošne rešitve.
    \end{itemize}

    \item Valovna enačba
    \begin{itemize}
        \item Izpelji valovno enačbo.
        \item Pojasni Dopplerjev pojav.
    \end{itemize}
\end{enumerate}