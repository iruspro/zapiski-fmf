\section{Vaje}
\subsection{Par točkastih električnih nabojev}
\begin{itemize}
    \item \(\ds F_{12}=F_{21} = \frac{1}{4 \pi \epsilon_0} \frac{|e_1e_2|}{r^2}\)
\end{itemize}
\subsection{Delo. Električna potencialna energija}
\begin{itemize}
    \item \(\ds W_p(\infty) = W_p(r) -  \frac{e_1 e_2}{4 \pi \epsilon_0} \int_{r}^{\infty} \frac{dr}{r^2} = W_p(r) + \frac{e_1e_2}{4\pi \epsilon_0} \frac{1}{r}\). Če si izberimo \(W_p(\infty) = 0\), potem \(\ds W_p(r) = -  \frac{e_1e_2}{4\pi \epsilon_0} \frac{1}{r}\)
\end{itemize}
\subsection{Električno polje}
\begin{itemize}
    \item Polje točkastega naboja: \(\ds E = \frac{F}{e_2} = \frac{e_1}{4 \pi \epsilon_0 r^2}\)
    \item \textbf{Gaussov izrek:} \(\ds \int_{V} \text{div} \vec{A} \cdot dV = \oint_S \vec{A} \cdot \vec{n} \, dS\), naboj v telesu \(V\): \[e = \epsilon_0 \oint_S \vec{E} \cdot d \vec{S}\]
\end{itemize}
\subsection{Električni potencial}
\begin{itemize}
    \item \(\ds W_p(\vec{r}_2)- W_p(\vec{r}_1) = -e \int_{\vec{r}_1}^{\vec{r}_2} \vec{E} \cdot \, d \vec{r} \lthen \Phi(\vec{r}_2) - \Phi(\vec{r}_1) = - \int_{\vec{r}_1}^{\vec{r}_2} \vec{E} \cdot \, d \vec{r} \) je razlika električnih potencialov ali \emph{električna napetost}.
\end{itemize}

\subsection{Vezja}
\paragraph{Pravila za obravnavo vezja} \ 
\begin{enumerate}
    \item \(\dot{q}_V = \sum_{i=1}^{k}I_i - \sum_{i=k+1}^{m}I_i\)
    \item \(\sum_{i} U_i{'} = 0\)
    \item \(U_R' = - (\hat{n}_R \cdot \hat{e}_R) RI_R\)
    \item \(U_C' = \frac{-q_{c,1}}{C} = \frac{q_{c,2}}{C}\), kjer je \(q_{c, 1}\) naboj na plošči kondenzatorja, ki jo prvič dotaknemo na poti
    \item \(U_L' = - (\hat{n}_L \cdot \hat{e}_L) L \dot{I}_L\)
\end{enumerate}
\paragraph{Postopek} \
\begin{itemize}
    \item Izberimo si orientacijo zanke okrog elementov. Ta orientacija nam implicira orientacijo \(\hat{e}_R, \hat{e}_L, \ldots\)
    \item Izberimo si orientacijo \(\hat{n}_R, \hat{n}_L, \ldots\)
\end{itemize}
\paragraph{Kontinuitetna enačba:} \(\ds I = \frac{de}{dt}\)
\begin{itemize}
    \item Praznjenje kondenzatorja: \(e = e_0 \exp(-\frac{t}{RC})\), kjer je \(R\) upor v vezji
    \item Polnjenje kondenzatorja: \(CU_g (1 - \exp(-\frac{t}{RC}))\)
\end{itemize}

\subsection{Magnetizem}
\begin{itemize}
    \item Gauss: \(\ds \oint \vec{B} \cdot d \vec{S} = 0\)
    \item Ampere: \(\ds \frac{1}{\mu_0} \oint \vec{B} \cdot d \vec{S} = I\) (objeti)
    \item \(\vec{B} = \mu_0 \vec{H}\), kjer je \(\vec{B}\) gostota magnetnega polja, \([\vec{B}] = \frac{\text{Vs}}{\text{m}^3} = T\) in \(\vec{H}\) jakost mag.\ polja.
    \item Biot-Sawartova enačba pove gostoto magnetnega polja na razdalji \(r\) od žici:
    \[\vec{B}(r) = \frac{\mu_0 I}{4 \pi} \int_{\text{po žici}} \frac{\vec{r} \times d \vec{s}}{r^3}\]
\end{itemize}