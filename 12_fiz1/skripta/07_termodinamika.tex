\section{Termodinamika}
\begin{enumerate}
    \item Temperatura
    \begin{itemize}
        \item Osnovni količini termodinamike. Termodinamska limita.
        \item Ali je termodinamsko ravnovesje tranzitivno?
        \item Plinski termometer. Dva varianta.
        \item \textbf{Definicija.} Trojna točka vode. Absolutna ničla.
    \end{itemize}

    \item Raztezanje snovi
    \begin{itemize}
        \item \textbf{Definicija.} Koeficient dolžinskega raztezka. Koeficient prostorninskega raztezka.
        \item Zveza med \(\alpha\) in \(\beta\) za trda telesa.
        \item Anomalija vode.
    \end{itemize}

    \item Kinetična teorija plinov
    \begin{itemize}
        \item \textbf{Definicija.} Tlak.
        \item Povprečna skupna kinetična energija.
        \item Zveza med \(p, V\) in \(\overline{W}_k\).
        \item \textbf{Definicija.} Intenzivna količina. Ekstenzivna količina.
        \item \textbf{Definicija.} Boltzmanova konstanta.
        \item Plinska enačba.
    \end{itemize}

    \item Ekviparcijski izrek
    \begin{itemize}
        \item Ekviparcijski izrek.
        \item Povprečna energija za molekulu: 1 atom, 2 atoma, 2 atoma visoka \(T\).
    \end{itemize}

    \item Energijski zakon
    \begin{itemize}
        \item Ohranitveni zakon.
        \item I.\ zakon termodinamike.
        \item Izraz za delo s stališča plina.
        \item Toplota. Specifična toplota vode.
        \item Notranja energija idealnega plina. Povezava med \(c_p\) in \(c_v\).
        \item Povezava med \(c_p, c_v\) in \(\kappa\).
    \end{itemize}

    \item Spremembe idealnega plina
    \begin{itemize}
        \item Izohorna sprememba.
        \item Izobarna sprememba.
        \item Izotermna sprememba.
        \item Adiabatna sprememba.
    \end{itemize}

    \item Fazne spremembe
    \begin{itemize}
        \item Izparilna toplota.
        \item Talilna toplota.
        \item Specifična izparilna in specifična talilna toplota vode.
    \end{itemize}

    \item Entropijski zakon
    \begin{itemize}
        \item Primer ireverzibilnih in reverzibilnih sprememb.
        \item Entropija.
        \item Izraz za spremembo entropije.
        \item II.\ zakon termodinamike.
    \end{itemize}

    \item Toplotni stroji
    \begin{itemize}
        \item Carnotov cikel. Izkoristek. Catrnotov izkoristek.
        \item Hladilnik. Učinkovitost.
    \end{itemize}
\end{enumerate}

\newpage
\subsection*{Izpitna vprašanja}
\begin{enumerate}
    \item Termodinamika
    \begin{itemize}
        \item Zapiši I.\ zakon termodinamike, ter obliko vseh členov za primer infinitezimalne spremembe plina pri konstantnem tlaku.
        \item Pozimi voda v jezeru zamrzne od površine navzdol. Zakaj?
        \item Navedi primer reverzibilnega in primer ireverzibilnega procesa.
        \item Zapiši izraz za delo, toploto in spremembo notranje energije pri izohorni spremembi.
        \item Termoelektrarna dela z močjo 600 MW. Predpostavimo, da je idealni toplotni stroj, ki toploto prejema pri višji temperaturi 227 °C in jo oddaja pri nižji temperaturi 27 °C. Kot toplotni rezervoar pri nižji temperaturi uporablja reko, ki teče mimo elektrarne s pretokom 1000 \(m^3/s\). Oceni, za koliko se segreje reka. Rezultat podaj na 1 decimalno mesto natančno.
    \end{itemize}
\end{enumerate}