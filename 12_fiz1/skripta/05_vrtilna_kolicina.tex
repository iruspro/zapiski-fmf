\section{Vrtilna količina}
\begin{enumerate}
    \item Vrtilna količina in navor za točkasta telesa
    \begin{itemize}
        \item \textbf{Definicija.} Navor. Vrtilna količina.
        \item Newtonov zakon za vrtenje točkastega telesa.
        \item Izrek o vrtilni količini. Izrek o ohranitve vrtilne količine.
        \item Newtonov zakon za vrtenje sistema točkastih teles.
    \end{itemize}

    \item Toga telesa
    \begin{itemize}
        \item \textbf{Definicija.} Togo telo.
        \item Kako lahko opišemo gibanje vsakega togega telesa?
        \item Zapiši \(\vec{r}\) in \(\vec{v}\) za poljubno točko glede na težišče.
        \item \textbf{Zgled.} Opiši kotaljenje brez drsanja na 3 različna načina: glede na težišče, glede na stičišče s tla, glede na zgornjo točko. 
        \item Zapiši izraz za navor na togo telo.
        \item \textbf{Zgled.} Izračunaj navor sile teže.
        \item Kdaj togo telo miruje?
    \end{itemize}

    \item Vrtilna količina togega telesa
    \begin{itemize}
        \item \todo{wtf}
        \item \textbf{Primer.} Kotaljenje valja po klancu.
    \end{itemize}

    \item Kinetična energija pri vrtenju
    \begin{itemize}
        \item Kinetična energija pri vrtenju okoli fiksne osi.
        \item Kinetična energija telesa pri vrtenju okoli težišča.
    \end{itemize}
\end{enumerate}

\newpage
\subsection*{Izpitna vprašanja}
\begin{enumerate}
    \item Vrtilna količina
    \begin{itemize}
        \item Zapiši Newtonov zakon za vrtenje točkastega telesa. Poimenuj vektorske količine, ki v njem nastopajo, in jih zapiši za točkast delec.
        \item Kroglica brez trenja kroži po neki plošči, v kateri je luknja. Luknja je postavljena v središču kroga, po obodu katerega kroži kroglica.Kroglica je povezana z vrvjo, ki gre skozi luknjo na sredini. Kroglica je na začetku oddaljena od središča za \(r\), ima pa konstantno velikost hitrosti \(v\). Katera sila nasprotuje sili vrvi na kroglico? Zapiši izraz zanjo. Nato silo vrvice infinitizimalno povečamo, tako da se kroglica začne gibati proti središču kroga. Ali se vrtilna količina in energija kroglice ohranita? Zapiši ustrezna izraza za energijo in vrtilno količino kroglice v končnem položaju.
        \item Na levi sliki označi vektorja \(\omega\) in \(\Gamma\) v primeru, ko je gred pri avtomobilu centrirana, na desni pa, ko gred ni centrirana. Z \(v\) je označena smer gibanja avtomobila.
        \item Zapiši zakon o ohranitvi vrtilne koli£ine (2.\ Newtonov zakon za vrtenje), pri čemer \(\Gamma\) izrazi v odvisnosti od časa. Natančno poimenuj količine, ki nastopajo v izrazu.
    \end{itemize}
\end{enumerate}