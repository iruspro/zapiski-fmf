\section{Gibalna količina}
\begin{enumerate}
    \item Ohranitev gibalne količine    
    \begin{itemize}
        \item \textbf{Izrek.} Izrek o gibalni količini. Sunek sile.
        \item \textbf{Izrek.} Izrek o ohranitvi gibalni količini.
        \item \textbf{Izrek.} Izrek o gibalni količini za težišče.
        \item \textbf{Poskus.} \todo{}
        \item \textbf{Zgled.} Recimo, da skočimo iz višine \(h = 5\) m, masa človeka pa \(m = 70\) kg. S kako silo na nas deluje tla, če se začnemo ustavljati za:
        \begin{itemize}
            \item \(y = 50\) cm;
            \item \(y = 1\) cm.
        \end{itemize}
    \end{itemize}

    \item Trki
    \begin{itemize}
        \item Neelstični (neprožni trk). Elastični trk.
        \item Kdaj se ohranja gibalna količina? Kdaj se ohranja kinetična energija?
        \item Neprožni trk: Škatla z maso \(M = 320\) g visi na vrvi z \(l = 5.65\) m, ki je na stropu. V njo izstrelimo izstrel z začetno hitrostjo \(v\) in maso \(m = 0.5\) g in škatla se odkloni na \(x = 17\) cm. Kolikšen je \(v\)?
        \item Elastični trk: Kroglo z maso \(m\) in hitrostjo \(v\) zakotalimo v mirujočo kroglo z maso \(M\). Določi hitrosti krogel po trku.
        \item Posebni primeri:
        \begin{itemize}
            \item \(m = M\);
            \item \(\frac{m}{M} \to 0\);
            \item \(\frac{m}{M} \to \infty\).
        \end{itemize}
    \end{itemize}

    \item Sila curka
    \begin{itemize}
        \item Izpelji izraz za silo curka pri pravokotnem odboju.
        \item \textbf{Primer.} Peltonova turbina. Maksimalna moč turbine. Izračunaj moč elektrarne, če \(\oldphi_V = 200 \ \text{m}^3/s\), \(h = 30\) m. 
        \item Izpelji enačbo za raketo.
        \item \textbf{Primer.} Izračunaj hitrost rakete, če \(m_0 = 3000\) T, \(v_0 = 2.6\) m/s, \(m_1 = 800\) T.
    \end{itemize}
\end{enumerate}

\newpage
\subsection*{Izpitna vprašanja}
\begin{enumerate}
    \item Gibalna količina
    \begin{itemize}
        \item Drsalka stoji ob ogradi drsališča, nakar se z rokami odrine od ograde (trenje med drsalkami in ledom je zanemarljivo). Katere sila podeli neničelno težiščno hitrost drsalki po odrivu? Koliko dela je na drsalki opravila ta sila?
        \item V strugi širine 30 m in globine 1 m teče voda s hitrostjo 1 m/s. Nato struga preide v sotesko, kjer je njena širina 5 m, globina pa 2 m. Kakšna je hitrost vode v soteski?
        \item Zapiši izraz za silo curka.
    \end{itemize}
\end{enumerate}