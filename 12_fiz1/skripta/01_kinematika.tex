\section{Kinematika}

\begin{enumerate}
    \item Kinematika
    
    Kinematika določa zveze med potjo, hitrostjo, časom in pospeškom.
    \begin{itemize}
        \item \textbf{Definicija.} Točkasto telo. Radij-vektor točke.
        \item \textbf{Definicija.} Hitrost. Pospešek.
        \item \textbf{Zgled.} Kako dobimo \(\vec{r}\) in \(\vec{v}\), če poznamo \(\vec{a}\)?
    \end{itemize}

    \item Premo gibanje (1D)
    \begin{itemize}
        \item \textbf{Definicija.} Enakomerno gibanje.
        \item \textbf{Trditev.} Odvisnost koordinate točke od časa pri enakomernem gibanju.
        \item \textbf{Definicija.} Enakomerno pospešeno gibanje.
        \item \textbf{Trditev.} Odvisnost koordinat in hitrosti točke od časa pri enakomerno pospešenem gibanju.
        \item \textbf{Trditev.} Zveza med \(a\) in \(v\) pri enakomerno pospešenem gibanju (brez časa).
        \item \textbf{Poskus.} Pospešek pri prostem padu. Vrv z utežmi: Kako naj razporedimo uteži na vrvi, da bomo med padanjem slišali zvok v enakih časovnih intervalih?
    \end{itemize}

    \item Ravninsko gibanje (2D)
    \begin{itemize}
        \item \textbf{Opomba.} Ali je gibanje v različnih smeri odvisno? 
        \item \textbf{Trditev.} Sprememba \(x\) in \(y\) koordinat v odvisnosti od časa pri poševnem metu.
        \item \textbf{Opomba.} Kaj je trajektorija gibanja? Eksplicitna rešitev.
        \item \textbf{Trditev.} Čas v katerem dosežemo največjo višino. Domet. Največja višina.
        \item \textbf{Opomba.} Pri kakšnem začetnem kotu dobimo maksimalni domet? Kaj je vodoravni met?
        \item \textbf{Poskus.} Razkopljeno gibanje: Eno kroglo pustimo, da prosto pada, drugo pa izstrelimo z hitrostjo \(v_0\) v vodoravni smeri. Katera krogla bo prva padla na tla? Izstrel: Kam moramo usmeriti izstrel, da izstrel zadene cilj, ki pada navpično navzdol?
    \end{itemize}
\end{enumerate}

\newpage
\subsection*{Izpitna vprašanja}
\begin{enumerate}
    \item Premo gibanje (1D)
    \begin{itemize}
        \item Skiciraj grafe ter napiši formule za \(a(t)\), \(v(t)\), \(x(t)\), če je gibanje enakomerno pospešeno.
    \end{itemize}

    \item Ravninsko gibanje (2D)
    
    \begin{itemize}
        \item Izpelji izraz za domet pri poševnem metu, pri čemer ga izrazi z začetno hitrostjo in kotom med začetno smerjo in vodoravnico.
    \end{itemize}
\end{enumerate}