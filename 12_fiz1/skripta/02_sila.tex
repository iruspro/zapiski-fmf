\section{Sila}
\begin{enumerate}
    \item Sile
    \begin{itemize}
        \item Kaj je sila? 
        \item Osnovne sile v naravi.
        \item Kavzalnost.
    \end{itemize}

    \item Newtonovi zakoni
    \begin{itemize}
        \item Ali je za enakomerno gibanje potrebna sila?
        \item \textbf{Zakon.} I.\ Newtonov zakon.
        \item \textbf{Zakon.} II.\ Newtonov zakon.
        \item \textbf{Poskus.} Merjenje \(g\).
        \item \textbf{Zakon.} III.\ Newtonov zakon.
    \end{itemize}

    \item Težišče
    \begin{itemize}
        \item \textbf{Zgled.} Zapiši Newtonov zakon za sistem dveh točkastih teles.
        \item \textbf{Definicija.} Težišče. Hitrost težišča. Pospešek težišča.
        \item \textbf{Zakon.} II.\ Newtonov zakon za težišče.
        \item \textbf{Definicija.} Skupna masa in težišče za zvezno telo.
        \item \textbf{Poskus.} Imamo dve vzmeti z isto maso. Ena je raztegnjena, druga pa ne. Kakšna bo prej padla na tla?
    \end{itemize}

    \item Neinercialni koordinatni sistem
    
    Kako se transformirajo vektorji, kadar gremo iz enega sistema v drug?
    \begin{itemize}
        \item Galilejeva transformacija.
        \item II.\ Newtonov zakon v inercialnih sistemih. Princip relativnosti.
        \item II.\ Newtonov zakon v linearno pospešenem sistemu.
        \item \textbf{Zgled.} \todo{}
        \item II.\ Newtonov zakon pri kroženju okoli fiksne osi. Tangentna sila. Coriolisova sila. Centrifugalna sila.
        \item \textbf{Zgled.} \todo{}
    \end{itemize}
\end{enumerate}

\newpage
\subsection*{Izpitna vprašanja}
\begin{enumerate}
    \item Newtonovi zakoni
    \begin{itemize}
        \item Napiši vse tri Newtonove zakone.
        \item Na stropu je z verigo z maso \(0.5\) kg pritrjen lestenec z maso \(2\) kg. Skiciraj situacijo, na njej označi vse sile, ki delujejo na lestenec in verigo ter napiši izraze za velikosti teh sil.
    \end{itemize}

    \item Težišče
    \begin{itemize}
        \item Zapiši izraz za lego težišča sistema točkastih teles.
        \item Skiciraj metlo in na njej približno označi težišče. Metlo nato prežagamo na pol skozi težišče. Katera polovica je težja?
    \end{itemize}

    \item Neinercialni sistemi
    \begin{itemize}
        \item Zakaj umetni satelit, ki kroži okoli Zemlje, ne pade na Zemljo zaradi gravitacijske sile?
        \item Anticiklon je območje visokega zračnega pritiska. V tlorisu skiciraj, kako se gibljejo zračne mase znotraj anticiklona na severni Zemeljski polobli. Zakaj, in katera sistemska sila igra pri tem vlogo?
    \end{itemize}
\end{enumerate}