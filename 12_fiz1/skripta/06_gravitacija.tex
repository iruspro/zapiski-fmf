\section{Gravitacija}
\begin{enumerate}
    \item Gravitacijska sila
    \begin{itemize}
        \item Gravitacijski zakon.
        \item Gravitacijska potencialna energija. Potencial objekta z maso \(m\).
    \end{itemize}

    \item Zemlja
    \begin{itemize}
        \item Izraz za gravitacijsko silo med Zemljo in telesom na površine:
        \begin{itemize}
            \item Tanka krogelna lupina;
            \item Polna homogena krogla.
        \end{itemize}
        \item Oblika Zemlje.
        \item Plimske sile.
    \end{itemize}

    \item Problem dveh teles
    \begin{itemize}
        \item Težišče sistema.
        \item Energija.
        \item Vrtilna količina.
        \item Oblika orbite. Klasifikacija glede na \(e\) in \(E\).
        \item I.\ Keplerjev zakon.
        \item Obhodni čas. II.\ Keplerjev zakon.
    \end{itemize}

    \item Gibanje Zemlje
    \begin{itemize}
        \item Precesija perihelija.
        \item Precesija osi.
    \end{itemize}
\end{enumerate}

\newpage
\subsection*{Izpitna vprašanja}
\begin{enumerate}
    \item Gravitacija
    \begin{itemize}
        \item Zapiši izraz za gravitacijsko silo med dvema točkastima telesoma.
        \item Klasificiraj obliko orbite nebesnega telesa v odvisnosti od njegove ekscentričnosti \(e\) in energije \(E\). 
        \item Katera sila je odgovorna za kroženje Lune okoli Zemlje? Zapiši formulo za razdalje Lune od Zemlje, pri čemer vse izrazi z gravitacijskim pospeškom na Zemlji, polmerom Zemlje in obodno hitrostjo Zemlje.
    \end{itemize}
\end{enumerate}