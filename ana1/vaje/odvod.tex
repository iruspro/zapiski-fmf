\section{ODVOD}

{\color{Purple} \subsubsection*{Iskanje globalnih ekstremov (nalogi $3-7$)}}
%
Vsaka zvezna funkcija $f$ na omejenem zaprtem intervalu $[a, b]$ doseže svojo minimalno in maksimalno vrednost. Če je funkcija $f$ še razmeroma gladka, lahko ti dve vrednosti poiščemo s pomočjo odvoda. \underline{Pri iskanju ekstremnih vrednosti} najprej zožimo nabor morebitnih kandidatov na naslednje tipe točk z intervala $[a,b]$:
\begin{itemize}
    \item stacionarne točke funkcije $f$ v $(a,b)$ ($x \in (a,b), \ f'(x)=0$).
    \item robni točki intervala $[a,b]$.
    \item točke na intervalu $(a,b)$, v katerih funkcija $f$ ni odvedljiva.
\end{itemize}
Nato pa izračunamo vrednosti $f$ v kandidatih in izberimo min/maks vrednost.

\underline{Ko optimiziramo kakšne geometrijske/fizikalne probleme}:
\begin{itemize}
    \item Formuliramo funkcijo, ki je odvisna samo od enega parametra in modelira naš problem.
    \item Poiščemo ekstreme te funkcije.
\end{itemize}
%
\underline{Ko iščemo ekstreme funkcij, ki so definirane z evklidsko razdaljo}, je ponavadi lažje obravnavati funkcijo $f=d^2$.
%
{\color{Purple} \subsubsection*{Geometrijski pomen odvoda. Tangente. Kot med funkcijama (nalogi $8-12$)}}
\textbf{Enačba tangente na graf funkcije $f$ v točki $(a, f(a))$:} $y = f'(a)(x-a) + f(a)$.

\textbf{Enačba normale na graf funkcije $f$ v točki $(a, f(a))$:} naklon: $k_t \cdot k_n = -1$; enačbo dobimo iz zveze $f(a) = k_n a + \textcolor{red}{b}$.

\underline{Kot med krivuljama} je kot med tangentoma v presičišču: $\displaystyle \tan \varphi = \frac{k_2 - k_1}{1+k_1 k_2}$

Enačbo krivulje lahko \underline{implicitno odvajamo}:
\begin{itemize}
    \item $x$-e odvajamo kot običajno.
    \item $y$-e odvajamo kot ponovadi + dodamo faktor $y'$.
\end{itemize}
Nato pa izrazimo $y'$. Koeficient $k$ bo odvisen od $x$ in $y$.
%
{\color{Purple} \subsubsection*{Zveznost. Odvedljivost. Zvezna odvedljivost (nalogi $16-18$)}}
Funkcija $f$ je zvezna v točki $a \Leftrightarrow \displaystyle f(a) = \lim_{x \to a} f(x)$. Zlepek funkcij $f_1$ in $f_2$ bo zvezen v točki $a \Leftrightarrow f_1(a) = f_2(a)$. Zlepek funkcij $f_1$ in~$f_2$ bo odvedlkiv v točki $a \Leftrightarrow f_1'(a) = f_2'(a)$. 

Funkcija $f$ je odvedljiva v točki $a$, ko obstaja limita $\displaystyle f'(a) = \lim_{h \to 0} \frac{f(a+h) -f(a)}{h}$.

{\color{Purple} \subsubsection*{Rollov in Lagrangeev izrek (nalogi $19-23$)}}

\textbf{Rollov izrek.} Naj bo funkcija $f: \, [a,b] \to \RR$ zvezna na $[a,b]$ in odvedljiva na $(a,b)$. Če je $f(a) = f(b)$, potem obstaja $c \in (a,b)$, za katero velja $f'(c) = 0$.

\textbf{Lagrangeev izrek.} Naj bo funkcija $f: \, [a,b] \to \RR$ zvezna na $[a,b]$ in odvedljiva na $(a,b)$. Potem obstaja $c \in (a,b)$, za katero velja $$\displaystyle f'(c) = \frac{f(b) - f(a)}{b-a} \quad \text{ali} \quad f(b) - f(a) = f'(c)(b-a).$$

Funkcija je Lipschitzeva, če obstaja konstanta $C$, da za vsaka $x,y \in \RR$ velja $|f(x)-f(y)| \leq C|x-y|$. Vsaka Lipschitzeva funkcija na~$\RR$ je enakomerno zvezna na $\RR$.

Če funkcija $f$ ima omejen odvod na $I$, potem je funkcija $f$ enakomerno zvezna na $I$.
%

{\color{Red} \subsubsection*{L'Hospitalovo pravilo (nalogi $24-25$)}}
TODO

{\color{Purple} \subsubsection*{Risanje grafov podanih eksplicitno (nalogi $26-27$)}}
\begin{enumerate}
    \item $D_f$, ničle, poli, lin. asimtote ($\displaystyle k = \lim_{x \to \infty} \frac{f(x)}{x}$, $\displaystyle b = \lim_{x \to \infty} (f(x) - kx)$), limite na robu $D_f$.
    \item \underline{1. odvod:} stacionarne točke ($f'(x)=0$, $f'(x)$ ni definirana), lokalni ekstremi, intervali naraščanja/padanja,~tangente~na~robu~$D_f$.
    \item \underline{2. odvod:} prevoji in intervali konveksnosti (вогнутости)/konkavnosti (выпуклости).
\end{enumerate}
%
%
{\color{Purple} \subsubsection*{Risanje parametrično podanih krivulj (naloge $28-32$)}}
\begin{enumerate}
    \item Skiciramo grafa obeh komponent.
    \item Tabeliramo položaje, ki ustrezajo stacionarnim točkam obeh komponent in po potrebi dodamo še kakšno točko.
    \item Z upoštevanjem grafov komponent skiciramo krivuljo (se splača pogledati intervali med stacionarnimi točkami + v neskončnosti).
\end{enumerate}

\begin{itemize}
    \item Krivulje, ki so podane parametrično, lahko tudi puskusimo zapisati eksplicitno/implicitno, da ugotovimo, po kateri trajektoriji se giblje točka ali kar koli drugega, kar je potrebno.    
    \item Če je neka komponenta gre proti neskončno se splača pogledati, ali ima krivulja kakšne asimptote tako, da pogledamo limite obeh komponent v "`zanimivih"' točkah.
    \item Poševno asimtoto dobimo tako, da izračunamo limite: $\displaystyle k(t) =\lim_{t \to a} \frac{\dot{y}(t)}{\dot{x}(t)}$ in $\displaystyle b = \lim_{t \to a} (y(t) - k(t) x(t))$, kjer je $a$ točka, v kateri komponenti gresta v neskončnost.
\end{itemize}

\paragraph*{Računanje tangent v parametrični obliki.} Vektor $\vec{r} \, '(a)$ je \emph{smerni vektor} tangente na krivuljo v položaju $\vec{r}(t)$ (v primeru, ko je~$\vec{r} \, '(a) \neq 0$). Če pišemo $\vec{r} \, '(a) = (\dot{x}, \, \dot{y})$, potem: (1) če je $\dot{x} = 0$, potem tangenta navpična; (2) če je $\dot{x} \neq 0$, potem $\displaystyle k = \frac{\dot{y}}{\dot{x}}$.
%
\newpage
{\color{Purple} \subsubsection*{Risanje krivulj podanih v polarne oblike (naloge $33-35$)}}

Funkcija $r = r(\varphi)$ nam pove, kako daleč od $(0,0)$ je točka na krivulji, ki leži v smeri $\varphi$.

\begin{enumerate}
    \item Skiciramo graf $r = r(\varphi)$.
    \item S pomočjo grafa analiziramo gibanje točke, ki kroži okoli izhodišča in mu približuje ali oddaljuje.
\end{enumerate}

\begin{itemize}
    \item Če je $r(\varphi) < 0$, v tej smeri ne narišemo krivulje.
    \item Asimptote dobimo tako, da zapišemo krivuljo v parametrični obliki s parametrom $\varphi$ in poglejmo, kaj se dogaja z $x(\varphi), \ y(\varphi)$.
    \item Polarna oblika je poseben primer parametrične oblike, kjer za parameter vzamemo polarni kot $\varphi$: \\ $x(\varphi) = r(\varphi) \cos (\varphi), \ y(\varphi)=r(\varphi) \sin (\varphi)$.
    \item Implicitno krivuljo lahko zapišemo v polarni obliki tako, da napišemo $x = r \cos \varphi, \ y = r \sin \varphi$ in poskusimo izraziti $r$ s $\varphi$.
    
\end{itemize}
%
\paragraph*{Računanje tangent v polarni obliki.} 1. Krivuljo parametiziramo. 2. Računamo tangento krivulje v parametrične oblike.
