\section{ŠTEVILA}


\begin{enumerate}
    \item Naravna števila
    \begin{itemize}
        \item Kaj so naravna števila. Množica naravnih števil. 
        \item \textbf{Peanovi aksoimi.} Aksiom popolne indukcije.
        \item Operaciji na množici naravnih števil.
        \item Urejenost množice naravnih števil.        
    \end{itemize}

    \item Cela števila
    \begin{itemize}
        \item Zakaj potrebujemo celi števili?
        \item Množica celih števil.
        \item Operacije na množice celih števil.
        \item Urejenost množice celih števil. Ali je dobra?
    \end{itemize}

    \item Racionalna števila
    \begin{itemize}
        \item Zakaj potrebujemo racionalna števila?
        \item Ulomek. Kadar dva različna ulomka predstavljata isto racionalno število?
        \item \colorbox{purple!30}{\textbf{Definicija.}} Racionalno število. Oznaka.
        \item Množica racionalnih števil.
        \item Seštevanje in množenje racionalnih števil. Ali sta dobro definirani?
        \item Ali na vsakem odprtem intervalu $(a,b), \ a,b \in \QQ$ leži racionalno število? Ali racionalna števila napolnijo številsko premico?
    \end{itemize}

    \item Lastnosti računskih operacij
    \begin{itemize}
        \item Aksiomi seštevanja.
        \item Kako rečemo množice $A$ z operacijo $+$, ki izpolnjuje aksiome seštevanja?
        \item \colorbox{blue!30}{\textbf{Trditev.}} Naj bo število $a \in A$. Koliko nasprotnih števil lahko ima $a$?
        \item \colorbox{blue!30}{\textbf{Trditev.}} Pravilo krajšanja za seštevanje.        
        \item \colorbox{orange!30}{\textbf{Posledica.}} Ali je $-0 = 0$?
        \begin{itemize}
            \item \colorbox{green!30}{\textbf{Dokaz.}} Vse dokažemo z uporabo aksiom.
        \end{itemize}
        \item \colorbox{purple!30}{\textbf{Definicija.}} Odštevanje. Ali je asociativno oz. komutativno? Ali je $(A, -)$ Abelova?
        \item Aksiomi množenja.
        \item Kako rečemo množice $A \setminus \set{0}$ z operacijo $\cdot$, ki izpolnjuje aksiome množenja?
        \item \colorbox{blue!30}{\textbf{Trditev.}} Naj bo število $a \in A, \ a \neq 0$. Koliko obratnih števil lahko ima $a$?
        \item \colorbox{blue!30}{\textbf{Trditev.}} Pravilo krajšanja za množenje. 
        \item \colorbox{orange!30}{\textbf{Posledica.}} Ali je $1^{-1} = 1$?
        \begin{itemize}
            \item \colorbox{green!30}{\textbf{Dokaz.}} Vse dokažemo z uporabo aksiom.
        \end{itemize}
        \item Aksioma polja.
        \item Kako rečemo množice $(A, +, \cdot)$, ki izpolnjuje aksiome seštevanja, množenja in polja?
        \item Aksioma urejenosti.
        \item Kako rečemo množice $(A, +, \cdot, <)$, ki izpolnjuje aksiome seštevanja, množenja, polja in urejenosti?
        \item \colorbox{yellow!30}{\emph{Primer.}} Ali je $(\QQ, +, \cdot, <)$ urejeni komutativni obseg?
        \item \colorbox{purple!30}{\textbf{Definicija.}} Urejenost v urejenem komutativnem obsegu.
        \item \colorbox{blue!30}{\textbf{Trditev.}} Naj bo $A$ urejeni obseg. Ali je $1$ pozitivno število? Ali ima $A$ neskončno elementov?
        \begin{itemize}
            \item \colorbox{green!30}{\textbf{Dokaz.}} (1) Enostavno s protislovjem z uporabo aksiom.
            
            (2) Poglejmo vsoto vsih pozitivnih členov.
        \end{itemize}
        \item \colorbox{blue!30}{\textbf{Trditev.}} Kaj sledi iz aksiome $A11$ za poljubni števili $a, b \in A$?      
        \item \colorbox{blue!30}{\textbf{Trditev.}} 4 lastnosti relacije $<$.
        \begin{itemize}
            \item \colorbox{green!30}{\textbf{Dokaz.}} Definicija urejenosti in aksiomi.
        \end{itemize}
        \item \colorbox{purple!30}{\textbf{Definicija.}} Relacija $\geq$.        
    \end{itemize}

    \newpage
    \item Dedekindov aksiom in realna števila
    \item[$\circ$] Realna števila
    \begin{itemize}
        \item Zakaj potrebujemo realna števila?
        \item \colorbox{yellow!30}{\emph{Primer.}} Ali je rešitev enačbe $x^2 = 2$ racionalno število?
        \item Dedekindov pristop za definicijo realnega števila. 
        \item \colorbox{purple!30}{\textbf{Definicija.}} Rez.
        \item \colorbox{yellow!30}{\emph{Primer.}} $A = \set{p \in \QQ; \ p<0}$. Ali je $A$ rez?
        \item \colorbox{purple!30}{\textbf{Definicija.}} Množica realnih števil.
        \item \colorbox{blue!30}{\textbf{Trditev.}} Preslikava, ki vloži množico racionalnih števil v množico realnih števil. Ali je injektivna?
        \item \colorbox{purple!30}{\textbf{Definicija.}} Vsota realnih števil. Oznaka.
        \item \colorbox{blue!30}{\textbf{Trditev.}} Ali je vsota realnih števil realno število?
        \begin{itemize}
            \item \colorbox{green!30}{\textbf{Dokaz.}} Pokažemo, da je rez.
        \end{itemize}
        \item \colorbox{purple!30}{\textbf{Definicija.}} Pozitivno realno število. 
        \item \colorbox{purple!30}{\textbf{Definicija.}} Produkt realnih števil.
        \item \colorbox{blue!30}{\textbf{Trditev.}} $(\RR, +, \cdot, <)$ izpolnjuje aksiome $A1-A4$.        
        \begin{itemize}
            \item \colorbox{green!30}{\textbf{Dokaz.}} Preverimo aksiome z uporabo definicji reza. 
            
            Enota je $0^*$, inverz od $A$ je             
            $-A = \set{p \in \QQ; \ \text{obstaja } r \in \QQ, r>0, -p-r \notin A}$.
        \end{itemize}
        \item \item \colorbox{blue!30}{\textbf{Trditev.}} $(\RR, +, \cdot, <)$ izpolnjuje aksiome $A1-A12$.
        \item \colorbox{blue!30}{\textbf{Trditev.}} Ali je urejeni obseg $\RR$ vsebuje urejeni obseg $\QQ$ kot podobseg? Zakaj?
        \begin{itemize}
            \item \colorbox{green!30}{\textbf{Dokaz.}} Čemu je enako $(p+q)^*, \ (p \cdot q)^*$ in kadar je $p < q$, $p, q \in \QQ$?
        \end{itemize}
    \end{itemize}
    \item[$\circ$] Dedekindov aksiom
    \begin{itemize}
        \item \colorbox{purple!30}{\textbf{Definicija.}} Navzgor (navzdol) omejena množica. Zgornja (spodnja) meja množice. 
        \item \colorbox{purple!30}{\textbf{Definicija.}} Natančna zgornja/spodnja meja množice (supremum/infimum). Kaj natančno velja za supremum/infimum?
        \item \colorbox{purple!30}{\textbf{Definicija.}} Maksimum/minimum množice.
        \item \colorbox{yellow!30}{\emph{Primer.}} Kako sta povezana maksumum in supremum množice?
        \item \colorbox{yellow!30}{\emph{Primer.}} Določi supremum množice, če obstaja.
        \begin{itemize}
            \item $B = \set{x \in \QQ, \ x<0} \subset \QQ$.
            \item $C = \set{x \in \QQ, \ x^2<0} \subset \QQ$.
        \end{itemize}
        \item \colorbox{blue!30}{\textbf{Dedekindov aksiom.}}
        \item \colorbox{yellow!30}{\emph{Opomba.}} Ali je $\QQ$ izpolnjuje Dedekindov aksiom?
        \item \colorbox{blue!30}{\textbf{Izrek.}} Ali je vsaka navzgor omejena podmnožica v $\RR$ ima supremum?
        \begin{itemize}
            \item \colorbox{green!30}{\textbf{Dokaz.}} Izberimo poljubno neprazno navzgor omejeno podmnožico $\mathcal{A}$ v $\RR$. Definiramo $C := \bigcup_{A \in \mathcal{A}}A$. Pokažemo, da je $C \in \RR$ in $C = \sup \mathcal{A}$.
        \end{itemize}
        \item \colorbox{orange!30}{\textbf{Posledica.}} Ali je vsaka navzdol omejena podmnožica v $\RR$ ima infimum?
        \item \colorbox{orange!30}{\textbf{Posledica.}} Ali je $\RR$ izpolnjuje Dedekindov aksiom?
        \item \colorbox{blue!30}{\textbf{Izrek.}} Ali je $(\RR, +, \cdot, <)$ izpolnjuje aksiome $A1-A13$? Ali vsebuje $\QQ$ kot podobseg?
    \end{itemize}

    \item[$\circ$] Posledice Dedekindovega aksioma
    \begin{itemize}
        \item \colorbox{orange!30}{\textbf{Posledica 1.}} Ali je množica celih števil navzgor omejena?
        \begin{itemize}
            \item \colorbox{green!30}{\textbf{Dokaz.}} Če je $M = \sup \ZZ$, potem $M-1$ ni $\sup \ZZ$.
        \end{itemize}
        \item \colorbox{orange!30}{\textbf{Posledica 2.}} Ali za vsak $a \in \RR$ obstaja $b \in \ZZ$, da velja $b>a$?
        \begin{itemize}
            \item \colorbox{green!30}{\textbf{Dokaz.}} Protislovje s prvo posledico.
        \end{itemize}
        \item \colorbox{orange!30}{\textbf{Posledica 3.}} Arhimedska lastnost.
        \begin{itemize}
            \item \colorbox{green!30}{\textbf{Dokaz.}} Sledi iz posledice $2$.
        \end{itemize}
        \item \colorbox{orange!30}{\textbf{Posledica 4.}} Naj bo $a \in \RR, a>0$. Ali obstaja $n \in \NN$ za katero valja $\frac{1}{n} < a$?
        \begin{itemize}
            \item \colorbox{green!30}{\textbf{Dokaz.}} Sledi iz posledice $2$.
        \end{itemize}
        \item \colorbox{orange!30}{\textbf{Posledica 5.}} Ali je $\QQ$ povsod gosta v $\RR$?
        \begin{itemize}
            \item \colorbox{green!30}{\textbf{Dokaz.}} Definicija realnega števila.
        \end{itemize}
    \end{itemize}

    \item[$\circ$] Intervali
    \begin{itemize}
        \item \colorbox{purple!30}{\textbf{Definicija.}} Zaprti (odprti) interval. Polodprti interval. Neskončni intervali.
        \item \colorbox{purple!30}{\textbf{Definicija.}} Naj bo $a \in \RR$. $\epsilon$-okolica števila $a$. Okolica števila $a$.
    \end{itemize}

    \newpage
    \item[$\circ$] Decimalni ulomki
    \begin{itemize}
        \item Ali vsakemu realnemu številu lahko priredimo neskončen decimanli zapis?
        \item Množica decimalnih približkov števila $x$.
        \item \colorbox{blue!30}{\textbf{Trditev.}} Čemu je enak supremum množice decimalnih približkov števila $x$?
        \begin{itemize}
            \item \colorbox{green!30}{\textbf{Dokaz.}} Po konstrukciji $x$ je zgornja meja. Recimo, da je $y = \sup \mathcal{A} < x$. Potem obstaja $p \in \NN$, da $x - y > \frac{1}{10^p}$. Vzemimo $z = n_0 + \frac{n_1}{10} + \ldots + \frac{n_p}{10^p}$ in izračunamo $x-z$.
        \end{itemize}
        \item \colorbox{purple!30}{\textbf{Definicija.}} Zapis števila $x$ kot neskončen decimalni ulomek.
        \item \colorbox{blue!30}{\textbf{Trditev.}} Ali je zapis števila $x \in \RR$ kot neskončen decimalni ulomek enoličen? Kaj velja za dva različna zapisa števila $x$?
        \begin{itemize}
            \item \colorbox{green!30}{\textbf{Dokaz.}} (1) Naj bo $\mathcal{A}$ množica decimalnih približkov števila $x$. Pokažemo, da je $y$ natančna zgornja meja od $\mathcal{A}$ tako, da pokažemo, da $y - \frac{1}{10^p}$ ni zgornja meja za noben $p in \NN$.
            
            (2) Obstaja najmanjši indeks $k \in \NN_0: \ n_k \neq m_k$. Recimo, da $n_k < m_k$. Pokažemo, da edina možnost je $m_k = n_k + 1$ in da $m_{k+l} = 0$ za vse $l \in \NN$ in da $n_{k+j} = 9$ za vse $j \in \NN$.
        \end{itemize}
        \item \colorbox{blue!30}{\textbf{Trditev.}} Karakterizacija racionalnega števila (decimalni zapis).
        \begin{itemize}
            \item \colorbox{green!30}{\textbf{Dokaz.}} $(\Rightarrow)$ Iz periodičnega decimalnega zapisa dobimo ulomek.
            
            $(\Leftarrow)$ Vemo, da $x \in \QQ \Rightarrow x = \frac{m}{n}$. Pogledamo ostanki pri deljenju.
        \end{itemize}
    \end{itemize}

    \item[$\circ$] Uporaba Dedekinovega aksioma za uvedbo korenov in logaritma
    \begin{itemize}
        \item \colorbox{blue!30}{\textbf{Izrek.}} Naj bo $x \in \RR, \ x>0,\ n \in \NN$. Koliko rešitev ima enačba $y^n = x$?
        \item \colorbox{blue!30}{\textbf{Izrek.}} Naj bo $x \in \RR, \ x>0$ in $b \in \RR, \ b>0, \ b \neq 1$. Koliko rešitev ima enačba $b^y = x$?
    \end{itemize}

    \item[$\circ$] Absolutna vrednost
    \begin{itemize}
        \item \colorbox{purple!30}{\textbf{Definicija.}} Absolutna vresnost števila $x \in \RR$.
        \item \colorbox{blue!30}{\textbf{Trditev.}} 8 lastnosti absolutne vrednosti.
        \item \colorbox{orange!30}{\textbf{Posledica.}} Oceni $||x|-|y||$ in $|x_1 + x_2 + \ldots + x_n|$.
        \begin{itemize}
            \item \colorbox{green!30}{\textbf{Dokaz.}} Najprej pokažemo, da $|x-y| \leq |x| + |y|$, nato pa pišemo $|x| = |(x+y) - y|$.
        \end{itemize}
    \end{itemize}

    \newpage
    \item Kompleksna števila
    \begin{itemize}
        \item Zakaj potrebujemo kompleksna števila?
        \item \colorbox{purple!30}{\textbf{Definicija.}} Kompleksno število. Množica kompleksnih števil.
        \item \colorbox{yellow!30}{\emph{Opomba.}} Kadar sta kompleksna števila enaKa? Seštevanje in množenje kompleksnih števil.
        \item \colorbox{blue!30}{\textbf{Izrek.}} Ali je $(\CC, +, \cdot)$ polje?
        \begin{itemize}
            \item \colorbox{green!30}{\textbf{Dokaz.}} Preverimo aksiome $A1-A8$.
        \end{itemize}
        \item \colorbox{yellow!30}{\emph{Opomba.}} Ali lahko v $\CC$ vpeljamo ureditve, za kateri bi bila urejeni obseg?
        \item \colorbox{blue!30}{\textbf{Trditev.}} Preslikava, ki vloži množico realnih števil v množico kompleksnih števil. Ali je injektina? S čim lahko enačimo urejene pare oblike $(a, 0)$? Zakaj?
        \item \colorbox{purple!30}{\textbf{Definicija.}} Imaginarna enota.
        \item \colorbox{yellow!30}{\emph{Opomba.}} Čemu je enako $i^2$? Algebraični zapis kompleksnega števila.
        \item \colorbox{purple!30}{\textbf{Definicija.}} Naj bo $z \in \CC$. Realni del in imaginarni del $z$. Konjugirano število $z$. Absolutna vresnost $z$.
        \item \colorbox{blue!30}{\textbf{Trditev.}} 2 lastnosti konjugiranja. Čemu je enako $\text{Re} z$ in $\text{Im} z$? Čemu je enako $|z|^2, z \in \CC$?
        \begin{itemize}
            \item \colorbox{green!30}{\textbf{Dokaz.}} Poračunamo.
        \end{itemize}
        \item \colorbox{blue!30}{\textbf{Trditev.}} 6 lastnosi absolutne vrednosti kompleksnega števila. 
        \begin{itemize}
            \item \colorbox{green!30}{\textbf{Dokaz.}} Poračunamo.
        \end{itemize}
        \item \colorbox{blue!30}{\textbf{Izrek.}} Osnovni izrek algebre. \textcolor{red}{[Brez dokaza]}
    \end{itemize}

    \item[$\circ$] Geometrijska interpretacija kompleksnega števila
    \begin{itemize}
        \item Kako predstavimo kompleksno število?
        \item S čim se ujema seštevanje kompleksnih števil?
        \item Kaj je absolutna vrednost kompleksnega števila?
        \item Kako lahko interpretiramo trikotniško neenakost?
    \end{itemize}
    \item[$\circ$] Polarni zapis kompleksnega števila števila
    \begin{itemize}
        \item Kako je podana lega točke v izbranem koordinatnem sistemu?
        \item Kako dobimo razdalje do izhodišča in polarni kot?
        \item Kaj vemo, če je kompleksno število podano v polarnem zapisu? 
        \item \colorbox{purple!30}{\textbf{Definicija.}} Polarni zapis kompleksenga števila. 
        \item \colorbox{purple!30}{\textbf{Definicija.}} Argument kompleksnega števila.
        \item \colorbox{yellow!30}{\emph{Opomba.}} Kaj lahko povemo o številu $\cos \phi + i \sin \phi$?
    \end{itemize}

    \item[$\circ$] Množenje kompleksnih števil
    \begin{itemize}
        \item Formula za množenje komplekšnih števil v polarnem zapisu.
    \end{itemize}

    \item[$\circ$] Potenciranje
    \begin{itemize}
        \item Moivreova formula za potenciranje kompleksnega števila.
    \end{itemize}

    \item[$\circ$] Konjugiranje
    \begin{itemize}
        \item Konjugiranje v polarnem zapisu.
    \end{itemize}

    \item[$\circ$] Korenjenje kompeksnih števil
    \begin{itemize}
        \item \colorbox{purple!30}{\textbf{Definicija.}} $n$-ti koreni kompleksnega števila $z$.
        \item Formula za korenjenje v polarnem zapisu.
        \item Kaj sestavljajo rešitve enačbe $w^n = z$?
        \item \colorbox{yellow!30}{\emph{Opomba.}} Kaj je rešitve enačbe $w^n = 1$?
    \end{itemize}
\end{enumerate}