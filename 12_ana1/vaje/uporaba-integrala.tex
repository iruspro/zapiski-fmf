\section{UPORABA INTEGRALA V GEOMETRIJI}
{\color{Purple} \subsection*{Ploščine likov (naloga $26$)}}
Ploščina lika, ki ga omejuje krivulja v polarni obliki: $\displaystyle S = \frac{1}{2} \int_{\alpha}^{\beta} (r(\phi))^2  \,d \phi $.

\textbf{Ideja:} Lik aproksimiramo z unijo trikotnikov, ki imajo vrh v izhodišču in seštejemo jih ploščine $dS = \frac{1}{2} r(\phi)^2 sin(d \phi)$.

Ploščina lika, ki ga omejuje krivulja v parametrični obliki: $\displaystyle S = \frac{1}{2} \int_{a}^{b} (x \dot{y} - \dot{x}y )  \,d t $.

\textbf{Ideja:} Lik aproksimiramo z unijo trikotnikov, ki imajo vrh v izhodišču in seštejemo jih ploščine $dS = \frac{1}{2} |\vec{r}(t) \times \vec{r} \, 'dt|$.


{\color{Purple} \subsection*{Dolžine krivulj (naloga $27$)}}
Dolžina krivulje, podane v parametrični obliki: $\displaystyle l = \int_{a}^{b} \sqrt{\dot{x}^2 + \dot{y}^2}  \,dt $. V formuli je $\displaystyle \sqrt{\dot{x}^2 + \dot{y}^2}$ hitrost, $dt$ pa čas $(ds = vdt)$.

Dolžina krivulje, podane v polarni obliki: $\displaystyle l = \int_{\alpha}^{\beta} \sqrt{r^2 + (r')^2}  \,d \phi $.

\textbf{Ideja:}  V začetni formuli vzamemo za parameter $t = \phi: \ x(\phi) = r(\phi) \cos \phi, \ y(\phi) = r(\phi) \sin \phi$.

Dolžina krivulje, podane v kartezični obliki (krivulja je graf): $\displaystyle l = \int_{a}^{b} \sqrt{1 + f'(x)^2}  \,d x $.

\textbf{Ideja:}  V začetni formuli vzamemo za parameter $t = x: \ x(x) = x, \ y(x) = f(x)$.

{\color{Purple} \subsection*{Površina in prostornina vrtenine (naloge $28-30$)}}
\textbf{\textcolor{Orange}{Opazka.}} Najprej ugotovimo, ali integral sploh obstaja.

Formula za volumen vrtenine  v kartezični obliki: $\displaystyle V = \pi \int_{a}^{b} f(x)^2  \,dx $.

\textbf{Ideja:}  Vrtenino aproksimiramo z unijo valjev. En tak majhen valj ima višino $dx$ in polmer osnovne ploskve $R = f(x)$. \newline Potem $dV = \pi R^2h = \pi f(x)^2dx$.

Formula za površino vrtenine v kartezični obliki: $\displaystyle P = 2 \pi \int_{a}^{b} f(x) \sqrt{1 + f'(x)^2}  \,dx $.

\textbf{Ideja:} Plašč vrtenine aproksimiramo z unijo plaščev prisekanih stožcev. \newline Potem $dP = \text{osnovnica} \cdot \text{širina} = 2 \pi R \, ds = 2 \pi f(x) \sqrt{1 + f'(x)^2}  \,dx $

Formula za površino vrtenine v polarni obliki: $\displaystyle P = 2 \pi \int_{\alpha}^{\beta} r(\phi) \sin \phi \sqrt{r^2 + (r')^2}  \,d \phi $

Formula za površino vrtenine v parametrični obliki: $\displaystyle P =  2 \pi \int_{t_1}^{t_2} y(t) \sqrt{\dot{x}^2 + \dot{y}^2}   \,dt $

\textbf{Ideja:}  Vzamemo formulo v kartezični obliki in iz njo izpeljamo nove.

Vrtenje okoli poljubne osi: 
\begin{enumerate}
    \item Parametiziramo točke na osi vrtenja
    \item Pogledamo kako daleč pravokotno na našo is točka na krivulje
    \item Uporabimo znano formula za volumen v kartezični verziji
\end{enumerate}
Guldinova formula: $V = 2 \pi d S$, kjer je $S$ ploščina lika, $d$ pa razdalja središča lika od osi vrtenja (za preproste like).


