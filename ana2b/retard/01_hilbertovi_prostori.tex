\section{Hilbertovi prostori}
\subsection{Vektorski prostori s skalarnim produktom}
Naj bo \(X\) vektorski prostor nad \(\R\) (ali nad \(\C\)).
\begin{definicija}
    \df{Skalarni produkt} \(\scp{\ }{}\) je preslikava \(\scp{\ }{}: X \times X \to \R \text{ (oz. \(\C\))}\) za katero velja:
    \begin{enumerate}
        \item \(\all{x \in X} \scp{x}{x} \geq 0\);
        \item \(\all{x \in X} \scp{x}{x} = 0 \liff x = 0\);
        \item \(\all{x, y \in X} \scp{x}{y} = \overline{\scp{y}{x}}\);
        \item \(\all{x,y,z \in X} \all{\lambda, \mu \in \R \text{ (oz. \(\C\))}} \scp{\lambda x + \mu y}{z} = \lambda \scp{x}{z} + \mu \scp{y}{z}\). 
    \end{enumerate}
\end{definicija}

\begin{opomba}
    1.\ -- 2.\ je \df{pozitivna definitnost} skalarnega produkta, 3.\ je \df{poševna simetričnost} (\df{simetričnost} nad \(\R\)), 4.\ je linearnost v prvem faktorju.
\end{opomba}

\begin{trditev}[Cauchy-Schwartzova neenakost]
    Naj bo \(\scp{\ }{}\) skalarni produkt na \(X\). Velja: 
    \[\all{x,y \in X} |\scp{x}{y}| \leq \sqrt{\scp{x}{x}} \cdot \sqrt{\scp{y}{y}} = \norm{x} \cdot \norm{y}.\]
\end{trditev}

\begin{proof}
    Nad \(\R\): Definiramo \(t \to \scp{x + ty}{x + ty} = f(t) \geq 0\).

    Nad \(\C\): Naj bo \(x, y \in X\). Obstaja \(\alpha \in \C, \ |\alpha| = 1\), da \(\scp{x}{y} = \alpha \cdot |\scp{x}{y}|\).
\end{proof}

\begin{definicija}
    Norma na vektorskem prostoru \(X\) je preslikava \(\norm{\ }: X \to \R\) za katero velja:
    \begin{enumerate}
        \item \(\all{x \in X} \norm{x} \geq 0\);
        \item \(\all{x \in X} \norm{x} = 0 \liff x = 0\);
        \item \(\all{\lambda \in \R \text{ (oz. \(\C\))}} \norm{\lambda x} = |\lambda| \cdot \norm{x}\);
        \item Trikotniška neenakost: \(\all{x, y \in X} \norm{x + y} \leq \norm{x} + \norm{y}\).
    \end{enumerate}
\end{definicija}

\begin{trditev}
    Naj bo \((X, \scp{\ }{})\) vektorski prostor s skalarnim produktom. Potem je \((X, \norm{\ })\), kjer je \(\all{x \in X} \norm{x} = \sqrt{\scp{x}{x}}\), vektorski prostor z normo.
\end{trditev}

\begin{proof}
    Preverimo lastnosti. Za trikotniško neenakost uporabimo CS neenakost.
\end{proof}

\begin{trditev}
    Naj bo \((X, \norm{\ })\) vektorski prostor s normo. Potem je \((X, d)\), kjer je \(\all{x, y \in X}d(x,y) = \norm{x - y}\), metrični prostor.
\end{trditev}

\begin{proof}
    Preverimo lastnosti. 
\end{proof}

\subsection{Hilbertovi prostori}
\begin{definicija}
    \df{Hilbertov prostor} je vektorski prostor \(X\) s skalarnim produktom \(\scp{\ }{}\), ki je v metriki, porojeni iz skalarnega produkta, poln metrični prostor.
\end{definicija}

\begin{opomba}
    \((X, \scp{\ }{}) \leadsto (X, \norm{\ }) \leadsto (X, d)\), kjer je \(\all{x, y \in X}d(x,y) = \norm{x - y}\).
\end{opomba}

\begin{opomba}
    \df{Banachov prostor} je vektorski prostor \(X\) z normo \(\norm{\ }\), ki je v metriki, porojeni iz norme, poln metrični prostor.
\end{opomba}

\begin{zgled}
\
\begin{enumerate}
    \item Naj bo \(X = \R^n\). Definiramo skalarni produkt. Naj bo \(x, y \in \R^n, \ x = (x_1 \ldots, x_n), \ y = (y_1, \ldots, y_n)\). \df{Standardni skalarni produkt} je \[x \cdot y = \sum_{k=1}^{n} x_ky_k.\]
    Ta skalarni produkt nam da normo \[\norm{x} = \sqrt{\sum_{k=1}^{n} x_k^2},\]
    ki porodi metriko \[d_2(x,y) = \sqrt{\sum_{k=1}^{n} (x_k - y_k)^2}.\]
    Vemo, da je \((\R^n, d_2)\) poln metrični prostor. Torej \((\R^n, \cdot)\) Hilbertov prostor.
    \newpage
    \item Na \(\R^n\) lahko definiramo tudi druge norme, npr.
    \begin{itemize}
        \item \(\norm{x}_\infty = \max \set{|x_1|, \ldots, |x_n|}\);
        \item \(\norm{x}_1 = |x_1| + \ldots + |x_n|\). 
    \end{itemize}
    Te dve normi ne prideta iz skalarnega produkta, ker za njih ne velja paralelogramsko pravilo.

    \((\R^n, \norm{x}_\infty)\) in \((\R^n, \norm{x}_1)\) sta Banachova prostora.
    \item Naj bo \(X = \C^n\). Definiramo skalarni produkt. Naj bo \(z, w \in \C^n, \ z = (z_1 \ldots, z_n), \ w = (w_1, \ldots, w_n)\). \df{Standardni skalarni produkt} je \[z \cdot w = \sum_{k=1}^{n} z_k\overline{w_k}.\]
    Ta skalarni produkt nam da normo \[\norm{z} = \sqrt{\sum_{k=1}^{n} |z_k|^2},\]
    ki porodi metriko \[d_2(z,w) = \sqrt{\sum_{k=1}^{n} |z_k - w_k|^2}.\]
    Vemo, da je \((\C^n, d_2)\) poln metrični prostor. Torej \((\C^n, \cdot)\) Hilbertov prostor.
\end{enumerate}
\end{zgled}

\subsection{Prostor \(L^2([a, b])\)}
\begin{trditev}
    Naj bo \(C([a,b])\) vektorski prostor nad \(\R\). Potem je s predpisom \[\all{f, g \in C([a,b])} \scp{f}{g} = \int_{a}^{b} f(x) g(x) \, dx\]
    definiran skalarni produkt na \(\cf{}\).
\end{trditev}

\begin{proof}
    Preverimo lastnosti.
\end{proof}

\begin{trditev}
    \((\cf{}, \scp{\ }{})\) ni Hilbertov prostor.
\end{trditev}

\begin{proof}
    Definiramo \(f_n(x) = \begin{cases}
        1; &\frac{1}{n} \leq x \leq 1 \\
        nx; &-\frac{1}{n} < x < \frac{1}{n} \\
        -1; &-1 \leq x \leq -\frac{1}{n}
    \end{cases}.\) Pokažemo, da je \((f_n)_n\) Cauchyjevo zaporedje v \(\cf{}\), ki nima limite.
\end{proof}

\begin{definicija}
    Naj bo \((M, d)\) metrični prostor. Pravimo, da lahko \df{napolnimo} prostor \(M\), če obstaja prostor \((\overline{M}, \overline{d})\), za kateri velja:
    \begin{enumerate}
        \item \((\overline{M}, \overline{d})\) je poln metrični prostor;
        \item \(M \subseteq \overline{M}\);
        \item \(\overline{d}|_{M \times M} = d\);
        \item \(M\) je gost v \(\overline{M}\), tj. \(\Cl M = \overline{M}\).
    \end{enumerate}
    Prostoru \(\overline{M}\) rečemo \df{napolnitev} prostora \(M\).
\end{definicija}

\begin{opomba}
    Ideja: \(\overline{M}\) je prostor vseh limit Cauchyjevih zaporedij v \(M\) (+ kvocient).
\end{opomba}

\begin{primer}
    Naj bo \(M = (0,1), \ d_2(x,y) = |x-y|\). Potem napolnitev \(\overline{M}\) prostora \(M\) je \(\overline{M} = [0, 1]\).
\end{primer}

\begin{opomba}
    Označili smo z \(L^1(A) = \setb{f: A \to \R}{\int_{A} |f| \, dx \ \text{obstaja}} /_\sim\) prostor vseh absolutno integrabilnih funkcij, kjer je  \(\all{f, g \in L^1} f \sim g \liff f = g \ \text{s.p.}\)
\end{opomba}

Vpeljemo zdaj s kvadratom integrabilne funkcije: 
\[L^2([a, b]) = \setb{f: [a,b] \to \R}{\int_{a}^{b} f^2(x) \, dx \ \text{obstaja}}/_\sim,\]
kjer je \(\all{f, g \in L^2} f \sim g \liff f = g \ \text{s.p.}\)

V tem prostoru gotovo so 
\begin{itemize}
    \item Zvezne funkcije: \(C([a,b]) \subseteq L^2([a,b])\);
    \item Odsekoma zvezni funkciji;
    \item \(f(x) = \frac{1}{\sqrt[4]{x-a}}\) itd.
\end{itemize}

\newpage
\paragraph{Cilj} Želimo posplošiti prostor \((\R, \cdot)\).

Naj bo \(f, g \in L^2\), potem \(|f \cdot g| \leq \frac{|f|^2 + |g|^2}{2} \lthen f \cdot g \in L^1([a, b])\). Torej lahko definiramo 
\[\scp{f}{g} = \int_{a}^{b}f(x)g(x) \, dx.\]

\begin{trditev}
    \(L^2([a,b])\) je vektorski prostor.
\end{trditev}

\begin{proof}
    Preverimo lastnosti.
\end{proof}

Torej \(L^2([a, b])\) je vektorski prostor nad \(\R\) s skalarnim produktom \[\scp{f}{g} = \int_{a}^{b}f(x)g(x) \, dx.\]

Očitno, da je \(C([a,b]) \leq L^2([a, b])\).

\begin{izrek}
    \(L^2([a,b])\) je Hilbertov in \(L^2([a,b])\) je napolnitev \(C([a,b])\).
\end{izrek}

\begin{opomba}
    \[\all{f \in L^2([a,b])} \some{f_n \in C([a,b])} \lim_{n \to \infty}  f_n = f\ \liff \lim_{n \to \infty} \norm{f_n - f} = 0 \liff \lim_{n \to \infty} \sqrt{\int_{a}^{b}(f_n(x) - f(x))^2} \, dx = 0\]
\end{opomba}

\begin{opomba}
    Nad \(\C\): \(f = u + iv, \ u, v: [a, b] \to \R\). Potem 
    \[\int_{a}^{b}f(x) \, dx = \int_{a}^{b}u(x) \, dx + i\int_{a}^{b}v(x) \, dx \]
    in 
    \[\scp{f}{g} = \int_{a}^{b}f(x) \overline{g(x)} \, dx.\]
\end{opomba}

\begin{zgled}
    Vzemimo \([0, 1]\). Definiramo \(f_n(x) = \begin{cases}
        \sqrt{n}; &0 < x \leq \frac{1}{n} \\ 0; &\text{sicer}
    \end{cases}.\)

    Velja: \(\lim_{n \to \infty} f_n(x) = 0\) za \(x \in [0, 1]\). Ali je \(\lim_{n \to \infty} f_n(x) = 0\) v \(L^2([0,1])\)?
\end{zgled}

\begin{zgled}
    \todo
\end{zgled}

\subsection{Ortogonalnost}
\begin{definicija}
    Naj bo \((X, \scp{\ }{})\) vektorski prostor s skalarnim produktom, \(A \subseteq X, \ A \neq \emptyset\). Naj bosta \(x, y \in X\).
    \begin{itemize}
        \item \(x\) je \df{pravokoten} na \(y\), če \(\scp{x}{y} = 0\), tj. \(x \perp y \liff \scp{x}{y} = 0\).
        \item \df{Ortogonalni komplement} množice \(A\) je \(A^\perp = \setb{x \in X}{\all{a \in A} x \perp a}\).
    \end{itemize}
\end{definicija}

\begin{trditev}
    \(A^\perp\) je vektorski podprostor v \(X\).
\end{trditev}

\begin{proof}
    Preverimo homogenost in linearnost.
\end{proof}

\begin{opomba}
    \(A \subseteq (A^\perp)^\perp\).
\end{opomba}

\begin{trditev}
    Naj bo \((X, \scp{\ }{})\) vektorski prostor s skalarnim produktom, \(v \in X\). Definiramo \(f: X \to \R, \ f(x) = \scp{x}{v}\). Potem \(f\) je zvezna na \(X\).
\end{trditev}

\begin{proof}
    Pokažemo, da je \(f\) Lipshitzeva.
\end{proof}

\begin{posledica}
    \(A^\perp\) je zaprt vektorski podprostor.
\end{posledica}

\begin{proof}
    Pokažemo, da je limita vsakega zaporedja v \(A^\perp\) tudi leži v \(A^\perp\).
\end{proof}

\begin{opomba}
    \(C([a, b]) \subseteq L^2([a,b])\) ni zaprt podprostor.
\end{opomba}

\begin{opomba}
    Če je \((X, \scp{\ }{})\) Hilbertov in \(A \subseteq X\) zaprt podprostor, potem \[(A^\perp)^\perp = A.\]
\end{opomba}

\begin{trditev}[Pitagorjev izrek]
    Naj bo \((X, \scp{\ }{})\) vektorski prostor s skalarnim produktom. Naj bodo \(x_1, \ldots, x_n \in X\) taki, da \(\all{i, j \in [n]} i \neq j \lthen x_j \perp x_j\). Tedaj 
    \[\norm{x_1 + \ldots + x_n}^2 = \norm{x_1}^2 + \ldots + \norm{x_n}^2.\]
\end{trditev}

\begin{proof}
    Izračunamo normo po definiciji.
\end{proof}

\newpage
\begin{definicija}
    Naj bo \((X, \scp{\ }{})\) vektorski prostor s skalarnim produktom in \(Y \leq X\) podprostor X. Naj bo \(x \in X\). \df{Pravokotna projekcija} vektorja \(x\) na podprostor \(Y\) (če obstaja) je tak vektor \(P_Y(x) \in Y\), da je 
    \[x - P_Y(x) \in Y^\perp.\]
\end{definicija}

\begin{trditev}
    Če je pravokotna projekcija \(x\) na \(Y\) obstaja, je enolično določena. Če obstaja, je to najboljša aproksimacija vektorja \(x\) z vektorji iz \(Y\), tj.
    \[\norm{x - P_y(x)} = \min_{w \in Y} \norm{x - w}.\]
\end{trditev}

\begin{proof}
    Enoličnost: Običajen način.

    Aproksimacija: Pitagorjev izrek.
\end{proof}

\begin{zgled}
    Naj bosta \(Y = C([a,b])\) in \(X = L^2([a, b])\). Če si izberimo \(f \in X \setminus Y\), potem \(f\) nima najboljše aproksimacije z zveznimi funkcijami.
\end{zgled}
