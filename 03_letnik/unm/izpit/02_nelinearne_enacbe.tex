\section{Nelinearne enačbe}
\s{Bisekcija}\\
\textbf{Izr.} \(\forall f \in C([a,b]) \land f(a) \cdot f(b) < 0 \lthen \some{c \in (a,b)} f(c) = 0\).

\boxedalgorithm{
  \(s \gets b - a\) \;
    \While{\(|s| > \eps\)}{  
        \(s \gets s / 2\) \;
        \(c \gets a + s\) \;
        \eIf{\(\sgn(f(a)) = \sgn(f(c))\)}{
            \(a \gets c\) \;
        }{
            \(b \gets c\) \;
        }
    }
}
\begin{itemize}
  \item Ne računa ničel sode stopnje ali kompleksne ničle.
  \item Lahko izračuna lihi pol. Da se izognemo težavam s poli, lahko definiramo novo funkcijo, ki ima enake ničle, vendar nima polov.
\end{itemize}
\s{Navadna iteracija}\\
Enačbo \(f(x) = 0\) prepišemo v ekvivalentno obliko \(g(x) = x\). Funkcijo \(g\) imenujemo \emph{iteracijska} funkcija, saj rešitev iščemo z iteracijo \[x_{r+1} = g(x_r),\ r = 0, 1, \ldots\]
Če je \(g\) na določenem intervalu. ki vsebuje \(x_0\) skrčitev, zaporedje \((x_r)_r\) konvergira k negibni točki \(g\), ki ustreza ničli funkcije \(f\).
\begin{itemize}
    \item Konvergenco preverjamo z odvodom: \((x_r)_r \to \alpha \liff |g'(\alpha)| < 1\). 
    \item Zaporedje \((x_r)_r \to \alpha\) z redom \(p\), če \(g^{(p)}(\alpha) \neq 0,\ g^{(i)}(\alpha) = 0,\ i < p\). V primeru \(p = 1\) mora veljati še \(|g'(\alpha)| < 1\).
\end{itemize}
\textbf{Ocena konvergence:} \(|x_{r+1} - \alpha| \leq \frac{q}{1-q}|x_{r+1}-x_r|\)
\columnbreak

\s{Tangentna metoda}\\
Je poseben primer navadne iteracije:
\(
    g(x) = x - \frac{f(x)}{f'(x)}.
\)
\begin{itemize}
    \item Naj bo \(f \in C^2\). Tedaj vse enostavne ničle so privlačne.
    \item \(\forall f \in C^2([a, \infty)) \land f' > 0 \land f'' > 0 \lthen \all{x_0 \in [a, \infty)} x_r \to \alpha\).
    \item \(\forall f \in C^2([a, \infty)) \land f' < 0 \land f'' < 0 \lthen \all{x_0 \in [a, \infty)} x_r \to \alpha\).
\end{itemize}
\textbf{Red konvergence} ugotovimo z odvodom. Prvi je \(g'(x) = \frac{f(x)f''(x)}{f'^2(x)}\).\\
\begin{itemize}
    \item Če je \(f'(\alpha) \neq 0\), konvergenca vsaj kvadratična in \(g''(\alpha) = \frac{f''(\alpha)}{f'(\alpha)}\).
    \item Če je \(f'(\alpha) = 0\), konvergenca linearna.
\end{itemize}
\s{Sekantna metoda}\\
Približki računamo z rekurzivno formulo: \(x_{r+1} = x_r - \frac{f(x_r)(x_r-x_{r-1})}{f(x_r)-f(x_{r-1})}\).
\s{Nasveti}
\begin{itemize}
    \item Pri računanju koeficientov treba paziti, da je \(\alpha\) res negibna točka.
    \item Poglejmo, če po \(k\) korakih pridemo v območje konvergence.
\end{itemize}