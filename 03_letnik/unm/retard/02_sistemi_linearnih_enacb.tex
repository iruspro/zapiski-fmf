\section{Sistemi linearnih enačb}
Naj bo \(\F \in \set{\R, \C}\).

\subsection{Uvod}
Rešujemo sistem linearnih enačb 
\begin{align*}
    a_{11}x_1 + \cdots + a_{1n}x_n &= b_1 \\
    \vdots \\
    a_{m1}x_1 + \cdots + a_{mn}x_n &= b_n,
\end{align*}
ki ga pišemo v obliki
\[Ax = b,\]
kjer 
\[
    A = \begin{bmatrix}
        a_{11} & \ldots & a_{1n} \\
        \vdots & & \vdots \\
        a_{m1} & \ldots & a_{mn} \\
    \end{bmatrix}, \quad 
    x = \begin{bmatrix}
        x_1 \\ \vdots \\ x_n
    \end{bmatrix}, \quad
    b = \begin{bmatrix}
        b_1 \\ \vdots \\ b_m
    \end{bmatrix}.
\]

\

Označimo z \(a_i\) \(i\)-ti stolpec matrike \(A\) in z \(\alpha_j^T\) \(j\)-to vrstico matrike \(A\). Naj bo 
\[
    e_i = [0\ \ldots\ 0\ 1_i\ 0\ \ldots\ 0]^T.
\]
Tedaj velja
\[
A e_i = a_i, \quad e_j^T A = \alpha_j^T \quad \text{in} \quad e_i^T A e_k = a_{ik}.
\]

\

Naj bo \(x, y \in \F^n\). Standardni skalarni produkt je 
\[
    \langle x, y \rangle = \sum_{i=1}^{n} x_i \overline{y_i} = y^H x,
\]
kjer je 
\[
    A^H = \overline{A^T}
\] 
hermitsko transponiranje.

\

Če imamo sistem \(y = Ax,\ A \in \R^{n \times n},\ x, y \in \R^n\), si to lahko predstavljamo:
\begin{enumerate}
    \item Po elementih, tj.\
    \[
        y_i = \sum_{k=1}^{n} a_{ik} x_k = \alpha_i^T X.
    \]
    \item Kot linearno kombinacijo stolpcev, tj.\ 
    \[
        y = \sum_{i=1}^{n} x_i a_i \lthen e_i^T y = \sum_{i=1}^{n} x_ke_i^T a_i.
    \]
\end{enumerate}

\newpage
Če imamo produkt \(C = A \cdot B\), kjer so \(A, B, C \in \R^n\), si to lahko predstavljamo:
\begin{enumerate}
    \item Po elementih, tj.\
    \[
        c_{ik} = \sum_{l=1}^{n} a_{il} b_{lk} = \alpha_i^T b_k.
    \]
    \item Po stolpcih, tj.\
    \[
        A \cdot [b_1\ \ldots\ b_n] = [A \cdot b_1\ \ldots\ A \cdot b_n] \lthen c_i = A \cdot b_i.
    \]
    \item Po vrsticah, tj.\
    \[
        \begin{bmatrix}
            \alpha_1^T \\ \vdots \\ \alpha_n^T
        \end{bmatrix} \cdot B = 
        \begin{bmatrix}
            \alpha_1^T \cdot B\\ \vdots \\ \alpha_n^T \cdot B
        \end{bmatrix} \lthen \gamma_i^T = \alpha_i^T \cdot B.
    \]
    \item Kot vsoto \(n\) matrik ranga \(1\) (diade), tj.\
    \[
        C = \sum_{l=1}^{n} a_l \beta_l^T.
    \]

    Vemo, da če \(x, y \in \R^n,\ x, y \neq 0\), potem \(xy^T\) matrika ranga 1.
\end{enumerate}

\

Spomnimo se nekaj osnovnih definicij in trditev. Naj bo \(A \in \F^{n \times n}\).
\begin{trditev}
    Naj bo \(A \in \F^{n \times n}\). NTSE:
    \begin{enumerate}
        \item \(A \in \F^{n \times n}\) je \emph{nesingularna}.
        \item \(\some{A^{-1}} A \cdot A^{-1} = A^{-1} \cdot A = I\).
        \item \(\det A \neq 0\).
        \item \(\mathop{rang} A = n\).
        \item \(\ker A = \set{0}\).
        \item Vse lastne vrednosti so neničelne.
    \end{enumerate}
\end{trditev}

\begin{definicija} Naj bo \(A \in \F^{n \times n}\).
    \begin{itemize}
        \item \(\lambda \in \F\) je \emph{lastna vrednost} matrike \(A\), če
        \[
            \some{x \in \F^n \setminus \set{0}} Ax = \lambda x.
        \]
        \item Matrika \(A\) je \emph{simetrična}, če 
        \[
            A^T = A,
        \]
        \emph{hermitska}, če
        \[
            A^H = A.
        \]
        \item Matrika \(A\) je \emph{simetrična pozitivno definitna}, če
        \[
            A = A^T \quad \text{in} \quad \all{x \in \F^n \setminus \set{0}} x^T A x > 0,
        \]
        \emph{simetrična pozitivno semidefinitna}, če
        \[
            A = A^T \quad \text{in} \quad \all{x \in \F^n \setminus \set{0}} x^T A x \geq 0,
        \]
        \emph{hermitska pozitivno semidefinitna}, če
        \[
            A = A^H \quad \text{in} \quad \all{x \in \F^n \setminus \set{0}} x^H A x > 0,
        \]
        \emph{hermitska pozitivno semidefinitna}, če
        \[
            A = A^H \quad \text{in} \quad \all{x \in \F^n \setminus \set{0}} x^H A x \geq 0,
        \]
    \end{itemize}
\end{definicija}

\subsection{Vektorske in matrične norme}
\begin{definicija}
    \emph{Vektorska norma} je preslikava \(\norm{\cdot}: \C^n \to \R\), za katero velja:
    \begin{enumerate}
        \item Pozitivna definitnost, tj.\ \(\all{x \in \C^n} \norm{x} \geq 0 \land \norm{x} = 0 \liff x = 0\).
        \item Homogenost, tj.\ \(\all{\lambda \in \C} \all{x \in \C} \norm{\lambda x} = |\lambda| \cdot \norm{x}\).
        \item Trikotniška neenakost, tj.\ \(\all{x, y \in \C^n} \norm{x + y} \leq \norm{x} + \norm{y}\).
    \end{enumerate}
\end{definicija}

Najbolj znane norme so 
\begin{align*}
    \norm{x}_1 &= \sum_{i=1}^{n} |x_i| \\
    \norm{x}_2 &= (\sum_{i=1}^{n} |x_i|^2)^{1/2} \\
    \norm{x}_\infty &= \max \setb{|x_i|}{i \in \set{1, \ldots, n}} \\
\end{align*}

Vse vektorske norme so ekvivalentne, tj.\
\[
    \all{\norm{\cdot}_a, \norm{\cdot}_b \in \R^{\C^n}} \all{x \in \C^n} \some{c_1, c_2 > 0} c_1 \norm{x}_a \leq \norm{x}_b \leq c_2 \norm{x}_a.
\]

Za \(\norm{\cdot}_1, \norm{\cdot}_2, \norm{\cdot}_\infty\) velja:
\begin{align*}
    \norm{x}_2 &\leq \norm{x}_1 \leq \sqrt{n} \cdot \norm{x}_2 \\
    \norm{x}_\infty &\leq \norm{x}_2 \leq \sqrt{n} \cdot \norm{x}_\infty \\
    \norm{x}_\infty &\leq \norm{x}_1 \leq n \cdot \norm{x}_\infty \\
\end{align*}

\begin{definicija}
    \emph{Matrična norma} je preslikava \(\norm{\cdot}: \C^{n \times n} \to \R\), za katero velja:
    \begin{enumerate}
        \item Pozitivna definitnost.
        \item Homogenost.
        \item Trikotniška neenakost.
        \item Submultiplikativnost, tj.\ \(\all{A, B \in \C^{n \times n}} \norm{A \cdot B} \leq \norm{A} \cdot \norm{B}\).
    \end{enumerate}
\end{definicija}

\begin{definicija}
    \emph{Vektorizacija} matrike \(A \in \F^{n \times n}\) je preslikava 
    \begin{align*}
        \text{vec:}\ &\C^{n \times n} \to C^{n^2} \\
        &A \mapsto [a_{11}\ \ldots\ a_{1n}\ a_{21}\ \ldots\ a_{nn}]^T
    \end{align*}
\end{definicija}
Sedaj lahko definiramo vektorske norme 
\begin{align*}
    N_1(A) &= \norm{\text{vec}(A)}_1 = \sum_{i, j = 1}^{n} |a_{ij}| \\ 
    N_2(A) &= \norm{\text{vec}(A)}_2 = (\sum_{i, j = 1}^{n} |a_{ij}|^2)^{1/2} \\ 
    N_\infty(A) &= \norm{\text{vec}(A)}_\infty = \max \setb{|x_i|}{i \in \set{1, \ldots, n}} \\ 
\end{align*}

\begin{trditev}
    Normi \(N_1\) in \(N_2\) sta matrični normi. Norma \(N_\infty\) ni matrična norma.
\end{trditev}

\begin{proof}
    \todo
\end{proof}

\begin{definicija}
    Matrična norma \(N_2\) je \emph{Frobeniusova norma}: 
    \[
        \norm{A}_F = \left(\sum_{i, j = 1}^{n} |a_{ij}|^2\right)^{1/2}.
    \]
\end{definicija}

\begin{lema}
    Naj bo \(\norm{\cdot}_v\) vektorska norma na \(\C^n\). Tedaj je 
    \[
        \norm{A}_v := \max_{x \neq 0} \frac{\norm{Ax}_v}{\norm{x}_v}
    \]
    matrična norma. Rečemo ji \emph{operatorska norma}.
\end{lema}

\begin{proof}
    \todo
\end{proof}

Sedaj lahko definiramo \emph{operatorsko \(p\)-normo}:
\[
    \norm{A}_p := \max_{x \neq 0} \frac{\norm{Ax}_p}{\norm{x}_p}\ \text{za} \ p \in \set{1, 2, \ldots, \infty}.
\]

\begin{lema}
    Naj bo \(A \in \C^{n \times n}\). Tedaj 
    \[
        \norm{A}_1 = \max \setb{\sum_{i=1}^{n} |a_{ij}|}{j \in \set{1, \ldots, n}} = \max \setb{\norm{a_j}_1}{j \in \set{1, \ldots, n}}.
    \]
\end{lema}

\begin{proof}
    \todo
\end{proof}

\begin{lema}
    Naj bo \(A \in \C^{n \times n}\). Tedaj 
    \[
        \norm{A}_\infty = \max \setb{\sum_{j=1}^{n} |a_{ij}|}{i \in \set{1, \ldots, n}} = \max \setb{\norm{\alpha_i^T}_1}{i \in \set{1, \ldots, n}}.
    \]
\end{lema}

\ 

Naj bo \(A \in \C^{n \times n}\). Matrika \(B = A^HA\) je hermitska in pozitivno semidefinitna, saj
\begin{enumerate}
    \item \(B^H = (A^HA)^H = A^H(A^H)^H = A^HA = B\) in 
    \item \(x^H B x = x^H A^H A x = (Ax)^H Ax = \norm{Ax}_2^2 \geq 0\).
\end{enumerate}

Vemo, da so torej vse njene lastne vrednosti nenegativne, zato jih lahko pišemo in uredimo kot 
\[
    \sigma_1^2 \geq \sigma_2^2 \geq \ldots \geq \sigma_n^2 \geq 0.
\]

Vrednostim \(\sigma_1^2, \sigma_2^2, \ldots, \sigma_n^2\) pravimo \emph{singularne vrednosti} matrike \(A\).

\begin{lema}
    Naj bo \(A \in \C^{n \times n}\). Tedaj 
    \[
        \norm{A}_2 = \sigma_1(A) = \sqrt{\lambda_{\max} (A^HA)}
    \]
\end{lema}

\begin{proof}
    \todo
\end{proof}

\begin{definicija}
    Norma \(\norm{\cdot}\) je \emph{spektralna norma}.
\end{definicija}

\begin{opomba}
    Vsako matriko \(A \in \C^{n \times n}\) lahko zapišemo v obliki
    \[
        A = U \Sigma V^T,
    \]
    kjer sta \(U, V\) ortogonalni matriki in 
    \[
        \Sigma = \begin{bmatrix}
            \sigma_1 \\
            & \ddots \\
            & & \sigma_n
        \end{bmatrix}.
    \]
    To je \emph{singularni razcep} matrike \(A\).
\end{opomba} 

\begin{opomba}
    Frobeniusova norma ni operatorska norma, saj je \(\norm{I}_F = \sqrt{n}\). Za operatorske norma pa velja, da je \(\norm{I} = 1\).
\end{opomba}

Tudi vse matrične norme so ekvivalentne. Velja:
\begin{align*}
    \frac{1}{\sqrt{n}} \cdot \norm{A}_F &\leq \norm{A}_2 \leq \norm{A}_F \\
    \frac{1}{\sqrt{n}} \cdot \norm{A}_1 &\leq \norm{A}_2 \leq \sqrt{n} \cdot \norm{A}_1 \\
    \frac{1}{\sqrt{n}} \cdot \norm{A}_\infty &\leq \norm{A}_2 \leq \sqrt{n} \cdot \norm{A}_\infty \\
\end{align*}

Poleg tega velja še:
\begin{align*}
    N_\infty(A) &\leq \norm{A}_2 \leq n \cdot N_\infty(A) \\
    \norm{A}_2 &\leq \sqrt{\norm{A}_1 \cdot \norm{A}_\infty} \\
    \all{i \in \set{1, \ldots, n}} \norm{a_i}_2, \norm{\alpha_i^T}_2 &\leq \norm{A}_2.
\end{align*}

\

Naj bo \(\norm{\cdot}_m\) operatorska norma z vektorsko normo \(\norm{\cdot}_v\). Tedaj
\[
    \all{A \in \C^{n \times n}} \all{x \in \C^n} \norm{Ax}_v \leq \norm{A}_m \cdot \norm{x}_v.
\]

\begin{definicija}
    Če za matrično normo \(\norm{\cdot}_m\) in vektorsko normo \(\norm{\cdot}_v\) za vsak par \(A \in C^{n \times n},\ x \in \C^n\) velja:
    \[
        \norm{Ax}_v \leq \norm{A}_m \cdot \norm{x}_v,
    \]
    pravimo, da sta normi \emph{usklajeni}.
\end{definicija}

\begin{lema}
Za vsako matrično normo \(\norm{\cdot}_m\) obstaja usklajena vektorska norma \(\norm{\cdot}_v\).
\end{lema}

\begin{proof}
    \todo
\end{proof}

\begin{lema}
    Naj bo \(\lambda \in \C\) lastna vrednost matrike \(A \in \C^{n \times n}\). Potem za poljubno matrično normo \(\norm{\cdot}_m\) velja:
    \[
        |\lambda| \leq  \norm{A}_m.
    \]
\end{lema}

\begin{proof}
    \todo
\end{proof}

\

Vse norme \(\norm{\cdot}_1, \norm{\cdot}_2, \norm{\cdot}_\infty, \norm{\cdot}_F\) se enostavno razširijo na pravokotne matrike \(A \in \C^{m \times n}\). Če sta \(A, B\) ustreznih velikosti, da obstaja produkt \(A \cdot B\), potem velja tudi submultiplikativnost. Ne veljajo pa vse prejšnje ocene za norme.

Velja pa 
\begin{align*}
    \norm{A}_2 &= \norm{A^H}_2 \\
    \norm{A}_F &= \norm{A^H}_F \\
    \norm{A}_1 &= \norm{A^H}_\infty
\end{align*}

\

Matrika \(U \in \C^{n \times n}\) je \emph{unitarna}, če \(U \cdot U^H = U^HU = I\).

\begin{lema}
    Normi \(\norm{\cdot}_2\) in \(\norm{\cdot}_F\) sta invariantni na množenje z unitarno matriko.
\end{lema}

\begin{proof}
    \todo
\end{proof}

\begin{lema}
    Naj za \(X\) velja \(\norm{X} < 1\). Tedaj 
    \begin{enumerate}
        \item Matrika \(I - X\) je obrnljiva.
        \item \((I-X)^{-1} = \sum_{k=0}^{\infty} X^k\).
        \item Če je \(\norm{I} = 1\), potem \(\norm{(I - X)^{-1}} \leq \frac{1}{1 - \norm{X}}\).
    \end{enumerate}
\end{lema}

\begin{proof}
    \todo
\end{proof}

\subsection{Občutljivost sistemov linearnih enačb}
\begin{definicija}
    Naj bo \(A \in \C^{n \times n}\). \emph{Občutljivost} ali \emph{pogojnostno število} matrike \(A\) je 
    \[
        \kappa(A) := \norm{A} \cdot \norm{A^{-1}}.
    \]
\end{definicija} 

\begin{opomba}
    Velja ocena:
    \[
    \all{A \in \C^{n \times n}} \norm{I} = \norm{A \cdot A^{-1}} \leq \norm{A} \cdot \norm{A^{-1}} = \kappa(A).
    \]
\end{opomba}

\begin{izrek}
    Naj bo \(A \in \C^{n \times n}\) nesingularna matrika in \(Ax = b\). Če \(A\) zmotimo v \(A + \Delta A\) in \(b\) v \(b + \Delta b\), kjer velja 
    \[
        \norm{\Delta A} \leq \frac{1}{\norm{A^{-1}}},
    \]
    potem za rešitev zmotenega sistema \((A + \Delta A)(x + \Delta x) = b + \Delta b\) velja
    \[
        \frac{\norm{\Delta x}}{\norm{x}} \leq \frac{\kappa(A)}{1 - \kappa(A) \cdot \frac{\norm{\Delta A}}{\norm{A}}} \left( \frac{\norm{\Delta A}}{\norm{A}} + \frac{\norm{\Delta b}}{\norm{b}} \right),
    \]
    kjer za \(\norm{\cdot}\) velja \(\norm{I} = 1\).
\end{izrek}

\begin{opomba}
    Velja:
    \[
        \kappa_2(A) = \frac{\sigma_1(A)}{\sigma_n(A)}.
    \]
\end{opomba}

\begin{opomba}
    Edine matrike z občutljivostjo \(1\) v \(\norm{\cdot}_2\) so s skalarjem pomnožene unitarne matrike.
\end{opomba}

\begin{definicija}
    Hilbertova matrika \(H \in \R^{n \times n}\) je matrika z elementi 
    \[
        h_{ij} = \frac{1}{i+j-1}.
    \]
\end{definicija}

\begin{opomba}
    Hilbertove matrike so zelo občutljive.
\end{opomba}

\todo

\subsection{LU razcep}

Želimo rešiti sistem linearnih enačb 
\[
    Ax = b,\ A \in \C^{n \times n},\ x, b \in \C^n,\ \det A \neq 0.
\]

Kakšni metodi so na voljo?

Naj bo 
\[
    A = \begin{bmatrix}
        \textcolor{blue}{a_{11}} & a_{12} & \ldots & a_{1n} \\
        a_{21} & a_{22} & \ldots & a_{2n} \\
        \vdots & \vdots & & \vdots \\
        a_{n1} & a_{n2} & \ldots & a_{nn} \\
    \end{bmatrix}, \quad
    a_{11} \neq 0. 
\]

Definiramo 
\[
L_1 = \begin{bmatrix}
    1 & \\
    -l_{21} & 1 \\
    -l_{31} & & \ddots \\
    \vdots && & \ddots \\
    -l_{n1} & & & & 1
\end{bmatrix}, \quad
l_{j1} = \frac{a_{j1}}{a_{11}},\ j \in \set{2, \ldots, n}.
\]

Potem 
\[
    L_1 A = \begin{bmatrix}
        a_{11} & a_{12} & \ldots & a_{1n} \\
        0 & \textcolor{blue}{a_{22}^{(1)}} & \ldots & a_{2n}^{(1)} \\
        0 & a_{32}^{(1)} & \ldots & a_{3n}^{(1)} \\
        \vdots & \vdots & & \vdots \\
        0 & a_{n2}^{(1)} & \ldots & a_{nn}^{(1)} \\        
    \end{bmatrix}, \quad
    a_{22} \neq 0. 
\]

Nadaljujemo in dobimo
\[
    L_i = \begin{bmatrix}
        1 & \\
        & \ddots \\
        & & 1 \\
        & & -l_{i+1, i} & 1 \\
        & & \vdots && & \ddots \\
        & & -l_{n,i} & & & & 1
    \end{bmatrix}, \quad
    l_{ji} = \frac{a_{ji}^{(i-1)}}{a_{ii}^{(i-1)}},\ j \in \set{i+1, \ldots, n}
\]
ter 
\[
    A^{(i-1)} = \begin{bmatrix}
        0 \\
        & \ddots \\
        & & 0 \\
        & & & a_{ii}^{(i-1)} & \ldots \\
        & & & \vdots \\
        & & & a_{ni}^{(i-1)} & \ldots
    \end{bmatrix}
\]

V končnem
\[
    L_{n-1} \cdots L_2L_1 A = \begin{bmatrix}
        a_{11} & a_{12} & a_{13} & \ldots & a_{1n} \\
        & a_{22}^{(1)} & a_{23}^{(1)} & \ldots & a_{2n}^{(1)} \\
        & & a_{33}^{(2)} & \ldots & a_{3n}^{(1)} \\
        & & & & a_{nn}^{(n-1)}
    \end{bmatrix}
\]

Definiramo \(L_{n-1} \cdots L_2L_1 A =: U\). Tedaj 
\[
    A = \underbrace{L_1^{-1}L_2^{-2} \cdots L_{n-1}^{-1}}_L U = LU.
\]

Kako dobimo matriko \(L\)?
\begin{align*}
    L_j &= I - l_je_j^T\ \text{je eliminacisjka matrika} \\ 
    &\lthen L_j^{-1} = I + l_j e_j^T,\ \text{kjer} \\
    l_j &= [0\ \ldots\ 0\ l_{j+1, j}\ \ldots\ l_{n, j}]^T \\
    &\lthen L = \begin{bmatrix}
        1 \\
        l_{21} & 1 \\
        l_{31} & l_{32} & 1 \\
        \vdots & \vdots & & \ddots \\
        l_{n1} & l_{n2} & \ldots & l_{n, n-1} & 1
    \end{bmatrix}
\end{align*}

S tem dobimo \emph{LU razcep brez pivotiranja}, kjer je \(L\) spodnje trikotna z \(1\) na diagonali in \(U\) zgornje trikotna matrika.

\begin{algorithm}
    \DontPrintSemicolon

    \label{lu-not-pivot}
    \caption{LU razcep}
    
    \KwData{\(A \in \C^{n \times n}\)}
    \KwResult{\(L \in \C^{n \times n},\ U \in \C^{n \times n}\)}

    \For{\(i = 1, \ldots, n-1\)}{
        \For{\(j = i+1, \ldots, n\)}{
            \(l_{ji} \gets a_{ji} / a_{ii}\) \;
            \For{\(k = i+1, \ldots, n\)}{
                \(a_{jk} \gets a_{jk} - l_{ji} a_{ik}\) \;
            }
        }
    }

    \(U \gets A\) \;
\end{algorithm}

Kako zdaj rešimo sistem \(Ax = b\)?
\[
    Ax = b \liff L\underbrace{Ux}_y = b.
\]
\paragraph{Postopek.}\ 
\begin{enumerate}
    \item \(A = LU\),
    \item Reši \(Ly = b\),
    \item Reši \(Ux = y\).
\end{enumerate}

Sistem \(Ly = b\) rešimo s \emph{premo substitucijo} (od zgoraj navzdol), tj. 
\[
    l_{j1} y_1 + \ldots + l_{j, j-1}y_{j-1} + y_i = b_j.
\]

\begin{algorithm}
    \DontPrintSemicolon

    \label{prema-sub}
    \caption{Prema substitucija}  
    
    \KwData{\(L \in \C^{n \times n},\ b \in \C^n\)}
    \KwResult{\(y \in \C^n\)}

    \For{\(j = 1, \ldots, n\)}{
        \(y_i \gets b_j - \sum_{k=1}^{j-1} l_{jk}y_k\) \;
    }
\end{algorithm}

Sistem \(Ux = y\) rešimo s \emph{obratno substitucijo} (od zgoraj navzdol), tj. 
\[
    u_{jj} x_j + \ldots + u_{j, n}x_{n} = y_j.
\]

\begin{algorithm}
    \DontPrintSemicolon

    \label{obratna-sub}
    \caption{Obratna substitucija}  
    
    \KwData{\(U \in \C^{n \times n},\ y \in \C^n\)}
    \KwResult{\(x \in \C^n\)}

    \For{\(j = n, n-1, \ldots, 1\)}{
        \(x_j \gets (1 / u_{jj}) \cdot (y_j - \sum_{k=j+1}^{n} u_{jk} x_k)\) \;
    }
\end{algorithm}

Elementom \(a_{11}, a_{22}^{(1)}, \ldots, a_{n-1, n-1}^{(n-2)}\), s katerimi delimo, pravimo \emph{pivotni elementi}.

\begin{izrek}
    Za \(A \in \C^{n \times n}\) NTSE:
    \begin{enumerate}
        \item Obstaja enoličen LU razcep \(A = LU\), kjer je \(L\) spodnje trikotna z 1 na diagonali in \(U\) nesingularna zgornje trikotna.
        \item \(\det(A_k) \neq 0\) za vse \(A_k = A(1:k, 1:k),\ k \in \set{1, \ldots, n}\).
        \item Vse vodilne podmatrike \(A\) so nesingularne.
    \end{enumerate}
\end{izrek}

\begin{zgled}
    Pri numeričnem računanju so težave lahko tudi zaradi pivotov, ki so sicer neničelni, a blizu \(0\). \todo
\end{zgled}

Rešitev za težave s (skoraj) ničelnimi pivoti je pivotiranje, kjer dopuščamo:
\begin{enumerate}
    \item Menjave vrstic - \emph{delno pivotiranje}.
    \item Menjave vrstic in stolpcev - \emph{kompletno pivotiranje}. 
\end{enumerate}

\subsubsection*{Delno pivotrianje}
V koraku \(i\) poiščemo največjega izmed elementov \(|a_{ij}|, |a_{i+1}, j|, \ldots, |a_{ni}|\) in zamenjamo ustrezni vrstici.

\begin{algorithm}
    \DontPrintSemicolon

    \label{lu-delni-pivot}
    \caption{LU razcep z delnim pivotiranjem}
    
    \KwData{\(A \in \C^{n \times n}\)}
    \KwResult{\(L \in \C^{n \times n},\ U \in \C^{n \times n}\)}    

    \For{\(i = 1, \ldots, n\)}{
        find \(p \in \set{i, \ldots, n}\), da \(|a_{pi}| = \max_{l \in \set{i, \ldots, n}} |a_{li}|\) \;
        swap the \(i\)-th and \(j\)-th rows \;
        \For{\(j = i+1, \ldots, n\)}{
            \(l_{ji} \gets a_{ji} / a_{ii}\) \;
            \For{\(k = i+1, \ldots, n\)}{
                \(a_{jk} \gets a_{jk} - l_{ji} a_{ik}\) \;
            }
        }
    }
\end{algorithm}

Dobimo razcep \(PA = LU\), kjer je \(P\) permutacijska matrika, ki ustreza zamenjavi vrstic.

Kako zdaj rešimo sistem?
\[
    Ax = b \lthen PAx = Pb \lthen LUx = Pb
\]
\paragraph{Postopek.}\ 
\begin{enumerate}
    \item \(PA = LU\),
    \item Reši \(Ly = Pb\),
    \item Reši \(Ux = y\).
\end{enumerate}

\begin{izrek}
    Če je \(A\) nesingularna, potem obstaja taka permutacijska matrika \(P\), da za \(PA\) obstaja \(LU\) razcep brez pivotiranja.
\end{izrek}

\begin{proof}
    \todo
\end{proof}

\subsubsection*{Kompletno pivotiranje}
Poiščemo največji element bloka in zamenjamo ustrezni vrstici in stolpca. Dobimo sistem 
\[
    \underbrace{PAQ}_{LU}\underbrace{Q^T x}_{\widetilde{x}} = Pb
\]

\paragraph{Postopek.} \
\begin{enumerate}
    \item \(PAQ = LU\),
    \item Reši \(Ly = Pb\),
    \item Reši \(U \widetilde{x} = y\),
    \item \(x = Q\widetilde{x}\).
\end{enumerate}

\subsection{Analiza zaokrožitvenih napak pri LU razcepu}
\begin{definicija}
    \emph{Pivotna rast} je 
    \[
        g(A) := \frac{\max_{i, j \in \set{1, \ldots, n}} |u_{ij}|}{\max_{i, j \in \set{1, \ldots, n}} |a_{ij}|}
    \]
\end{definicija}
\todo

\subsection{Sistemi posebne oblike}
Recimo, da rešujemo sistem \(Ax = b\), kjer ima \(A\) posebno obliko, npr.\ diagonalna, trikotna itn.

\subsubsection*{Simetrična pozitivno definitna matrika}
\begin{izrek}
    Naj bo \(A \in \R^{n \times n}\). Tedaj 
    \begin{enumerate}
        \item Če je \(A\) s.p.d., so vse vodilne podmatrike \(A_k = A(1:k, 1:k)\) tudi s.p.d.
        \item Če je \(A\) s.p.d., obstaja enoličen LU razcep \(A = LU\), kjer je \(u_{ii} > 0\) za \(i \in [n]\).
        \item \(A\) je s.p.d. natanko tedaj, ko obstaja enolična spodnja trikotna matrika \(V,\ v_{ii} > 0\) za \(i \in [n]\), da je 
        \[
            A = V \cdot V^T.
        \]
        To je \emph{razcep Choleskega}. Matriko \(V\) pa imenujemo \emph{faktor Choleskega}.
    \end{enumerate}
\end{izrek}

\begin{proof}
    \todo
\end{proof}

Kako rešimo sistem s pomočjo razcepa Choleskega?
\paragraph{Postopek.}\
\begin{enumerate}
    \item \(A = V V^T\),
    \item \(Vy = b\),
    \item \(V^Tx = y\).
\end{enumerate}

\begin{algorithm}
    \DontPrintSemicolon

    \label{cholsky}
    \caption{Razcep Choleskega}
    
    \KwData{\(A \in \C^{n \times n}\)}
    \KwResult{\(V \in \C^{n \times n}\)}    

    \For{\(i = 1, \ldots, n\)}{
        \(v_{ii} \gets (a_{ii} = \sum_{k=1}^{i-1} v_{ik}^2)^{1/2}\) \;
        \For{\(j = i+1, \ldots, n\)}{
            \(v_{ji} \gets (1/v_{ii}) \cdot (a_{ij} - \sum_{k=1}^{i-1} v_{ik}v_{jk})\) \;            
        }
    }
\end{algorithm}

