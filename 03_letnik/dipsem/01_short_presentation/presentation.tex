\documentclass[10pt]{beamer}
% Možne velikosti pisav so: 8pt, 9pt, 10pt, 11pt, 12pt, 14pt, 17pt, 20pt

\usepackage[T1]{fontenc}
\usepackage[utf8]{inputenc}
\usepackage[slovene]{babel}

% To potrebujemo, da bomo lahko delali zapiske za predavatelja
\usepackage{pgfpages}

% Izberemo, ali bomo naredili samo prosojnice, samo zapiske ali oboje

\setbeameroption{hide notes}                        % samo prosojnice
%\setbeameroption{show only notes}                   % samo zapiski
% \setbeameroption{show notes on second screen=right}  % oboje

% Minimalistični still
\mode<presentation>
% \usetheme{Goettingen}
% \usecolortheme{rose}

% Malo manj dolgočasna pisava
\usepackage{palatino}
\usefonttheme{serif}

% Izklopimo navigacijske simbole, ker so neuporabni
\setbeamertemplate{navigation symbols}{}

% Aktiviramo oštevilčenje strani
\setbeamertemplate{footline}[frame number]{}

% Stil za zapiske
\setbeamertemplate{note page}{\pagecolor{yellow!5}\insertnote}

\usepackage{xcolor}
\usepackage{amsmath}
\usepackage{amssymb}
\usepackage{bbold}

% environments
\usepackage{amsthm}

\theoremstyle{definition}{
    \newtheorem{definicija}{Definicija}[section]
}

\theoremstyle{plain} {
    \newtheorem{izrek}[definicija]{Izrek}
    \newtheorem{trditev}[definicija]{Trditev}
    \newtheorem{posledica}[definicija]{Posledica}
    \newtheorem{lema}[definicija]{Lema}
    \newtheorem{aksiom}[definicija]{Aksiom}
}

\theoremstyle{remark}{    
    \newtheorem{opomba}{Opomba}
    \newtheorem{primer}{Primer}
    \newtheorem{zgled}{Zgled}
}

% default sets
\newcommand{\N}{\mathbb{N}}
\newcommand{\Z}{\mathbb{Z}}
\newcommand{\Q}{\mathbb{Q}}
\newcommand{\R}{\mathbb{R}}
\newcommand{\C}{\mathbb{C}}
\newcommand{\F}{\mathbb{F}}
\newcommand{\HH}{\mathbb{H}}

% logic
\newcommand{\all}[1]{\forall #1 \,.\,}
\newcommand{\some}[1]{\exists #1 \,.\,}
\newcommand{\exactlyone}[1]{\exists! #1 \,.\,}
\newcommand{\lthen}{\implies}
\newcommand{\liff}{\iff}

% sets
\newcommand{\set}[1]{\left\{#1\right\}}
\newcommand{\setb}[2]{\set{#1 \,|\, #2}}

% mappings
\newcommand{\img}[1]{#1_{*}}
\newcommand{\invimg}[1]{#1^{*}}









\DeclareMathOperator{\lin}{Lin}
\DeclareMathOperator{\rang}{rang}

\DeclareMathOperator{\sgn}{sgn}  % sign
\DeclareMathOperator{\id}{id}  % identity func
\DeclareMathOperator{\im}{im}  % slika

\DeclareMathOperator{\Int}{Int}  % interior
\DeclareMathOperator{\Cl}{Cl}  % closure

\DeclareMathOperator{\FV}{FV}  % Fourierjeva vrsta

\DeclareMathOperator{\eval}{eval}
% math symbols
\newcommand{\wt}[1]{\widetilde{#1}}

% Greece letters
\let\oldphi\phi
\let\oldtheta\theta
\newcommand{\eps}{\varepsilon}
\renewcommand{\phi}{\varphi}
\renewcommand{\theta}{\vartheta}

% style
\newcommand{\ds}{\displaystyle}
\newcommand{\mc}[1]{\mathcal{#1}}

% other
\newcommand{\todo}[1]{\textcolor{red}{TODO: #1}}


% Graphics
\usepackage{pgfplots}
\pgfplotsset{width=8cm,compat=1.9}

\begin{document}

\title{Izračun skupnih negibnih točk odsekoma linearnih funkcij}
\author{Ruslan Urazbakhtin}
\date{24.\ november 2025}

\begin{frame}
    \titlepage

    \note{
        Dober dan. Moja diplomska tema je ">Izračun skupnih negibnih točk odsekoma linearnih funkcij."<
        Danes bi rad na kratko predstavil, kaj sploh so odsekoma linearne funkcije, katere med njimi imajo negibne točke in zakaj, ter s kakšnimi težavami se srečamo pri njihovem izračunu.
    }
\end{frame}

\begin{frame}
    \frametitle{Odsekoma linearne funkcije}

    \only<1>{\note{
        Preden se lotimo uradne definicije, si oglejmo preprost primer odsekoma linearne funkcije, ki preslika interval \([0,1]\) vase. Vidimo, da ta funkcija ni zvezna, vendar je linearna (afina) na vsakem posameznem kosu domene. Da bi takšno funkcijo natančno opisali, je dovolj, da podamo njen predpis, kar najlažje naredimo po posameznih kosih.
        \begin{figure}
        \centering
            \begin{tikzpicture}
            \begin{axis}[
                width=6cm, height=6cm,
                xmin=0, xmax=1.1, ymin=0, ymax=1.1,
                axis lines=center,
                % xlabel={\(x\)}, ylabel=\(f(x)\),
                xtick={0,1}, ytick={0, 1},
                grid=none,
                clip=false
            ]
                \addplot[
                    domain=0:0.5,
                    color=blue
                    ]
                {0.5*x + 0.25};

                \draw[->, color=blue] (axis cs:0.5,0.5) -- ++(0.05,0.025);

                % \addplot[
                %     domain=0.5:0.75,
                %     color=blue
                % ]
                % {0.8};

                % \draw[->, color=blue] (axis cs:0.75,0.8) -- ++(0.05,0.00);

                \addplot[
                    domain=0.5:1,
                    color=blue
                ]
                {0.5*x + 0.5};

                \addplot[
                    only marks,
                    mark=*,
                ] coordinates {(0.5,0)} node[below] {$a$};

                % \addplot[
                %     domain=0:1,
                %     color=red
                % ]
                % {x};
            \end{axis}
            \end{tikzpicture}
        \end{figure}    

        \[
            f(x) = \begin{cases}
                q_0 + q_1 x; &0 \leq x,\ x < a; \\
                q_0' + q_1' x; &a \leq x,\ x \leq 1.
            \end{cases}
        \]
    }}

    \only<2>{\begin{definicija}
        \emph{Linearni izraz} v spremenljivkah \(x_1, x_2, \ldots, x_n\) je izraz
        \[
            q_0 + q_1x_1 + q_2x_2 + \ldots + q_nx_n,
        \]
        kjer so \(q_0, \ldots, q_n \in \R\).
    \end{definicija}
    
    \begin{definicija}
        \emph{Pogojni linearni izraz} je par \(C \vdash e\), kjer je \(e\) linearni izraz in \(C\) končna množica neenačb med linearni izrazi, torej vsak element množice \(C\) je ene izmed oblik
        \[
            e_1 \leq e_2, \quad e_1 < e_2.
        \]
        Za vsak \(\vec{r} \in \R^n\) z \(C(\vec{r})\) označimo konjunkcijo neenačb, kjer nadomestimo spremenljivke v \(C\) z števili \(r_1, \ldots, r_n\).
    \end{definicija}
    
    \note{\begin{itemize}
            \item Označi vse nove termine v primeru.
            \item Z pomočjo pogojnega linearnega izraza lahko opišemo en kos funkcije:
            \begin{itemize}
                \item Domena kosa je \(\setb{\vec{r} \in \R^n}{C(\vec{r}) = \top}\);
                \item Linearni izraz \(e\) podaja linearno funkcijo nad domeno.
            \end{itemize}
        \end{itemize}}
    }

    \only<3>{\begin{definicija}
        Pravimo, da je funkcija \(f: [0,1]^n \to [0,1]\) \emph{odsekoma linearna}, če obstaja končna množica \(\mc{F}\) pogojnih linearnih izrazov v spremenljivkah \(x_1, \ldots, x_n\), da velja:
        \begin{enumerate}
            \item \(\all{\vec{r} \in [0,1]^n} \some{(C \vdash e) \in \mc F} C(\vec{r}) = \top\) in
            \item \(\all{\vec{r} \in [0,1]^n} \all{(C \vdash e) \in \mc F} C(\vec{r}) = \top \lthen f(\vec{r}) = e(\vec{r})\).
        \end{enumerate}
        Pravimo, da množica \(\mc{F}\) \emph{predstavlja} funkcijo \(f\).
    \end{definicija}
    \note{1.\ točka pomeni celovitost, 2.\ točka pa enoličnost. \\
            \textbf{Vprašanje:} Kakšni so zadostni pogoji za funkcijo \(f\), da ima negibno točko?}
    }
\end{frame}

\begin{frame}
    \frametitle{Izrek Knaster-Tarskega}

    \only<1>{\note{Če bi bila funkcija \(f\) zvezna, bi lahko uporabili Brouwerjev izrek o negibni točki. Vendar pa so odsekoma linearne funkcije v splošnem nezvezne, zato potrebujemo drug pristop. Na srečo obstaja ustrezen izrek v teoriji delno urejenih množic in mrež, ki nam bo v pomoč.}}

    % \only<2>{
    %     \framesubtitle{Delna urejenost}
    %     \begin{definicija}
    %         Naj bo \(E\) neka množica. Relacija \(\leq\) na množici \(E\) je \emph{delna urejenost}, če je refleksivna, antisimetrična in tranzitivna. Pravimo, da je \(E\) \emph{delno urejena} množica z relacijo \(\leq\).            
    %     \end{definicija}

    %     \note{
    %         \begin{itemize}
    %             \item \textbf{Primer} (delna urejenost): Relacija \(\subseteq\) na \(P(E)\).
    %             \item Naj bo \(X \subseteq E\). Označimo z \(\bigwedge X\) največjo spodnjo mejo (infimum) množice \(X\) ter z \(\bigvee X\) najmanjšo zgornjo mejo (supremum) množice \(X\), če ti obstajata.
    %         \end{itemize}
    %     }
    % }
    
    % \only<3>{
    %     \framesubtitle{Monotone preslikave}
    %     \begin{definicija}
    %         Preslikava \(f: E \to F\) med delno urejenima množicama \((E, \leq_E)\) in \((F, \leq_F)\) je \emph{monotona} (ali \emph{naraščajoča}), ko velja 
    %         \[
    %             \all{x, y \in E} x \leq_E y \lthen f(x) \leq_F f(y).
    %         \]
    %     \end{definicija}
    % }

    % \only<4>{
    %     \framesubtitle{Mreže}
    %     \begin{definicija}
    %         Naj bo \((E, \leq)\) delno urejena množica. Množica \(E\) je \emph{polna mreža}, če vsaka podmnožica \(X \subseteq E\) ima največjo spodnjo mejo \(\bigwedge X\) in najmanjšo zgornjo mejo \(\bigvee X\).
    %     \end{definicija}

    %     \note{\textbf{Primer} (polne mreže):
    %     \begin{itemize}
    %         \item Potenčna množica \(P(E)\) urejena z relacijo \(\subseteq\).
    %         \item Interval \([0,1]\) urejeni z relacijo \(\leq\).
    %         \item Produkt polnih mrež je polna mreža, če za relacijo na produkte vzamemo produktno urejenost.
    %     \end{itemize}        
    %     }
    % }

    \only<2>{
        \begin{definicija}
            Naj bo \(S\) poljubna množica in \(f: S \to S\) preslikava. \emph{Negibna točka} preslikave \(f\) je element \(x \in S\), da velja \(f(x) = x\). Množico vseh negibnih točk preslikave \(f\) označimo z \(\Fix(f)\).
        \end{definicija}

        \begin{izrek}[Knaster-Tarski]
            Naj bo \((S, \leq)\) polna mreža in \(f: S \to S\) monotona preslikava. Tedaj sta \(\inf (\Fix(f))\) in \(\sup (\Fix(f))\) elementi \(\Fix(f)\).
        \end{izrek}

        \note{
            \begin{itemize}
                \item Se spomni vse definicije.
                \item Povej, da je \([0, 1]\) polna mreža in da je monotonost zadostni pogoj za preslikavo \(f\), da ima negibno točko.
                \item Postavi problem v algoritmični obliki.
            \end{itemize}
        }
    }
\end{frame}

% \begin{frame}
%     \frametitle{Algoritmi}
%     \only<1-2>{
%         \framesubtitle{Izačun ene negibne točke}

%         Naj bo \(Y\) polna mreža in naj bo \[f: X \times Y \to Y\] monotona preslikava. Želimo določiti preslikavo \[f^\dagger: X \to Y.\]

%         \pause

%         Za končni \(X\) in \(Y\) imamo učinkovit algoritem. Ta algoritem lahko posplošimo tudi na naš primer, kjer je \(Y = [0, 1]\) in je \(f\) odsekoma linearna funkcija.
%     }

%     \only<3>{
%         \framesubtitle{Izračun skupnih negibnih točk}
%         Naj bodo \(Y_1, \ldots, Y_n\) polne mreže in naj bo \[F: X \times Y_1 \times \cdots \times Y_n \to Y_1 \times \cdots \times Y_n\]
%         monotona preslikava. Želimo določiti preslikavo \[F^\dagger: X \to Y_1 \times \cdots \times Y_n.\]
%     }    
% \end{frame}
\end{document}
