\section{Kriptografija}
\subsection{Šifriranje}
\begin{definicija}
    \emph{Kriptosistem} (oz.\ \emph{šifra}) je peterka \((\B, \CC, \K, \E, \D)\), kjer
    \begin{itemize}
        \item \(\B\) je končna množica besedil;
        \item \(\CC\) je množica kriptogramov (angl.\ ciphertext);
        \item \(\K\) je množica ključev;
        \item \(\E = \setb{E_k: \B \to \CC}{k \in \K}\) je množica \emph{šifrirnih} funkcij razreda \(\mathcal{O}(n^p)\);
        \item \(\D = \setb{D_k: \CC \to \B}{k \in \K}\) je množica \emph{dešifrirnih} funkcij razreda \(\mathcal{O}(n^p)\).
    \end{itemize}
    Pri tem za kriptosistem mora veljati \emph{pravilnost}, tj.\
    \[
        \all{m \in \B} \all{k \in \K} \some{k' \in \K} D_{k'}(E_k(m)) = m.
    \]
\end{definicija}
%
\begin{opomba} 
    Za vse \(k \in \K\) je funkcija \(E_k\) injektivna \textcolor{red}{(zakaj?)}, sledi da \(|\B| \leq |C|\).
\end{opomba}

\subsubsection{Varnost}
\textbf{Kerckhoffovo načelo.} Kriptosistem naj bo varen, če tudi napadalec pozna sistem, ne pa skrivnega ključa.

\subsection{Popolna tajnost}
Označimo z 
\begin{itemize}
    \item \(X_\B\) slučajno spremenljivko izbire besedila;
    \item \(X_\CC\) slučajno spremenljivko izbire kriptograma.
\end{itemize}
Predpostavimo, da je \(\all{c \in \CC} P(X_\CC = c) > 0\).

\begin{definicija}
    Kriptosistem \emph{ima lastnost popolne tajnosti (LPT)}, če 
    \[
        \all{m \in \B} \all{c \in \CC} P(X_\B = m\, |\, X_\CC = c) = P(X_\B = m).
    \]
\end{definicija}

\begin{lema}
    Kriptosistem ima LPT \(\liff P(X_\CC = c\, |\, X_\B = m) = P(X_\CC = c)\)
\end{lema}

\begin{proof}
    \todo
\end{proof}

\begin{opomba}
    Če kriptosistem ima LPT, potem
    \[
        \all{m_1, m_2 \in \B} \all{c \in \CC} P_{k \leftarrow K}(E_k(m_1) = c) = P_{k \leftarrow K}(E_k(m_2) = c)
    \]
\end{opomba}

\subsubsection{Vernamova šifra (OTP: one-time pad)}
Naj bodo \(\B = \CC = \K = \set{0, 1}^\lambda,\ \lambda > 0\). Ključi izbiramo enakomerno naključno. Definiramo 
\begin{itemize}
    \item \(E_k(m) = m \oplus k\); 
    \item \(D_k(c) = c \oplus k\).
\end{itemize}

\begin{trditev}
    Vernamova šifra je pravilna in ima LPT.
\end{trditev}

\begin{proof}
    \todo
\end{proof}

Vernamova šifra ima LPT, ampak tudi slabosti:
\begin{itemize}
    \item Ključ lahko uporabimo samo enkrat:
    \[
        E_k(m_1) = m_1 \oplus k,\ E_k(m_2) = m_2 \oplus k \lthen m_1 \oplus m_2 \oplus (k \oplus k) = m_1 \oplus m_2.
    \]
    Iz \(m_1 \oplus m_2\) ponavadi se da dobiti neko informacijo.
    \item Ključi so enako dolgi kot besedilo, kar povzroči \(2x\) porabo prostora.
\end{itemize}

\

Izkaže se, da vsak kriptosistem, ki ima LPT, ima dolge ključe, saj
\begin{trditev}
    Če ima kriptosistem LPT, potem
    \[
        |\B| \leq |\CC| \leq |\K|.
    \]
\end{trditev}

\begin{proof}
    \todo
\end{proof}

\begin{opomba}
    Recimo, da \(\B = \set{0, 1}^\lambda\) ter \(\K = \set{0, 1}^t\). Tedaj \(|\B| = 2^\lambda\) in \(|\K| = 2^t\). Če ima kriptosistem LPT, potem \(|\B| \leq |\K| \lthen \lambda \leq t\). Torej vsak ključ je dolg vsaj \(\lambda\).
\end{opomba}

\subsection{Tokovne šifre}