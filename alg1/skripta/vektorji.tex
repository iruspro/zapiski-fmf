\section{VEKTORJI V $\RR^3$}

\begin{enumerate}
    \item Koordinatni sistem
    \begin{itemize}
        \item Model za realna števila. Bijekcija med $\RR$ in realno osjo.
        \item Model za $\RR^2$. Običajen koordinatni sistem. Abscisna in ordinatna os. Bijekcija med $\RR^2$ in ravnino. Razdalja med točkama $T_1(x_1, y_1)$ in $T_2(x_2, y_2)$.
        \item Model za $\RR^3$. Pozitivno orientiran koordinatni sistem. Bijekcija med $\RR^3$ in prostorom. Razdalja med točkama $T_1(x_1, y_1, z_1)$ in $T_2(x_2, y_2, z_2)$.
        \item Definicija seštevanja in množenja s skalarjem na množici $\RR^3$.
    \end{itemize}

    \item Vektorji
    \begin{itemize}
        \item \colorbox{purple!30}{\textbf{Definicija.}} Krajevni vektor točke $a$. Oznaka. Bijekcija med $\RR^3$ in množico krajevnih vektorjev.
        \item \colorbox{purple!30}{\textbf{Definicija.}} Vektor.
        \item \colorbox{yellow!30}{\emph{Opomba.}} Nenatančna definicija vektorja. Kadar sta dva vektorja enaka? Oznaka za vektor od točke $A$ do točke $B$.
        \item Bijekcija med množico vektorjev in množico krajevnih vektorjev. Bijekcija med $\RR^3$ in množico vektorjev.
        \item Definicija seštevanja in množenja s skalarjem na množici vektorjev.
        \item \colorbox{purple!30}{\textbf{Definicija.}} Ničelni vektor. Nasprotni vektor. S čim sta določena ta vektorja?
        \item Definicija odštevanja vektorjev.
        \item \colorbox{blue!30}{\textbf{Trditev.}} 8 lastnosi operacij z vektorji (aksiomi za vektorski prostor).
        \begin{itemize}
            \item \colorbox{green!30}{\textbf{Dokaz.}} Napišemo po komponentah in poračunamo.
        \end{itemize}
        \item \colorbox{purple!30}{\textbf{Definicija.}} Linearna kombinacija vektorjev $\vec{a}_1, \vec{a}_2, \ldots, \vec{a}_n$.
        \item \colorbox{purple!30}{\textbf{Definicija.}} Linearno odvisni vektorji. Linearno neodvisni vektorji.
        \item \colorbox{yellow!30}{\emph{Opomba.}} Kadar sta sva vektorja $\vec{a}$ in $\vec{b}$ linearno odvisna?
        \item \colorbox{blue!30}{\textbf{Trditev.}} Ali je ravnina, napeta na linearno neodvisna vektorja $\vec{a}$ in $\vec{b}$, je natanko množica vseh vektorjev oblike $\alpha \vec{a} + \beta \vec{b}$?
        \begin{itemize}
            \item \colorbox{green!30}{\textbf{Dokaz.}} Potegnimo ustrezne premice.
        \end{itemize}
        \item \colorbox{orange!30}{\textbf{Posledica.}} Karakterizacija linearne odvisnosti treh krajevnih vektorjev (ravnine).
        \item \colorbox{purple!30}{\textbf{Definicija.}} Baza prostora $\RR^3$.
        \item \colorbox{yellow!30}{\emph{Opomba.}} Baza prostora $\RR^2$.
        \item \colorbox{blue!30}{\textbf{Trditev.}} Razvoj po baze v $\RR^3$. 
        \begin{itemize}
            \item \colorbox{green!30}{\textbf{Dokaz.}} Obstoj: Potegnimo ustrezne premice.            
            Enoličnost: Definicija linearne neodvisnosti.
        \end{itemize}
        \item \colorbox{orange!30}{\textbf{Posledica.}} Kaj lahko povemo o linearni odvisnosti 4 vektorjev v $\RR^3$?
        \item \colorbox{yellow!30}{\emph{Primer.}} \textbf{Standardna baza} prostora $\RR^3$. \textbf{Standardna baza} prostora $\RR^2$. 
        \item \colorbox{blue!30}{\textbf{Trditev.}} Karakterizacija linearne neodvisnosti treh vektorjev v $\RR^3$ (linearna kombinacija).
        \begin{itemize}
            \item \colorbox{green!30}{\textbf{Dokaz.}} Definicija linearne neodvisnosti.
        \end{itemize}
    \end{itemize}

    \item Skalarni produkt
    \begin{itemize}
        \item \colorbox{purple!30}{\textbf{Definicija.}} Skalarni produkt.
        \item \colorbox{blue!30}{\textbf{Trditev.}} 4 lastnosti skalarnega produkta.
        \begin{itemize}
            \item \colorbox{green!30}{\textbf{Dokaz.}} Vektorji napišemo po komponentah in poračunamo.
        \end{itemize}
        \item \colorbox{yellow!30}{\emph{Opomba.}} Ali je skalarni produkt asociativen?
        \item \colorbox{purple!30}{\textbf{Definicija.}} Dolžina ali norma vektorja.
        \item \colorbox{yellow!30}{\emph{Opomba.}} Kaj je dolžina krajevnega vektorja?
        \item \colorbox{blue!30}{\textbf{Izrek.}} Formula za skalarni produkt.
        \begin{itemize}
            \item \colorbox{green!30}{\textbf{Dokaz.}} Cosinusni izrek.
        \end{itemize}
        \item Dogovor o vektorju $\vec{0}$.
        \item \colorbox{orange!30}{\textbf{Posledica.}} Karakterizacija pravokotnosti vektorjev s skalarnim produktom.
        \item \colorbox{yellow!30}{\emph{Primer.}} Izračunaj kot med vektorjema $\vec{a}=(1,1,2)$ in $\vec{b} = (1, 0, 1)$.
        \item \colorbox{yellow!30}{\emph{Primer.}} Naj bosta $\vec{a}_1=(x_1,y_1,0)$ in $\vec{a}_2=(x_2,y_2,0)$ vektorja v ravnini $z=0$. Z vektorjema $\vec{a}_1$ in $\vec{a}_2$ izrazi ploščino paralelograma, napetega na $\vec{a}_1$ in $\vec{a}_2$. Kaj je in kaj pove dobljeni izraz?
        \item \colorbox{orange!30}{\textbf{Posledica.}} Karakterizacija linearne odvisnosti dveh vektorjev v ravnini.
    \end{itemize}

    \newpage
    \item Vektorski produkt
    \begin{itemize}
        \item \colorbox{purple!30}{\textbf{Definicija.}} Vektorski produkt.
        \item Dogovor o vektorju $\vec{0}$.
        \item \colorbox{orange!30}{\textbf{Posledica.}} Karakterizacija vzporednosti vektorjev s vektorskim produktom.
        \item Naj bo $\vec{a}_1, \vec{a}_2 \in \RR^3$. Izračunaj koordinate vekotrja $\vec{a}_1 \times \vec{a}_2$. Kaj je rezultat?
        \begin{itemize}
            \item \colorbox{green!30}{\textbf{Izračun.}} Izračunali bomo $z_3 = ||\vec{a}_1 \times \vec{a}_2|| \cdot \vec{k}$. Proiciramo paralelogram napeti na vektorja $\vec{a}_1$ in $\vec{a}_2$ in poiščemo zvezo med ploščino originalnega in proiciranega paralelograma (pomagamo si s ploščinoma trikotnikov).
        \end{itemize}
        \item \colorbox{blue!30}{\textbf{Trditev.}} 3 lastnosti vektorskega produkta.
        \begin{itemize}
            \item \colorbox{green!30}{\textbf{Dokaz.}} Preverimo z računom.
        \end{itemize}
        \colorbox{yellow!30}{\emph{Primer.}} Izračunaj ploščino trikotnika  s oglišči $A(1,0,2), \ B(2,2,0), \ C(3,-2,1)$.
    \end{itemize}

    \item Mešani produkt
    \begin{itemize}
        \item \colorbox{purple!30}{\textbf{Definicija.}} Mešani produkt. Oznaka.
        \item Čemu je enak mešani produkt?
        \item Geometrijska interpretacija mešanega produkta.
        \item \colorbox{orange!30}{\textbf{Posledica.}} Ciklična zamena komponent v mešanem produktu.
        \item \colorbox{orange!30}{\textbf{Posledica.}} Čemu je enak mešani produkt?
        \item \colorbox{orange!30}{\textbf{Posledica.}} Karakterizacija linearne odvisnosti 3 vektorjev c mešanim produktom.
        \item \colorbox{blue!30}{\textbf{Trditev.}} 2 lastnosti mešanega produkta.
        \begin{itemize}
            \item \colorbox{green!30}{\textbf{Dokaz.}} Preverimo z računom ali upoštevamo lastnosti skalarnega in vektorskega proudkta.
        \end{itemize}
    \end{itemize}

    \item Dvojni vektorski produkt
    \begin{itemize}
        \item Formula za dvojni vektorski produkt $(\vec{a} \times \vec{b}) \times \vec{c}$.
        \begin{itemize}
            \item \colorbox{green!30}{\textbf{Dokaz.}} Preverimo z računom.
        \end{itemize}
        \item Kaj če v zgornjo formulo vstavimo $\vec{c} \times \vec{d}$ namesto $\vec{c}$?
        \item Langrangeeva identiteta $(\vec{a} \times \vec{b}) \cdot (\vec{c} \times \vec{d})$.
        \item Kaj če v zgornjo formulo vstavimo $\vec{c} = \vec{a}$ in $\vec{d} = \vec{b}$?
    \end{itemize}

    \item Premice in ravnine v $\RR^3$
    \item[$\circ$] Enačba ravnine
    \begin{itemize}
        \item \colorbox{purple!30}{\textbf{Definicija.}} Enačba ravnine. 
        \item \colorbox{purple!30}{\textbf{Definicija.}} Normala ravnine.
        \item S čim je enolično določena ravnina? Ali je z normalo?
        \item \colorbox{blue!30}{\textbf{Izpeljava.}} Vektorska enačba ravnine. Vektorska enačba ravnine po komponentah.
        \begin{itemize}
            \item \colorbox{green!30}{\textbf{Dokaz.}} Skalarni produkt pravokotnih vektorjev.
        \end{itemize}
        \item Kaj je enačba ravnine? Ali je vsaka linearna enačba v spremenljivkah $x, y, z$ enačba neke ravnine? Kako dobimo normalo iz enačbe ravnine? V čim se razlekujejo enačbe ravnin? Ali je enačba z ravnino enolično določena?
        \item Enotska normala. Normalna enačba ravnine.
        \item \colorbox{blue!30}{\textbf{Izpeljava.}} Enačba ravnine, podane s 3 nekolinearnimi točkami.
        \begin{itemize}
            \item \colorbox{green!30}{\textbf{Dokaz.}} Normalo dobimo z vektorskim produktom.
        \end{itemize}
        \item \colorbox{yellow!30}{\emph{Primer.}} Določi enačbo ravnine skozi točke $T_0(1, -1, 3), \ T_1(2,0,5), \ T_2(-3, -1, 4)$.
    \end{itemize}

    \item[$\circ$] Razdalja do ravnine
    
    Zanima nas razdalja $\Delta$ med točko $T_1$ in ravnino $\Sigma$.
    \begin{itemize}
        \item \colorbox{purple!30}{\textbf{Definicija.}} Razdalja med $T_1$ in ravnino $\Sigma$.
        \item Kaj je $\Delta$? Zakaj?
        \item \colorbox{blue!30}{\textbf{Izpeljava.}} Formula za razdaljo med točko in ravnino.
        \begin{itemize}
            \item \colorbox{green!30}{\textbf{Dokaz.}} Pravokotni trikotnik + formula za skalarni produkt.
        \end{itemize}        
        \item Kako dobimo razdaljo med točko in ravnino? Kadar je točka pripada ravnine?
        \item \colorbox{purple!30}{\textbf{Definicija.}} Razdalja med ravninama, ki se sekata. Razdalja med vzporednima ravninama. Razdalja med ravnino in premico, ke se sekata. Razdalja med ravnino in njej vzporedno premico. 
    \end{itemize}

    \newpage
    \item[$\circ$] Enačba premice
    \begin{itemize}
        \item Kaj je premica (ali je nek presek)? Kaj je enačba premice?
        \item \colorbox{purple!30}{\textbf{Definicija.}} Smerni vektor.
        \item Ali je vzporedne premice imajo isti smerni vektor? S čim je enolično določena premica?
        \item \colorbox{blue!30}{\textbf{Izpeljava.}} Parametrična vektorska enačba premice. Parameter.
        \begin{itemize}
            \item \colorbox{green!30}{\textbf{Dokaz.}} Kako pridemo do vsake točke na premice?
        \end{itemize}
        \item Enačba premice brez parametra. Kako preberemo smerni vektor?
        \item \colorbox{blue!30}{\textbf{Izpeljava.}} Enačba premice skozi točki $T_0, \ T_1$.
        \begin{itemize}
            \item \colorbox{green!30}{\textbf{Dokaz.}} Za smerni vektor vzamemo $\overrightarrow{T_1 T_0}$
        \end{itemize}
    \end{itemize}

    \item[$\circ$] Razdalja do premice
    
    Zanima nas razdalja $\Delta$ med točko $T_1$ in premico.
    \begin{itemize}
        \item Čemu je enaka razdalja med točko $T_1$ in premico?
        \item \colorbox{blue!30}{\textbf{Izpeljava.}} Formula za razdaljo med točko $T_1$ in premico.
        \begin{itemize}
            \item \colorbox{green!30}{\textbf{Dokaz.}} Pravokotni trikotnik in vektorski produkt.
        \end{itemize}
        \item \colorbox{yellow!30}{\emph{Primer.}} Izračunaj razdaljo med točko $A(1,1,1)$ in premico z enačbo $\frac{x}{2} = \frac{y+1}{-2}=z+1$.
        \item \colorbox{purple!30}{\textbf{Definicija.}} Razdalja med premicama, ki se sekata. Razdalja med vzporednima premicama.
        \item \colorbox{blue!30}{\textbf{Izpeljava.}} Razdalja med mimobežnima premicama.
        \begin{itemize}
            \item \colorbox{green!30}{\textbf{Dokaz.}} Skozi $T_1$ potegnimo premico $p_2'$, ki je vzporedna $p_2$. Podobno naredimo skozi točko $T_2$. Sekajoči premici določata dve vzporedni ravnini, torej razdalja med premicama je vsaj razdalja med ravninama. Z pravoktnimi projekcijama premic $p_1$ in $p_2$ na ustrezne ravnine pokažemo, da je razdalja med premicama kvečjemu razdalja med presečičsama danih premic in pravokotnih projekcij. Od tod sledi, da je razdalja med premicama je razdalja med vzporednima ravninama.
        \end{itemize}
        \item \colorbox{orange!30}{\textbf{Posledica.}} Kadar premici se sekata?
        \item \colorbox{yellow!30}{\emph{Primer.}} Poišči razdaljo med premicama $x=1, \ y-1 = - z$ in $\frac{1-x}{4}=\frac{y+1}{3}= \frac{z+1}{4}$.
    \end{itemize}
\end{enumerate}

\newpage
\