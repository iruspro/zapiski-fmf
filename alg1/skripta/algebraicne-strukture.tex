\section{OSNOVNE ALGEBRAIČNE STRUKTURE}

\begin{enumerate}
    \item[$\circ$] Ponovitev preslikav
    \begin{itemize}
        \item \colorbox{purple!30}{\textbf{Definicija.}} Injektivna, surjektivna, bijektivna preslikava.
        \item \colorbox{purple!30}{\textbf{Definicija.}} Predpis za inverzno preslikavo.
        \item \colorbox{purple!30}{\textbf{Definicija.}} Kompozitum preslikav.
        \item \colorbox{purple!30}{\textbf{Definicija.}} Identična preslikava (identiteta) na $A$.
        \item \colorbox{blue!30}{\textbf{Trditev.}} Karakterizacija bijektivnosti preslikave (obstoj inverza).
        \item \colorbox{blue!30}{\textbf{Trditev.}} Kompozicija injektivnih/surjektivnih/bijektivnih preslikav. Kaj če je kompozicija preslikav injektivna/surjektivna?
        \item \colorbox{purple!30}{\textbf{Definicija.}} Slika in praslika.
    \end{itemize}
    \item Operacije
    \begin{itemize}
        \item \colorbox{purple!30}{\textbf{Definicija.}} Dvočlena (binarna) notranja operacija na množice $A$. Kompozitum elementov. Konkretna operacija.
        \item \colorbox{yellow!30}{\emph{Primer.}} Ali je operacija?
        \begin{itemize}
            \item $A=\NN$, $\circ$ seštevanje ali množenje. Ali je odštevanje operacija na $\NN$?
            \item $A=\RR$, $\circ$ seštevanje ali množenje ali odštevanje. Ali je deljenje operacija na $\RR$?
            \item $A=\RR^3$, $\circ$ seštevanje ali odštevanje ali vektorski produkt. Ali je skalarni produkt operacija na $\RR^3$?
            \item Naj bo $A \neq \emptyset$ in $F$ množica vseh preslikav $A \to A$, $\circ$ je kompozitum preslikav.
        \end{itemize}
        \item \colorbox{purple!30}{\textbf{Definicija.}} $n$-člena notranja operacija na množice $A$. 
        \item \colorbox{purple!30}{\textbf{Definicija.}} Dvočlena zunanja operacija na množice $A$. 
        \item \colorbox{yellow!30}{\emph{Primer.}} Ali je množenje s skalarjem zunanja operacija na $\RR^3$?
        \item \colorbox{purple!30}{\textbf{Definicija.}} Kadar pravimo, da množica $A$ ima algebraično strukturo?
        \item \colorbox{purple!30}{\textbf{Definicija.}} Asociativna operacija.
        \item \colorbox{blue!30}{\textbf{Trditev.}} Asociativna operacija in oklepaji.
        \item \colorbox{yellow!30}{\emph{Opomba.}} Kako pogosto rečemo asociativne operacije? Ali lahko spuščamo oklepaji?
        \item \colorbox{purple!30}{\textbf{Definicija.}} Komutativna operacija.
        \item Kako pogosto rečemo komutativne in asociativne operacije?
        \item \colorbox{yellow!30}{\emph{Primer.}} Ali je komutatvni/asociativni naslednji operaciji?
        \begin{itemize}
            \item Seštevanje in množenje na $\RR$.
            \item Odštevanje na $\RR$.            
            \item Vektorski produkt na $\RR^3$.            
            \item Kompozitum preslikav iz $A$ v $A$.
        \end{itemize}
        \item \colorbox{purple!30}{\textbf{Definicija.}} Enota ali nevtralni element za operacijo $\circ$. Leva/desna enota.
        \item \colorbox{yellow!30}{\emph{Primer.}} Določi enote, če obstajajo.
        \begin{itemize}
            \item Enota v $(\RR, +)$.
            Enota v $(\RR, \cdot)$.
            Enota v $(\RR^3, \times)$.
            Enota v $(F = \set{f: A \to A}, \circ)$.
        \end{itemize}
        \item \colorbox{blue!30}{\textbf{Trditev.}} Kaj če obstaja leva in desna enota za operacijo $\circ$?
        \begin{itemize}
            \item \colorbox{green!30}{\textbf{Dokaz.}} Enostavno izračunamo $e \circ f$.
        \end{itemize}
        \item \colorbox{orange!30}{\textbf{Posledica.}} Koliko enot lahko ima neka operacija $\circ$?
        \item \colorbox{purple!30}{\textbf{Definicija.}} Levi/desni inverz elementa $a$. Inverz elementa $a$. Kaj pravimo, če je $a$ ima inverz?
        \item \colorbox{yellow!30}{\emph{Opomba.}} Ali inverz nujno obstaja?
        \item \colorbox{blue!30}{\textbf{Trditev.}} Kaj če je obstaja levi in desni inverz elementa $a$ za neko asociativno operacijo z enoto?
        \begin{itemize}
            \item \colorbox{green!30}{\textbf{Dokaz.}} Enostavno.
        \end{itemize}
        \item \colorbox{orange!30}{\textbf{Posledica.}} Koliko inverzov lahko ima element $a$?
        \item Oznaka za inverz elementa $a$.
    \end{itemize}

    \newpage
    \item Grupe
    \begin{itemize}
        \item \colorbox{purple!30}{\textbf{Definicija.}} Grupoid. Polgrupa. Monoid.
        \item \colorbox{yellow!30}{\emph{Primer.}}  Ugotovi, ali je grupoid, polgrupa, monoid.
        \begin{itemize}
            \item $(\NN, +), \ (\ZZ, +), \ (\QQ, +),  \ (\RR, +)$.
            $(\NN, \cdot), \ (\ZZ, \cdot), \ (\QQ, \cdot),  \ (\RR, \cdot)$.
            $(\ZZ, -)$.
            $(\RR^3, \times)$.
            \item $(F = \set{f: A \to A}, \circ)$.
        \end{itemize}
        \item \colorbox{blue!30}{\textbf{Trditev.}} Kompozitum obrnljivih elementov v monoidu.
        \begin{itemize}
            \item \colorbox{green!30}{\textbf{Dokaz.}} Z računom pokažemo, da $(a \circ b)^{-1} = b^{-1} \circ a^{-1}$.
        \end{itemize}
        \item \colorbox{purple!30}{\textbf{Definicija.}} Grupa.
        \item \colorbox{yellow!30}{\emph{Primer.}}  Ugotovi, ali je grupa.
        \begin{itemize}
            \item $(\ZZ, +), \ (\QQ, +),  \ (\RR, +)$.
            $(\QQ, \cdot)$.
            $(\QQ \setminus \set{0}, \cdot),  \ (\RR \setminus \set{0}, \cdot), \ (\CC \setminus \set{0}, \cdot), (\set{1, -1, i, -i}, \cdot)$.
            $(\ZZ \setminus \set{0}, \cdot)$.            
            \item Naj bo $F$ množica vseh preslikav iz $A$ v $A$. Ali je $(F, \circ)$ grupa?  
            \item Naj bo $A$ množica in naj bo $S(A)$ množica vseh bijektivnih preslikav $A \to A$. Ali je $(S(A), \circ)$ grupa?
        \end{itemize}
        \item \colorbox{purple!30}{\textbf{Definicija.}} Multiplikativni zapis operacije: operacija, enota, potenca ($a^n, \ a^0, \ a^{-n}$).
        \item \colorbox{blue!30}{\textbf{Trditev.}} 2 lastnosti potenc elementa $a$ v grupi.
        \begin{itemize}
            \item \colorbox{green!30}{\textbf{Dokaz.}} Kot za znane formule o potencah števil.
        \end{itemize}
        \item \colorbox{orange!30}{\textbf{Posledica.}} $a^{-n} = (a^{n})^{-1}$.
        \begin{itemize}
            \item \colorbox{green!30}{\textbf{Dokaz.}} Pokažimo, da je inverz.
        \end{itemize}
        \item \colorbox{purple!30}{\textbf{Definicija.}} Komutativna ali Abelova grupa.
        \item \colorbox{yellow!30}{\emph{Primer.}}  Komutativni grupi.
        \begin{itemize}
            \item $(\ZZ, +), \ (\QQ, +),  \ (\RR, +)$.
            $(\QQ \setminus \set{0}, \cdot),  \ (\RR \setminus \set{0}, \cdot), \ ((0, \infty), \cdot)$.
        \end{itemize} 
        \item \colorbox{purple!30}{\textbf{Definicija.}} Aditivni zapis operacije (operacija, enota, vsota $n$ elementov, inverz).
        \item \colorbox{blue!30}{\textbf{Trditev.}} 3 lastnosti vsote elementov v Abelovi grupi.
        \begin{itemize}
            \item \colorbox{green!30}{\textbf{Dokaz.}} Prve dve kot prej, tretjo izračunamo.
        \end{itemize}
        \item \colorbox{yellow!30}{\emph{Opomba.}} Ali velja $(a \cdot b)^n = a^n \cdot b^n$, če $G$ ni komutativna?
        \item \colorbox{blue!30}{\textbf{Trditev.}} Ali lahko krajšamo v grupi?
        \begin{itemize}
            \item \colorbox{green!30}{\textbf{Dokaz.}} Pomnožimo z ustreznim inverzom.
        \end{itemize}
        \item \colorbox{yellow!30}{\emph{Opomba.}} Ali v grupi vedno velja $ab = ca \Rightarrow b = c$? Ali v splošnem velja krajšanje v polgrupah?
        \item \colorbox{purple!30}{\textbf{Definicija.}} Tabela množenja za končne grupe. Ali se lahko kakšen element v vrstici ali stolpcu ponovi?
        \item \colorbox{blue!30}{\textbf{Trditev.}} Karakterizacija komutativnosti končne grupe s tabelo množenja. 
    \end{itemize}

    \item[$\circ$] Grupe majhnih moči
    \begin{itemize}
        \item $n=1$. \textbf{Trivialna grupa.}
        \item \colorbox{purple!30}{\textbf{Definicija.}} Grupa $(\ZZ_n, +)$ ostankov za seštevanje po modulu $n$. Ali je komutativna?
        \item $n=2, \ n=3, \ n = 4, \ n= 5$.
        \item \colorbox{yellow!30}{\emph{Opomba.}} Ali so vse grupe moči največ $5$ komutativni?
    \end{itemize}

    \item[$\circ$] Grupe permutacij
    \begin{itemize}
        \item \colorbox{purple!30}{\textbf{Definicija.}} Permutacija.  Simetrična grupa reda $n$. Oznaka.
        \item \colorbox{yellow!30}{\emph{Opomba.}} Koliko elementov ima $S_n$? Ali je $S_n$ komutativna? Kako pišemo permutacije?
        \item \colorbox{yellow!30}{\emph{Primer.}} Elementi grupi $S_3$. Ali je $S_3$ komutativna? Tabela množenja za $S_3$.
        \item \colorbox{purple!30}{\textbf{Definicija.}} Cikel dolžine $k$. Transpozicija. Disjunktna cikla. Oznaka.
        \item \colorbox{yellow!30}{\emph{Opomba.}} Ali disjunktna cikla vedno komutirata? Cikel dolžine $1$. 
        \item \colorbox{yellow!30}{\emph{Primer.}} Ali so vsi elementi $S_3$ cikli? Kaj pa s elementi $S_4$?
        \item \colorbox{blue!30}{\textbf{Izrek.}} Permutacija kot produkt paroma disjunktnih ciklov.
        \begin{itemize}
            \item \colorbox{green!30}{\textbf{Dokaz.}} Obstoj: S strogo indukcijo na $n$. Oznaka $x_i = \pi^i(1)$ za vsak $i \in \NN_0$. Ali se eden iz med $x_i$ zagotovo ponovi?
            
            Enoličnost: Naj bo $\pi = \sigma_1 \ldots \sigma_m = \rho_1 \ldots \rho_l$. Ker cikli komutirajo lahko izberimo tak vrstni red, da se $\sigma_1$ in $\rho_1$ začneta z $1$, $\sigma_2$ in $\rho_2$ začneta z najmanjšim številom, ki se ne pojavi v $\sigma_1$ in $\rho_1$ in tako naprej.
        \end{itemize}
        \item \colorbox{yellow!30}{\emph{Primer.}} Zapiši kot produkt disjunktnih ciklov: $ \begin{pmatrix}
            1 & 2 & 3 & 4 & 5 & 6 & 7 & 8 & 9 \\
            3 & 9 & 1 & 6 & 4 & 5 & 2 & 7 & 8 
            \end{pmatrix}  $.
        \item \colorbox{blue!30}{\textbf{Trditev.}} Ali je vsak cikel produkt transpozicij?
        \begin{itemize}
            \item \colorbox{green!30}{\textbf{Dokaz.}} $(a_1 \ a_2 \ \ldots \ a_k) = (a_1 \ a_k)(a_1 \ a_{k-1}) \ldots (a_1 a_2)$.
        \end{itemize}
        \item \colorbox{orange!30}{\textbf{Posledica.}} Ali je vsaka permutacija produkt transpozicij? Ali je ta produkt komutativen oziroma enoličen?
        
        \newpage
        \item \colorbox{purple!30}{\textbf{Definicija.}} Znak permutacije.
        \item \colorbox{yellow!30}{\emph{Opomba.}} Zakaj je definicija dobra?
        \item \colorbox{yellow!30}{\emph{Primer.}} Določi znak permutacije: $ \begin{pmatrix}
            1 & 2 & 3 & 4 & 5 & 6 & 7 & 8 & 9 \\
            3 & 9 & 1 & 6 & 4 & 5 & 2 & 7 & 8 
            \end{pmatrix}  $.
        \item \colorbox{blue!30}{\textbf{Trditev.}} Kaj se zgodi z znakom permutacije, če jo z levo pomnožimo z transpozicijo?
        \begin{itemize}
            \item \colorbox{green!30}{\textbf{Dokaz.}} Za števili iz transpozicije $\tau$ imamo dve možnosti:
            
            (1) Števili se pojavita v istem ciklu.

            (2) Števili se pojavita v dveh različnih ciklih.
        \end{itemize}
        \item \colorbox{orange!30}{\textbf{Posledica.}} Znak permutacije v odvisnosti od števila transpozicij v razcepu.
        \begin{itemize}
            \item \colorbox{green!30}{\textbf{Dokaz.}} Z indukcijo na $M$ na podlagi prejšnje trditve.
        \end{itemize}
        \item \colorbox{orange!30}{\textbf{Posledica.}} Kaj se zgodi z znakom permutacije, če jo z desno pomnožimo z transpozicijo?
        \item \colorbox{orange!30}{\textbf{Posledica.}} Ali lahko iste permutacije zapišemo enkrat kot produkt sodega števila transpozicij, drugič pa kot produkt lihega števila transpozicij?
        \begin{itemize}
            \item \colorbox{green!30}{\textbf{Dokaz.}} Definicija znaka permutacije.
        \end{itemize}
        \item \colorbox{purple!30}{\textbf{Definicija.}} Soda/liha permutacija.
        \item \colorbox{orange!30}{\textbf{Posledica.}} Znak produkta permutacij.
        \begin{itemize}
            \item \colorbox{green!30}{\textbf{Dokaz.}} Razcepimo permutaciji na produkt transpozicij in izračunamo $s(\pi) s(\sigma)$.
        \end{itemize}
        \item \colorbox{blue!30}{\textbf{Trditev.}} Produkt sodih permutacij. Inverz sode permutacije.
        \begin{itemize}
            \item \colorbox{green!30}{\textbf{Dokaz.}} Izračunamo znak produkta in inverza.
        \end{itemize}
        \item \colorbox{purple!30}{\textbf{Definicija.}} Alternirajoča grupa reda $n$. Oznaka.
        \item \colorbox{yellow!30}{\emph{Opomba.}} Naj bo $n>1$ in $\tau \in S_n$ transpozicija. Ali je preslikava $f: \ A_n \to \text{lihe permutaciji}, \ f(\pi) = \tau \circ \pi$ bijektivna? Koliko elementov ima $A_n$?
    \end{itemize}

    \item[$\circ$] Podgrupe
    \begin{itemize}
        \item \colorbox{purple!30}{\textbf{Definicija.}} Podgrupa grupe $G$.
        \item \colorbox{blue!30}{\textbf{Trditev.}} Ali je podgrupa vedno vsebuje enoto grupe $G$?
        \begin{itemize}
            \item \colorbox{green!30}{\textbf{Dokaz.}} Definicija podgrupe.
        \end{itemize}
        \item \colorbox{orange!30}{\textbf{Posledica.}} Ali je podgrupa grupe $G$ grupa?
        \begin{itemize}
            \item \colorbox{green!30}{\textbf{Dokaz.}} Definicija podgrupe in prejšnja trditev.
        \end{itemize}
        \item \colorbox{yellow!30}{\emph{Primer.}}  Ugotovi ali je podgrupa:
        \begin{itemize}
            \item $(\ZZ, +)$ v  $(\QQ, +)$  v $(\RR, +)$.
            $(\QQ \setminus \set{0}, \cdot)$ v $(\RR \setminus \set{0}, \cdot)$.
            \item $(A_n, \circ)$ v $(S_n, \circ)$.
            Naj bo $\tau$ transpozicija. $\set{id, \tau}$ v $(S_n, \circ)$.
            \item \textbf{Neprava podgrupa. Trivialna podgrupa.}
        \end{itemize}
        \item \colorbox{blue!30}{\textbf{Trditev.}} Karakterizacija podgrupe multiplikativno pisane grupe.
        \begin{itemize}
            \item \colorbox{green!30}{\textbf{Dokaz.}} Definicija podgrupe.
        \end{itemize}
        \item \colorbox{orange!30}{\textbf{Posledica.}} Karakterizacija podgrupe Abelove aditivno pisane grupe.
        \item \colorbox{blue!30}{\textbf{Trditev.}} Ali je presek poljibne družine podgrup podgrupa?
        \begin{itemize}
            \item \colorbox{green!30}{\textbf{Dokaz.}} Enostavno z uporabo karakterizacije.
        \end{itemize}
        \item \colorbox{yellow!30}{\emph{Opomba.}} Ali je unija podgrup podgrupa? $G = S_3, \ H_1 = \set{\id, (1 \ 2)}, H_2 = \set{\id, (1 \ 3)}$.
    \end{itemize}

    \item[$\circ$] Homomorfizem grup
    \begin{itemize}
        \item \colorbox{purple!30}{\textbf{Definicija.}} Homomorfizem grup.
        \item \colorbox{yellow!30}{\emph{Primer.}}  Ugotovi ali je homomorfizem:
        \begin{itemize}
            \item $f: (\ZZ,+) \to (\QQ \setminus \set{0}, \cdot), \ f(x) := 2^x$.
            \item $s: S_n \to (\set{1, -1}, \cdot)$. Preslikava $s$ je znak permutacije.                        
            \item $g: (\CC \setminus \set{0}, \cdot) \to (\RR \setminus \set{0}, \cdot), \ g(z) := |z|$.
            \item Naj bo $G$ in $H$ poljubni grupi in je $e$ enota v $H$. $h: G \to H, \ h(x) := e$.
            Identiteta $id_G$.
            \item Naj bo $G$ poljubna grupa in $a \in G$. Definiramo preslikavo $f_a: G \to G$ s predpisom $f_a(x) := axa^{-1}$.
        \end{itemize}
        \item \colorbox{purple!30}{\textbf{Definicija.}} Izomorfizem. Monomorfizem. Epimorfizem. Endomorfizem. Avtomorfizem.
        \item \colorbox{yellow!30}{\emph{Primer.}} $f_a: G \to G$. Ali je to avtomorfizem? \textbf{Notranji avtomorfizem grupe $G$.}
        \item \colorbox{blue!30}{\textbf{Trditev.}} Kompozitum homomorfizmov. Inverz izomorfizma.
        \begin{itemize}
            \item \colorbox{green!30}{\textbf{Dokaz.}} (1) Enostavno po definiciji homomorfizma in kompozituma.
            
            (2) Enostavno po definiciji homomorfizma in inverzne preslikave.
        \end{itemize}
        \item \colorbox{blue!30}{\textbf{Trditev.}} Kam homomorfizem slika enoto in inverz?
        \begin{itemize}
            \item \colorbox{green!30}{\textbf{Dokaz.}} (1) Zapišemo $f(e) = f(e \cdot e) = f(e) \circ f(e)$ in pomnožimo z $f(e)^{-1}$ 
            
            (2) Pokažemo, da je $f(a^{-1})$ inverz od $f(a)$.
        \end{itemize}
        \item \colorbox{purple!30}{\textbf{Definicija.}} Izomorfni grupi. Oznaka. 
        \item \colorbox{yellow!30}{\emph{Opomba.}} Ali je izomorfnost ekvivalenčna relacija?
        \item \colorbox{blue!30}{\textbf{Trditev.}} Kam homomorfizem slika podgrupo?
        \begin{itemize}
            \item \colorbox{green!30}{\textbf{Dokaz.}} Pokažemo, da iz $x, y \in f(H)$ sledi, da $xy^{-1} \in f(H)$.
        \end{itemize}
        \item \colorbox{purple!30}{\textbf{Definicija.}} Slika homomorfizma. Jedro homomorfizma. Oznaki.
        \item \colorbox{yellow!30}{\emph{Opomba.}} Ali lahko $\ker f = \emptyset$?
        \item \colorbox{blue!30}{\textbf{Trditev}} o slike in jedru homomorfizma (ali je neki podgrupi?).
        \begin{itemize}
            \item \colorbox{green!30}{\textbf{Dokaz.}} Trditev za sliko sledi iz prejšnje trditve. Za jedro uporabimo karakterizacijo podgrupe.
        \end{itemize}
        \item \colorbox{yellow!30}{\emph{Opomba.}} Kadar je homomorfizem surjektiven?
        \item \colorbox{blue!30}{\textbf{Izrek.}} Karakterizacija injektivnosti homomorfizma. \textbf{Trivialno jedro.}
        \begin{itemize}
            \item \colorbox{green!30}{\textbf{Dokaz.}} Definicija injektivnosti, jedra, homomorfizma in trditev o slike enote in inverza.
        \end{itemize}
        \item \colorbox{yellow!30}{\emph{Primer.}} Ali je naslednja homomorfizma injektivna?
        \begin{itemize}
            \item $f: (\ZZ,+) \to (\QQ \setminus \set{0}, \cdot), \ f(x) := 2^x$.
            \item $g: (\CC \setminus \set{0}, \cdot) \to (\RR \setminus \set{0}, \cdot), \ g(z) := |z|$. Ali je enotska krožnica grupa za množenje.
        \end{itemize}
    \end{itemize}

    \item Kolobarji
    \begin{itemize}
        \item \colorbox{purple!30}{\textbf{Definicija.}} Kolobar. Komutativen kolobar. Kolobar z enoto.
        \item \colorbox{blue!30}{\textbf{Trditev.}} 2 lastnosti kolobarja.
        \begin{itemize}
            \item \colorbox{green!30}{\textbf{Dokaz.}} Množenje z $0$: Distributivnost.
            
            Množenje z obratnim elementom: Pokažemo, da je res obratni element.
        \end{itemize}
        \item \colorbox{yellow!30}{\emph{Primer.}}  Ugotovi ali je kolobar. Ali je komativen in ima enoto?
        \begin{itemize}
            \item  $(\ZZ, +, \cdot), \  (\QQ, +, \cdot), \ (\RR, +, \cdot), \ (\text{soda števila}, +, \cdot)$.
            \item Definicija množenja v $\ZZ_n$. $(\ZZ_n, +, \cdot)$. 
            \item Definiramo množenje v $\RR^3$ s predpisom $(x_1, y_1, z_1) \cdot (x_2, y_2, z_2) = (x_1 x_2, x_1y_2 + y_1 z_1, z_1 z_2)$. $(\RR^3, +, \cdot)$.
            \item Naj bo $\Ff$ množica vseh funkcij $f: \RR \to \RR$. Na $\Ff$ definiramo seštevanje in množenje po točkah: $(f~+~g)(x)=f(x)+g(x), \ (f \cdot g)(x) = f(x) \cdot g(x)$. Ali je $(\Ff, +, \cdot)$ kolobar?
            \item Množica polinomov za običajno seštevanje in množenje polinomov.
            \item Naj bo $A$ Abelova grupa in $\End (A)$ množica vseh endomorfizmov $A \to A$. Na $\End (A)$ definiramo seštevanje po točkah. Ali je $(\End (A), +, \circ)$ kolobar?
        \end{itemize}
        \item \colorbox{purple!30}{\textbf{Definicija.}} Delitelja niča. Levi/desni delitelj niča.
        \item \colorbox{yellow!30}{\emph{Primer.}} Poišči deljitelja niča v $\ZZ_6$.
        \item \colorbox{purple!30}{\textbf{Definicija.}} Obseg. Polje.
        \item \colorbox{yellow!30}{\emph{Opomba.}} Karakterizacija obsega (kadar je kolobar obseg?). Ali treba predpostaviti, da $1 \neq 0$?
        \item \colorbox{yellow!30}{\emph{Opomba.}} Ali lahko definiramo deljenje v polju? Ali lahko tudi definiramo deljenje v obsegu?
        \item \colorbox{yellow!30}{\emph{Primer.}} Številski obsegi: $\QQ, \ \RR, \ \CC$.
        \item \colorbox{blue!30}{\textbf{Trditev.}} Ali (levi ali desni) delitelj niča lahko bi bil obrnljiv?
        \begin{itemize}
            \item \colorbox{green!30}{\textbf{Dokaz.}} Enostavno s protislovjem.
        \end{itemize}
        \item \colorbox{orange!30}{\textbf{Posledica.}} Ale je v obsegu delitelji niča?
        \item \colorbox{blue!30}{\textbf{Trditev.}} Kadar je $(\ZZ_n, +, \cdot)$ obseg?
        \begin{itemize}
            \item \colorbox{green!30}{\textbf{Dokaz.}} $(\Rightarrow)$ S protislovjem pokažemo, da ima obseg delitelje niča.
            
            $(\Leftarrow)$ Naj bo $a \in \ZZ_n \setminus \set{0}$. Iščemo $a^{-1}$. Izmed števil $1 \cdot a, 2 \cdot a, \ldots, (n-1) \cdot a$ najdemo tisto, ki je enako $1$ (ali se lahko kakšen ostanek ponovi?). 
        \end{itemize}
        \item \colorbox{purple!30}{\textbf{Definicija.}} Podkolobar. 
        \item \colorbox{yellow!30}{\emph{Primer.}} $\ZZ$ je podkolobar v $\QQ$. Trivialni kolobar.
        \item \colorbox{purple!30}{\textbf{Definicija.}} Homomorfizem kolobarjev.
        \item \colorbox{purple!30}{\textbf{Definicija.}} Jedro in slika homomorfizma.
        \item \colorbox{blue!30}{\textbf{Trditev}} o slike in jedru homomorfizma (ali lahko neki podkolobarji)?
        \begin{itemize}
            \item \colorbox{green!30}{\textbf{Dokaz.}} Treba pokazati le zaprtost za množenje.
        \end{itemize}
        \item \colorbox{blue!30}{\textbf{Trditev.}} Karakterizacija injektivnosti homomorfizma kolobarjev.
        \begin{itemize}
            \item \colorbox{green!30}{\textbf{Dokaz.}} Homomorfizem kolobarjev je tudi homomorfizem grup.
        \end{itemize}
        \item \colorbox{orange!30}{\textbf{Posledica.}} Naj bo $f: O \to K$ homomorfizem kolobarjev, kjer je $O$ obseg. Kaj lahko povemo o $f$?
        \begin{itemize}
            \item \colorbox{green!30}{\textbf{Dokaz.}} Recimo da $f$ ni ničeln in s protislovjem pokažemo, da je $\ker f = \set{0}$.
        \end{itemize} 
        \item \colorbox{purple!30}{\textbf{Definicija.}} Podobseg. 
        \item \colorbox{yellow!30}{\emph{Primer.}} $\QQ$ je podobseg v $\RR$.
    \end{itemize}
    
    \newpage
    \item Vektorski prostor
    \begin{itemize}
        \item \colorbox{purple!30}{\textbf{Definicija.}} Vektorski prostor nad poljem $\FF$. Operacija množenje s skalarji.
        \item \colorbox{yellow!30}{\emph{Primer.}} Ugotovi ali je vektorski prostor.       
            \begin{itemize}
                \item $\RR^2$ in $\RR^3$ nad $\RR$ za običajne operacije po kompinentah.
                \item $\FF^n$ nad $\FF$ za običajne operacije po kompinentah.
                \item $\CC$ nad $\RR$ za običajno seštevanje in množenje kompleksnih števil z realnimi..
                \item Množica $\mathcal{F}$ vseh funkcij iz $M \neq \emptyset$ v $\RR$ nad $\RR$ za seštevanje funkcij in množenje funkcij s skalarji po točkah.                
                \item Prostor polinomov z realnimi koeficienti nad $\RR$ za običajno seštevanje polinomov in množenje polinomov s števili. \emph{Oznake.} $\RR[x], \ \CC[x], \ \RR[x,y]$.
            \end{itemize}
        \item \colorbox{blue!30}{\textbf{Trditev.}} 4 lastnosti vektorskega prostora.
        \begin{itemize}
            \item \colorbox{green!30}{\textbf{Dokaz.}} (1) - (2). Ustrezna distributivnost iz definiciji.\
            
            (3) Pokažimo, da je inverz. 

            (4) Uporabimo lastnosti polja z predpostavko, da $\alpha \neq 0$.
        \end{itemize}
        \item \colorbox{purple!30}{\textbf{Definicija.}} Vektorski podprostor.
        \item \colorbox{yellow!30}{\emph{Primer.}} Pokaži, da je vektorski podprostori.        
            \begin{itemize}
                \item Trivialni podprostor. Vektorski prostor kot podprostor samega sebe.
                \item  Vse premice skozi izhodišče in vse ravnine skozi izshodišče v $\RR^3$.
                \item  Polinome stopnje največ $n$ v $\RR[x]$.
            \end{itemize}
        \item \colorbox{blue!30}{\textbf{Trditev.}} Ali je vektorski podprostor vektorski prostor?
        \begin{itemize}
            \item \colorbox{green!30}{\textbf{Dokaz.}} Pokažemo, da je $(W, +)$ Abelova grupa ter preverimo ostale predpostavke.
        \end{itemize}
        \item \colorbox{purple!30}{\textbf{Definicija.}} Linearna kombinacija vektorjev $\x{n}$.
        \item \colorbox{blue!30}{\textbf{Trditev.}} Karakterizacija vektorskega podprostora (tvorenje linarnih kombinacij).
        \begin{itemize}
            \item \colorbox{green!30}{\textbf{Dokaz.}} Definicija vektorskega podprostora + indukcija.
        \end{itemize}
        \item \colorbox{orange!30}{\textbf{Posledica.}} Karakterizacija vektorskega podprostora (linearna kombinacija vektorjev $x$ in $y$).
        \begin{itemize}
            \item \colorbox{green!30}{\textbf{Dokaz.}} Definicija vektorskega podprostora.
        \end{itemize}
        \item \colorbox{blue!30}{\textbf{Trditev.}} Ali je presek poljubne družine vektorskih podprostorov vektorski podprostor v $V$?
        \begin{itemize}
            \item \colorbox{green!30}{\textbf{Dokaz.}} Treba dokazati samo zaprtost za množenje s skalarji.
        \end{itemize}
        \item \colorbox{orange!30}{\textbf{Posledica.}} Ali obstaja najmanjši vektorski podprostor v $V$, ki vsebuje množico $M \subseteq V$?
        \begin{itemize}
            \item \colorbox{green!30}{\textbf{Dokaz.}} Pokažemo, da je presek ustrezne družine najmanjši.
        \end{itemize}
        \item \colorbox{purple!30}{\textbf{Definicija.}} Linearna ogrinjača ali linearna lupina. Oznaka.
        \item \colorbox{blue!30}{\textbf{Trditev.}} Opis množice $\Lin M = W$.
        \begin{itemize}
            \item \colorbox{green!30}{\textbf{Dokaz.}} Ena vsebovanost je očitna. Za drugo pokažemo, da je $W \leq V$ ter upoštevamo definicijo $\Lin M$.
        \end{itemize}
        \item \colorbox{yellow!30}{\emph{Primer.}} Določi linearne ogrinjače:        
        \begin{itemize} 
            \item $\Lin \set{v_1}$.
            $\Lin \set{v_1, v_2}$.            
            Naštej vektorski podprostori v $\RR^3$.
        \end{itemize}
        \item \colorbox{yellow!30}{\emph{Opomba.}} Ali je unija vektorskih podprostorov vektorski podprostor?
        \item \colorbox{purple!30}{\textbf{Definicija.}} Vsota podprostorov. Oznaka.
        \item \colorbox{blue!30}{\textbf{Trditev.}} Opis vsote podprostorov (kaj vsebuje ta množica?).
        \begin{itemize}
            \item \colorbox{green!30}{\textbf{Dokaz.}} Definicija vsote podprostorov + definicija in opis linearne ogrinjače.
        \end{itemize}
        \item \colorbox{purple!30}{\textbf{Definicija.}} Direktna vsota podprostorov. Oznaka.
        \item \colorbox{blue!30}{\textbf{Izrek.}} Karakterizacija direktne vsote podprostorov.
        \begin{itemize}
            \item \colorbox{green!30}{\textbf{Dokaz.}} $(\Rightarrow)$ Vzamemo vektor $x$ iz preseka in pokažemo z definicijo direktne vsote, da je $x=0$.
            
            $(\Leftarrow)$ Recimo, da $x_1 + \ldots + x_k = y_1 + \ldots + y_k$. Vzemimo poljuben $j$ in zapišemo zadnjo enakost kot $x_j - y_j = (y_1 - x_1) + \ldots + (y_{j-1} - x_{j-1}) +  (y_{j+1} - x_{j+1}) + \ldots + (y_k - x_k)$. 
        \end{itemize}
        \item \colorbox{orange!30}{\textbf{Posledica.}} Kadar je vsota podprostorov $W_1$ in $W_2$ direktna?
        \item \colorbox{yellow!30}{\emph{Primer.}} Kaj je vsota poljubne premice skozi izhodišče in poljubne ravnine skozi izhodišče, tako da premica ne leži na ravnine? Ali je ta vsota direktna? Ali je vsota dveh ravnin skozi izhodišče direktna?
    \end{itemize}

    \newpage
    \item Linearne preslikave
    \begin{itemize}
        \item \colorbox{purple!30}{\textbf{Definicija.}} Linearna preslikava. Oznaka. Oznaka za množico vseh linearnih preslikav iz $U$ v $V$. 
        \item \colorbox{purple!30}{\textbf{Definicija.}} Endomorfizem prostora $U$. Oznaka za množico vseh endomorfizmov prostora.
        \item \colorbox{purple!30}{\textbf{Definicija.}} Linearen funkcional. Oznaka za množico vseh linearnih funkcionalov.
        \item \colorbox{yellow!30}{\emph{Primer.}} Ugotovi ali je preslikava linearna:   
        \begin{itemize}
            \item Ničelna preslikava.
            \item Identiteta. Ali je endomorfizem?
            \item Projekcija na $xy$-ravnino: $\mathcal{A}: \RR^3 \to \RR^2, \ \mathcal{A} \, (x_1, x_2, x_3) := (x_1, x_2)$.
            \item Fiksiramo $\vec{a} \in \RR^3$: $\mathcal{A}: \RR^3 \to \RR^3, \ \mathcal{A} \, \vec{x} := \vec{x} \times \vec{a}$.            
            \item Naj bo $\Ff$ prostor vseh integrabilnih funkcij, definiranih na $[a,b]$. $\Aa: \Ff \to \RR, \ \Aa \, f := \int_{a}^{b} f(x) \,dx $. Ali je linearen funkcional?            
            \item Naj bo $\RR[x]$ prostor vseh polinomov. $\Aa: \RR[x] \to \RR[x], \ \Aa \, f := f'$.
        \end{itemize}
        \item \colorbox{blue!30}{\textbf{Trditev.}} Karakterizacija linearnosti preslikave (3 ekvivalentne trditve).
        \begin{itemize}
            \item \colorbox{green!30}{\textbf{Dokaz.}} $(2) \Rightarrow (3)$ in $(3) \Rightarrow (1)$ sta očitni. $(1) \Rightarrow (2)$ z indukcijo na $n$.
        \end{itemize}
        \item \colorbox{purple!30}{\textbf{Definicija.}} Jedro in slika linearne preslikave. Oznaki.
        \item \colorbox{yellow!30}{\emph{Opomba.}} Ali je linearna preslikava homomorfizem Abelovih grup? Kaj to pove o $\ker \mathcal{A}$ in $\im \mathcal{A}$?
        \item \colorbox{blue!30}{\textbf{Trditev.}} Ali je jedro in slika linearne preslikave vektorski podprostori?
        \begin{itemize}
            \item \colorbox{green!30}{\textbf{Dokaz.}} Zaprtost za seštevanje sledi iz prejšnje opombe. Enostavno pokažemo še zaprtost za množenje s skalarji.
        \end{itemize}
        \item \colorbox{blue!30}{\textbf{Trditev.}} Kadar je linearna preslikava injektivna?
        \begin{itemize}
            \item \colorbox{green!30}{\textbf{Dokaz.}} Ker je linearna preslikava homomorfizem Abelovih grup, trditev sledi.
        \end{itemize}
        \item Definicija seštevanja in množenja s skalarji na množice $L(U, V)$.
        \item \colorbox{blue!30}{\textbf{Trditev.}} Linearnost preslikav $\Aa + \Bb$, $\alpha \Aa$.
        \begin{itemize}
            \item \colorbox{green!30}{\textbf{Dokaz.}} Po definiciji ali s karakterizacijo linearnosti.
        \end{itemize}
        \item \colorbox{blue!30}{\textbf{Trditev.}} Ali je $L(U, V)$ vektorski prostor nad $\FF$ za zgoraj definirani operaciji?
        \begin{itemize}
            \item \colorbox{green!30}{\textbf{Dokaz.}} Najprej pokažemo, da je množica vseh homomorfizmov Abelovih grup $(U, +)$, $(V,+)$ je Abelova (isto kot za endomorfizme v poglavju o kolobarjih) in grupa $L(U, V)$ podgrupa tej grupe. Aksiome pa preverimo z računom.
        \end{itemize}
        \item \colorbox{blue!30}{\textbf{Trditev}} o kompozicije linearnih preslikav in inverze bijektivne linearne preslikave.
        \begin{itemize}
            \item \colorbox{green!30}{\textbf{Dokaz.}} Aditivnost sledi iz tega, da so linearne preslikave tudi homomorfizmi Abelovih grup. Homogenost kompozituma pokažemo enostavno, inverza pa z uporabo $\Aa^{-1}$ na izrazu za $\Aa (\alpha y)$, kjer $y = \Aa^{-1}x$.
        \end{itemize}
        \item \colorbox{purple!30}{\textbf{Definicija.}} Izomorfizem vektorskih prostorov. Izomorfni vektorski prostori. Oznaka. 
        \item \colorbox{yellow!30}{\emph{Opomba.}} Ali je izomorfnost vektorskih prostorov ekvivalenčna relacija?
        \item \colorbox{blue!30}{\textbf{Trditev.}} Ali je $(L(V), +, \circ)$ kolobar? Ali ima enoto?
        \begin{itemize}
            \item \colorbox{green!30}{\textbf{Dokaz.}} Preveriti treba distributivnost.
        \end{itemize}
    \end{itemize}

    \item Algebra
    \begin{itemize}
        \item \colorbox{purple!30}{\textbf{Definicija.}} Algebra nad poljem $\FF$. Komutativna algebra. Algebra z enoto.
        \item \colorbox{yellow!30}{\emph{Primer.}} Ugotovi ali je algebra in ali je komutativna in ima enoto:
        \begin{itemize}
            \item $\FF$ nad $\FF$, če množenje s skalarji definiramo kot običajno množenje v $\FF$.
            \item $\CC$ nad $\RR$.            
            \item Naj bo $M$ neprazna množica in $\Ff$ množica vseh funkcij iz $M$ v $\RR$. Na $\Ff$ vse operacije definiramo po točkah. Ali je $\Ff$ algebra nad $\RR$?            
            \item $L(V)$ nad $\FF$ za seštevanje in množenje s skalarji po točkah in kompozitum.
        \end{itemize}
    \end{itemize}
\end{enumerate}