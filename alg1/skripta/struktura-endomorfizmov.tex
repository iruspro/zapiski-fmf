\section{STRUKTURA ENDOMORFIZMOV}

\textbf{Problem.} Naj bo $\mathcal{A}$ endomorfizem prostora $V$. Kakšno bazo $\mathcal{B}$ prostora $V$ moramo izbrati, da bo matrika $\mathcal{A}_{\mathcal{B},\mathcal{B}}$ čim bolj enostavna?

\begin{enumerate}

    \item Lastne vrednosti in lastni vektorji
    \begin{itemize}
        \item \colorbox{purple!30}{\textbf{Definicija.}} Kadar endomorfizem $\Aa \in L(V)$ se da diagonalizirati? \textcolor{red}{[Ponovitev]}
        \item \colorbox{purple!30}{\textbf{Definicija.}} Naj bo $\Aa \in L(V)$. Lastni vektor endomorfizma $\Aa$. Lastna vrednost endomorfizma $\Aa$ za lastni vektor $v$. \textcolor{red}{[Ponovitev]}
        \item \colorbox{yellow!30}{\emph{Opomba.}} Koliko lastnih vrednosti pripadajo lastnemu vektorju? Koliko lastnih vektorjev pripadajo lastni vrednosti? \textcolor{red}{[Ponovitev]}
        \item \colorbox{blue!30}{\textbf{Trditev.}} Ali je množica $\set{x \in V; \mathcal{A} x = \lambda x}$, kjer $\lambda$ lastna vrednost endomorfizma $\mathcal{A} \in L(V)$, vektorski podprostor prostora $V$?
        \begin{itemize}
            \item \colorbox{green!30}{\textbf{Dokaz.}} Pokažemo, da $Ax = \lambda x \Leftrightarrow x \in \ker (\Aa - \lambda \id)$.
        \end{itemize}
        \item \colorbox{purple!30}{\textbf{Definicija.}} Lastni podprostor endomorfizma $\mathcal{A}$ za lastno vrednost $\lambda$. Geometrijska večkratnost lastne vrednosti $\lambda$.
        \item \colorbox{yellow!30}{\emph{Opomba.}} Čemu je enak lastni podprostor endomorfizma $\mathcal{A}$ za lastno vrednost $\lambda?$ Kaj ga sestavljajo?
        \item \colorbox{blue!30}{\textbf{Trditev.}} Karakterizicija lastne vrednosti.
        \begin{itemize}
            \item \colorbox{green!30}{\textbf{Dokaz.}} Pokažemo, da $\lambda$ je lastna vrednost endomorfizma $\Aa \Leftrightarrow \det (\Aa_{B,B} - \lambda I) = 0$.
        \end{itemize}
        \item \colorbox{yellow!30}{\emph{Primer.}} Poišči lastne vrednosti in lastne vektorji matrike $\begin{bmatrix}
            -1 & -2 & -2 \\
            -2 & -1 & -2 \\
            2 & 2 & 3
        \end{bmatrix}$.
        \item \colorbox{blue!30}{\textbf{Trditev.}} Karakterizacija diagonalizacije endomorfizma (baza prostora $V$).
        \begin{itemize}
            \item \colorbox{green!30}{\textbf{Dokaz.}} Po definiciji matrike linearne preslikave, diagonalizaciji endomorfizma, lastnih vektorjev in lastnih vrednosti.
        \end{itemize}
        \item \colorbox{orange!30}{\textbf{Posledica.}} Karakterizacija diagonalizacije endomorfizma (število linearno neodvisnih lastnih vektorjev).
        \item \colorbox{yellow!30}{\emph{Opomba.}} Recimo, da se endomorfizem da diagonalizirati. Kaj so členi pripadajoče diagonalne matrike?
        \item \colorbox{yellow!30}{\emph{Primer.}} Naj bo $A \in \CC^{n \times n}$ matrika, ki jo identificiramo s preslikavo $\CC^n \to \CC^n, \ x \mapsto Ax$. \textbf{Kadar matrika se da diagonalizirati?}  Recimo, da je $A = PDP^{-1}$ za neko diagonalno matriko $D$ in neko obrnljivo matriko $P$. Kaj so diagonalni členi matrike $D$? Kaj je $P$?        
        \begin{itemize}
            \item \colorbox{green!30}{\textbf{Dokaz.}} Izračunamo $A = PDP^{-1} \Leftrightarrow AP = DP \Leftrightarrow \text{ za vsak } i = 1, \ldots, n \text{ velja: } APe_i = PDe_i$.
        \end{itemize}
        \item \colorbox{blue!30}{\textbf{Trditev.}} Ali so linearno neodvisni lastni vektorji, ki pripadajo različnim lastnim vrednostim?
        \begin{itemize}
            \item \colorbox{green!30}{\textbf{Dokaz.}} S protislovjem pokažemo, da je eden izmed lastnih vektorjev ničeln.
        \end{itemize}
        \item \colorbox{orange!30}{\textbf{Posledica.}} Kaj velja, če ima endomorfizem prostora $V$, za kateri velja $\dim V = n$, $n$ različnih lastnih vrednosti?
    \end{itemize}

    \item Karakteristični in minimalni polinom
    \begin{itemize}
        \item Naj bo $\Aa \in L(V)$ endomorfizem in $A = \Aa_{B, B}$ matrika tega enodmorfizma glede na neko bazo $B$ prostora $V$. 
        
        Ali je $\det (A - \lambda I)$ polinom v $\lambda$? Kakšne stopnje, kaj je vodilni koeficient in kaj je konstantni člen?
        
        \textbf{Kroneckerjev delta.}
        \item \colorbox{purple!30}{\textbf{Definicija.}} Karakteristični polinom matrike $A$. Oznaka.
        \item \colorbox{blue!30}{\textbf{Trditev.}} Ali podobne matrike imata enak karakteristični polinom?
        \begin{itemize}
            \item \colorbox{green!30}{\textbf{Dokaz.}} Naj bo $B = P^{-1}AP$. Izračunamo $\Delta_B(\lambda)$.
        \end{itemize}
        \item \colorbox{purple!30}{\textbf{Definicija.}} Karakteristični polinom endomorfizma $\Aa \in L(V)$. Ali je definicija smiselna?
        \item \colorbox{blue!30}{\textbf{Izrek.}} Kako povezane lastne vrednosti endomorfizma $\Aa \in L(V)$ in njegov karakteristini polinom?
        \item \colorbox{yellow!30}{\emph{Primer.}} $V = \RR^2, \ \Aa: \RR^2 \to \RR^2$ rotacija za $\pi/2$ v pozitivni smeri. Kaj je lastni vektorji? Kaj so ničle karakterističnega polinoma?
        \item[\textbf{!}] Od zdaj naprej in do konca poglavja o strukturi endomorfizmov bomo predpostavili, da je $\FF = \CC$. 
        
        Zakaj to smiselno?
        \item \colorbox{blue!30}{\textbf{Osnovni izrek algebre.}} \textcolor{red}{[Dokaz pri Analize 2]}
        \item \colorbox{orange!30}{\textbf{Posledica.}} Kakšne je oblike vsak nekonstanten polinom $p \in \CC[\lambda]$?
        \item \colorbox{orange!30}{\textbf{Posledica.}} Kadaj je $\alpha \in \CC$ lastna vrednost endomorfizma $\Aa \in L(V)$?
        \item \colorbox{yellow!30}{\emph{Opomba.}} Nad kakšnim poljem bomo obravnavali matrike?
        \item \colorbox{purple!30}{\textbf{Definicija.}} Lastne vrednosti matrike $A$. 
        \item \colorbox{yellow!30}{\emph{Opomba.}} Ali lastne vrednosti matrike je lastne vrednoti nekega endomorfizma?
        \item \colorbox{purple!30}{\textbf{Definicija.}} Algebraične večkratnosti lastnih vrednosti. Spekter matrike/endomorfizma. 
        Oznaka.  
        
        \newpage
        \item \colorbox{purple!30}{\textbf{Definicija.}} Kadar matrika $A$ se da diagonalizirati?
        \item \colorbox{yellow!30}{\emph{Opomba.}} Karakterizacija diagonalizacije matrike. (Ali mora biti podobna neke diagonalne?) Kaj je v tem primeru na diagonali matrike $D$? Kaj so stolpce matrike $P$?
        \item \colorbox{blue!30}{\textbf{Trditev.}} (1) Kakšna zveza med algebraično in geometrično večkratnostjo lastne vrednosti matrike $A$?
        
        (2) Karakterizacija diagonalizaciji matrike $A$.        
        \begin{itemize}
            \item \colorbox{green!30}{\textbf{Dokaz.}} (1) Naj bo $m = \dim(\ker(A- \alpha I))$ geometrična večkratnost lastne vrednosti $\alpha$. Dopolnimo bazo $\ker(A- \alpha I)$ do baze $B$ za $\CC^n$ in z pomočjo matrike $A_{B, B}$ izračunamo $\Delta_A(\lambda)$.
            
            (2) $(\Rightarrow)$ Naj bodo $\alpha_1, \ldots, \alpha_k$ vse različne lastne vrednosti matrike $A$, $n_1, \ldots, n_k$ algebraične večkratnosti in $m_1, \ldots, m_k$ geometrične večkratnosti. Ocenimo zgoraj in spodaj izraz $n_1 + \ldots + n_k$ (koliko linearno neodvisnih lastnih vektorjev ima matrika $A$, če se da diagonalizirati?).

            $(\Leftarrow)$ Za vsak $i$ izberimo bazo jedra $\ker(A - \alpha_i I): \set{v_{i,1}, v_{i, 2}, \ldots, v_{i, n_i}}$. Pokažemo, da so vektorji $v_{ij}$ tvorijo bazo prostora $\CC^n$.
        \end{itemize}
        \item \colorbox{yellow!30}{\emph{Opomba.}} Ali enako velja za endomorfizmi?
        \item \colorbox{purple!30}{\textbf{Definicija.}} Vrednost polinoma $p \in \CC[\lambda]$ v matrike $A \in F^{n \times n}$. 
        
        Vrednost polinoma $p \in \CC[\lambda]$ v endomorfizmu $\Aa \in L(V)$.
        \item \colorbox{purple!30}{\textbf{Definicija.}} Matrični polinom.
        \item \colorbox{blue!30}{\textbf{Izrek.}} Cayley-Hamiltonov izrek.
        \begin{itemize}
            \item \colorbox{green!30}{\textbf{Dokaz.}} Za vsak $\lambda \in \CC$ naj bo $B(\lambda) = \widetilde{A - \lambda I}$ prirejenka matrike $A - \lambda I$.
            
            Vemo: $B(\lambda)^T \cdot (A - \lambda I) = \det (A - \lambda I) \cdot I = \Delta_A(\lambda) \cdot I$ za vsak $\lambda \in \CC$. Z primejavo koeficientov matričnih polinomov pri posameznih potencah pridemo do rezultata.
        \end{itemize}
        \item \colorbox{orange!30}{\textbf{Posledica.}} Naj bo $A \in \CC^{n \times n}$ obrnljiva. Ali obstaja polinom $p$, da je $A^{-1} = p(A)$. Kakšne stopnje ta polinom?
        \begin{itemize}
            \item \colorbox{green!30}{\textbf{Dokaz.}} Z uporabo prejšnjega izreka dobimo $A^{-1}$.
        \end{itemize}
        \item \colorbox{purple!30}{\textbf{Definicija.}} Minimalni polinom matrike $A \in \CC^{n \times n}$.
        \item \colorbox{yellow!30}{\emph{Opomba.}} Ali je minimalni polinom obstaja? Ali je enoličen?
        \item \colorbox{blue!30}{\textbf{Trditev.}} Ali podobni matriki imata enak minimalni polinom?
        \begin{itemize}
            \item \colorbox{green!30}{\textbf{Dokaz.}} Naj bo $B = P^{-1}AP$ in $p(\lambda) = a_m\lambda^m + a_{m-1}\lambda^{m-1}+ \ldots + a_1 + \lambda a_0$ poljuben polinom. Z~izračunom $p(B)$ pokažemo, da je $p(A) = 0 \Leftrightarrow p(B) = 0$.
        \end{itemize}
        \item \colorbox{purple!30}{\textbf{Definicija.}} Minimalni polinom endomorfizma $\Aa \in L(V)$.
        \item \colorbox{yellow!30}{\emph{Opomba.}} Ali je definicja dobra? Zakaj?
        \item \colorbox{blue!30}{\textbf{Trditev.}} Opis minimalnega polinoma endomorfizma $\Aa \in L(V)$.
        \begin{itemize}
            \item \colorbox{green!30}{\textbf{Dokaz.}} Najprej pokažemo, da za poljuben polinom $p \in \CC[\lambda]$ velja: $\Phi(p(\Aa)) = p(\Phi(\Aa))$. Nato pokažemo, da ima minimalni polinom vodilni koeficient $1$, $\Aa$ za ničlo in, da minimalne možne stopnje med vsemi neničelnimi polinomi.
        \end{itemize}
        \item \colorbox{yellow!30}{\emph{Opomba.}} Ali vse, kar velja za minimalni polinomi matrik, velja tudi za minimalne polinome endomorfizmov, in obratno?
        \item \colorbox{blue!30}{\textbf{Izrek.}} Osnovni izrek o deljenju polinomov. \textbf{Kvocient} in \textbf{ostanek}. \textcolor{red}{[Dokaz v gimnaziji]}
        \item \colorbox{blue!30}{\textbf{Izrek.}} Naj bo $A \in \CC^{n \times n}$ in $p \in \CC[\lambda]$ tak polinom, da $p(A) = 0$. V kakšni zvezi sta $p$ in $m_A$?
        \begin{itemize}
            \item \colorbox{green!30}{\textbf{Dokaz.}} Po osnovnemu izreku o deljenju polinomov obstajata $k, r \in \CC[\lambda]$, da $p(\lambda) = k(\lambda) \cdot m_A(\lambda) + r(\lambda)$. Treba je dokazati, da $r \equiv 0$, zato poglejmo vrednost v matrike $A$.
        \end{itemize}
        \item \colorbox{orange!30}{\textbf{Posledica.}} Ali $m_A \, | \, \Delta_a$?
        \begin{itemize}
            \item \colorbox{green!30}{\textbf{Dokaz.}} Prejšnja trditev + Cayley-Hamiltonov izrek.
        \end{itemize}
        \item \colorbox{orange!30}{\textbf{Posledica.}} Ali vse ničle minimalnega polinoma so tudi ničle karakterističnega polinoma?
        \item \colorbox{blue!30}{\textbf{Izrek.}} Ali vse ničle karakterističnega polinoma so tudi ničle minimalnega polinoma?
        \begin{itemize}
            \item \colorbox{green!30}{\textbf{Dokaz.}} Naj bo $\alpha$ ničla $\Delta_A(\lambda)$. Po osnovnemu izreku o deljenju polinomov, obstajata $k, c \in \CC[\lambda]$, da $m_A(\lambda) = k(\lambda) \cdot (\lambda - \alpha) + c$. Radi bi pokazali, da je $c = 0$. Zato poglejmo vrednost polinomov v matriki $A$ in upoštevamo, da je $\alpha$ lastna vrednost matrike $A$.
        \end{itemize}
        \item \colorbox{orange!30}{\textbf{Posledica.}} Naj bo $A \in \CC^{n \times n}$ kakšne oblike je njen karakteristični in kakšne oblike je njen minimalni polinom?
        \item \colorbox{yellow!30}{\emph{Primer.}} Poišči minimalni polinom matrike $\begin{bmatrix}
            1 & 0 & 0 & 0 \\
            0 & 0 & 1 & 0 \\
            0 & 0 & 0 & 0 \\
            0 & 0 & 0 & 0
        \end{bmatrix}$.
    \end{itemize}

    \newpage
    \item Korenski podprostori
    
    V nadaljevanju bomo poiskali lepo matriko, ki pripada endomorfizmu, v primeru, ko se endomorfizem ne da diagonalizirati. Za začetek bomo pokazali, da endomorfizmu vedno pripada bločno diagonalna matrika, kjer ima vsak blok le eno lastno vrednost.
    \begin{itemize}
        \item \colorbox{purple!30}{\textbf{Definicija.}} Naj bo $\Aa \in L(V)$. Invarianten podprostor za endomorfizem $\Aa$.
        \item \colorbox{yellow!30}{\emph{Primer.}} Ugotovi, ali je podprostor invarianten.
        \begin{itemize}
            \item $\set{0}, V$ za poljuben endomorfizem $\Aa \in L(V)$.
            \item $\ker \Aa$ in $\im \Aa$ za poljuben endomorfizem $\Aa \in L(V)$.
            \item Lastni podprostori.
            \item Naj bo $V = \RR^3$ in $\Aa: \RR^3 \to \RR^3$ naj bo rotacija za $\frac{\pi}{2}$ okrog $z$-osi. Ali je $z$-os invarianten podprostor? Ali je ravnina $z=0$ invarianten podprostor? Ali je lastni podprostor?
        \end{itemize}
        \item \colorbox{yellow!30}{\emph{Opomba.}} Naj bo $\Aa \in L(V)$ in $U \leq V$. Ali zožitev $A|_U \in L(U, V)$? Kaj če je $U$ invarianten za $\Aa$?
        \item \colorbox{blue!30}{\textbf{Trditev.}} Naj bo $U \subseteq V$ invarianten podprostor za endomorfizem $\Aa \in L(V)$ in naj bo $B_U = \set{u_1, \ldots, u_m}$ baza prostora $U$. Dopolnimo jo do baze $B = \set{u_1, \ldots, u_m, v_1, \ldots v_k}$ prostora $V$. Kakšne oblike potem $A_{B, B}$?
        \begin{itemize}
            \item \colorbox{green!30}{\textbf{Dokaz.}} Napišemo matriko $\Aa_{B, B}$. Lahko si pomagamo z matriko zožitve $\Aa|_U$ glede na bazo $B_U$.
        \end{itemize}
        \item \colorbox{yellow!30}{\emph{Primer.}} Naj bo $V$ prostor polinomov stopnje največ $2$ in $\Aa \in L(V)$ odvajanje. $\Lin \set{1}$ je jedro, torej je invarianten podprostor. Baza jedra je $\set{1}$, ki jo dopolnimo do baze $B = \set{1, x, x^2}$ prostora $V$. Ali je matrika $\Aa_{B, B}$ res bločno zgornje trikotna? 
        \item \colorbox{yellow!30}{\emph{Primer.}} Naj bo $V = \RR^3$, $\Aa \in L(V)$ rotacija za $\frac{\pi}{2}$ okrog $z$-osi. netrivialna invariantna podprostora: $U_1 = z \text{-os}$, $U_2 = \text{ravnina } z = 0$. Naj bodo $B_1 = \set{(0, 0, 1)}, \ B_2 = \set{(1, 0, 0), (0, 1, 0)}$. Potem $B = B_1 \cup B_2$ baza za $\RR^3$. Določi $(\Aa|_{U_1})_{B_1, B_1}$, $(\Aa|_{U_2})_{B_2, B_2}$ in $\Aa_{B, B}$.
        \item \colorbox{blue!30}{\textbf{Trditev.}} Naj bo $\Aa \in L(V)$ in naj bo $V = V_1 \oplus V_2 \oplus \ldots \oplus V_k$, kjer so $V_1, \ldots, V_k$ invariantni za $\Aa$. Kakšne oblike potem $A_{B, B}$ glede na neko bazo $B$ prostora $V$? \textbf{Direktna vsota matrik}
        \begin{itemize}
            \item \colorbox{green!30}{\textbf{Dokaz.}} Napišemo matriko $\Aa_{B, B}$. Lahko si pomagamo z matrikami zožitev $\Aa|_{V_i}$ glede na bazo $B_i$.
        \end{itemize}
        \item[\textbf{!}] Do konca poglavja o korenskih podprostorov fiksiramo naslednje oznake:
        
        $V$ naj bo $n$-razsežen vektorski prostor nad $\CC$. $\Aa \in L(V)$ naj bo endomorfizem prostora $V$.

        $\lambda_1, \ldots, \lambda_2$ naj bodo vse različne lastne vrednosti endomorfizma $\Aa$.

        $\Delta_{\Aa}(\lambda) = (-1)^n(\lambda - \lambda_1)^{n_1}(\lambda - \lambda_2)^{n_2} \ldots (\lambda - \lambda_k)^{n_k}$.

        $m_{\Aa}(\lambda) = (\lambda - \lambda_1)^{m_1}(\lambda - \lambda_2)^{m_2} \ldots (\lambda - \lambda_k)^{m_k}$.
        \item \colorbox{purple!30}{\textbf{Definicija.}} Korenski podprostor za endomorfizem $\Aa$, ki pripada lastne vrednosti $\lambda_j$.
        \item \colorbox{yellow!30}{\emph{Opomba.}} Ali je korenski podprostor res vektorski podprostor $V$? Ali je korenski podprostor, ki pripada lastne vrednosti $\lambda_j$, vsebuje ustrezni lastni podprostor? Ali je vsak korenski podprostor netrivialen?
    \end{itemize}
    Radi bi uporabili prejšnjo trditev, kjer bomo za podprostore $V_j$ vzeli korenske podprostore.
    \begin{itemize}
        \item \colorbox{yellow!30}{\emph{Opomba.}} Ali polinomi v istem endomorfizmu komutirata?
        \item \colorbox{blue!30}{\textbf{Trditev.}} Ali je vsak korenski podprostor $W_j$ invarianten za endomorfizem $\Aa$?
        \begin{itemize}
            \item \colorbox{green!30}{\textbf{Dokaz.}} Pokažemo, da endomorfizmi $\Aa$ in $(\Aa - \lambda_j \id)^{m_j}$ komutirata (glavni argument: 
            
            $\Aa^k \circ \Aa^m = \Aa^m \circ \Aa^k$).
        \end{itemize}
        \item \colorbox{blue!30}{\textbf{Izrek.}} Naj bo $\FF$ poljubno polje in $d(\lambda)$ največji skupni delitelj polinomov $p_1(\lambda), \ldots, p_k(\lambda) \in \FF[\lambda]$. Kaj potem lahko zapišemo $d(\lambda)$? \textcolor{red}{[Dokaz pri Algebra 2]}
        \item \colorbox{blue!30}{\textbf{Izrek.}} Ali je prostor $V$ je direktna vsota korenskih podprostorov?
        \begin{itemize}
            \item \colorbox{green!30}{\textbf{Dokaz.}} Za vsak $i = 1, \ldots, k$, definiramo $p_j(\lambda) = \Pi_{i \neq j}(\lambda - \lambda_i)^{m_i} \in \CC[\lambda]$. Pokažemo, da obstajajo polinomi $q_1(\lambda), \ldots, q_k(\lambda)$, da velja: $p_1(\lambda)q_1(\lambda) + \ldots + p_k(\lambda)q_k(\lambda) = 1$ in vstavimo v ta enakost $\Aa$, dobimo enakost (*).
            
            Obstoj zapisa: Naj bo $x \in V$. Definiramo $x_j = p_j(\Aa)q_j(\Aa)x$. Izračunamo $x_1+x_2 + \ldots + x_k$ in pokažemo, da $x_j \in W_j$ za vsak $j = 1, \ldots, k$.

            Enoličnost: Recimo, da $x_1 + \ldots + x_k = y_1 + \ldots + y_k$. Definiramo $z_i = y_i - z_i$. Pokažemo, da je $z_i = 0$, zato izračunamo $p_j(\Aa)z_i, \ i \neq j$ in $p_j(\Aa)z_j$. Nato v enakost (*) vstavimo $z_i$.
        \end{itemize}
        \item \colorbox{orange!30}{\textbf{Posledica.}} Ali je v neki bazi prostora $V$ endomorfizmu $\Aa$ pripada bločno diagonalna matrika $[A_j]$, kjer je za vsak $j$, $A_j$ matrika, ki pripada zožitvi $\Aa|_{W_j}$ glede na neko bazo podprostra $W_j$? 
        \item \colorbox{blue!30}{\textbf{Izrek.}} Za vsak $j = 1, \ldots, k$ naj bo $\Aa_j = \Aa|_{W_j}$. Kaj potem $\Delta_{\Aa_j}(\lambda)$ in $m_{\Aa_j}(\lambda)$?
        \begin{itemize}
            \item \colorbox{green!30}{\textbf{Dokaz.}} Vemo, da $\Delta_{\Aa_j}(\lambda) = \Delta_{A_j}(\lambda)$. Izračunamo $\Delta_\Aa(\lambda)$ (*).
            
            Pokažemo, da je $A_j$ ničla polinoma $(\lambda - \lambda_j)^{m_j}$. Dobimo obliko karakteristinega in minimalnega polinoma.

            S pomočjo (*) pokažemo enakost za karakteristični polinom. 
            
            S pomočjo polinoma $p(\lambda) = (\lambda - \lambda_1)^{s_j} \ldots (\lambda - \lambda_k)^{s_k}$ pokažemo enakost za minimalni polinom.
        \end{itemize}
        \item \colorbox{orange!30}{\textbf{Posledica.}} Kakšne velikosti so matrike $A_j$?
        \newpage
        \item \colorbox{orange!30}{\textbf{Posledica.}} Kadar endomorfizem $A \in L(V)$ se da diagonalizirati (ničle polinoma $m_{\Aa}(\lambda)$)? 
        \begin{itemize}
            \item \colorbox{green!30}{\textbf{Dokaz.}} $(\Rightarrow)$ Naj bodo $A$ pripadajoča diagonalna matrika in $\lambda_1, \ldots, \lambda_k$ vse različne lastne vrednosti. Definiramo $p(\lambda) = (\lambda - \lambda_1) \ldots (\lambda - \lambda_k)$ in pokažemo, da $m_{\Aa}(\lambda) \, | \, p(\lambda)$ in $p(\lambda) \, | \, m_{\Aa}(\lambda)$.
            
            $(\Leftarrow)$ Naj bo $\Aa_j = \Aa|_{W_j}$. Pokažemo, da $A_j$ se da diagonalizirati za vsak $j = 1, \ldots, k$.
        \end{itemize}       
    \end{itemize}

    \item Jordanova kanonična forma
    
    Vsaka matrika podobna neki bločno diagonalni matriki, kjer ima vsak diagonalni blok le eno lastno vrednost. CIlj tega poglavja je najti čim enostavnejšo obliko teh diagonalnih blokov.

    \item[$\circ$] Enodmorfizmi z eno samo lastno vrednostjo
    
    $V$ naj bo $n$-razsežen vektorski prostor nad $\CC$ in $\Aa \in L(V)$ naj bo endomorfizem z eno samo lastno vrednostjo $\rho$. Potem $\Delta_{\Aa}(\lambda) = (-1)^n (\lambda - \rho)^n$. Minimalni polinom je oblike $m_{\Aa} (\lambda) = (\lambda - \rho)^r$ za nek $r$. 
    
    Velja: $(\Aa - \rho \id)^r = 0$ in $(\Aa - \rho \id)^{r-1} \neq 0$. Od $\Aa$ odštejemo $\rho \id$ in definiramo endomorfizem $\mathcal{N} = \Aa - \rho \id$. Velja: $\mathcal{N}^r = 0$ in $\mathcal{N}^{r-1} \neq 0$.

    \begin{itemize}
        \item \colorbox{purple!30}{\textbf{Definicija.}} Nilpotenten endomorfizem. Indeks nilpotentnosti.
        \item \colorbox{yellow!30}{\emph{Opomba.}} Recimo, da je $N$ matrika za $\mathcal{N}$ glede na neko bazo. Kaj potem $N - \rho I$?
        \item \colorbox{blue!30}{\textbf{Lema.}} Naj bo $\mathcal{N}$ nilpotenten endomorfizem in naj bo $V_j = \ker \mathcal{N}^j$ za vsak $j \in \NN_0$. 3 lastnosti $V_j$.
        \begin{itemize}
            \item \colorbox{green!30}{\textbf{Dokaz.}} Enostavno.
        \end{itemize}     
        \item \colorbox{blue!30}{\textbf{Lema.}} Naj bo $j \geq 2$ in naj bo $B = \set{v_1, \ldots, v_k} \subseteq V_j$ linearno neodvisna množica za katero velja 
        
        $(\Lin B) \cap V_{j-1} = \set{0}$. Kaj potem velja za množico $\mathcal{N}(B) = \set{\mathcal{N}v_1, \ldots, \mathcal{N}v_k}$?
        \begin{itemize}
            \item \colorbox{green!30}{\textbf{Dokaz.}} Enostavno z uporabo prejšnje leme.
        \end{itemize}   
        \item \colorbox{orange!30}{\textbf{Posledica.}} Ali je vsebovanost v predprejšnje leme stroga?
        \begin{itemize}
            \item \colorbox{green!30}{\textbf{Dokaz.}} Z obratno indukcijo pokažemo, da za vsak $j = 1, \ldots, r$ obstaja linearno neodvisna množica $B_j \subseteq V_j$, da je $(\Lin B_j) \cap V_{j-1} = \set{0}$.
        \end{itemize} 
    \end{itemize}

    \item[$\circ$] Konstrukcija Jordanove baze
    \begin{itemize}
        \item Naj bo $\mathcal{N} \in L(V)$ nilpotenten endomorfizem z indeksom nilpotentnosti $r$ in naj bo $V_j = \ker \mathcal{N}^j$ za vsak $j \geq 0$.  Konstrukcija Jordanove baze za nilpotentne endomorfizme. 
        \begin{itemize}
            \item \colorbox{green!30}{\textbf{Ideja.}} Za vsak $j = 1, \ldots, r$ najdemo $B_j \subseteq V_j$, kjer $B_j = \set{v_1^{j}, \ldots, v_{s_j}^j}$ taka linearno neodvisna podmnožica, da velja $U_j = \Lin B_j$ in $V_j = U_j \oplus V_{j-1}$. To lahko naredimo z obratno indukcijo z uporabo lem.
        \end{itemize} 
        \item \colorbox{purple!30}{\textbf{Definicija.}} Naj bo $\mathcal{N} \in L(V)$ nilpotenten endomorfizem Jordanova baza in Jordanova matrika enodmorfizma $\mathcal{N}$. Jordanova celica reda $t$. Oznaka.
        \item \colorbox{yellow!30}{\emph{Opomba.}} Kakšne oblike je $J(\mathcal{N})$? Ali je $J(\mathcal{N})$ neka direktna vsota? 
        \item \colorbox{yellow!30}{\emph{Opomba.}} Kaj lahko povemo o velikosti Jordanovih celic? Čemu je enaka velikost največje celice? Čemu je enako število celic?
        \item \colorbox{purple!30}{\textbf{Definicija.}} Naj bo $\Aa \in L(V)$ endomorfizem z eno samo lastno vrednostjo $\rho$. Jordanova baza in Jordanova matrika endomorfizma $\Aa$. Jordanova celica za lastno vrednost $\rho$.
        \item \colorbox{yellow!30}{\emph{Opomba.}} Čemu je enako število Jordanovih celic? Čemu je enaka velikost največje celice?
        \item \colorbox{purple!30}{\textbf{Definicija.}} Naj bo $\Aa \in L(V)$ enodmorfizem. Jordanova baza in Jordanova matrika endomorfizma $\Aa$.
        \item \colorbox{purple!30}{\textbf{Definicija.}} Jordanova baza in Jordanova matrika matrike $A \in \CC^{n \times n}$.
        \item \colorbox{yellow!30}{\emph{Opomba.}} Ali je $J(A)$ neka direktna vsota?
        \item \colorbox{yellow!30}{\emph{Opomba.}} Naj bo $\lambda_i$ neka lastna vrednost matrike $A$. Koliko krat pojavi $\lambda_i$ da diagonali? Čemu je enaka velikost največje celice $J_t(\lambda_i)$ za fiksen $i$? Čemu je enako število celic $J_t(\lambda_i)$ za fiksen $i$?
        \item \colorbox{yellow!30}{\emph{Opomba.}} Ali je Jordanova baza enolična? Ali je Jordanova matrika enolična? \textbf{Jordanova kanonična forma.}
        \item \colorbox{yellow!30}{\emph{Opomba.}} Čemu je enako število Jordanovih celic v $J(\Aa)$, ki pripada lastne vrednosti $\lambda_i$, velikosti vsaj $t \times t$? 
        \item \colorbox{yellow!30}{\emph{Primer.}} Določi Jordanovo formo in Jordanovo bazo matrike $\begin{bmatrix}
            0 & 0 & 0 & 1 & 1 & 1 \\
            1 & 0 & 0 & -1 & 0 & 0 \\
            -1 & 0 & 0 & -1 & -3 & -3 \\
            1 & 0 & 1 & 1 & 2 & 1 \\
            1 & 0 & 0 & 0 & 2 & 2 \\
            -1 & 0 & 0 & 0 & -1 & -1
        \end{bmatrix}$
    \end{itemize}

    \newpage
    \item Funkcije matrik 
    \begin{itemize}
        \item Naj bo $q$ polinom in $A \in \CC^{n \times n}$ matrika. Naj bo $J(A)$ Jordanova matrika matrike $A$ in $P$ prehodna matrika, da je $A = P J(A) P^{-1}$. Kaj potem $q(A)$? Ali je matrika $q(J(A))$ bločno diagonalna? Kaj potrebno vedeti, da bi znali izračunati $q(J(A))$ in $q(A)$?
        \item Kako zgleda polinom v Jordanovi celici $q(J_t(\rho))$?
        \begin{itemize}
            \item \colorbox{green!30}{\textbf{Ideja.}} Jordanovo celico $J_t(\rho) \in \CC^{t \times t}$ pišemo v obliki $J_t(\rho) = \rho I + N$, kjer je $N$ nilpotentna Jordanova celica velikosti $t \times t$. Polinom $q$ zapišemo v bazi $\set{1, \lambda - \rho, (\lambda - \rho)^2, \ldots, (\lambda - \rho)^k}$ in izračunamo koeficiente $a_0, a_1, \ldots, a_k$ (kot pri razvoju v $n$-ti Taylorjevi polinom).
        \end{itemize}         
        \item \colorbox{purple!30}{\textbf{Definicija.}} Analitična funkcija $f: D \to \CC$ v Jordanovi celici: $f(J_t(\rho))$.
        \item \colorbox{purple!30}{\textbf{Definicija.}} Analitična funkcija $f: D \to \CC$ v matriki: $f(A)$.
        \item \colorbox{yellow!30}{\emph{Opomba.}} Ali se ta definicija ujema z definicijo polinoma v matriki? 
        
        Ali za vsako (lepo) funkcijo $f$ obstaja polinom $q$ (odvisen of $f$ in od $A$), da je $f(A) = q(A)$?

        Jordanova baza ni enolična. Ali je definicija dobra?
        \item \colorbox{blue!30}{\textbf{Trditev.}} Naj bo $\mathcal{F}$ algebra vseh kompleksnih funkcij. ki jih mogoče v okolici spektra fiksne matrike $A$ razviti v Taylorjevo vrsto, in naj bo $\Phi: \ \mathcal{F} \to \CC^{n \times n}, \ \Phi(f) = f(A)$. Kaj potem velja za $\Phi$?
        \item \colorbox{yellow!30}{\emph{Primer.}} Naj bo $J = \begin{bmatrix}
            0 & 1 & 0 \\
            0 & 0 & 1 \\
            0 & 0 & 0
        \end{bmatrix}, \ f(z) = \sin z, \ g(z) = \cos z$. Izračunaj $f(J), \ g(J)$ in $f^2(J) + g^2(J)$.
        \item \colorbox{blue!30}{\textbf{Izrek.}} Izrek o preslikave spektra.
        \begin{itemize}
            \item \colorbox{green!30}{\textbf{Dokaz.}} Zaradi enakosti $f(A) = Pf(J(A))P^{-1}$ dovolj, da izrek dokažemo za vse Jordanove matrike. Kaj je $f(J(A))$?
        \end{itemize} 
    \end{itemize}    
\end{enumerate}

\newpage
\