\section{KONČNORAZSEŽNI VEKTORSKI PROSTORI}

\begin{enumerate}

    \item Baza in razsežnost
    \begin{itemize}
        \item \colorbox{purple!30}{\textbf{Definicija.}} Ogrodje.
        \item \colorbox{yellow!30}{\emph{Primer.}}       
        \begin{itemize}
            \item Vektorji $e_j$. Ogrodje prostora $\FF^n$.
            \item Ogrodje prostora $\RR[x]$.
            \item Ali je prejšnje ogrogje ogrodje za prostor vseh konvergentnih potenčnih vrst?
        \end{itemize}
        \item \colorbox{purple!30}{\textbf{Definicija.}} Končnorazsežen vektorski prostor.
        \item \colorbox{yellow!30}{\emph{Primer.}}       
        \begin{itemize}
            \item Ali je $\FF^n$ končnorazsežen?
            \item Ali je $\RR[x]$ končnorazsežen?
            \item Ali je prostor polinomov stopnje največ $n$ končnorazsežen?
            \item Ali je prostor vseh funkcij $\RR \to \RR$ končnorazsežen?
        \end{itemize}
        \item \colorbox{blue!30}{\textbf{Trditev.}} Kako dobimo čim manjše ogrodje?
        \begin{itemize}
            \item \colorbox{green!30}{\textbf{Dokaz.}} Pokažemo, da lahko vsak vektor iz $V$ zapišemo kot linearno kombinacijo elementov iz manšjega ogrodja.
        \end{itemize}
        \item \colorbox{purple!30}{\textbf{Definicija.}} Linearno neodvisni vektorji (za končno množico vektorjev). Linearno odvisni vektorji.
        \item \colorbox{yellow!30}{\emph{Opomba.}} Kdaj je neskončna množica vektorjev linearno neodvisna?
        \item \colorbox{yellow!30}{\emph{Primer.}} Ugotovi, ali so vektorji linearno neodvisni:
        \begin{itemize}
            \item $(1,1,-1), \ (1,1,2)$ in $(1, 1, -2)$.
            \item $e_1, e_2, \ldots, e_n \in \FF^n$.
            \item Funkciji $f(x) = e^x$ in $g(x) = e^{2x}$. $f(x) = \cos^2 x, \ g(x) = \sin^2 x, \ g(x) = 1$.
        \end{itemize}
        \item \colorbox{yellow!30}{\emph{Opomba.}} Ali definicija linearne neodvisnosti v $\RR^3$ se ujema z zgornjo definicijo?
        \item \colorbox{blue!30}{\textbf{Trditev.}} Karakterizacija linearne neodvisnosti vektorjev (linearna kombinacija).
        \begin{itemize}
            \item \colorbox{green!30}{\textbf{Dokaz.}} Definicija linearne neodvisnosti.
        \end{itemize}
        \item \colorbox{blue!30}{\textbf{Trditev}} o linearno odvisnih vektorjih (ali lahko vsaj en izmed teh vektorjev izrazimo z ostali?).
        \begin{itemize}
            \item \colorbox{green!30}{\textbf{Dokaz.}} Definicija linearne neodvisnosti.
        \end{itemize}
        \item \colorbox{purple!30}{\textbf{Definicija.}} Baza vektorskega prostora.
        \item \colorbox{yellow!30}{\emph{Primer.}}       
        \begin{itemize}
            \item Baza v $\RR^3$.
            \textbf{Standardna baza prostora $\FF^n$.}
            \item Baza v $\RR[x]$.
        \end{itemize}
        \item \colorbox{yellow!30}{\emph{Opomba.}} Ali baza vedno obstaja?
        \item \colorbox{blue!30}{\textbf{Izrek.}} Ali ima vsak netrivialen končnorazsežen vektorski prostor bazo?
        \begin{itemize}
            \item \colorbox{green!30}{\textbf{Dokaz.}} Izberimo kakšno končno ogrodje. Od leve proti desne zaporedoma odstranjujemo vektorje, ki so linearne kombinacije prejšnjih.
        \end{itemize}
        \item \colorbox{yellow!30}{\emph{Opomba.}} Ali ima vsak neskončnorazsežen vektorski prostor bazo?
        \item \colorbox{blue!30}{\textbf{Izrek.}} Karakterizacija baze.
        \begin{itemize}
            \item \colorbox{green!30}{\textbf{Dokaz.}} Definicija baze in linearne neodvisnosti.
        \end{itemize}
        \item \colorbox{blue!30}{\textbf{Trditev}} o moči vsake linearne neodvisne podmnožice vektorskega prostora. 
        \begin{itemize}
            \item \colorbox{green!30}{\textbf{Dokaz.}} Dokažemo z nadomeščanjem vektorjev v ogrodju: v ogrodje dobavimo vektor iz linearno neodvisne podmnožice. Spet dobimo ogrodje, ki je linearno odvisno. Torej obstaja vektor, ki je linearna kombinacija prejšnih. Ga odstranimo.
        \end{itemize}
        \item \colorbox{orange!30}{\textbf{Posledica.}} Ali imajo vse baze končnorazsežnega vektorskega prostora isto moč?
        \begin{itemize}
            \item \colorbox{green!30}{\textbf{Dokaz.}} Definicija baze + prejšna trditev.
        \end{itemize}
        \item \colorbox{purple!30}{\textbf{Definicija.}} Razsežnost ali dimenzija končnorazsežnega vektorskega prostora. Oznaka.
        \item \colorbox{yellow!30}{\emph{Opomba.}} Ali je definicija dobra?
        
        \item \colorbox{blue!30}{\textbf{Trditev.}} Ali lahko vsako linearno neodvisno podmnožico vektorskega prostora dopolnimo do baze?
        \begin{itemize}
            \item \colorbox{green!30}{\textbf{Dokaz.}} Izberimo kakšno ogrodje in kot prej z nadomeščanjem vektorjev iz linearno neodvisne podmnožice pridemo do baze.
        \end{itemize}
        \item \colorbox{yellow!30}{\emph{Primer.}} Dopolni vektorja $(1,1,1)$ in $(0,1,1)$ do baze prostora $\RR^3$.     
        
        \newpage
        \item \colorbox{orange!30}{\textbf{Posledica.}} (1) Naj bo $\dim V = n$ in $\vecs{v}{n}$ so linearno neodvisni. Ali je potem tvorijo bazo prostora $V$? (2) Naj bo $W \leq V$, $V$ končnorazsežen in $W \neq V$. Kaj lahko povemo o $\dim W$ in $\dim V$?        
        \begin{itemize}
            \item \colorbox{green!30}{\textbf{Dokaz.}} (1) Dopolnimo $\vecs{v}{n}$ do baze.
            
            (2) Dopolnimo bazo prostora $W$ do baze prosotra $V$.
        \end{itemize}
        \item \colorbox{blue!30}{\textbf{Trditev.}} Naj bosta $U$ in $W$ podprostora prostora $V$. Čemu je enaka $\dim (U+W)$?
        \begin{itemize}
            \item \colorbox{green!30}{\textbf{Dokaz.}} Dopolnimo bazo prostora $U \cap W, \ B_{U \cap W} = \set{v_1, \ldots, v_k}$ do baz prostora $U$,
            
            $B_U = \set{v_1, \ldots, v_k, u_1, \ldots, u_n}$ in $W, \ B_W = \set{v_1, \ldots, v_k, w_1, \ldots, w_m}$. Pokažemo, da 
            
            $B = \set{v_1, \ldots, v_k, u_1, \ldots, u_n, w_1, \ldots, w_m}$ baza prostora $U+W$ (zakaj to dovolj?).
        \end{itemize}
        \item \colorbox{orange!30}{\textbf{Posledica.}} Čemu je enaka $\dim (U \oplus  W)$?
        \begin{itemize}
            \item \colorbox{green!30}{\textbf{Dokaz.}} Čemu je enak presek dveh prostorov, če je vsota direktna?
        \end{itemize}
        \item \colorbox{blue!30}{\textbf{Trditev.}} Direktni komplement $U$ podprostora $W$ prostora $V$.
        \begin{itemize}
            \item \colorbox{green!30}{\textbf{Dokaz.}} Če je $W = \set{0}$ ali $W = V$, potem enostavno dobimo $U$.
            
            Če $W \neq V$ in $W \neq \set{0}$, potem ima bazo. Dopolnimo jo do baze prostora $V$. Definiramo $U$ kot linearno ogrinjačo dodatnih vektorjev in po definiciji pokažemo, da $V = W \oplus U$.
        \end{itemize}
        \item \colorbox{blue!30}{\textbf{Trditev.}} Naj bo $\Aa: U \to V$ lin. preslikava. Kaj lahko povemo, če je $\Aa$ injektivna/surjektivna/bijektivna?
        \begin{itemize}
            \item \colorbox{green!30}{\textbf{Dokaz.}} (1) Po definiciji linearne neodvisnosti z uporabo definiciji linearne preslikave ter lastnosti injektivne linearne preslikave.
            
            (2) Definicija in opis ogrodja + definicija surjektivne preslikave.
        \end{itemize}
        \item \colorbox{blue!30}{\textbf{Izrek.}} (1) Izomorfnost $V$, kjer $\dim V = n$ in $\FF^n$. 
        
        (2) Karakterizacija izomorfnosti končnorazsežnih vektorskih prostorov.        
        \begin{itemize}
            \item \colorbox{green!30}{\textbf{Dokaz.}} (1) Pokažemo, da je preslikava $\Phi: \ \FF^n \mapsto V, \ \Phi(\alpha_1, \ldots, \alpha_n) = \alpha_1 v_1 + \ldots + \alpha_n v_n$, kjer so $v_1, \ldots, v_n$ bazni vektorji, izomorfizem.
            
            (2) Uporabimo prejšnjo trditev + točko (1).
        \end{itemize}
        \item \colorbox{blue!30}{\textbf{Izrek.}} Naj bo $\mathcal{A}: \ U \mapsto V$ linearna preslikava. Dimenzijska enačba.
        \begin{itemize}
            \item \colorbox{green!30}{\textbf{Dokaz.}} Izberimo bazi prostorov $\ker \mathcal{A}$ in $\im \mathcal{A}$. Ker vektorji, ki pripadajo baze prostora $\im \mathcal{A}$, ležijo v $\im \mathcal{A}$, obstajajo praslike teh vektorjev, ki ležijo v $U$. Pokažemo, da množica, ki vsebuje bazne vektorje jedra in praslike baznih vektorjev slike, sestavljajo bazo prostora $U$.
            
            Linearna neodvisnost: definicija jedra, linearne neodvisnosti ter linearne preslikave.

            Ogrodje: definicija slike, jedra baze ter linearne preslikave.
        \end{itemize}
        \item \colorbox{orange!30}{\textbf{Posledica.}} Karakterizacija injektivnosti/surjektivnosti/bijektivnosti endomorfizma $\Aa \in L(U)$.
        \begin{itemize}
            \item \colorbox{green!30}{\textbf{Dokaz.}} Uporabimo prejšne trditve.
        \end{itemize}
        \item \colorbox{purple!30}{\textbf{Definicija.}} Rang preslikave $\Aa$. Oznaka.
    \end{itemize}

    \item Linearne preslikave in matrike
    \begin{itemize}
        \item \colorbox{purple!30}{\textbf{Definicija.}} Matrika reda $m \times n$ nad poljem $\FF$. Členi matrike. 
        \item Običajne oznake (matrika, členi matrike, krajša oblika za zapis matrike z splošnimi členi, vrstica matrike, stolpec matrike, množica vseh matrik reda $m \times n$ nad poljem $\FF$).
        \item \colorbox{purple!30}{\textbf{Definicija.}} Vsota matrik. Množenje matrik s skalarji.
        \item \colorbox{blue!30}{\textbf{Trditev.}} Ali je $\FF^{m \times n}$ vektorski prostor nad $\FF$ za prej definirane operacije? \textbf{Ničelna matrika.}
        \begin{itemize}
            \item \colorbox{green!30}{\textbf{Dokaz.}} Enostavno preverimo aksiome.
        \end{itemize}
    \end{itemize}

    \item[$\circ$] Kako so matrike povezane z linearnimi preslikavami?      
    \begin{itemize}          
        \item \colorbox{blue!30}{\textbf{Trditev.}} Naj bo $V$ in $U$ vektorska prostora nad istem poljem $\FF$, $\Aa: V \to U$ linearna preslikava. S čim je preslikava $\Aa$ že enolično določena?
        \begin{itemize}
            \item \colorbox{green!30}{\textbf{Dokaz.}} Najprej pokažemo, da sploh lahko definiramo preslikavo. Nato pa pokažemo še enoličnost.
        \end{itemize}
        \item \colorbox{purple!30}{\textbf{Definicija.}} Matrika, prirejena preslikave. Oznaka.
        \item \colorbox{yellow!30}{\emph{Opomba.}} Od česa je odvisna matrika iz prejšne definiciji? Poudaren zapis.
        \item \colorbox{yellow!30}{\emph{Primer.}} Naj bo $V$ prostor polinomov stopnje največ 2, $W$ prostor polinomov stopnje največ $1$ in $\Aa: \ V \mapsto W$ odvajanje. Določi matriko preslikave, glede na baze $\Bb_V = \set{1, x, x^2}$ in $\Bb_W = \set{1, x}$.
        \item \colorbox{yellow!30}{\emph{Opomba.}} Izbrani bazi sta "`urejeni"'. Kaj se zgodi, če zamenjamo vrstni red vektorjev v kakšni bazi?
        \item \colorbox{yellow!30}{\emph{Primer.}} Naj bosta $V$ vektorski prostor, $B_v$ baza prostora $V$. Določi matriko za $\id: V \to V$ glede na dano bazo. 
        \item \colorbox{purple!30}{\textbf{Definicija.}} \textbf{Identična matrika (identiteta).} Oznaka.
        \item \colorbox{purple!30}{\textbf{Definicija.}} Diagonalna matrika. Zapis.
        \item \colorbox{purple!30}{\textbf{Definicija.}} Zgornje trikotna matrika. Spodnje trikotna matrika.
        \item \colorbox{purple!30}{\textbf{Definicija.}} Bločno zgornje trikotna matrika. Bločno spodnje trikotna matrika. Bločno diagonalna matrika.
        
        \newpage
        \item \colorbox{blue!30}{\textbf{Trditev.}} Naj bosta $V$ in $W$ vektorska prostora z bazama $B_v$ in $B_w$. Kaj lahko povemo o preslikavi \begin{align*}
            \Phi: \Lin(V, W) &\to \FF^{m \times n} \\
            \Aa &\mapsto \Aa_{B_w, B_v}?
        \end{align*}
        \begin{itemize}
            \item \colorbox{green!30}{\textbf{Dokaz.}} Injektivnost pokažemo po definiciji z uporabo prejšnje trditve. Za surjektivnost definiramo $\Aa (x) = \sum_{i}^{n} \alpha_i \sum_{j}^{m} a_{ji}w_j$, kjer so $w_j$ bazni vektrorji prostora $W$, $\alpha_i$ pa so koeficienti pri razvoju vektorja $x$ po bazi prostora $V$.
        \end{itemize}
        \item \colorbox{blue!30}{\textbf{Trditev.}} Matrike znamo seštevati in množiti s skalarji. Linearne preslikave tudi znamo seštevati in množiti s skalarji. Kako so operacije povezane?
        \begin{itemize}
            \item \colorbox{green!30}{\textbf{Dokaz.}} Poračunamo.
        \end{itemize}
        \item \colorbox{orange!30}{\textbf{Posledica.}}  Kaj trditev pove o preslikave $\Phi: \Lin(V, W) \to \FF^{m \times n}$? Kaj to pomeni?
        \item \colorbox{purple!30}{\textbf{Definicija.}} Elementarne matrike (matrične enote).
        \item \colorbox{blue!30}{\textbf{Trditev.}} (1) Čemu je enaka dimenzija prostora $\FF^{m \times n}$? 
                
        (2) Naj bosta $\dim V = n$ in $\dim W = m$. Čemu je enaka dimenzija prostora $L(V, W)$?
        \begin{itemize}
            \item \colorbox{green!30}{\textbf{Dokaz.}} Po definiciji dokažemo, da je $\set{E_{ij}; \ 1 \leq i \leq m, \ 1 \leq j \leq n}$ baza prostora $F^{m \times n}$.
        \end{itemize}
        \item \colorbox{yellow!30}{\emph{Posebna primera.}} 
        \begin{itemize}
            \item $m=1$. Izomorfnost $L(V, \FF)$ in $V$.
            \item $n=1$. Izomorfnost $L(\FF, W)$ in $W$. Definicija danega izomorfizma brez izbire baz. Prostor stolpcev. Standardna baza prostora stolpcev. 
        \end{itemize}
    \end{itemize}

    \item[$\circ$] Množenje matrik
    
    Linearne preslikave komponiramo in kompozitum je spet linearna preslikava. Radi bi definirali množenje matrik tako, da bo usrezalo kompozitumu preslikav. Poseben preimer množenja matrik bo množenje matrike s stolpcem:
    \begin{itemize}
        \item \colorbox{blue!30}{\textbf{Izpeljava.}} Smiselna definicija produkta matrike in stolpca (vektroja). 
        \item \colorbox{purple!30}{\textbf{Definicija.}} Produkt matrike $A \in \FF^{m \times n}$ in vektorja $x \in \FF^n$.
        \item \colorbox{yellow!30}{\emph{Opomba.}} Kadar lahko pomnožimo matriko z vektorjem? Kakšne velikosti dobimo matriko, če pomnožimo matriko z vektorjem? Kaj dobimo na $i$-tem komponentu produkta?
        \item \colorbox{yellow!30}{\emph{Opomba.}} Ali je za vsako matriko $A \in \FF^{m \times n}$ preslikava \begin{align*}
            A: \FF^n &\to \FF^m \\
            x &\mapsto Ax 
        \end{align*}
        linearna? Kakšna matrika pripada tej preslikave glede na standardne bazi prostorov $\FF^n$ in $\FF^m$? Čemu v tem primeru ustreza množenje matrike $A$ z vektorjem $x$?
        \item Oznaka za stolpec koeficientov iz razvoja vektorja $x \in V$, po bazi $B_v$.
        \item \colorbox{blue!30}{\textbf{Trditev.}} Naj bo $\Aa$ linearna preslikava med vektorskoma prostoroma $V$ in $W$. Ali izračun preslikave v vektorju $x$ ustreza množenju matrike preslikave glede na izbrani bazi prostorov s vektorjem $x$?
        \begin{itemize}
            \item \colorbox{green!30}{\textbf{Dokaz.}} Izberimo bazi prostora in uvedemo oznako za matriko preslikave, nato poračunamo kot pri izpeljave definiciji za množenje.
        \end{itemize}
        \item \colorbox{yellow!30}{\emph{Primer.}} Naj bo $V$ prostor polinomov stopnje največ 2, $W$ prostor polinomov stopnje največ $1$ in $\Aa: \ V \mapsto W$ odvajanje. Izberimo običajne baze $\Bb_V = \set{1, x, x^2}$ in $\Bb_W = \set{1, x}$. Preveri prejšno trditev s polinimom $p(x) = ax^2 + bx + c$.
        \item \colorbox{orange!30}{\textbf{Posledica.}} Naj bo $V$ in $W$ vektorska prostora z bazama $B_v = \set{v_1, \ldots, v_n}$ in $B_w = \set{w_1, \ldots, w_m}$. Naj bosta $\phi_n: \FF^n \to V$ in $\phi_m: \FF^m \to W$ izomorfizma, definirama s prespisoma $\phi_n \begin{bmatrix}
            x_1 \\ \vdots \\ x_n
        \end{bmatrix} = x_1 v_1 + x_2 v_2 + \ldots + x_n v_n$ in $\phi_m \begin{bmatrix}
            y_1 \\ \vdots \\ y_y
        \end{bmatrix} = y_1 w_1 + y_2 w_2 + \ldots + y_m w_m$. Naj bo $\Aa: V \to W$ linearna preslikava in $A = \Aa_{B_w, B_v}$. $A$ gledamo kot linearno preslikavo $A: \FF^n \to \FF^m$, $x \mapsto Ax$. Dokaži, da diagram     
            \xymatrix{
                V \ar@{->}[r]^{\mathcal{A}} & W \\
                \mathbb{F}^n \ar@{->}[u]^{\phi_n} \ar@{->}[r]_{A} & \mathbb{F}^m \ar@{->}[u]_{\phi_m}
            }  komutira.  
        \begin{itemize}
            \item \colorbox{green!30}{\textbf{Dokaz.}} Pokažemo, da je $\phi_m \circ A = \Aa \circ \phi_n$. Kako dobimo vektorji $x \in V$ in $\Aa x \in W$ z izomorfizmoma $\phi_n$ in $\phi_m$?
        \end{itemize}        
    \end{itemize}    
    Naj bosta $A, B$ matriki, ki ju obravnavamo kot linearni preslikavi med prostoroma stolpcev. Radi bi definirali produkt matrik $A \cdot B$, ki bo usrezal kompozitumu preslikav $A \circ B$:
    \begin{itemize}
        \item Kdaj lahko definiramo produkt? Izpeljava (kam preslika matrika vektorji standardne baze?). Kaj je element $c_{ik}$ v produktu matrik?
        \item \colorbox{purple!30}{\textbf{Definicija.}} Produkt matrik $A \in \FF^{m \times n}$ in $B \in \FF^{n \times p}$.
        \item \colorbox{yellow!30}{\emph{Primer.}} Izračunaj produkti $AB, \ BA, \ CD, \ DC$, če obstajajo. Ali je produkt matrik komutativen?
        
        $A = \begin{bmatrix}
            1 & -1 \\
            2 & 3 \\
            0 & -4
        \end{bmatrix}, \ B = \begin{bmatrix}
            1 & -1 & 2 \\
            -3 & 1 & 0 \\
            -1 & 1 & 2
        \end{bmatrix}, \ C = \begin{bmatrix}
            2 & 1 \\
            -1 & 3
        \end{bmatrix}, \ D = \begin{bmatrix}
            0 & 1 \\
            0 & -2
        \end{bmatrix}$.
        \item \colorbox{blue!30}{\textbf{Trditev.}} Kako med sabo povezane kompozitum preslikav in množenje matrik?
        \begin{itemize}
            \item \colorbox{green!30}{\textbf{Dokaz.}} Izračunamo matriko preslikave $\Aa \circ B$ glede na izbrani bazi.
        \end{itemize}  
        \item \colorbox{orange!30}{\textbf{Posledica.}} Ali je množenje matrik asociativno?
        \begin{itemize}
            \item \colorbox{green!30}{\textbf{Dokaz.}} Matrike $A, B, C$ identificiramo s preslikavi med prostorama stolpcev.
        \end{itemize}  
        \item \colorbox{orange!30}{\textbf{Posledica.}} Enota za množenje matrik.
        \begin{itemize}
            \item \colorbox{green!30}{\textbf{Dokaz.}} Pokažemo, da $I_n \in \FF^{n \times n}$ enota za množenje.
        \end{itemize}  
        \item \colorbox{orange!30}{\textbf{Posledica.}} (1) Ali je $\FF^{n \times n}$ algebra?
        
        (2) Naj bo $V$ $n$-razsežen vektorski prostor. Izomorfizem algeber $L(V)$ in $\FF^{n \times n}$.
        \begin{itemize}
            \item \colorbox{green!30}{\textbf{Dokaz.}} (1) Nekaj že vemo, ostalo preverimo z računom.
            
            (2) Preverimo še multiplikativnost.
        \end{itemize}  
        \item \colorbox{purple!30}{\textbf{Definicija.}} (1) Kadar pravimo, da je endomorfizem $\Aa \in L(V)$ obrnljiv?
        
        (2) Kadar pravimo, da je kvadratna matrika $A \in \FF^{n \times n}$ obrnljiva? Inverz matrike $A$. Oznaka.
        \item \colorbox{yellow!30}{\emph{Primer.}} Kaj je $\begin{bmatrix}
            a & b \\ c & d
        \end{bmatrix}^{-1}$? 
        \item \colorbox{blue!30}{\textbf{Trditev.}} Ali je za obrnlivost endomorfizma $\Aa \in L(V)$ dovolj, da obstaja $\Bb \in L(V)$, da $\Aa \circ B = \id$?
        \begin{itemize}
            \item \colorbox{green!30}{\textbf{Dokaz.}} Dimenzijska enačba.
        \end{itemize}  
        \item \colorbox{orange!30}{\textbf{Posledica.}} Kam slika izomorfizem $\Phi: L(V) \to \FF^{n \times n}$ obrnljive endomorfizme? Kaj velja za vsak obrnljiv endomorfizem?
        \item \colorbox{orange!30}{\textbf{Posledica.}} Kadar je matrika $A \in \FF^{n \times n}$ obrnljiva?
        \item \colorbox{purple!30}{\textbf{Definicija.}} Transponirana matrika matrike $A$. Kaj je transponiranje?
        \item \colorbox{yellow!30}{\emph{Primer.}} Določi $\begin{bmatrix}
            1 & 2 \\ -1 & 3 \\ 0 & 5
        \end{bmatrix}^T$.
        \item \colorbox{blue!30}{\textbf{Trditev.}} 3 lastnosti transponiranja.
        \begin{itemize}
            \item \colorbox{green!30}{\textbf{Dokaz.}} Račun.
        \end{itemize}
    \end{itemize}

    \item Kvocientne vektorski prostori
    \item[$\circ$] Ponovitev relacij
    \begin{itemize}
        \item \colorbox{purple!30}{\textbf{Definicija.}} Relacija med množicama $A$ in $B$. Elementa v relaciji (oznaka). Relacija na množici $A$.
        \item \colorbox{purple!30}{\textbf{Definicija.}} Refleksivna, simetrična, antisimetrična in tranzitivna relacija.
        \item \colorbox{purple!30}{\textbf{Definicija.}} Ekvivalenčna relacija.
        \item \colorbox{purple!30}{\textbf{Definicija.}} Relacija delne urejenosti.
        \item \colorbox{yellow!30}{\emph{Primer.}} Ali je relacija vsebovanosti na množici vektorskih podprostorov danega vektorskega prostora relacija delne urejenosti?        
        \item \colorbox{purple!30}{\textbf{Definicija.}} Primerljiva in neprimerljiva elementa.
        \item \colorbox{yellow!30}{\emph{Primer.}} Naj bo $A$ množica vektorskih podprostorov prostora $\RR^3$ in $R$ relacija vsebovanosti. Ali je primerljiva $\set{0}$ in $x$-os ter $x$-os in $y$-os?
        \item \colorbox{purple!30}{\textbf{Definicija.}} Linearna urejenost.
        \item \colorbox{purple!30}{\textbf{Definicija.}} Maksimalni in minimalni element. Največji in najmanjši element za relacijo delne urejenosti. Obstoj in zveza med njimi.
        \item \colorbox{yellow!30}{\emph{Primer.}} Naj bo $V$ vektorski prostor in $M \subseteq V$ poljubna množica. Naj bo $A$ množica vseh vektorskih podprostorov prostora $V$, ki vsebujejo $M$. Množico $A$ delno uredimo z inkluzijo. Določi najmanjši element, če obstaja.
    \end{itemize}

    \newpage
    \item[$\circ$] Ekvivalenčna relacija in kvocietntne množice
    \begin{itemize}
        \item \colorbox{purple!30}{\textbf{Definicija.}} Ekvivalenčni razred elementa $a \in A$. Oznaka. Predstavnik ekvivalenčnega razreda.
        \item \colorbox{blue!30}{\textbf{Izrek.}} Na kaj ekvivalenčna relacija razdeli množico $A$? Karakterizacija ekvivalentnosti elementov.
        \begin{itemize}
            \item \colorbox{green!30}{\textbf{Dokaz.}} LMN.
        \end{itemize}
        \item \colorbox{blue!30}{\textbf{Izrek.}} Obrat prejšnega izreka.
        \begin{itemize}
            \item \colorbox{green!30}{\textbf{Dokaz.}} LMN.
        \end{itemize}
        \item \colorbox{purple!30}{\textbf{Definicija.}} Kvocientna ali faktorska množica množice $A$ po relacije $\sim$. Oznaka.
        \item \colorbox{purple!30}{\textbf{Definicija.}} Kvocientna preslikava.
        \item \colorbox{yellow!30}{\emph{Opomba.}} Ali je kvocientna preslikava surjektivna?
        \item \colorbox{yellow!30}{\emph{Primer.}} Določi ekvivalenčne razrede ter pri 1. in 3. primeru kvocientno množico..
        \begin{itemize}
            \item Na $\ZZ \times \NN$ definiramo relacijo $\sim: (m,n) \sim (p, q) \Leftrightarrow mq = np$.
            \item Na množice vseh usmerjenih daljic v $\RR^3$ definiramo, da sta usmerjeni daljici v relaciji $\sim$, kadar eno od njiju dobimo z vzporednim premikom druge. 
            \item Naj bo $n \in \NN$ in $\equiv$ relacija na $\ZZ$, definirana s predpisom $a \equiv b \Leftrightarrow n | a -b$. 
            \item Na $\RR^2$ definiramo relacijo $(x,y) \sim (z,w) \Leftrightarrow y = w$.
        \end{itemize}
        \item \colorbox{blue!30}{\textbf{Izrek.}} Naj bo $f: A \to B$ preslikava. Na $A$ definiramo relacijo $\sim$ s predpisom $x \sim y \Leftrightarrow f(x) = f(y)$.
        \begin{enumerate}
            \item[(1)] Ali je $\sim$ ekvivalenčna relacija?
            \item[(2)] Naj bo $q: A \to A/_\sim$ kvocientna preslikava. Kaj lahko povemo o diagramu \xymatrix{
                A \ar@{->}[rr]^{f} \ar@{->}[rd]_{q} &  & B \\
                 & A/_\sim \ar@{->}[ru]_{p} & 
                }?
            \item[(3)] Ali je $p$ injektivna? Čemu je enaka $Z_p$?            
        \end{enumerate}
        \begin{itemize}
            \item \colorbox{green!30}{\textbf{Dokaz.}} LMN. Glavna stvar izreka je dokaz, da je s predpisom $p([a]) = f(a)$ preslikava $p$ dobro definirana.
        \end{itemize}
    \end{itemize}

    \item[$\circ$] Usklajenost operacije z ekvivalenčno relacijo
    \begin{itemize}
        \item \colorbox{purple!30}{\textbf{Definicija.}} Usklajenost operacije z ekvivalenčno relacijo.
        \item Recimo, da je operacija $\circ$ usklajena z ekvivalenčno relacijo $\sim$. Definicija operacije na množice $A/_\sim$. Ali je definicija dobra?
        \item \colorbox{yellow!30}{\emph{Primer.}}  
        \begin{itemize}
            \item Ali je operaciji seštevanje in množenje racionalnih števil usklajeni z relacijo $\sim$ na množice $\ZZ \times \NN$?
            \item Ali je operacija seštevanje vektorjev usklajena z relacijo $\sim$ na množici vseh vektorjev?
        \end{itemize}
    \end{itemize}
    
   
    \item[$\circ$] Kvocientne grupe Abelovih grup
    
    Naj bo $(G,+)$ Abelova grupa in $H$ njena podgrupa. Na $G$ definiramo relacijo $\sim$ s predpisom $a \sim b \Leftrightarrow a - b \in H$.
    \begin{itemize}
        \item \colorbox{blue!30}{\textbf{Trditev.}} Ali je $\sim$ ekvivalenčna relacija na $G$?
        \begin{itemize}
            \item \colorbox{green!30}{\textbf{Dokaz.}} Z uporabo definicije podrupe preverimo lastnosti ekvivalenčne relacije.
        \end{itemize}
        \item \colorbox{yellow!30}{\emph{Opomba.}} Predpis za ekvivalenčno relacijo na $G$, če je $(G, \cdot)$ poljubna grupa in $H$ njena podgrupa.  
        \item \colorbox{blue!30}{\textbf{Trditev.}} Ali je operacija $+$ na $G$ usklajena z relacijo $\sim$?
        \begin{itemize}
            \item \colorbox{green!30}{\textbf{Dokaz.}} Treba pokazati, da je $a \sim b \text{ in } c \sim d \Rightarrow a + c \sim b + d$.
        \end{itemize}
        \item \colorbox{yellow!30}{\emph{Opomba.}} Naj bo $(G, \cdot)$ poljubna grupa in $H$ njena podgrupa. Ali je množenje iz prejšnje opombe nujno usklajeno z relacijo $\sim$?
        \item \colorbox{purple!30}{\textbf{Definicija.}} Definicija operacije $+$ na množice $G/_\sim$.
        \item \colorbox{blue!30}{\textbf{Trditev.}} Ali je $(G/_\sim, +)$ Abelova? Ali je kvocientna preslikava homomorfizem grup?
        \begin{itemize}
            \item \colorbox{green!30}{\textbf{Dokaz.}} Zakaj operacija dobro definirana? Ostale aksiome preverimo z računom.
        \end{itemize}
        \item \colorbox{purple!30}{\textbf{Definicija.}} Kvocientna ali faktorska grupa grupe $G$ po podgrupi $H$. Oznaki za kvocientno grupo in ekvivalenčni razred.
        \item \colorbox{yellow!30}{\emph{Opomba.}} Zakaj $[a]$ označimo z $a+H$?
        \item \colorbox{yellow!30}{\emph{Primer.}}
        \begin{itemize}
            \item Izomorfnost $G/_{\set{0}}$ in $G$.
            \item Kaj lahko povemo o grupi $G/_G$?
            \item Naj bo $n \in \NN, \ n \neq 1$. Označimo z $n\ZZ = \set{na; \ a \in Z}$. Kaj je $\ZZ/_{n\ZZ}$?
        \end{itemize}
        \item \colorbox{yellow!30}{\emph{Opomba.}} Naj bosta $G, H$ grupi, $G$ Abelova, in naj bo $f: G \to H$ homomorfizem grup. Kaj je relacija $\sim$, definirana s prespisom $x \sim y \Leftrightarrow f(x) = f(y)$? Kaj to pomeni o $G/_\sim$?
        
        \newpage
        \item \colorbox{blue!30}{\textbf{Izrek.}} Naj bosta $G, H$ grupi, $G$ Abelova, in naj bo $f: G \to H$ homomorfizem grup. Kaj lahko povemo o diagramu \xymatrix{
            G \ar@{->}[rr]^{f} \ar@{->}[rd]_{q} &  & H \\
             & G/_{\ker f} \ar@{->}[ru]_{p} & 
            }?
        \begin{itemize}
            \item \colorbox{green!30}{\textbf{Dokaz.}} Vse že vemo iz LMN, dokazati le treba, da je $p$ homomorfizem grup.
        \end{itemize}
    \end{itemize}

    \item[$\circ$] Kvocientni vektorski prostori
    
    Naj bo $V$ vektorski prostor nad $\FF$ in $W \leq V$. $(V,+)$ je Abelova grupa in $(W, +)$ njena podgrupa, zato lahko definiramo ekvivalenčno relacijo $\sim$ s prespisom $x \sim y \Leftrightarrow x-y \in W$, in kvocientno Abelovo grupo $V/_W$. Na $V/_W$ bi radi definirali množenje s skalarji tako, da bo $V/_W$ vektorski prostor.
    \begin{itemize}
        \item \colorbox{blue!30}{\textbf{Trditev.}} Ali je množenje s skalarji na $V$ usklajeno z ekvivalenčno relacijo?
        \begin{itemize}
            \item \colorbox{green!30}{\textbf{Dokaz.}} Treba pokazati, da če $x \sim y \text{ in } \alpha \in \FF \Rightarrow \alpha x \sim \alpha y$.
        \end{itemize}
        \item \colorbox{orange!30}{\textbf{Posledica.}} Ali je na $V/_W$ s predpisom $\alpha [x] = [\alpha x]$ dobro definirano množenje s skalarji?
        \begin{itemize}
            \item \colorbox{green!30}{\textbf{Dokaz.}} Treba pokazati, da če $[x] = [y] \Rightarrow [\alpha x] = [\alpha y]$.
        \end{itemize}
        \item \colorbox{blue!30}{\textbf{Trditev.}} Naj bo $V$ vektorski prostor nad $\FF$ in $W \leq V$. Na $V/_W$ definiramo operaciji s predpisom 
        
        $[x] + [y] = [x+y]$ in $\alpha [x] = [\alpha x]$.
        
        (1) Ali je $V/_W$ vektorski prostor nad $\FF$?

        (2) Ali je kvocientna preslikava linearna?
        \begin{itemize}
            \item \colorbox{green!30}{\textbf{Dokaz.}} (1) Vemo že, da sta operacije dobro definirane in da je $(V/_W, +)$ Abelova grupa. Preverimo še ostale aksiome za vektorski prostor.
            
            (2) Vemo že, da je kvocientna preslikava homomorfizem grup $(V,+)$ in $(V/_W, +)$. Preverimo še homogenost.
        \end{itemize}
        \item \colorbox{purple!30}{\textbf{Definicija.}} Kvocientni ali faktorski vektorski prostor prostora $V$ po podprostoru $W$.
        \item \colorbox{yellow!30}{\emph{Opomba.}} Kaj so ekvivalenčni razredi?  Oznaka. \textbf{Afin podprostor}. 
        \item \colorbox{yellow!30}{\emph{Primer.}} Afini podprostori v $\RR^3$.
        \item \colorbox{blue!30}{\textbf{Izrek.}} Naj bosta $V, \ U$ vektorska prostora nad $\FF$, in naj bo $\Aa: V \to U$ linearna preslikava. Kaj lahko povemo o diagramu \xymatrix{
            V \ar@{->}[rr]^{\Aa} \ar@{->}[rd]_{q} &  & U \\
             & V/_{\ker f} \ar@{->}[ru]_{\overline{\Aa}} & 
            }?
        \begin{itemize}
            \item \colorbox{green!30}{\textbf{Dokaz.}} Vse že vemo iz LMN in izreka v primeru grup, dokazati le treba, da je $\overline{\Aa}$ homogena.
        \end{itemize}
        \item \colorbox{orange!30}{\textbf{Posledica.}} Ali je vektorska prostora $V/_{\ker \Aa}$ in $\im \Aa$ izomorfna?
        \begin{itemize}
            \item \colorbox{green!30}{\textbf{Dokaz.}} $U$ zamenjamo z $\im \Aa$ in uporabimo izrek.
        \end{itemize}
        \colorbox{yellow!30}{\emph{Primer.}} $\Aa: \RR^3 \to \RR^3, \ \Aa(x,y,z) = (x, y, 0)$. Kaj je $\im \Aa$ in $\ker \Aa$? Kaj je ekvivalenčni razredi v $\RR^3/_{\ker \Aa}$? Čemu je izomorfen $\RR^3/_{\ker \Aa}$? Kam slika izomorfizem navpične premice?
        \item \colorbox{orange!30}{\textbf{Posledica.}} Naj bo $V$ končnorazsežen vektorski prostor nad $\FF$. Naj bo $\Aa: \ V \to U$ linearna preslikava. Čemu je enaka $\dim (V/_{\ker \Aa})$?
        \begin{itemize}
            \item \colorbox{green!30}{\textbf{Dokaz.}} Uporabimo prejšnjo posledico in dimenzijsko enačbo.
        \end{itemize}
        \item \colorbox{orange!30}{\textbf{Posledica.}} Naj bo $V = V_1 \oplus V_2$. Kaj lahko povemo o $V/_{V_1}$ in $V/_{V_2}$? \textbf{Projekcija na $V_2$ vzdolž $V_1$}.
        \begin{itemize}
            \item \colorbox{green!30}{\textbf{Dokaz.}} Definiramo preslikavo $P: \ V \to V_2, \ P(v) = P(x_1 + x_2) = x_2$. Pokažemo, da je ta preslikava linearna ter velja $\im P = V_2, \ \ker P = V_1$. Uporabimo izrek in predpjejšnjo posledico.
        \end{itemize}
        \item \colorbox{orange!30}{\textbf{Posledica.}} Naj bo $V$ končnorazsežen vektorski prostor nad $\FF$ in $W \leq V$. Ali je $V/_W$ končnorazsežen? Čemu je enaka $\dim(V/_W)$?
        \begin{itemize}
            \item \colorbox{green!30}{\textbf{Dokaz.}} Uporabimo izrek o obstoju direktnega komplementa, prejšnjo posledico ter dimenzijsko enačbo.
        \end{itemize}
    \end{itemize}
    
    
    \newpage
    \item Prehod na novi bazi
    
    Naj bo $\Aa: V \to W$ linearna preslikava. Matrika $A$, ki pripada preslikave $\Aa$ ni odvisna samo od preslikave~$\Aa$, ampak tudi od baz $B_v$ in $B_w$. Recimo, da si izberimo še drugi dve bazi $B_{v'}$ in $B_{w'}$ in definiramo matriko $A' = \Aa_{B_{w'}, B_{v'}}$. Zanima nas, v kakšni zvezi sta $A$ in $A'$.
    \begin{itemize}
        \item \colorbox{purple!30}{\textbf{Definicija.}} Prehodna matrika med bazama $B_v$ in $B_{v'}$.
        \item \colorbox{yellow!30}{\emph{Opomba.}} Kako dobimo prehodno matriko $P$?
        \item \colorbox{blue!30}{\textbf{Trditev.}} Kateri preslikavi pripada $P$? Ali je $P$ obrnljiva? 
        \begin{itemize}
            \item \colorbox{green!30}{\textbf{Dokaz.}} Uporabimo zvezo med množenjem matrik in kompozitumom preslikav.
        \end{itemize}
        \item \colorbox{yellow!30}{\emph{Opomba.}} Naj bo $V = \FF^n$ in $B_v$ standardna baza. Kaj so stolpci prehodne matrike med bazama $B_v$ in $B_{v'}$?
        \item \colorbox{blue!30}{\textbf{Trditev.}} Naj bo $\Aa: V \to W$ linearna preslikava, $A = \Aa_{B_{w}, B_{v}}$ in $A' = \Aa_{B_{w'}, B_{v'}}$. V kakšni zvezi s prehodnimi matrikami sta $A$ in $A'$?
        \begin{itemize}
            \item \colorbox{green!30}{\textbf{Dokaz.}} Izračunamo $Q^{-1}AP$ z uporabo zveze med kompozitumom preslikav in množenjem matrik.
        \end{itemize}
        \item \colorbox{purple!30}{\textbf{Definicija.}} Ekvivalentni matriki. Oznaka.
        \item \colorbox{yellow!30}{\emph{Opomba.}} Ali sta matriki, ki pripada isti linearni preslikavi ekvivalentni?
        \item \colorbox{blue!30}{\textbf{Trditev.}} Ali je ekvivalentnost matrik ekvivalenčna relacija?
        \begin{itemize}
            \item \colorbox{green!30}{\textbf{Dokaz.}} Enostavno preverimo lastnosti.
        \end{itemize}
        \item \colorbox{blue!30}{\textbf{Trditev.}} Karakterizacija ekvivalentnosti matrik (obstoj linearne preslikave).
        \begin{itemize}
            \item \colorbox{green!30}{\textbf{Dokaz.}} $(\Leftarrow)$ Že vemo.
            
            $(\Rightarrow)$ Definiramo preslikavo $\Aa: \FF^n \to \FF^m$ s predpisom $\Aa x = Ax$. Vemo že, da tej preslikave glede na standardne baze prostorov $S_n$ in $S_m$ pripada matrika $A$. Ker $A \sim A', \ A' = Q^{-1} A P$. Definiramo $B_m = \mathcal{Q}(S_m)$ in $B_n = \mathcal{P}(S_n)$, kjer $\mathcal{Q}: \FF^m \to \FF^m, \ \mathcal{Q}(x) = Qx$ in $\mathcal{P}$ definirana podobno. Pokažemo, da je $B_m$ in $B_n$ bazi in izračunamo $\Aa_{B_m, B_n} = (\id \circ \Aa \circ \id)_{B_m, B_n}$.
        \end{itemize}
        \item \colorbox{purple!30}{\textbf{Definicija.}} Rang, jedro in slika matrike $A \in \FF^{m \times n}$. 
        \item \colorbox{blue!30}{\textbf{Trditev.}} Naj bo $\Aa: V \to W$ linearna preslikava in $A$ matrika te preslikave glede na poljubni bazi. Kakšna je zveza med $\rang \Aa$ in $\rang A$?
        \begin{itemize}
            \item \colorbox{green!30}{\textbf{Dokaz.}} Vemo, da diagram     
            \xymatrix{
                V \ar@{->}[r]^{\mathcal{A}} & W \\
                \mathbb{F}^n \ar@{->}[u]^{\phi_n} \ar@{->}[r]_{A} & \mathbb{F}^m \ar@{->}[u]_{\phi_m}
            }  komutira. S pomočjo njo pokažemo, da je $\im A = \phi^{-1}_m(\im \Aa)$, nato pokažemo, da $\dim \im A = \dim \im \Aa$.
        \end{itemize}
        
        \item \colorbox{orange!30}{\textbf{Posledica.}} V kakšni zvezi so rangi ekvivalentnih matrik?
        \begin{itemize}
            \item \colorbox{green!30}{\textbf{Dokaz.}} Karakterizacija ekvivalentnosti matrik + prejšnja trditev.
        \end{itemize}
        \item \colorbox{orange!30}{\textbf{Posledica.}} Kaj se zgodi z rangom matrike $A$, če jo pomnožimo (z leve ali z desne) z obrnljivo matriko?
        \begin{itemize}
            \item \colorbox{green!30}{\textbf{Dokaz.}} Pokažemo, da je $A \sim AP$ in $A \sim PA$.
        \end{itemize}
        \item \colorbox{blue!30}{\textbf{Trditev.}} Čemu je enaka slika matrike?
        \begin{itemize}
            \item \colorbox{green!30}{\textbf{Dokaz.}} Naj bo $A \in \FF^{m \times n}$ poljubna. Izračunamo $\im A$ s pomočjo slik baznih vektorjev prostora $\FF^n$.
        \end{itemize}
        \item \colorbox{blue!30}{\textbf{Izrek.}} Kakšne matrike je ekvivalentna vsaka matrika $A \in \FF^{m \times n}$? Neenakost za $\rang A$.
        \begin{itemize}
            \item \colorbox{green!30}{\textbf{Dokaz.}} Izberimo bazo za $\im A$, kjer $A: \FF^n \to \FF^m, \ A(x) = Ax$. Neenakost pokažemo s pomočjo dimenzijske enačbe ter definiciji slike.
            
            Dopolnimo bazo za $\im A$ do baze $B_m$ za $\FF^m$. V dokazu dimenzijske enačbe smo pokazali, da obstaja baza $B_n$ za $\FF^n$, da velja: $Av_i = w_i, \ i = 1, \ldots, r$ in $Av_i = 0, \ i > r$. Kakšna matrika je $A_{B_m, B_n}$?

            Predpostavimo, da $A \sim \begin{bmatrix}
                I_s & 0 \\ 0 & 0
            \end{bmatrix}$ in izračunamo $\rang \begin{bmatrix}
                I_s & 0 \\ 0 & 0
            \end{bmatrix}$.
        \end{itemize}        
        \item \colorbox{orange!30}{\textbf{Posledica.}} Karakterizacija ekvivalentnosti matrik iste velikosti (rang).
        \begin{itemize}
            \item \colorbox{green!30}{\textbf{Dokaz.}} $(\Rightarrow)$ Že vemo.
            
            $(\Leftarrow)$ $\sim$ je ekvivalenčna + trditev.
        \end{itemize}
        \item \colorbox{blue!30}{\textbf{Lema.}} Recimo, da je $P \in \Fnn$ obrnljiva. Kaj lahko povemo o $P^{T}$? Čemu je enako $(P^T)^{-1}$.
        \begin{itemize}
            \item \colorbox{green!30}{\textbf{Dokaz.}} Uporabimo transponiranje na enakosti $PP^{-1} = P^{-1}P = I$
        \end{itemize}
        \item \colorbox{orange!30}{\textbf{Posledica.}} V kakšni zvezi sta $\rang A$ in $\rang A^{T}$?
        \begin{itemize}
            \item \colorbox{green!30}{\textbf{Dokaz.}} Uporabimo transponiranje na enakosti $A = Q^{-1} \begin{bmatrix}
                I_r & 0 \\ 0 & 0
            \end{bmatrix} P$.
        \end{itemize}

        \newpage
        \item \colorbox{blue!30}{\textbf{Trditev.}} Naj bo $A \in \Fmn$. V kakšni zvezi so števila: največje možno število linearno neodvisnih stolpcev,  največje možno število linearno neodvisnih vrstic, $\rang A$?
        \begin{itemize}
            \item \colorbox{green!30}{\textbf{Dokaz.}} Naj bo matrika $A$ ima $r$ linearno neodvisnih stolpcev. S protislovjem pokažemo, da oni tvorijo bazo prostora $\im A$. Druga enakost sledi iz prejšnje posledice.
        \end{itemize}
    \end{itemize}
    
    \item[$\circ$] Izračun ranga.
    \begin{itemize}
        \item Permutacijska matrika $P_{p, q}$. Matrika $P_{\alpha, p}$. Matrika $P = I + \alpha E_{p, q}, \ p \neq q$.
        \item \colorbox{blue!30}{\textbf{Trditev.}} Ali so matriki $P_{p, q}$, $P_{\alpha, p}$ in $P = I + \alpha E_{p, q}, \ p \neq q$ obrnljivi?
        \begin{itemize}
            \item \colorbox{green!30}{\textbf{Dokaz.}} Inverz od $P_{p, q}$ je $P_{p, q}$. Inverz od $P_{\alpha, p}$ je $P_{\alpha^{-1}, p}$. 
            
            Inverz od $P = I + \alpha E_{p, q}, \ p \neq q$ je $P = I - \alpha E_{p, q}, \ p \neq q$.
        \end{itemize}
        \item \colorbox{blue!30}{\textbf{Trditev.}} Naj bo $A \in \Fmn$. Kaj se zgodi, če pomnožimo (z desne ali z leve) $A$ z $P_{p, q}$, $P_{\alpha, p}$ ali $P = I + \alpha E_{p, q}, \ p \neq q$?
        \begin{itemize}
            \item \colorbox{green!30}{\textbf{Dokaz.}} Pomnožimo z desne in izračunamo rezultat. Pri množenji z leve upoštevamo, da $AB = (AB)^{TT}$.
        \end{itemize}
        \item \colorbox{blue!30}{\textbf{Izrek.}} Ali lahko iz vsake matrike $A \in \Fmn$ pridelamo neko lepo matriko, da bi enostavno izračunali $\rang A$?
        \begin{itemize}
            \item \colorbox{green!30}{\textbf{Dokaz.}} Z indukcijo na $k$ dokažemo, da lahko dobimo neko matriko oblike $A_k = \begin{bmatrix}
                I_k & 0 \\ 0 & A_k'
            \end{bmatrix}$ za vsak $k \in \set{0, 1, \ldots, r}$.
        \end{itemize}
        \item \colorbox{yellow!30}{\emph{Primer.}} Izračunaj rang matrike $A = \begin{bmatrix}
            1 & 0 & 2 \\ -1 & 1 & -3 \\
            0 & 1 & -1 \\ 1 & 2 & 0
        \end{bmatrix}$.

        \item \colorbox{yellow!30}{\emph{Opomba.}} Kaj je običajno dovolj za računanje ranga?
    \end{itemize}

    \item[$\circ$] Reševanje sistema linearnih enačb
    
    Radi bi rešili sistem linaernih enačb
    \begin{align*}
        &a_{11}x_1 + a_{12}x_2 + \ldots + a_{1n}x_n = b_1, \\
        &a_{21}x_1 + a_{22}x_2 + \ldots + a_{2n}x_n = b_2, \\
        &\vdots \\
        &a_{m1}x_1 + a_{m2}x_2 + \ldots + a_{mn}x_n = b_m,
    \end{align*}
    kjer so $a_{ij} \in \FF$ in $b_i \in \FF$ dani skalarji, $\x{n}$ pa neznanki
    \begin{itemize}
        \item \colorbox{purple!30}{\textbf{Definicija.}} Matrika sistema. Vektor neznank. Vektor desni strani.
        \item \colorbox{yellow!30}{\emph{Opomba.}} Čemu je ekvivalenten sistem? Kakšnim enačbam bomo rekli sistem enačb?
        \item \colorbox{purple!30}{\textbf{Definicija.}} Homogen sistem. Nehomogen sistem. Protisloven sistem. Neprotisloven sistem.
        \item \colorbox{purple!30}{\textbf{Definicija.}} Razširjena matrika.
        \item \colorbox{blue!30}{\textbf{Kronecker-Capellijev izrek.}} Karakterizacija neprotislovnosti sistema. 
        \begin{itemize}
            \item \colorbox{green!30}{\textbf{Dokaz.}} Vemo: Rang matrike je maksimalno število linearno neodvisnih stolpcev ter slika matrike je linearna ogrinjača stolpcev. Po definiciji neprotislovnega sistema z ekvivalentnimi perehodami pokažemo, da trditev velja.
        \end{itemize}
        \item Ali je homogen sistem vedno rešljiv? Kaj so rešitve homogenega sistema? Kdaj je homogen sistem ima edino rešitev? Kaj je v splošnem rešitev homogenega sistema? Čemu je enaka dimenzija prostora rešitev?
        \item \colorbox{blue!30}{\textbf{Trditev.}} Recimo, da je nehomogen sistem neprotisloven. Čemu je potem enaka množica rešitev sistema? \textbf{Partikularna rešitev.}
        \begin{itemize}
            \item \colorbox{green!30}{\textbf{Dokaz.}} Pokažemo, da sta množici rešitev sistema $Ax = b$ in vektorjev oblike $v+u$, kjer je $v$ partikularna rešitev in $u \in \ker A$ enaki.
        \end{itemize}
    \end{itemize}
    \begin{itemize}
        \item \colorbox{blue!30}{\textbf{Gaussova eliminacija}.} Vektorska oblika rešitev. Kako rešujemo sistemi v praksi?
        \begin{itemize}
            \item \colorbox{green!30}{\textbf{Dokaz.}} Ali množenje razširjene matrike sistema z obrnljivo matriko z leve spremenijo rešitve?
        \end{itemize}
        \item \colorbox{yellow!30}{\emph{Primer.}} Reši sistem:
        \begin{align*}
            &x_1 + 2x_2 + 3x_3 + 4x_4 = -1, \\
            &8x_1 + 7x_2 + 6x_3 + 5x_4= 10, \\            
            &9x_1 + 10x_2 +  11x_3 + 12x_4 = 7.
        \end{align*}
        \item Reševanje več sistemov hkrati. Kako poiščemo inverz matrike $A \in \Fnn$?
        \item \colorbox{yellow!30}{\emph{Primer.}} Izračunaj $\begin{bmatrix}
            1 & 2 & -1 \\
            2 & 1 & 0 \\
            2 & -1 & 1
        \end{bmatrix}^{-1}$.
    \end{itemize}

    \item[$\circ$] Podobnost matrik 
    
    Naj bo $\Aa$ endomorfizem prostora $V$.  
    \begin{itemize}
        \item Običajno v $V$ izberemo eno bazo $B$ in napišemo matriko $\Aa_{B, B}$, in ne izberemo dveh različnih baz $B_1$ in $B_2$ in napišemo matriko $\Aa_{B_1, B_2}$. Zakaj to smiselno?
        \item Imejmo endomorfizem $\Aa: V \to V$ in bazi $B, B'$. Endomorfizmu priredimo matriki $A = \Aa_{B, B}$ in $A' = \Aa_{B', B'}$. V kakšni zvezi potem $A$ in $A'$?
        \item \colorbox{purple!30}{\textbf{Definicija.}} Podobnost matrik.
        \item \colorbox{yellow!30}{\emph{Opomba.}} Ali vsaki podobni matriki tudi ekvivalentni? Ali velja obrat?
        \item \colorbox{blue!30}{\textbf{Trditev.}} Ali je podobnost ekvivalenčna relacija?
        \begin{itemize}
            \item \colorbox{green!30}{\textbf{Dokaz.}} Kot pri ekvivalentnosti.
        \end{itemize}
        \item \colorbox{blue!30}{\textbf{Trditev.}} Karakterizacija podobnosti matrik z endomorfizmami.
        \begin{itemize}
            \item \colorbox{green!30}{\textbf{Dokaz.}} Kot pri ekvivalentnosti.
        \end{itemize}
    \end{itemize}
    Zanima nas, kakšna je najbolj enostavna matrika, ki pripada endomorfizmu $\Aa$ prostora $V$, torej kakšna je najbol enostavna matrika oblike $\Aa_{B, B}$, kjer je $B$ baza prostora $V$.
    \begin{itemize}
        \item Ali je vsaka matrika podobna matrike oblike $\begin{bmatrix}
            I_k & 0 \\ 0 & 0
        \end{bmatrix}$?
        \item Ali morda pa je matrika podobna neki diagonalni matriki?
        \item \colorbox{purple!30}{\textbf{Definicija.}} Kadar endomorfizem $\Aa \in L(V)$ se da diagonalizirati?
        \item \colorbox{yellow!30}{\emph{Primer.}} Ali se da diagonalizirati matrika $A = \begin{bmatrix}
            0 & 1 \\ 0 & 0
        \end{bmatrix}$?
        \begin{itemize}
            \item \colorbox{green!30}{\textbf{Dokaz.}} Recimo, da $A = P^{-1}DP$, za neko diagonalno matriko $D$ in obrnljivo matriko $P$.
        \end{itemize}
        \item Recimo, da se endomorfizem $\Aa: V \to V$ diagonalizira v bazi $B$ in naj bo $A = \Aa_{B, B}, \ B = \set{v_1, \ldots, v_n}$. Kam slika endomorfizem $\Aa$ vektorji iz baze $B$?
        \item \colorbox{purple!30}{\textbf{Definicija.}} Naj bo $\Aa \in L(V)$. Lastni vektor endomorfizma $\Aa$. Lastna vrednost endomorfizma $\Aa$ za lastni vektor $v$.
        \item \colorbox{blue!30}{\textbf{Trditev.}} Ali je lastna vrednost enolično določena z lastnim vektorjem?
        \begin{itemize}
            \item \colorbox{green!30}{\textbf{Dokaz.}} Enostavno.
        \end{itemize}
        \item \colorbox{blue!30}{\textbf{Trditev.}} Ali je lastni vektor enolično določen z lastno vrednostjo?
        \begin{itemize}
            \item \colorbox{green!30}{\textbf{Dokaz.}} Recimo, da je $\Aa v = \lambda v$. Kaj vemo o vektorju $\alpha v$?
        \end{itemize}
        \item \colorbox{blue!30}{\textbf{Trditev.}} Kako poiščemo lastni vrednosti endomorfizma $\Aa$? Kadar je $\lambda$ lastna vrednost endomorfizma $\Aa$?
        \begin{itemize}
            \item \colorbox{green!30}{\textbf{Dokaz.}} Pokažemo, da je $\lambda$ lastna vrednost endomorfizma $\Aa \Leftrightarrow \Aa - \lambda \id$ ni bijektivna.
        \end{itemize}
        \item \colorbox{blue!30}{\textbf{Trditev.}} Karakterizacija ni bijektivnosti preslikave $\Aa - \lambda \id$.
        \begin{itemize}
            \item \colorbox{green!30}{\textbf{Dokaz.}} Izomorfizem med prostorama endomorfizmov in matrik.
        \end{itemize}
    \end{itemize}
\end{enumerate}

\newpage
\

