\documentclass[11pt, a4paper]{article}

\usepackage[utf8]{inputenc}
\usepackage[T1]{fontenc}
\usepackage[slovene]{babel}
\usepackage{lmodern}

\usepackage{xcolor}
\usepackage{bbold}
\usepackage{amsmath}
\usepackage{amssymb}
\usepackage{amsthm}
\usepackage{xypic}

\usepackage{enumitem}
\setlist{nosep}

\setlength{\parindent}{0mm}

\usepackage{geometry}
\geometry{
    a4paper,
    left=5mm,
    right=5mm,
    top=5mm,
    bottom=15mm
}

% Standardne množice
\newcommand{\NN}{\mathbb{N}}
\newcommand{\ZZ}{\mathbb{Z}}
\newcommand{\QQ}{\mathbb{Q}}
\newcommand{\RR}{\mathbb{R}}
\newcommand{\CC}{\mathbb{C}}
\newcommand{\FF}{\mathbb{F}}

%%% Quantifiers
\newcommand{\all}[1]{\forall #1 \,.\,}
\newcommand{\some}[1]{\exists #1 \,.\,}
\newcommand{\exactlyone}[1]{\exists! #1 \,.\,}

% Implikacija
\newcommand{\lthen}{\Rightarrow}
\newcommand{\liff}{\Leftrightarrow}

%%% Množice
\newcommand{\set}[1]{\left\{#1\right\}}
\newcommand{\setb}[2]{\set{#1 \ | \  #2}}

%% Preslikave
\newcommand{\img}[1]{#1_{*}}
\newcommand{\invimg}[1]{#1^{*}}
\newcommand{\lin}[1]{\mathcal{#1}}

%% Odvod
\newcommand{\podv}[2]{\frac{\partial #1}{\partial #2}}

%% Metrični prostori
\newcommand{\norm}[1]{||#1||}

\DeclareMathOperator{\grad}{grad}




%--------------------------------------------------------------------
%-- Okolja

\theoremstyle{plain}{
    \newtheorem{izrek}{Izrek}[section]
    \newtheorem{trditev}[izrek]{Trditev}
    \newtheorem{posledica}{Posledica}[izrek]
    \newtheorem{lema}[izrek]{Lema}
    \newtheorem{aksiom}[izrek]{Aksiom}
}

\theoremstyle{definition}{
    \newtheorem{definicija}{Definicija}[section]
    \newtheorem*{primer}{Primer}
    \newtheorem*{opomba}{Opomba}
    \newtheorem*{pojmi}{Ključni pojmi}
}

\newenvironment{ideja}{\begin{proof}[Ideja dokaza]}{\end{proof}}


\begin{document}

\section{VEKTORJI V $\RR^3$}

\begin{enumerate}
    \item Koordinatni sistem
    \begin{itemize}
        \item Model za realna števila. Bijekcija med $\RR$ in realno osjo.
        \item Model za $\RR^2$. Običajen koordinatni sistem. Abscisna in ordinatna os. Bijekcija med $\RR^2$ in ravnino. Razdalja med točkama $T_1(x_1, y_1)$ in $T_2(x_2, y_2)$.
        \item Model za $\RR^3$. Pozitivno orientiran koordinatni sistem. Bijekcija med $\RR^3$ in prostorom. Razdalja med točkama $T_1(x_1, y_1, z_1)$ in $T_2(x_2, y_2, z_2)$.
        \item Definicija seštevanja in množenja s skalarjem na množici $\RR^3$.
    \end{itemize}

    \item Vektorji
    \begin{itemize}
        \item \colorbox{purple!30}{\textbf{Definicija.}} Krajevni vektor točke $a$. Oznaka. Bijekcija med $\RR^3$ in množico krajevnih vektorjev.
        \item \colorbox{purple!30}{\textbf{Definicija.}} Vektor.
        \item \colorbox{yellow!30}{\emph{Opomba.}} Nenatančna definicija vektorja. Kadar sta dva vektorja enaka? Oznaka za vektor od točke $A$ do točke $B$.
        \item Bijekcija med množico vektorjev in množico krajevnih vektorjev. Bijekcija med $\RR^3$ in množico vektorjev.
        \item Definicija seštevanja in množenja s skalarjem na množici vektorjev.
        \item \colorbox{purple!30}{\textbf{Definicija.}} Ničelni vektor. Nasprotni vektor. S čim sta določena ta vektorja?
        \item Definicija odštevanja vektorjev.
        \item \colorbox{blue!30}{\textbf{Trditev.}} 8 lastnosi operacij z vektorji (aksiomi za vektorski prostor).
        \begin{itemize}
            \item \colorbox{green!30}{\textbf{Dokaz.}} Napišemo po komponentah in poračunamo.
        \end{itemize}
        \item \colorbox{purple!30}{\textbf{Definicija.}} Linearna kombinacija vektorjev $\vec{a}_1, \vec{a}_2, \ldots, \vec{a}_n$.
        \item \colorbox{purple!30}{\textbf{Definicija.}} Linearno odvisni vektorji. Linearno neodvisni vektorji.
        \item \colorbox{yellow!30}{\emph{Opomba.}} Kadar sta sva vektorja $\vec{a}$ in $\vec{b}$ linearno odvisna?
        \item \colorbox{blue!30}{\textbf{Trditev.}} Ali je ravnina, napeta na linearno neodvisna vektorja $\vec{a}$ in $\vec{b}$, je natanko množica vseh vektorjev oblike $\alpha \vec{a} + \beta \vec{b}$?
        \begin{itemize}
            \item \colorbox{green!30}{\textbf{Dokaz.}} Potegnimo ustrezne premice.
        \end{itemize}
        \item \colorbox{orange!30}{\textbf{Posledica.}} Karakterizacija linearne odvisnosti treh krajevnih vektorjev (ravnine).
        \item \colorbox{purple!30}{\textbf{Definicija.}} Baza prostora $\RR^3$.
        \item \colorbox{yellow!30}{\emph{Opomba.}} Baza prostora $\RR^2$.
        \item \colorbox{blue!30}{\textbf{Trditev.}} Razvoj po baze v $\RR^3$. 
        \begin{itemize}
            \item \colorbox{green!30}{\textbf{Dokaz.}} Obstoj: Potegnimo ustrezne premice.            
            Enoličnost: Definicija linearne neodvisnosti.
        \end{itemize}
        \item \colorbox{orange!30}{\textbf{Posledica.}} Kaj lahko povemo o linearni odvisnosti 4 vektorjev v $\RR^3$?
        \item \colorbox{yellow!30}{\emph{Primer.}} \textbf{Standardna baza} prostora $\RR^3$. \textbf{Standardna baza} prostora $\RR^2$. 
        \item \colorbox{blue!30}{\textbf{Trditev.}} Karakterizacija linearne neodvisnosti treh vektorjev v $\RR^3$ (linearna kombinacija).
        \begin{itemize}
            \item \colorbox{green!30}{\textbf{Dokaz.}} Definicija linearne neodvisnosti.
        \end{itemize}
    \end{itemize}

    \item Skalarni produkt
    \begin{itemize}
        \item \colorbox{purple!30}{\textbf{Definicija.}} Skalarni produkt.
        \item \colorbox{blue!30}{\textbf{Trditev.}} 4 lastnosti skalarnega produkta.
        \begin{itemize}
            \item \colorbox{green!30}{\textbf{Dokaz.}} Vektorji napišemo po komponentah in poračunamo.
        \end{itemize}
        \item \colorbox{yellow!30}{\emph{Opomba.}} Ali je skalarni produkt asociativen?
        \item \colorbox{purple!30}{\textbf{Definicija.}} Dolžina ali norma vektorja.
        \item \colorbox{yellow!30}{\emph{Opomba.}} Kaj je dolžina krajevnega vektorja?
        \item \colorbox{blue!30}{\textbf{Izrek.}} Formula za skalarni produkt.
        \begin{itemize}
            \item \colorbox{green!30}{\textbf{Dokaz.}} Cosinusni izrek.
        \end{itemize}
        \item Dogovor o vektorju $\vec{0}$.
        \item \colorbox{orange!30}{\textbf{Posledica.}} Karakterizacija pravokotnosti vektorjev s skalarnim produktom.
        \item \colorbox{yellow!30}{\emph{Primer.}} Izračunaj kot med vektorjema $\vec{a}=(1,1,2)$ in $\vec{b} = (1, 0, 1)$.
        \item \colorbox{yellow!30}{\emph{Primer.}} Naj bosta $\vec{a}_1=(x_1,y_1,0)$ in $\vec{a}_2=(x_2,y_2,0)$ vektorja v ravnini $z=0$. Z vektorjema $\vec{a}_1$ in $\vec{a}_2$ izrazi ploščino paralelograma, napetega na $\vec{a}_1$ in $\vec{a}_2$. Kaj je in kaj pove dobljeni izraz?
        \item \colorbox{orange!30}{\textbf{Posledica.}} Karakterizacija linearne odvisnosti dveh vektorjev v ravnini.
    \end{itemize}

    \newpage
    \item Vektorski produkt
    \begin{itemize}
        \item \colorbox{purple!30}{\textbf{Definicija.}} Vektorski produkt.
        \item Dogovor o vektorju $\vec{0}$.
        \item \colorbox{orange!30}{\textbf{Posledica.}} Karakterizacija vzporednosti vektorjev s vektorskim produktom.
        \item Naj bo $\vec{a}_1, \vec{a}_2 \in \RR^3$. Izračunaj koordinate vekotrja $\vec{a}_1 \times \vec{a}_2$. Kaj je rezultat?
        \begin{itemize}
            \item \colorbox{green!30}{\textbf{Izračun.}} Izračunali bomo $z_3 = ||\vec{a}_1 \times \vec{a}_2|| \cdot \vec{k}$. Proiciramo paralelogram napeti na vektorja $\vec{a}_1$ in $\vec{a}_2$ in poiščemo zvezo med ploščino originalnega in proiciranega paralelograma (pomagamo si s ploščinoma trikotnikov).
        \end{itemize}
        \item \colorbox{blue!30}{\textbf{Trditev.}} 3 lastnosti vektorskega produkta.
        \begin{itemize}
            \item \colorbox{green!30}{\textbf{Dokaz.}} Preverimo z računom.
        \end{itemize}
        \colorbox{yellow!30}{\emph{Primer.}} Izračunaj ploščino trikotnika  s oglišči $A(1,0,2), \ B(2,2,0), \ C(3,-2,1)$.
    \end{itemize}

    \item Mešani produkt
    \begin{itemize}
        \item \colorbox{purple!30}{\textbf{Definicija.}} Mešani produkt. Oznaka.
        \item Čemu je enak mešani produkt?
        \item Geometrijska interpretacija mešanega produkta.
        \item \colorbox{orange!30}{\textbf{Posledica.}} Ciklična zamena komponent v mešanem produktu.
        \item \colorbox{orange!30}{\textbf{Posledica.}} Čemu je enak mešani produkt?
        \item \colorbox{orange!30}{\textbf{Posledica.}} Karakterizacija linearne odvisnosti 3 vektorjev c mešanim produktom.
        \item \colorbox{blue!30}{\textbf{Trditev.}} 2 lastnosti mešanega produkta.
        \begin{itemize}
            \item \colorbox{green!30}{\textbf{Dokaz.}} Preverimo z računom ali upoštevamo lastnosti skalarnega in vektorskega proudkta.
        \end{itemize}
    \end{itemize}

    \item Dvojni vektorski produkt
    \begin{itemize}
        \item Formula za dvojni vektorski produkt $(\vec{a} \times \vec{b}) \times \vec{c}$.
        \begin{itemize}
            \item \colorbox{green!30}{\textbf{Dokaz.}} Preverimo z računom.
        \end{itemize}
        \item Kaj če v zgornjo formulo vstavimo $\vec{c} \times \vec{d}$ namesto $\vec{c}$?
        \item Langrangeeva identiteta $(\vec{a} \times \vec{b}) \cdot (\vec{c} \times \vec{d})$.
        \item Kaj če v zgornjo formulo vstavimo $\vec{c} = \vec{a}$ in $\vec{d} = \vec{b}$?
    \end{itemize}

    \item Premice in ravnine v $\RR^3$
    \item[$\circ$] Enačba ravnine
    \begin{itemize}
        \item \colorbox{purple!30}{\textbf{Definicija.}} Enačba ravnine. 
        \item \colorbox{purple!30}{\textbf{Definicija.}} Normala ravnine.
        \item S čim je enolično določena ravnina? Ali je z normalo?
        \item \colorbox{blue!30}{\textbf{Izpeljava.}} Vektorska enačba ravnine. Vektorska enačba ravnine po komponentah.
        \begin{itemize}
            \item \colorbox{green!30}{\textbf{Dokaz.}} Skalarni produkt pravokotnih vektorjev.
        \end{itemize}
        \item Kaj je enačba ravnine? Ali je vsaka linearna enačba v spremenljivkah $x, y, z$ enačba neke ravnine? Kako dobimo normalo iz enačbe ravnine? V čim se razlekujejo enačbe ravnin? Ali je enačba z ravnino enolično določena?
        \item Enotska normala. Normalna enačba ravnine.
        \item \colorbox{blue!30}{\textbf{Izpeljava.}} Enačba ravnine, podane s 3 nekolinearnimi točkami.
        \begin{itemize}
            \item \colorbox{green!30}{\textbf{Dokaz.}} Normalo dobimo z vektorskim produktom.
        \end{itemize}
        \item \colorbox{yellow!30}{\emph{Primer.}} Določi enačbo ravnine skozi točke $T_0(1, -1, 3), \ T_1(2,0,5), \ T_2(-3, -1, 4)$.
    \end{itemize}

    \item[$\circ$] Razdalja do ravnine
    
    Zanima nas razdalja $\Delta$ med točko $T_1$ in ravnino $\Sigma$.
    \begin{itemize}
        \item \colorbox{purple!30}{\textbf{Definicija.}} Razdalja med $T_1$ in ravnino $\Sigma$.
        \item Kaj je $\Delta$? Zakaj?
        \item \colorbox{blue!30}{\textbf{Izpeljava.}} Formula za razdaljo med točko in ravnino.
        \begin{itemize}
            \item \colorbox{green!30}{\textbf{Dokaz.}} Pravokotni trikotnik + formula za skalarni produkt.
        \end{itemize}        
        \item Kako dobimo razdaljo med točko in ravnino? Kadar je točka pripada ravnine?
        \item \colorbox{purple!30}{\textbf{Definicija.}} Razdalja med ravninama, ki se sekata. Razdalja med vzporednima ravninama. Razdalja med ravnino in premico, ke se sekata. Razdalja med ravnino in njej vzporedno premico. 
    \end{itemize}

    \newpage
    \item[$\circ$] Enačba premice
    \begin{itemize}
        \item Kaj je premica (ali je nek presek)? Kaj je enačba premice?
        \item \colorbox{purple!30}{\textbf{Definicija.}} Smerni vektor.
        \item Ali je vzporedne premice imajo isti smerni vektor? S čim je enolično določena premica?
        \item \colorbox{blue!30}{\textbf{Izpeljava.}} Parametrična vektorska enačba premice. Parameter.
        \begin{itemize}
            \item \colorbox{green!30}{\textbf{Dokaz.}} Kako pridemo do vsake točke na premice?
        \end{itemize}
        \item Enačba premice brez parametra. Kako preberemo smerni vektor?
        \item \colorbox{blue!30}{\textbf{Izpeljava.}} Enačba premice skozi točki $T_0, \ T_1$.
        \begin{itemize}
            \item \colorbox{green!30}{\textbf{Dokaz.}} Za smerni vektor vzamemo $\overrightarrow{T_1 T_0}$
        \end{itemize}
    \end{itemize}

    \item[$\circ$] Razdalja do premice
    
    Zanima nas razdalja $\Delta$ med točko $T_1$ in premico.
    \begin{itemize}
        \item Čemu je enaka razdalja med točko $T_1$ in premico?
        \item \colorbox{blue!30}{\textbf{Izpeljava.}} Formula za razdaljo med točko $T_1$ in premico.
        \begin{itemize}
            \item \colorbox{green!30}{\textbf{Dokaz.}} Pravokotni trikotnik in vektorski produkt.
        \end{itemize}
        \item \colorbox{yellow!30}{\emph{Primer.}} Izračunaj razdaljo med točko $A(1,1,1)$ in premico z enačbo $\frac{x}{2} = \frac{y+1}{-2}=z+1$.
        \item \colorbox{purple!30}{\textbf{Definicija.}} Razdalja med premicama, ki se sekata. Razdalja med vzporednima premicama.
        \item \colorbox{blue!30}{\textbf{Izpeljava.}} Razdalja med mimobežnima premicama.
        \begin{itemize}
            \item \colorbox{green!30}{\textbf{Dokaz.}} Skozi $T_1$ potegnimo premico $p_2'$, ki je vzporedna $p_2$. Podobno naredimo skozi točko $T_2$. Sekajoči premici določata dve vzporedni ravnini, torej razdalja med premicama je vsaj razdalja med ravninama. Z pravoktnimi projekcijama premic $p_1$ in $p_2$ na ustrezne ravnine pokažemo, da je razdalja med premicama kvečjemu razdalja med presečičsama danih premic in pravokotnih projekcij. Od tod sledi, da je razdalja med premicama je razdalja med vzporednima ravninama.
        \end{itemize}
        \item \colorbox{orange!30}{\textbf{Posledica.}} Kadar premici se sekata?
        \item \colorbox{yellow!30}{\emph{Primer.}} Poišči razdaljo med premicama $x=1, \ y-1 = - z$ in $\frac{1-x}{4}=\frac{y+1}{3}= \frac{z+1}{4}$.
    \end{itemize}
\end{enumerate}

\newpage
\

\newpage
\section{OSNOVNE ALGEBRAIČNE STRUKTURE}

\begin{enumerate}
    \item[$\circ$] Ponovitev preslikav
    \begin{itemize}
        \item \colorbox{purple!30}{\textbf{Definicija.}} Injektivna, surjektivna, bijektivna preslikava.
        \item \colorbox{purple!30}{\textbf{Definicija.}} Predpis za inverzno preslikavo.
        \item \colorbox{purple!30}{\textbf{Definicija.}} Kompozitum preslikav.
        \item \colorbox{purple!30}{\textbf{Definicija.}} Identična preslikava (identiteta) na $A$.
        \item \colorbox{blue!30}{\textbf{Trditev.}} Karakterizacija bijektivnosti preslikave (obstoj inverza).
        \item \colorbox{blue!30}{\textbf{Trditev.}} Kompozicija injektivnih/surjektivnih/bijektivnih preslikav. Kaj če je kompozicija preslikav injektivna/surjektivna?
        \item \colorbox{purple!30}{\textbf{Definicija.}} Slika in praslika.
    \end{itemize}
    \item Operacije
    \begin{itemize}
        \item \colorbox{purple!30}{\textbf{Definicija.}} Dvočlena (binarna) notranja operacija na množice $A$. Kompozitum elementov. Konkretna operacija.
        \item \colorbox{yellow!30}{\emph{Primer.}} Ali je operacija?
        \begin{itemize}
            \item $A=\NN$, $\circ$ seštevanje ali množenje. Ali je odštevanje operacija na $\NN$?
            \item $A=\RR$, $\circ$ seštevanje ali množenje ali odštevanje. Ali je deljenje operacija na $\RR$?
            \item $A=\RR^3$, $\circ$ seštevanje ali odštevanje ali vektorski produkt. Ali je skalarni produkt operacija na $\RR^3$?
            \item Naj bo $A \neq \emptyset$ in $F$ množica vseh preslikav $A \to A$, $\circ$ je kompozitum preslikav.
        \end{itemize}
        \item \colorbox{purple!30}{\textbf{Definicija.}} $n$-člena notranja operacija na množice $A$. 
        \item \colorbox{purple!30}{\textbf{Definicija.}} Dvočlena zunanja operacija na množice $A$. 
        \item \colorbox{yellow!30}{\emph{Primer.}} Ali je množenje s skalarjem zunanja operacija na $\RR^3$?
        \item \colorbox{purple!30}{\textbf{Definicija.}} Kadar pravimo, da množica $A$ ima algebraično strukturo?
        \item \colorbox{purple!30}{\textbf{Definicija.}} Asociativna operacija.
        \item \colorbox{blue!30}{\textbf{Trditev.}} Asociativna operacija in oklepaji.
        \item \colorbox{yellow!30}{\emph{Opomba.}} Kako pogosto rečemo asociativne operacije? Ali lahko spuščamo oklepaji?
        \item \colorbox{purple!30}{\textbf{Definicija.}} Komutativna operacija.
        \item Kako pogosto rečemo komutativne in asociativne operacije?
        \item \colorbox{yellow!30}{\emph{Primer.}} Ali je komutatvni/asociativni naslednji operaciji?
        \begin{itemize}
            \item Seštevanje in množenje na $\RR$.
            \item Odštevanje na $\RR$.            
            \item Vektorski produkt na $\RR^3$.            
            \item Kompozitum preslikav iz $A$ v $A$.
        \end{itemize}
        \item \colorbox{purple!30}{\textbf{Definicija.}} Enota ali nevtralni element za operacijo $\circ$. Leva/desna enota.
        \item \colorbox{yellow!30}{\emph{Primer.}} Določi enote, če obstajajo.
        \begin{itemize}
            \item Enota v $(\RR, +)$.
            Enota v $(\RR, \cdot)$.
            Enota v $(\RR^3, \times)$.
            Enota v $(F = \set{f: A \to A}, \circ)$.
        \end{itemize}
        \item \colorbox{blue!30}{\textbf{Trditev.}} Kaj če obstaja leva in desna enota za operacijo $\circ$?
        \begin{itemize}
            \item \colorbox{green!30}{\textbf{Dokaz.}} Enostavno izračunamo $e \circ f$.
        \end{itemize}
        \item \colorbox{orange!30}{\textbf{Posledica.}} Koliko enot lahko ima neka operacija $\circ$?
        \item \colorbox{purple!30}{\textbf{Definicija.}} Levi/desni inverz elementa $a$. Inverz elementa $a$. Kaj pravimo, če je $a$ ima inverz?
        \item \colorbox{yellow!30}{\emph{Opomba.}} Ali inverz nujno obstaja?
        \item \colorbox{blue!30}{\textbf{Trditev.}} Kaj če je obstaja levi in desni inverz elementa $a$ za neko asociativno operacijo z enoto?
        \begin{itemize}
            \item \colorbox{green!30}{\textbf{Dokaz.}} Enostavno.
        \end{itemize}
        \item \colorbox{orange!30}{\textbf{Posledica.}} Koliko inverzov lahko ima element $a$?
        \item Oznaka za inverz elementa $a$.
    \end{itemize}

    \newpage
    \item Grupe
    \begin{itemize}
        \item \colorbox{purple!30}{\textbf{Definicija.}} Grupoid. Polgrupa. Monoid.
        \item \colorbox{yellow!30}{\emph{Primer.}}  Ugotovi, ali je grupoid, polgrupa, monoid.
        \begin{itemize}
            \item $(\NN, +), \ (\ZZ, +), \ (\QQ, +),  \ (\RR, +)$.
            $(\NN, \cdot), \ (\ZZ, \cdot), \ (\QQ, \cdot),  \ (\RR, \cdot)$.
            $(\ZZ, -)$.
            $(\RR^3, \times)$.
            \item $(F = \set{f: A \to A}, \circ)$.
        \end{itemize}
        \item \colorbox{blue!30}{\textbf{Trditev.}} Kompozitum obrnljivih elementov v monoidu.
        \begin{itemize}
            \item \colorbox{green!30}{\textbf{Dokaz.}} Z računom pokažemo, da $(a \circ b)^{-1} = b^{-1} \circ a^{-1}$.
        \end{itemize}
        \item \colorbox{purple!30}{\textbf{Definicija.}} Grupa.
        \item \colorbox{yellow!30}{\emph{Primer.}}  Ugotovi, ali je grupa.
        \begin{itemize}
            \item $(\ZZ, +), \ (\QQ, +),  \ (\RR, +)$.
            $(\QQ, \cdot)$.
            $(\QQ \setminus \set{0}, \cdot),  \ (\RR \setminus \set{0}, \cdot), \ (\CC \setminus \set{0}, \cdot), (\set{1, -1, i, -i}, \cdot)$.
            $(\ZZ \setminus \set{0}, \cdot)$.            
            \item Naj bo $F$ množica vseh preslikav iz $A$ v $A$. Ali je $(F, \circ)$ grupa?  
            \item Naj bo $A$ množica in naj bo $S(A)$ množica vseh bijektivnih preslikav $A \to A$. Ali je $(S(A), \circ)$ grupa?
        \end{itemize}
        \item \colorbox{purple!30}{\textbf{Definicija.}} Multiplikativni zapis operacije: operacija, enota, potenca ($a^n, \ a^0, \ a^{-n}$).
        \item \colorbox{blue!30}{\textbf{Trditev.}} 2 lastnosti potenc elementa $a$ v grupi.
        \begin{itemize}
            \item \colorbox{green!30}{\textbf{Dokaz.}} Kot za znane formule o potencah števil.
        \end{itemize}
        \item \colorbox{orange!30}{\textbf{Posledica.}} $a^{-n} = (a^{n})^{-1}$.
        \begin{itemize}
            \item \colorbox{green!30}{\textbf{Dokaz.}} Pokažimo, da je inverz.
        \end{itemize}
        \item \colorbox{purple!30}{\textbf{Definicija.}} Komutativna ali Abelova grupa.
        \item \colorbox{yellow!30}{\emph{Primer.}}  Komutativni grupi.
        \begin{itemize}
            \item $(\ZZ, +), \ (\QQ, +),  \ (\RR, +)$.
            $(\QQ \setminus \set{0}, \cdot),  \ (\RR \setminus \set{0}, \cdot), \ ((0, \infty), \cdot)$.
        \end{itemize} 
        \item \colorbox{purple!30}{\textbf{Definicija.}} Aditivni zapis operacije (operacija, enota, vsota $n$ elementov, inverz).
        \item \colorbox{blue!30}{\textbf{Trditev.}} 3 lastnosti vsote elementov v Abelovi grupi.
        \begin{itemize}
            \item \colorbox{green!30}{\textbf{Dokaz.}} Prve dve kot prej, tretjo izračunamo.
        \end{itemize}
        \item \colorbox{yellow!30}{\emph{Opomba.}} Ali velja $(a \cdot b)^n = a^n \cdot b^n$, če $G$ ni komutativna?
        \item \colorbox{blue!30}{\textbf{Trditev.}} Ali lahko krajšamo v grupi?
        \begin{itemize}
            \item \colorbox{green!30}{\textbf{Dokaz.}} Pomnožimo z ustreznim inverzom.
        \end{itemize}
        \item \colorbox{yellow!30}{\emph{Opomba.}} Ali v grupi vedno velja $ab = ca \Rightarrow b = c$? Ali v splošnem velja krajšanje v polgrupah?
        \item \colorbox{purple!30}{\textbf{Definicija.}} Tabela množenja za končne grupe. Ali se lahko kakšen element v vrstici ali stolpcu ponovi?
        \item \colorbox{blue!30}{\textbf{Trditev.}} Karakterizacija komutativnosti končne grupe s tabelo množenja. 
    \end{itemize}

    \item[$\circ$] Grupe majhnih moči
    \begin{itemize}
        \item $n=1$. \textbf{Trivialna grupa.}
        \item \colorbox{purple!30}{\textbf{Definicija.}} Grupa $(\ZZ_n, +)$ ostankov za seštevanje po modulu $n$. Ali je komutativna?
        \item $n=2, \ n=3, \ n = 4, \ n= 5$.
        \item \colorbox{yellow!30}{\emph{Opomba.}} Ali so vse grupe moči največ $5$ komutativni?
    \end{itemize}

    \item[$\circ$] Grupe permutacij
    \begin{itemize}
        \item \colorbox{purple!30}{\textbf{Definicija.}} Permutacija.  Simetrična grupa reda $n$. Oznaka.
        \item \colorbox{yellow!30}{\emph{Opomba.}} Koliko elementov ima $S_n$? Ali je $S_n$ komutativna? Kako pišemo permutacije?
        \item \colorbox{yellow!30}{\emph{Primer.}} Elementi grupi $S_3$. Ali je $S_3$ komutativna? Tabela množenja za $S_3$.
        \item \colorbox{purple!30}{\textbf{Definicija.}} Cikel dolžine $k$. Transpozicija. Disjunktna cikla. Oznaka.
        \item \colorbox{yellow!30}{\emph{Opomba.}} Ali disjunktna cikla vedno komutirata? Cikel dolžine $1$. 
        \item \colorbox{yellow!30}{\emph{Primer.}} Ali so vsi elementi $S_3$ cikli? Kaj pa s elementi $S_4$?
        \item \colorbox{blue!30}{\textbf{Izrek.}} Permutacija kot produkt paroma disjunktnih ciklov.
        \begin{itemize}
            \item \colorbox{green!30}{\textbf{Dokaz.}} Obstoj: S strogo indukcijo na $n$. Oznaka $x_i = \pi^i(1)$ za vsak $i \in \NN_0$. Ali se eden iz med $x_i$ zagotovo ponovi?
            
            Enoličnost: Naj bo $\pi = \sigma_1 \ldots \sigma_m = \rho_1 \ldots \rho_l$. Ker cikli komutirajo lahko izberimo tak vrstni red, da se $\sigma_1$ in $\rho_1$ začneta z $1$, $\sigma_2$ in $\rho_2$ začneta z najmanjšim številom, ki se ne pojavi v $\sigma_1$ in $\rho_1$ in tako naprej.
        \end{itemize}
        \item \colorbox{yellow!30}{\emph{Primer.}} Zapiši kot produkt disjunktnih ciklov: $ \begin{pmatrix}
            1 & 2 & 3 & 4 & 5 & 6 & 7 & 8 & 9 \\
            3 & 9 & 1 & 6 & 4 & 5 & 2 & 7 & 8 
            \end{pmatrix}  $.
        \item \colorbox{blue!30}{\textbf{Trditev.}} Ali je vsak cikel produkt transpozicij?
        \begin{itemize}
            \item \colorbox{green!30}{\textbf{Dokaz.}} $(a_1 \ a_2 \ \ldots \ a_k) = (a_1 \ a_k)(a_1 \ a_{k-1}) \ldots (a_1 a_2)$.
        \end{itemize}
        \item \colorbox{orange!30}{\textbf{Posledica.}} Ali je vsaka permutacija produkt transpozicij? Ali je ta produkt komutativen oziroma enoličen?
        
        \newpage
        \item \colorbox{purple!30}{\textbf{Definicija.}} Znak permutacije.
        \item \colorbox{yellow!30}{\emph{Opomba.}} Zakaj je definicija dobra?
        \item \colorbox{yellow!30}{\emph{Primer.}} Določi znak permutacije: $ \begin{pmatrix}
            1 & 2 & 3 & 4 & 5 & 6 & 7 & 8 & 9 \\
            3 & 9 & 1 & 6 & 4 & 5 & 2 & 7 & 8 
            \end{pmatrix}  $.
        \item \colorbox{blue!30}{\textbf{Trditev.}} Kaj se zgodi z znakom permutacije, če jo z levo pomnožimo z transpozicijo?
        \begin{itemize}
            \item \colorbox{green!30}{\textbf{Dokaz.}} Za števili iz transpozicije $\tau$ imamo dve možnosti:
            
            (1) Števili se pojavita v istem ciklu.

            (2) Števili se pojavita v dveh različnih ciklih.
        \end{itemize}
        \item \colorbox{orange!30}{\textbf{Posledica.}} Znak permutacije v odvisnosti od števila transpozicij v razcepu.
        \begin{itemize}
            \item \colorbox{green!30}{\textbf{Dokaz.}} Z indukcijo na $M$ na podlagi prejšnje trditve.
        \end{itemize}
        \item \colorbox{orange!30}{\textbf{Posledica.}} Kaj se zgodi z znakom permutacije, če jo z desno pomnožimo z transpozicijo?
        \item \colorbox{orange!30}{\textbf{Posledica.}} Ali lahko iste permutacije zapišemo enkrat kot produkt sodega števila transpozicij, drugič pa kot produkt lihega števila transpozicij?
        \begin{itemize}
            \item \colorbox{green!30}{\textbf{Dokaz.}} Definicija znaka permutacije.
        \end{itemize}
        \item \colorbox{purple!30}{\textbf{Definicija.}} Soda/liha permutacija.
        \item \colorbox{orange!30}{\textbf{Posledica.}} Znak produkta permutacij.
        \begin{itemize}
            \item \colorbox{green!30}{\textbf{Dokaz.}} Razcepimo permutaciji na produkt transpozicij in izračunamo $s(\pi) s(\sigma)$.
        \end{itemize}
        \item \colorbox{blue!30}{\textbf{Trditev.}} Produkt sodih permutacij. Inverz sode permutacije.
        \begin{itemize}
            \item \colorbox{green!30}{\textbf{Dokaz.}} Izračunamo znak produkta in inverza.
        \end{itemize}
        \item \colorbox{purple!30}{\textbf{Definicija.}} Alternirajoča grupa reda $n$. Oznaka.
        \item \colorbox{yellow!30}{\emph{Opomba.}} Naj bo $n>1$ in $\tau \in S_n$ transpozicija. Ali je preslikava $f: \ A_n \to \text{lihe permutaciji}, \ f(\pi) = \tau \circ \pi$ bijektivna? Koliko elementov ima $A_n$?
    \end{itemize}

    \item[$\circ$] Podgrupe
    \begin{itemize}
        \item \colorbox{purple!30}{\textbf{Definicija.}} Podgrupa grupe $G$.
        \item \colorbox{blue!30}{\textbf{Trditev.}} Ali je podgrupa vedno vsebuje enoto grupe $G$?
        \begin{itemize}
            \item \colorbox{green!30}{\textbf{Dokaz.}} Definicija podgrupe.
        \end{itemize}
        \item \colorbox{orange!30}{\textbf{Posledica.}} Ali je podgrupa grupe $G$ grupa?
        \begin{itemize}
            \item \colorbox{green!30}{\textbf{Dokaz.}} Definicija podgrupe in prejšnja trditev.
        \end{itemize}
        \item \colorbox{yellow!30}{\emph{Primer.}}  Ugotovi ali je podgrupa:
        \begin{itemize}
            \item $(\ZZ, +)$ v  $(\QQ, +)$  v $(\RR, +)$.
            $(\QQ \setminus \set{0}, \cdot)$ v $(\RR \setminus \set{0}, \cdot)$.
            \item $(A_n, \circ)$ v $(S_n, \circ)$.
            Naj bo $\tau$ transpozicija. $\set{id, \tau}$ v $(S_n, \circ)$.
            \item \textbf{Neprava podgrupa. Trivialna podgrupa.}
        \end{itemize}
        \item \colorbox{blue!30}{\textbf{Trditev.}} Karakterizacija podgrupe multiplikativno pisane grupe.
        \begin{itemize}
            \item \colorbox{green!30}{\textbf{Dokaz.}} Definicija podgrupe.
        \end{itemize}
        \item \colorbox{orange!30}{\textbf{Posledica.}} Karakterizacija podgrupe Abelove aditivno pisane grupe.
        \item \colorbox{blue!30}{\textbf{Trditev.}} Ali je presek poljibne družine podgrup podgrupa?
        \begin{itemize}
            \item \colorbox{green!30}{\textbf{Dokaz.}} Enostavno z uporabo karakterizacije.
        \end{itemize}
        \item \colorbox{yellow!30}{\emph{Opomba.}} Ali je unija podgrup podgrupa? $G = S_3, \ H_1 = \set{\id, (1 \ 2)}, H_2 = \set{\id, (1 \ 3)}$.
    \end{itemize}

    \item[$\circ$] Homomorfizem grup
    \begin{itemize}
        \item \colorbox{purple!30}{\textbf{Definicija.}} Homomorfizem grup.
        \item \colorbox{yellow!30}{\emph{Primer.}}  Ugotovi ali je homomorfizem:
        \begin{itemize}
            \item $f: (\ZZ,+) \to (\QQ \setminus \set{0}, \cdot), \ f(x) := 2^x$.
            \item $s: S_n \to (\set{1, -1}, \cdot)$. Preslikava $s$ je znak permutacije.                        
            \item $g: (\CC \setminus \set{0}, \cdot) \to (\RR \setminus \set{0}, \cdot), \ g(z) := |z|$.
            \item Naj bo $G$ in $H$ poljubni grupi in je $e$ enota v $H$. $h: G \to H, \ h(x) := e$.
            Identiteta $id_G$.
            \item Naj bo $G$ poljubna grupa in $a \in G$. Definiramo preslikavo $f_a: G \to G$ s predpisom $f_a(x) := axa^{-1}$.
        \end{itemize}
        \item \colorbox{purple!30}{\textbf{Definicija.}} Izomorfizem. Monomorfizem. Epimorfizem. Endomorfizem. Avtomorfizem.
        \item \colorbox{yellow!30}{\emph{Primer.}} $f_a: G \to G$. Ali je to avtomorfizem? \textbf{Notranji avtomorfizem grupe $G$.}
        \item \colorbox{blue!30}{\textbf{Trditev.}} Kompozitum homomorfizmov. Inverz izomorfizma.
        \begin{itemize}
            \item \colorbox{green!30}{\textbf{Dokaz.}} (1) Enostavno po definiciji homomorfizma in kompozituma.
            
            (2) Enostavno po definiciji homomorfizma in inverzne preslikave.
        \end{itemize}
        \item \colorbox{blue!30}{\textbf{Trditev.}} Kam homomorfizem slika enoto in inverz?
        \begin{itemize}
            \item \colorbox{green!30}{\textbf{Dokaz.}} (1) Zapišemo $f(e) = f(e \cdot e) = f(e) \circ f(e)$ in pomnožimo z $f(e)^{-1}$ 
            
            (2) Pokažemo, da je $f(a^{-1})$ inverz od $f(a)$.
        \end{itemize}
        \item \colorbox{purple!30}{\textbf{Definicija.}} Izomorfni grupi. Oznaka. 
        \item \colorbox{yellow!30}{\emph{Opomba.}} Ali je izomorfnost ekvivalenčna relacija?
        \item \colorbox{blue!30}{\textbf{Trditev.}} Kam homomorfizem slika podgrupo?
        \begin{itemize}
            \item \colorbox{green!30}{\textbf{Dokaz.}} Pokažemo, da iz $x, y \in f(H)$ sledi, da $xy^{-1} \in f(H)$.
        \end{itemize}
        \item \colorbox{purple!30}{\textbf{Definicija.}} Slika homomorfizma. Jedro homomorfizma. Oznaki.
        \item \colorbox{yellow!30}{\emph{Opomba.}} Ali lahko $\ker f = \emptyset$?
        \item \colorbox{blue!30}{\textbf{Trditev}} o slike in jedru homomorfizma (ali je neki podgrupi?).
        \begin{itemize}
            \item \colorbox{green!30}{\textbf{Dokaz.}} Trditev za sliko sledi iz prejšnje trditve. Za jedro uporabimo karakterizacijo podgrupe.
        \end{itemize}
        \item \colorbox{yellow!30}{\emph{Opomba.}} Kadar je homomorfizem surjektiven?
        \item \colorbox{blue!30}{\textbf{Izrek.}} Karakterizacija injektivnosti homomorfizma. \textbf{Trivialno jedro.}
        \begin{itemize}
            \item \colorbox{green!30}{\textbf{Dokaz.}} Definicija injektivnosti, jedra, homomorfizma in trditev o slike enote in inverza.
        \end{itemize}
        \item \colorbox{yellow!30}{\emph{Primer.}} Ali je naslednja homomorfizma injektivna?
        \begin{itemize}
            \item $f: (\ZZ,+) \to (\QQ \setminus \set{0}, \cdot), \ f(x) := 2^x$.
            \item $g: (\CC \setminus \set{0}, \cdot) \to (\RR \setminus \set{0}, \cdot), \ g(z) := |z|$. Ali je enotska krožnica grupa za množenje.
        \end{itemize}
    \end{itemize}

    \item Kolobarji
    \begin{itemize}
        \item \colorbox{purple!30}{\textbf{Definicija.}} Kolobar. Komutativen kolobar. Kolobar z enoto.
        \item \colorbox{blue!30}{\textbf{Trditev.}} 2 lastnosti kolobarja.
        \begin{itemize}
            \item \colorbox{green!30}{\textbf{Dokaz.}} Množenje z $0$: Distributivnost.
            
            Množenje z obratnim elementom: Pokažemo, da je res obratni element.
        \end{itemize}
        \item \colorbox{yellow!30}{\emph{Primer.}}  Ugotovi ali je kolobar. Ali je komativen in ima enoto?
        \begin{itemize}
            \item  $(\ZZ, +, \cdot), \  (\QQ, +, \cdot), \ (\RR, +, \cdot), \ (\text{soda števila}, +, \cdot)$.
            \item Definicija množenja v $\ZZ_n$. $(\ZZ_n, +, \cdot)$. 
            \item Definiramo množenje v $\RR^3$ s predpisom $(x_1, y_1, z_1) \cdot (x_2, y_2, z_2) = (x_1 x_2, x_1y_2 + y_1 z_1, z_1 z_2)$. $(\RR^3, +, \cdot)$.
            \item Naj bo $\Ff$ množica vseh funkcij $f: \RR \to \RR$. Na $\Ff$ definiramo seštevanje in množenje po točkah: $(f~+~g)(x)=f(x)+g(x), \ (f \cdot g)(x) = f(x) \cdot g(x)$. Ali je $(\Ff, +, \cdot)$ kolobar?
            \item Množica polinomov za običajno seštevanje in množenje polinomov.
            \item Naj bo $A$ Abelova grupa in $\End (A)$ množica vseh endomorfizmov $A \to A$. Na $\End (A)$ definiramo seštevanje po točkah. Ali je $(\End (A), +, \circ)$ kolobar?
        \end{itemize}
        \item \colorbox{purple!30}{\textbf{Definicija.}} Delitelja niča. Levi/desni delitelj niča.
        \item \colorbox{yellow!30}{\emph{Primer.}} Poišči deljitelja niča v $\ZZ_6$.
        \item \colorbox{purple!30}{\textbf{Definicija.}} Obseg. Polje.
        \item \colorbox{yellow!30}{\emph{Opomba.}} Karakterizacija obsega (kadar je kolobar obseg?). Ali treba predpostaviti, da $1 \neq 0$?
        \item \colorbox{yellow!30}{\emph{Opomba.}} Ali lahko definiramo deljenje v polju? Ali lahko tudi definiramo deljenje v obsegu?
        \item \colorbox{yellow!30}{\emph{Primer.}} Številski obsegi: $\QQ, \ \RR, \ \CC$.
        \item \colorbox{blue!30}{\textbf{Trditev.}} Ali (levi ali desni) delitelj niča lahko bi bil obrnljiv?
        \begin{itemize}
            \item \colorbox{green!30}{\textbf{Dokaz.}} Enostavno s protislovjem.
        \end{itemize}
        \item \colorbox{orange!30}{\textbf{Posledica.}} Ale je v obsegu delitelji niča?
        \item \colorbox{blue!30}{\textbf{Trditev.}} Kadar je $(\ZZ_n, +, \cdot)$ obseg?
        \begin{itemize}
            \item \colorbox{green!30}{\textbf{Dokaz.}} $(\Rightarrow)$ S protislovjem pokažemo, da ima obseg delitelje niča.
            
            $(\Leftarrow)$ Naj bo $a \in \ZZ_n \setminus \set{0}$. Iščemo $a^{-1}$. Izmed števil $1 \cdot a, 2 \cdot a, \ldots, (n-1) \cdot a$ najdemo tisto, ki je enako $1$ (ali se lahko kakšen ostanek ponovi?). 
        \end{itemize}
        \item \colorbox{purple!30}{\textbf{Definicija.}} Podkolobar. 
        \item \colorbox{yellow!30}{\emph{Primer.}} $\ZZ$ je podkolobar v $\QQ$. Trivialni kolobar.
        \item \colorbox{purple!30}{\textbf{Definicija.}} Homomorfizem kolobarjev.
        \item \colorbox{purple!30}{\textbf{Definicija.}} Jedro in slika homomorfizma.
        \item \colorbox{blue!30}{\textbf{Trditev}} o slike in jedru homomorfizma (ali lahko neki podkolobarji)?
        \begin{itemize}
            \item \colorbox{green!30}{\textbf{Dokaz.}} Treba pokazati le zaprtost za množenje.
        \end{itemize}
        \item \colorbox{blue!30}{\textbf{Trditev.}} Karakterizacija injektivnosti homomorfizma kolobarjev.
        \begin{itemize}
            \item \colorbox{green!30}{\textbf{Dokaz.}} Homomorfizem kolobarjev je tudi homomorfizem grup.
        \end{itemize}
        \item \colorbox{orange!30}{\textbf{Posledica.}} Naj bo $f: O \to K$ homomorfizem kolobarjev, kjer je $O$ obseg. Kaj lahko povemo o $f$?
        \begin{itemize}
            \item \colorbox{green!30}{\textbf{Dokaz.}} Recimo da $f$ ni ničeln in s protislovjem pokažemo, da je $\ker f = \set{0}$.
        \end{itemize} 
        \item \colorbox{purple!30}{\textbf{Definicija.}} Podobseg. 
        \item \colorbox{yellow!30}{\emph{Primer.}} $\QQ$ je podobseg v $\RR$.
    \end{itemize}
    
    \newpage
    \item Vektorski prostor
    \begin{itemize}
        \item \colorbox{purple!30}{\textbf{Definicija.}} Vektorski prostor nad poljem $\FF$. Operacija množenje s skalarji.
        \item \colorbox{yellow!30}{\emph{Primer.}} Ugotovi ali je vektorski prostor.       
            \begin{itemize}
                \item $\RR^2$ in $\RR^3$ nad $\RR$ za običajne operacije po kompinentah.
                \item $\FF^n$ nad $\FF$ za običajne operacije po kompinentah.
                \item $\CC$ nad $\RR$ za običajno seštevanje in množenje kompleksnih števil z realnimi..
                \item Množica $\mathcal{F}$ vseh funkcij iz $M \neq \emptyset$ v $\RR$ nad $\RR$ za seštevanje funkcij in množenje funkcij s skalarji po točkah.                
                \item Prostor polinomov z realnimi koeficienti nad $\RR$ za običajno seštevanje polinomov in množenje polinomov s števili. \emph{Oznake.} $\RR[x], \ \CC[x], \ \RR[x,y]$.
            \end{itemize}
        \item \colorbox{blue!30}{\textbf{Trditev.}} 4 lastnosti vektorskega prostora.
        \begin{itemize}
            \item \colorbox{green!30}{\textbf{Dokaz.}} (1) - (2). Ustrezna distributivnost iz definiciji.\
            
            (3) Pokažimo, da je inverz. 

            (4) Uporabimo lastnosti polja z predpostavko, da $\alpha \neq 0$.
        \end{itemize}
        \item \colorbox{purple!30}{\textbf{Definicija.}} Vektorski podprostor.
        \item \colorbox{yellow!30}{\emph{Primer.}} Pokaži, da je vektorski podprostori.        
            \begin{itemize}
                \item Trivialni podprostor. Vektorski prostor kot podprostor samega sebe.
                \item  Vse premice skozi izhodišče in vse ravnine skozi izshodišče v $\RR^3$.
                \item  Polinome stopnje največ $n$ v $\RR[x]$.
            \end{itemize}
        \item \colorbox{blue!30}{\textbf{Trditev.}} Ali je vektorski podprostor vektorski prostor?
        \begin{itemize}
            \item \colorbox{green!30}{\textbf{Dokaz.}} Pokažemo, da je $(W, +)$ Abelova grupa ter preverimo ostale predpostavke.
        \end{itemize}
        \item \colorbox{purple!30}{\textbf{Definicija.}} Linearna kombinacija vektorjev $\x{n}$.
        \item \colorbox{blue!30}{\textbf{Trditev.}} Karakterizacija vektorskega podprostora (tvorenje linarnih kombinacij).
        \begin{itemize}
            \item \colorbox{green!30}{\textbf{Dokaz.}} Definicija vektorskega podprostora + indukcija.
        \end{itemize}
        \item \colorbox{orange!30}{\textbf{Posledica.}} Karakterizacija vektorskega podprostora (linearna kombinacija vektorjev $x$ in $y$).
        \begin{itemize}
            \item \colorbox{green!30}{\textbf{Dokaz.}} Definicija vektorskega podprostora.
        \end{itemize}
        \item \colorbox{blue!30}{\textbf{Trditev.}} Ali je presek poljubne družine vektorskih podprostorov vektorski podprostor v $V$?
        \begin{itemize}
            \item \colorbox{green!30}{\textbf{Dokaz.}} Treba dokazati samo zaprtost za množenje s skalarji.
        \end{itemize}
        \item \colorbox{orange!30}{\textbf{Posledica.}} Ali obstaja najmanjši vektorski podprostor v $V$, ki vsebuje množico $M \subseteq V$?
        \begin{itemize}
            \item \colorbox{green!30}{\textbf{Dokaz.}} Pokažemo, da je presek ustrezne družine najmanjši.
        \end{itemize}
        \item \colorbox{purple!30}{\textbf{Definicija.}} Linearna ogrinjača ali linearna lupina. Oznaka.
        \item \colorbox{blue!30}{\textbf{Trditev.}} Opis množice $\Lin M = W$.
        \begin{itemize}
            \item \colorbox{green!30}{\textbf{Dokaz.}} Ena vsebovanost je očitna. Za drugo pokažemo, da je $W \leq V$ ter upoštevamo definicijo $\Lin M$.
        \end{itemize}
        \item \colorbox{yellow!30}{\emph{Primer.}} Določi linearne ogrinjače:        
        \begin{itemize} 
            \item $\Lin \set{v_1}$.
            $\Lin \set{v_1, v_2}$.            
            Naštej vektorski podprostori v $\RR^3$.
        \end{itemize}
        \item \colorbox{yellow!30}{\emph{Opomba.}} Ali je unija vektorskih podprostorov vektorski podprostor?
        \item \colorbox{purple!30}{\textbf{Definicija.}} Vsota podprostorov. Oznaka.
        \item \colorbox{blue!30}{\textbf{Trditev.}} Opis vsote podprostorov (kaj vsebuje ta množica?).
        \begin{itemize}
            \item \colorbox{green!30}{\textbf{Dokaz.}} Definicija vsote podprostorov + definicija in opis linearne ogrinjače.
        \end{itemize}
        \item \colorbox{purple!30}{\textbf{Definicija.}} Direktna vsota podprostorov. Oznaka.
        \item \colorbox{blue!30}{\textbf{Izrek.}} Karakterizacija direktne vsote podprostorov.
        \begin{itemize}
            \item \colorbox{green!30}{\textbf{Dokaz.}} $(\Rightarrow)$ Vzamemo vektor $x$ iz preseka in pokažemo z definicijo direktne vsote, da je $x=0$.
            
            $(\Leftarrow)$ Recimo, da $x_1 + \ldots + x_k = y_1 + \ldots + y_k$. Vzemimo poljuben $j$ in zapišemo zadnjo enakost kot $x_j - y_j = (y_1 - x_1) + \ldots + (y_{j-1} - x_{j-1}) +  (y_{j+1} - x_{j+1}) + \ldots + (y_k - x_k)$. 
        \end{itemize}
        \item \colorbox{orange!30}{\textbf{Posledica.}} Kadar je vsota podprostorov $W_1$ in $W_2$ direktna?
        \item \colorbox{yellow!30}{\emph{Primer.}} Kaj je vsota poljubne premice skozi izhodišče in poljubne ravnine skozi izhodišče, tako da premica ne leži na ravnine? Ali je ta vsota direktna? Ali je vsota dveh ravnin skozi izhodišče direktna?
    \end{itemize}

    \newpage
    \item Linearne preslikave
    \begin{itemize}
        \item \colorbox{purple!30}{\textbf{Definicija.}} Linearna preslikava. Oznaka. Oznaka za množico vseh linearnih preslikav iz $U$ v $V$. 
        \item \colorbox{purple!30}{\textbf{Definicija.}} Endomorfizem prostora $U$. Oznaka za množico vseh endomorfizmov prostora.
        \item \colorbox{purple!30}{\textbf{Definicija.}} Linearen funkcional. Oznaka za množico vseh linearnih funkcionalov.
        \item \colorbox{yellow!30}{\emph{Primer.}} Ugotovi ali je preslikava linearna:   
        \begin{itemize}
            \item Ničelna preslikava.
            \item Identiteta. Ali je endomorfizem?
            \item Projekcija na $xy$-ravnino: $\mathcal{A}: \RR^3 \to \RR^2, \ \mathcal{A} \, (x_1, x_2, x_3) := (x_1, x_2)$.
            \item Fiksiramo $\vec{a} \in \RR^3$: $\mathcal{A}: \RR^3 \to \RR^3, \ \mathcal{A} \, \vec{x} := \vec{x} \times \vec{a}$.            
            \item Naj bo $\Ff$ prostor vseh integrabilnih funkcij, definiranih na $[a,b]$. $\Aa: \Ff \to \RR, \ \Aa \, f := \int_{a}^{b} f(x) \,dx $. Ali je linearen funkcional?            
            \item Naj bo $\RR[x]$ prostor vseh polinomov. $\Aa: \RR[x] \to \RR[x], \ \Aa \, f := f'$.
        \end{itemize}
        \item \colorbox{blue!30}{\textbf{Trditev.}} Karakterizacija linearnosti preslikave (3 ekvivalentne trditve).
        \begin{itemize}
            \item \colorbox{green!30}{\textbf{Dokaz.}} $(2) \Rightarrow (3)$ in $(3) \Rightarrow (1)$ sta očitni. $(1) \Rightarrow (2)$ z indukcijo na $n$.
        \end{itemize}
        \item \colorbox{purple!30}{\textbf{Definicija.}} Jedro in slika linearne preslikave. Oznaki.
        \item \colorbox{yellow!30}{\emph{Opomba.}} Ali je linearna preslikava homomorfizem Abelovih grup? Kaj to pove o $\ker \mathcal{A}$ in $\im \mathcal{A}$?
        \item \colorbox{blue!30}{\textbf{Trditev.}} Ali je jedro in slika linearne preslikave vektorski podprostori?
        \begin{itemize}
            \item \colorbox{green!30}{\textbf{Dokaz.}} Zaprtost za seštevanje sledi iz prejšnje opombe. Enostavno pokažemo še zaprtost za množenje s skalarji.
        \end{itemize}
        \item \colorbox{blue!30}{\textbf{Trditev.}} Kadar je linearna preslikava injektivna?
        \begin{itemize}
            \item \colorbox{green!30}{\textbf{Dokaz.}} Ker je linearna preslikava homomorfizem Abelovih grup, trditev sledi.
        \end{itemize}
        \item Definicija seštevanja in množenja s skalarji na množice $L(U, V)$.
        \item \colorbox{blue!30}{\textbf{Trditev.}} Linearnost preslikav $\Aa + \Bb$, $\alpha \Aa$.
        \begin{itemize}
            \item \colorbox{green!30}{\textbf{Dokaz.}} Po definiciji ali s karakterizacijo linearnosti.
        \end{itemize}
        \item \colorbox{blue!30}{\textbf{Trditev.}} Ali je $L(U, V)$ vektorski prostor nad $\FF$ za zgoraj definirani operaciji?
        \begin{itemize}
            \item \colorbox{green!30}{\textbf{Dokaz.}} Najprej pokažemo, da je množica vseh homomorfizmov Abelovih grup $(U, +)$, $(V,+)$ je Abelova (isto kot za endomorfizme v poglavju o kolobarjih) in grupa $L(U, V)$ podgrupa tej grupe. Aksiome pa preverimo z računom.
        \end{itemize}
        \item \colorbox{blue!30}{\textbf{Trditev}} o kompozicije linearnih preslikav in inverze bijektivne linearne preslikave.
        \begin{itemize}
            \item \colorbox{green!30}{\textbf{Dokaz.}} Aditivnost sledi iz tega, da so linearne preslikave tudi homomorfizmi Abelovih grup. Homogenost kompozituma pokažemo enostavno, inverza pa z uporabo $\Aa^{-1}$ na izrazu za $\Aa (\alpha y)$, kjer $y = \Aa^{-1}x$.
        \end{itemize}
        \item \colorbox{purple!30}{\textbf{Definicija.}} Izomorfizem vektorskih prostorov. Izomorfni vektorski prostori. Oznaka. 
        \item \colorbox{yellow!30}{\emph{Opomba.}} Ali je izomorfnost vektorskih prostorov ekvivalenčna relacija?
        \item \colorbox{blue!30}{\textbf{Trditev.}} Ali je $(L(V), +, \circ)$ kolobar? Ali ima enoto?
        \begin{itemize}
            \item \colorbox{green!30}{\textbf{Dokaz.}} Preveriti treba distributivnost.
        \end{itemize}
    \end{itemize}

    \item Algebra
    \begin{itemize}
        \item \colorbox{purple!30}{\textbf{Definicija.}} Algebra nad poljem $\FF$. Komutativna algebra. Algebra z enoto.
        \item \colorbox{yellow!30}{\emph{Primer.}} Ugotovi ali je algebra in ali je komutativna in ima enoto:
        \begin{itemize}
            \item $\FF$ nad $\FF$, če množenje s skalarji definiramo kot običajno množenje v $\FF$.
            \item $\CC$ nad $\RR$.            
            \item Naj bo $M$ neprazna množica in $\Ff$ množica vseh funkcij iz $M$ v $\RR$. Na $\Ff$ vse operacije definiramo po točkah. Ali je $\Ff$ algebra nad $\RR$?            
            \item $L(V)$ nad $\FF$ za seštevanje in množenje s skalarji po točkah in kompozitum.
        \end{itemize}
    \end{itemize}
\end{enumerate}

\newpage
\section{KONČNORAZSEŽNI VEKTORSKI PROSTORI}

\begin{enumerate}

    \item Baza in razsežnost
    \begin{itemize}
        \item \colorbox{purple!30}{\textbf{Definicija.}} Ogrodje.
        \item \colorbox{yellow!30}{\emph{Primer.}}       
        \begin{itemize}
            \item Vektorji $e_j$. Ogrodje prostora $\FF^n$.
            \item Ogrodje prostora $\RR[x]$.
            \item Ali je prejšnje ogrogje ogrodje za prostor vseh konvergentnih potenčnih vrst?
        \end{itemize}
        \item \colorbox{purple!30}{\textbf{Definicija.}} Končnorazsežen vektorski prostor.
        \item \colorbox{yellow!30}{\emph{Primer.}}       
        \begin{itemize}
            \item Ali je $\FF^n$ končnorazsežen?
            \item Ali je $\RR[x]$ končnorazsežen?
            \item Ali je prostor polinomov stopnje največ $n$ končnorazsežen?
            \item Ali je prostor vseh funkcij $\RR \to \RR$ končnorazsežen?
        \end{itemize}
        \item \colorbox{blue!30}{\textbf{Trditev.}} Kako dobimo čim manjše ogrodje?
        \begin{itemize}
            \item \colorbox{green!30}{\textbf{Dokaz.}} Pokažemo, da lahko vsak vektor iz $V$ zapišemo kot linearno kombinacijo elementov iz manšjega ogrodja.
        \end{itemize}
        \item \colorbox{purple!30}{\textbf{Definicija.}} Linearno neodvisni vektorji (za končno množico vektorjev). Linearno odvisni vektorji.
        \item \colorbox{yellow!30}{\emph{Opomba.}} Kdaj je neskončna množica vektorjev linearno neodvisna?
        \item \colorbox{yellow!30}{\emph{Primer.}} Ugotovi, ali so vektorji linearno neodvisni:
        \begin{itemize}
            \item $(1,1,-1), \ (1,1,2)$ in $(1, 1, -2)$.
            \item $e_1, e_2, \ldots, e_n \in \FF^n$.
            \item Funkciji $f(x) = e^x$ in $g(x) = e^{2x}$. $f(x) = \cos^2 x, \ g(x) = \sin^2 x, \ g(x) = 1$.
        \end{itemize}
        \item \colorbox{yellow!30}{\emph{Opomba.}} Ali definicija linearne neodvisnosti v $\RR^3$ se ujema z zgornjo definicijo?
        \item \colorbox{blue!30}{\textbf{Trditev.}} Karakterizacija linearne neodvisnosti vektorjev (linearna kombinacija).
        \begin{itemize}
            \item \colorbox{green!30}{\textbf{Dokaz.}} Definicija linearne neodvisnosti.
        \end{itemize}
        \item \colorbox{blue!30}{\textbf{Trditev}} o linearno odvisnih vektorjih (ali lahko vsaj en izmed teh vektorjev izrazimo z ostali?).
        \begin{itemize}
            \item \colorbox{green!30}{\textbf{Dokaz.}} Definicija linearne neodvisnosti.
        \end{itemize}
        \item \colorbox{purple!30}{\textbf{Definicija.}} Baza vektorskega prostora.
        \item \colorbox{yellow!30}{\emph{Primer.}}       
        \begin{itemize}
            \item Baza v $\RR^3$.
            \textbf{Standardna baza prostora $\FF^n$.}
            \item Baza v $\RR[x]$.
        \end{itemize}
        \item \colorbox{yellow!30}{\emph{Opomba.}} Ali baza vedno obstaja?
        \item \colorbox{blue!30}{\textbf{Izrek.}} Ali ima vsak netrivialen končnorazsežen vektorski prostor bazo?
        \begin{itemize}
            \item \colorbox{green!30}{\textbf{Dokaz.}} Izberimo kakšno končno ogrodje. Od leve proti desne zaporedoma odstranjujemo vektorje, ki so linearne kombinacije prejšnjih.
        \end{itemize}
        \item \colorbox{yellow!30}{\emph{Opomba.}} Ali ima vsak neskončnorazsežen vektorski prostor bazo?
        \item \colorbox{blue!30}{\textbf{Izrek.}} Karakterizacija baze.
        \begin{itemize}
            \item \colorbox{green!30}{\textbf{Dokaz.}} Definicija baze in linearne neodvisnosti.
        \end{itemize}
        \item \colorbox{blue!30}{\textbf{Trditev}} o moči vsake linearne neodvisne podmnožice vektorskega prostora. 
        \begin{itemize}
            \item \colorbox{green!30}{\textbf{Dokaz.}} Dokažemo z nadomeščanjem vektorjev v ogrodju: v ogrodje dobavimo vektor iz linearno neodvisne podmnožice. Spet dobimo ogrodje, ki je linearno odvisno. Torej obstaja vektor, ki je linearna kombinacija prejšnih. Ga odstranimo.
        \end{itemize}
        \item \colorbox{orange!30}{\textbf{Posledica.}} Ali imajo vse baze končnorazsežnega vektorskega prostora isto moč?
        \begin{itemize}
            \item \colorbox{green!30}{\textbf{Dokaz.}} Definicija baze + prejšna trditev.
        \end{itemize}
        \item \colorbox{purple!30}{\textbf{Definicija.}} Razsežnost ali dimenzija končnorazsežnega vektorskega prostora. Oznaka.
        \item \colorbox{yellow!30}{\emph{Opomba.}} Ali je definicija dobra?
        
        \item \colorbox{blue!30}{\textbf{Trditev.}} Ali lahko vsako linearno neodvisno podmnožico vektorskega prostora dopolnimo do baze?
        \begin{itemize}
            \item \colorbox{green!30}{\textbf{Dokaz.}} Izberimo kakšno ogrodje in kot prej z nadomeščanjem vektorjev iz linearno neodvisne podmnožice pridemo do baze.
        \end{itemize}
        \item \colorbox{yellow!30}{\emph{Primer.}} Dopolni vektorja $(1,1,1)$ in $(0,1,1)$ do baze prostora $\RR^3$.     
        
        \newpage
        \item \colorbox{orange!30}{\textbf{Posledica.}} (1) Naj bo $\dim V = n$ in $\vecs{v}{n}$ so linearno neodvisni. Ali je potem tvorijo bazo prostora $V$? (2) Naj bo $W \leq V$, $V$ končnorazsežen in $W \neq V$. Kaj lahko povemo o $\dim W$ in $\dim V$?        
        \begin{itemize}
            \item \colorbox{green!30}{\textbf{Dokaz.}} (1) Dopolnimo $\vecs{v}{n}$ do baze.
            
            (2) Dopolnimo bazo prostora $W$ do baze prosotra $V$.
        \end{itemize}
        \item \colorbox{blue!30}{\textbf{Trditev.}} Naj bosta $U$ in $W$ podprostora prostora $V$. Čemu je enaka $\dim (U+W)$?
        \begin{itemize}
            \item \colorbox{green!30}{\textbf{Dokaz.}} Dopolnimo bazo prostora $U \cap W, \ B_{U \cap W} = \set{v_1, \ldots, v_k}$ do baz prostora $U$,
            
            $B_U = \set{v_1, \ldots, v_k, u_1, \ldots, u_n}$ in $W, \ B_W = \set{v_1, \ldots, v_k, w_1, \ldots, w_m}$. Pokažemo, da 
            
            $B = \set{v_1, \ldots, v_k, u_1, \ldots, u_n, w_1, \ldots, w_m}$ baza prostora $U+W$ (zakaj to dovolj?).
        \end{itemize}
        \item \colorbox{orange!30}{\textbf{Posledica.}} Čemu je enaka $\dim (U \oplus  W)$?
        \begin{itemize}
            \item \colorbox{green!30}{\textbf{Dokaz.}} Čemu je enak presek dveh prostorov, če je vsota direktna?
        \end{itemize}
        \item \colorbox{blue!30}{\textbf{Trditev.}} Direktni komplement $U$ podprostora $W$ prostora $V$.
        \begin{itemize}
            \item \colorbox{green!30}{\textbf{Dokaz.}} Če je $W = \set{0}$ ali $W = V$, potem enostavno dobimo $U$.
            
            Če $W \neq V$ in $W \neq \set{0}$, potem ima bazo. Dopolnimo jo do baze prostora $V$. Definiramo $U$ kot linearno ogrinjačo dodatnih vektorjev in po definiciji pokažemo, da $V = W \oplus U$.
        \end{itemize}
        \item \colorbox{blue!30}{\textbf{Trditev.}} Naj bo $\Aa: U \to V$ lin. preslikava. Kaj lahko povemo, če je $\Aa$ injektivna/surjektivna/bijektivna?
        \begin{itemize}
            \item \colorbox{green!30}{\textbf{Dokaz.}} (1) Po definiciji linearne neodvisnosti z uporabo definiciji linearne preslikave ter lastnosti injektivne linearne preslikave.
            
            (2) Definicija in opis ogrodja + definicija surjektivne preslikave.
        \end{itemize}
        \item \colorbox{blue!30}{\textbf{Izrek.}} (1) Izomorfnost $V$, kjer $\dim V = n$ in $\FF^n$. 
        
        (2) Karakterizacija izomorfnosti končnorazsežnih vektorskih prostorov.        
        \begin{itemize}
            \item \colorbox{green!30}{\textbf{Dokaz.}} (1) Pokažemo, da je preslikava $\Phi: \ \FF^n \mapsto V, \ \Phi(\alpha_1, \ldots, \alpha_n) = \alpha_1 v_1 + \ldots + \alpha_n v_n$, kjer so $v_1, \ldots, v_n$ bazni vektorji, izomorfizem.
            
            (2) Uporabimo prejšnjo trditev + točko (1).
        \end{itemize}
        \item \colorbox{blue!30}{\textbf{Izrek.}} Naj bo $\mathcal{A}: \ U \mapsto V$ linearna preslikava. Dimenzijska enačba.
        \begin{itemize}
            \item \colorbox{green!30}{\textbf{Dokaz.}} Izberimo bazi prostorov $\ker \mathcal{A}$ in $\im \mathcal{A}$. Ker vektorji, ki pripadajo baze prostora $\im \mathcal{A}$, ležijo v $\im \mathcal{A}$, obstajajo praslike teh vektorjev, ki ležijo v $U$. Pokažemo, da množica, ki vsebuje bazne vektorje jedra in praslike baznih vektorjev slike, sestavljajo bazo prostora $U$.
            
            Linearna neodvisnost: definicija jedra, linearne neodvisnosti ter linearne preslikave.

            Ogrodje: definicija slike, jedra baze ter linearne preslikave.
        \end{itemize}
        \item \colorbox{orange!30}{\textbf{Posledica.}} Karakterizacija injektivnosti/surjektivnosti/bijektivnosti endomorfizma $\Aa \in L(U)$.
        \begin{itemize}
            \item \colorbox{green!30}{\textbf{Dokaz.}} Uporabimo prejšne trditve.
        \end{itemize}
        \item \colorbox{purple!30}{\textbf{Definicija.}} Rang preslikave $\Aa$. Oznaka.
    \end{itemize}

    \item Linearne preslikave in matrike
    \begin{itemize}
        \item \colorbox{purple!30}{\textbf{Definicija.}} Matrika reda $m \times n$ nad poljem $\FF$. Členi matrike. 
        \item Običajne oznake (matrika, členi matrike, krajša oblika za zapis matrike z splošnimi členi, vrstica matrike, stolpec matrike, množica vseh matrik reda $m \times n$ nad poljem $\FF$).
        \item \colorbox{purple!30}{\textbf{Definicija.}} Vsota matrik. Množenje matrik s skalarji.
        \item \colorbox{blue!30}{\textbf{Trditev.}} Ali je $\FF^{m \times n}$ vektorski prostor nad $\FF$ za prej definirane operacije? \textbf{Ničelna matrika.}
        \begin{itemize}
            \item \colorbox{green!30}{\textbf{Dokaz.}} Enostavno preverimo aksiome.
        \end{itemize}
    \end{itemize}

    \item[$\circ$] Kako so matrike povezane z linearnimi preslikavami?      
    \begin{itemize}          
        \item \colorbox{blue!30}{\textbf{Trditev.}} Naj bo $V$ in $U$ vektorska prostora nad istem poljem $\FF$, $\Aa: V \to U$ linearna preslikava. S čim je preslikava $\Aa$ že enolično določena?
        \begin{itemize}
            \item \colorbox{green!30}{\textbf{Dokaz.}} Najprej pokažemo, da sploh lahko definiramo preslikavo. Nato pa pokažemo še enoličnost.
        \end{itemize}
        \item \colorbox{purple!30}{\textbf{Definicija.}} Matrika, prirejena preslikave. Oznaka.
        \item \colorbox{yellow!30}{\emph{Opomba.}} Od česa je odvisna matrika iz prejšne definiciji? Poudaren zapis.
        \item \colorbox{yellow!30}{\emph{Primer.}} Naj bo $V$ prostor polinomov stopnje največ 2, $W$ prostor polinomov stopnje največ $1$ in $\Aa: \ V \mapsto W$ odvajanje. Določi matriko preslikave, glede na baze $\Bb_V = \set{1, x, x^2}$ in $\Bb_W = \set{1, x}$.
        \item \colorbox{yellow!30}{\emph{Opomba.}} Izbrani bazi sta "`urejeni"'. Kaj se zgodi, če zamenjamo vrstni red vektorjev v kakšni bazi?
        \item \colorbox{yellow!30}{\emph{Primer.}} Naj bosta $V$ vektorski prostor, $B_v$ baza prostora $V$. Določi matriko za $\id: V \to V$ glede na dano bazo. 
        \item \colorbox{purple!30}{\textbf{Definicija.}} \textbf{Identična matrika (identiteta).} Oznaka.
        \item \colorbox{purple!30}{\textbf{Definicija.}} Diagonalna matrika. Zapis.
        \item \colorbox{purple!30}{\textbf{Definicija.}} Zgornje trikotna matrika. Spodnje trikotna matrika.
        \item \colorbox{purple!30}{\textbf{Definicija.}} Bločno zgornje trikotna matrika. Bločno spodnje trikotna matrika. Bločno diagonalna matrika.
        
        \newpage
        \item \colorbox{blue!30}{\textbf{Trditev.}} Naj bosta $V$ in $W$ vektorska prostora z bazama $B_v$ in $B_w$. Kaj lahko povemo o preslikavi \begin{align*}
            \Phi: \Lin(V, W) &\to \FF^{m \times n} \\
            \Aa &\mapsto \Aa_{B_w, B_v}?
        \end{align*}
        \begin{itemize}
            \item \colorbox{green!30}{\textbf{Dokaz.}} Injektivnost pokažemo po definiciji z uporabo prejšnje trditve. Za surjektivnost definiramo $\Aa (x) = \sum_{i}^{n} \alpha_i \sum_{j}^{m} a_{ji}w_j$, kjer so $w_j$ bazni vektrorji prostora $W$, $\alpha_i$ pa so koeficienti pri razvoju vektorja $x$ po bazi prostora $V$.
        \end{itemize}
        \item \colorbox{blue!30}{\textbf{Trditev.}} Matrike znamo seštevati in množiti s skalarji. Linearne preslikave tudi znamo seštevati in množiti s skalarji. Kako so operacije povezane?
        \begin{itemize}
            \item \colorbox{green!30}{\textbf{Dokaz.}} Poračunamo.
        \end{itemize}
        \item \colorbox{orange!30}{\textbf{Posledica.}}  Kaj trditev pove o preslikave $\Phi: \Lin(V, W) \to \FF^{m \times n}$? Kaj to pomeni?
        \item \colorbox{purple!30}{\textbf{Definicija.}} Elementarne matrike (matrične enote).
        \item \colorbox{blue!30}{\textbf{Trditev.}} (1) Čemu je enaka dimenzija prostora $\FF^{m \times n}$? 
                
        (2) Naj bosta $\dim V = n$ in $\dim W = m$. Čemu je enaka dimenzija prostora $L(V, W)$?
        \begin{itemize}
            \item \colorbox{green!30}{\textbf{Dokaz.}} Po definiciji dokažemo, da je $\set{E_{ij}; \ 1 \leq i \leq m, \ 1 \leq j \leq n}$ baza prostora $F^{m \times n}$.
        \end{itemize}
        \item \colorbox{yellow!30}{\emph{Posebna primera.}} 
        \begin{itemize}
            \item $m=1$. Izomorfnost $L(V, \FF)$ in $V$.
            \item $n=1$. Izomorfnost $L(\FF, W)$ in $W$. Definicija danega izomorfizma brez izbire baz. Prostor stolpcev. Standardna baza prostora stolpcev. 
        \end{itemize}
    \end{itemize}

    \item[$\circ$] Množenje matrik
    
    Linearne preslikave komponiramo in kompozitum je spet linearna preslikava. Radi bi definirali množenje matrik tako, da bo usrezalo kompozitumu preslikav. Poseben preimer množenja matrik bo množenje matrike s stolpcem:
    \begin{itemize}
        \item \colorbox{blue!30}{\textbf{Izpeljava.}} Smiselna definicija produkta matrike in stolpca (vektroja). 
        \item \colorbox{purple!30}{\textbf{Definicija.}} Produkt matrike $A \in \FF^{m \times n}$ in vektorja $x \in \FF^n$.
        \item \colorbox{yellow!30}{\emph{Opomba.}} Kadar lahko pomnožimo matriko z vektorjem? Kakšne velikosti dobimo matriko, če pomnožimo matriko z vektorjem? Kaj dobimo na $i$-tem komponentu produkta?
        \item \colorbox{yellow!30}{\emph{Opomba.}} Ali je za vsako matriko $A \in \FF^{m \times n}$ preslikava \begin{align*}
            A: \FF^n &\to \FF^m \\
            x &\mapsto Ax 
        \end{align*}
        linearna? Kakšna matrika pripada tej preslikave glede na standardne bazi prostorov $\FF^n$ in $\FF^m$? Čemu v tem primeru ustreza množenje matrike $A$ z vektorjem $x$?
        \item Oznaka za stolpec koeficientov iz razvoja vektorja $x \in V$, po bazi $B_v$.
        \item \colorbox{blue!30}{\textbf{Trditev.}} Naj bo $\Aa$ linearna preslikava med vektorskoma prostoroma $V$ in $W$. Ali izračun preslikave v vektorju $x$ ustreza množenju matrike preslikave glede na izbrani bazi prostorov s vektorjem $x$?
        \begin{itemize}
            \item \colorbox{green!30}{\textbf{Dokaz.}} Izberimo bazi prostora in uvedemo oznako za matriko preslikave, nato poračunamo kot pri izpeljave definiciji za množenje.
        \end{itemize}
        \item \colorbox{yellow!30}{\emph{Primer.}} Naj bo $V$ prostor polinomov stopnje največ 2, $W$ prostor polinomov stopnje največ $1$ in $\Aa: \ V \mapsto W$ odvajanje. Izberimo običajne baze $\Bb_V = \set{1, x, x^2}$ in $\Bb_W = \set{1, x}$. Preveri prejšno trditev s polinimom $p(x) = ax^2 + bx + c$.
        \item \colorbox{orange!30}{\textbf{Posledica.}} Naj bo $V$ in $W$ vektorska prostora z bazama $B_v = \set{v_1, \ldots, v_n}$ in $B_w = \set{w_1, \ldots, w_m}$. Naj bosta $\phi_n: \FF^n \to V$ in $\phi_m: \FF^m \to W$ izomorfizma, definirama s prespisoma $\phi_n \begin{bmatrix}
            x_1 \\ \vdots \\ x_n
        \end{bmatrix} = x_1 v_1 + x_2 v_2 + \ldots + x_n v_n$ in $\phi_m \begin{bmatrix}
            y_1 \\ \vdots \\ y_y
        \end{bmatrix} = y_1 w_1 + y_2 w_2 + \ldots + y_m w_m$. Naj bo $\Aa: V \to W$ linearna preslikava in $A = \Aa_{B_w, B_v}$. $A$ gledamo kot linearno preslikavo $A: \FF^n \to \FF^m$, $x \mapsto Ax$. Dokaži, da diagram     
            \xymatrix{
                V \ar@{->}[r]^{\mathcal{A}} & W \\
                \mathbb{F}^n \ar@{->}[u]^{\phi_n} \ar@{->}[r]_{A} & \mathbb{F}^m \ar@{->}[u]_{\phi_m}
            }  komutira.  
        \begin{itemize}
            \item \colorbox{green!30}{\textbf{Dokaz.}} Pokažemo, da je $\phi_m \circ A = \Aa \circ \phi_n$. Kako dobimo vektorji $x \in V$ in $\Aa x \in W$ z izomorfizmoma $\phi_n$ in $\phi_m$?
        \end{itemize}        
    \end{itemize}    
    Naj bosta $A, B$ matriki, ki ju obravnavamo kot linearni preslikavi med prostoroma stolpcev. Radi bi definirali produkt matrik $A \cdot B$, ki bo usrezal kompozitumu preslikav $A \circ B$:
    \begin{itemize}
        \item Kdaj lahko definiramo produkt? Izpeljava (kam preslika matrika vektorji standardne baze?). Kaj je element $c_{ik}$ v produktu matrik?
        \item \colorbox{purple!30}{\textbf{Definicija.}} Produkt matrik $A \in \FF^{m \times n}$ in $B \in \FF^{n \times p}$.
        \item \colorbox{yellow!30}{\emph{Primer.}} Izračunaj produkti $AB, \ BA, \ CD, \ DC$, če obstajajo. Ali je produkt matrik komutativen?
        
        $A = \begin{bmatrix}
            1 & -1 \\
            2 & 3 \\
            0 & -4
        \end{bmatrix}, \ B = \begin{bmatrix}
            1 & -1 & 2 \\
            -3 & 1 & 0 \\
            -1 & 1 & 2
        \end{bmatrix}, \ C = \begin{bmatrix}
            2 & 1 \\
            -1 & 3
        \end{bmatrix}, \ D = \begin{bmatrix}
            0 & 1 \\
            0 & -2
        \end{bmatrix}$.
        \item \colorbox{blue!30}{\textbf{Trditev.}} Kako med sabo povezane kompozitum preslikav in množenje matrik?
        \begin{itemize}
            \item \colorbox{green!30}{\textbf{Dokaz.}} Izračunamo matriko preslikave $\Aa \circ B$ glede na izbrani bazi.
        \end{itemize}  
        \item \colorbox{orange!30}{\textbf{Posledica.}} Ali je množenje matrik asociativno?
        \begin{itemize}
            \item \colorbox{green!30}{\textbf{Dokaz.}} Matrike $A, B, C$ identificiramo s preslikavi med prostorama stolpcev.
        \end{itemize}  
        \item \colorbox{orange!30}{\textbf{Posledica.}} Enota za množenje matrik.
        \begin{itemize}
            \item \colorbox{green!30}{\textbf{Dokaz.}} Pokažemo, da $I_n \in \FF^{n \times n}$ enota za množenje.
        \end{itemize}  
        \item \colorbox{orange!30}{\textbf{Posledica.}} (1) Ali je $\FF^{n \times n}$ algebra?
        
        (2) Naj bo $V$ $n$-razsežen vektorski prostor. Izomorfizem algeber $L(V)$ in $\FF^{n \times n}$.
        \begin{itemize}
            \item \colorbox{green!30}{\textbf{Dokaz.}} (1) Nekaj že vemo, ostalo preverimo z računom.
            
            (2) Preverimo še multiplikativnost.
        \end{itemize}  
        \item \colorbox{purple!30}{\textbf{Definicija.}} (1) Kadar pravimo, da je endomorfizem $\Aa \in L(V)$ obrnljiv?
        
        (2) Kadar pravimo, da je kvadratna matrika $A \in \FF^{n \times n}$ obrnljiva? Inverz matrike $A$. Oznaka.
        \item \colorbox{yellow!30}{\emph{Primer.}} Kaj je $\begin{bmatrix}
            a & b \\ c & d
        \end{bmatrix}^{-1}$? 
        \item \colorbox{blue!30}{\textbf{Trditev.}} Ali je za obrnlivost endomorfizma $\Aa \in L(V)$ dovolj, da obstaja $\Bb \in L(V)$, da $\Aa \circ B = \id$?
        \begin{itemize}
            \item \colorbox{green!30}{\textbf{Dokaz.}} Dimenzijska enačba.
        \end{itemize}  
        \item \colorbox{orange!30}{\textbf{Posledica.}} Kam slika izomorfizem $\Phi: L(V) \to \FF^{n \times n}$ obrnljive endomorfizme? Kaj velja za vsak obrnljiv endomorfizem?
        \item \colorbox{orange!30}{\textbf{Posledica.}} Kadar je matrika $A \in \FF^{n \times n}$ obrnljiva?
        \item \colorbox{purple!30}{\textbf{Definicija.}} Transponirana matrika matrike $A$. Kaj je transponiranje?
        \item \colorbox{yellow!30}{\emph{Primer.}} Določi $\begin{bmatrix}
            1 & 2 \\ -1 & 3 \\ 0 & 5
        \end{bmatrix}^T$.
        \item \colorbox{blue!30}{\textbf{Trditev.}} 3 lastnosti transponiranja.
        \begin{itemize}
            \item \colorbox{green!30}{\textbf{Dokaz.}} Račun.
        \end{itemize}
    \end{itemize}

    \item Kvocientne vektorski prostori
    \item[$\circ$] Ponovitev relacij
    \begin{itemize}
        \item \colorbox{purple!30}{\textbf{Definicija.}} Relacija med množicama $A$ in $B$. Elementa v relaciji (oznaka). Relacija na množici $A$.
        \item \colorbox{purple!30}{\textbf{Definicija.}} Refleksivna, simetrična, antisimetrična in tranzitivna relacija.
        \item \colorbox{purple!30}{\textbf{Definicija.}} Ekvivalenčna relacija.
        \item \colorbox{purple!30}{\textbf{Definicija.}} Relacija delne urejenosti.
        \item \colorbox{yellow!30}{\emph{Primer.}} Ali je relacija vsebovanosti na množici vektorskih podprostorov danega vektorskega prostora relacija delne urejenosti?        
        \item \colorbox{purple!30}{\textbf{Definicija.}} Primerljiva in neprimerljiva elementa.
        \item \colorbox{yellow!30}{\emph{Primer.}} Naj bo $A$ množica vektorskih podprostorov prostora $\RR^3$ in $R$ relacija vsebovanosti. Ali je primerljiva $\set{0}$ in $x$-os ter $x$-os in $y$-os?
        \item \colorbox{purple!30}{\textbf{Definicija.}} Linearna urejenost.
        \item \colorbox{purple!30}{\textbf{Definicija.}} Maksimalni in minimalni element. Največji in najmanjši element za relacijo delne urejenosti. Obstoj in zveza med njimi.
        \item \colorbox{yellow!30}{\emph{Primer.}} Naj bo $V$ vektorski prostor in $M \subseteq V$ poljubna množica. Naj bo $A$ množica vseh vektorskih podprostorov prostora $V$, ki vsebujejo $M$. Množico $A$ delno uredimo z inkluzijo. Določi najmanjši element, če obstaja.
    \end{itemize}

    \newpage
    \item[$\circ$] Ekvivalenčna relacija in kvocietntne množice
    \begin{itemize}
        \item \colorbox{purple!30}{\textbf{Definicija.}} Ekvivalenčni razred elementa $a \in A$. Oznaka. Predstavnik ekvivalenčnega razreda.
        \item \colorbox{blue!30}{\textbf{Izrek.}} Na kaj ekvivalenčna relacija razdeli množico $A$? Karakterizacija ekvivalentnosti elementov.
        \begin{itemize}
            \item \colorbox{green!30}{\textbf{Dokaz.}} LMN.
        \end{itemize}
        \item \colorbox{blue!30}{\textbf{Izrek.}} Obrat prejšnega izreka.
        \begin{itemize}
            \item \colorbox{green!30}{\textbf{Dokaz.}} LMN.
        \end{itemize}
        \item \colorbox{purple!30}{\textbf{Definicija.}} Kvocientna ali faktorska množica množice $A$ po relacije $\sim$. Oznaka.
        \item \colorbox{purple!30}{\textbf{Definicija.}} Kvocientna preslikava.
        \item \colorbox{yellow!30}{\emph{Opomba.}} Ali je kvocientna preslikava surjektivna?
        \item \colorbox{yellow!30}{\emph{Primer.}} Določi ekvivalenčne razrede ter pri 1. in 3. primeru kvocientno množico..
        \begin{itemize}
            \item Na $\ZZ \times \NN$ definiramo relacijo $\sim: (m,n) \sim (p, q) \Leftrightarrow mq = np$.
            \item Na množice vseh usmerjenih daljic v $\RR^3$ definiramo, da sta usmerjeni daljici v relaciji $\sim$, kadar eno od njiju dobimo z vzporednim premikom druge. 
            \item Naj bo $n \in \NN$ in $\equiv$ relacija na $\ZZ$, definirana s predpisom $a \equiv b \Leftrightarrow n | a -b$. 
            \item Na $\RR^2$ definiramo relacijo $(x,y) \sim (z,w) \Leftrightarrow y = w$.
        \end{itemize}
        \item \colorbox{blue!30}{\textbf{Izrek.}} Naj bo $f: A \to B$ preslikava. Na $A$ definiramo relacijo $\sim$ s predpisom $x \sim y \Leftrightarrow f(x) = f(y)$.
        \begin{enumerate}
            \item[(1)] Ali je $\sim$ ekvivalenčna relacija?
            \item[(2)] Naj bo $q: A \to A/_\sim$ kvocientna preslikava. Kaj lahko povemo o diagramu \xymatrix{
                A \ar@{->}[rr]^{f} \ar@{->}[rd]_{q} &  & B \\
                 & A/_\sim \ar@{->}[ru]_{p} & 
                }?
            \item[(3)] Ali je $p$ injektivna? Čemu je enaka $Z_p$?            
        \end{enumerate}
        \begin{itemize}
            \item \colorbox{green!30}{\textbf{Dokaz.}} LMN. Glavna stvar izreka je dokaz, da je s predpisom $p([a]) = f(a)$ preslikava $p$ dobro definirana.
        \end{itemize}
    \end{itemize}

    \item[$\circ$] Usklajenost operacije z ekvivalenčno relacijo
    \begin{itemize}
        \item \colorbox{purple!30}{\textbf{Definicija.}} Usklajenost operacije z ekvivalenčno relacijo.
        \item Recimo, da je operacija $\circ$ usklajena z ekvivalenčno relacijo $\sim$. Definicija operacije na množice $A/_\sim$. Ali je definicija dobra?
        \item \colorbox{yellow!30}{\emph{Primer.}}  
        \begin{itemize}
            \item Ali je operaciji seštevanje in množenje racionalnih števil usklajeni z relacijo $\sim$ na množice $\ZZ \times \NN$?
            \item Ali je operacija seštevanje vektorjev usklajena z relacijo $\sim$ na množici vseh vektorjev?
        \end{itemize}
    \end{itemize}
    
   
    \item[$\circ$] Kvocientne grupe Abelovih grup
    
    Naj bo $(G,+)$ Abelova grupa in $H$ njena podgrupa. Na $G$ definiramo relacijo $\sim$ s predpisom $a \sim b \Leftrightarrow a - b \in H$.
    \begin{itemize}
        \item \colorbox{blue!30}{\textbf{Trditev.}} Ali je $\sim$ ekvivalenčna relacija na $G$?
        \begin{itemize}
            \item \colorbox{green!30}{\textbf{Dokaz.}} Z uporabo definicije podrupe preverimo lastnosti ekvivalenčne relacije.
        \end{itemize}
        \item \colorbox{yellow!30}{\emph{Opomba.}} Predpis za ekvivalenčno relacijo na $G$, če je $(G, \cdot)$ poljubna grupa in $H$ njena podgrupa.  
        \item \colorbox{blue!30}{\textbf{Trditev.}} Ali je operacija $+$ na $G$ usklajena z relacijo $\sim$?
        \begin{itemize}
            \item \colorbox{green!30}{\textbf{Dokaz.}} Treba pokazati, da je $a \sim b \text{ in } c \sim d \Rightarrow a + c \sim b + d$.
        \end{itemize}
        \item \colorbox{yellow!30}{\emph{Opomba.}} Naj bo $(G, \cdot)$ poljubna grupa in $H$ njena podgrupa. Ali je množenje iz prejšnje opombe nujno usklajeno z relacijo $\sim$?
        \item \colorbox{purple!30}{\textbf{Definicija.}} Definicija operacije $+$ na množice $G/_\sim$.
        \item \colorbox{blue!30}{\textbf{Trditev.}} Ali je $(G/_\sim, +)$ Abelova? Ali je kvocientna preslikava homomorfizem grup?
        \begin{itemize}
            \item \colorbox{green!30}{\textbf{Dokaz.}} Zakaj operacija dobro definirana? Ostale aksiome preverimo z računom.
        \end{itemize}
        \item \colorbox{purple!30}{\textbf{Definicija.}} Kvocientna ali faktorska grupa grupe $G$ po podgrupi $H$. Oznaki za kvocientno grupo in ekvivalenčni razred.
        \item \colorbox{yellow!30}{\emph{Opomba.}} Zakaj $[a]$ označimo z $a+H$?
        \item \colorbox{yellow!30}{\emph{Primer.}}
        \begin{itemize}
            \item Izomorfnost $G/_{\set{0}}$ in $G$.
            \item Kaj lahko povemo o grupi $G/_G$?
            \item Naj bo $n \in \NN, \ n \neq 1$. Označimo z $n\ZZ = \set{na; \ a \in Z}$. Kaj je $\ZZ/_{n\ZZ}$?
        \end{itemize}
        \item \colorbox{yellow!30}{\emph{Opomba.}} Naj bosta $G, H$ grupi, $G$ Abelova, in naj bo $f: G \to H$ homomorfizem grup. Kaj je relacija $\sim$, definirana s prespisom $x \sim y \Leftrightarrow f(x) = f(y)$? Kaj to pomeni o $G/_\sim$?
        
        \newpage
        \item \colorbox{blue!30}{\textbf{Izrek.}} Naj bosta $G, H$ grupi, $G$ Abelova, in naj bo $f: G \to H$ homomorfizem grup. Kaj lahko povemo o diagramu \xymatrix{
            G \ar@{->}[rr]^{f} \ar@{->}[rd]_{q} &  & H \\
             & G/_{\ker f} \ar@{->}[ru]_{p} & 
            }?
        \begin{itemize}
            \item \colorbox{green!30}{\textbf{Dokaz.}} Vse že vemo iz LMN, dokazati le treba, da je $p$ homomorfizem grup.
        \end{itemize}
    \end{itemize}

    \item[$\circ$] Kvocientni vektorski prostori
    
    Naj bo $V$ vektorski prostor nad $\FF$ in $W \leq V$. $(V,+)$ je Abelova grupa in $(W, +)$ njena podgrupa, zato lahko definiramo ekvivalenčno relacijo $\sim$ s prespisom $x \sim y \Leftrightarrow x-y \in W$, in kvocientno Abelovo grupo $V/_W$. Na $V/_W$ bi radi definirali množenje s skalarji tako, da bo $V/_W$ vektorski prostor.
    \begin{itemize}
        \item \colorbox{blue!30}{\textbf{Trditev.}} Ali je množenje s skalarji na $V$ usklajeno z ekvivalenčno relacijo?
        \begin{itemize}
            \item \colorbox{green!30}{\textbf{Dokaz.}} Treba pokazati, da če $x \sim y \text{ in } \alpha \in \FF \Rightarrow \alpha x \sim \alpha y$.
        \end{itemize}
        \item \colorbox{orange!30}{\textbf{Posledica.}} Ali je na $V/_W$ s predpisom $\alpha [x] = [\alpha x]$ dobro definirano množenje s skalarji?
        \begin{itemize}
            \item \colorbox{green!30}{\textbf{Dokaz.}} Treba pokazati, da če $[x] = [y] \Rightarrow [\alpha x] = [\alpha y]$.
        \end{itemize}
        \item \colorbox{blue!30}{\textbf{Trditev.}} Naj bo $V$ vektorski prostor nad $\FF$ in $W \leq V$. Na $V/_W$ definiramo operaciji s predpisom 
        
        $[x] + [y] = [x+y]$ in $\alpha [x] = [\alpha x]$.
        
        (1) Ali je $V/_W$ vektorski prostor nad $\FF$?

        (2) Ali je kvocientna preslikava linearna?
        \begin{itemize}
            \item \colorbox{green!30}{\textbf{Dokaz.}} (1) Vemo že, da sta operacije dobro definirane in da je $(V/_W, +)$ Abelova grupa. Preverimo še ostale aksiome za vektorski prostor.
            
            (2) Vemo že, da je kvocientna preslikava homomorfizem grup $(V,+)$ in $(V/_W, +)$. Preverimo še homogenost.
        \end{itemize}
        \item \colorbox{purple!30}{\textbf{Definicija.}} Kvocientni ali faktorski vektorski prostor prostora $V$ po podprostoru $W$.
        \item \colorbox{yellow!30}{\emph{Opomba.}} Kaj so ekvivalenčni razredi?  Oznaka. \textbf{Afin podprostor}. 
        \item \colorbox{yellow!30}{\emph{Primer.}} Afini podprostori v $\RR^3$.
        \item \colorbox{blue!30}{\textbf{Izrek.}} Naj bosta $V, \ U$ vektorska prostora nad $\FF$, in naj bo $\Aa: V \to U$ linearna preslikava. Kaj lahko povemo o diagramu \xymatrix{
            V \ar@{->}[rr]^{\Aa} \ar@{->}[rd]_{q} &  & U \\
             & V/_{\ker f} \ar@{->}[ru]_{\overline{\Aa}} & 
            }?
        \begin{itemize}
            \item \colorbox{green!30}{\textbf{Dokaz.}} Vse že vemo iz LMN in izreka v primeru grup, dokazati le treba, da je $\overline{\Aa}$ homogena.
        \end{itemize}
        \item \colorbox{orange!30}{\textbf{Posledica.}} Ali je vektorska prostora $V/_{\ker \Aa}$ in $\im \Aa$ izomorfna?
        \begin{itemize}
            \item \colorbox{green!30}{\textbf{Dokaz.}} $U$ zamenjamo z $\im \Aa$ in uporabimo izrek.
        \end{itemize}
        \colorbox{yellow!30}{\emph{Primer.}} $\Aa: \RR^3 \to \RR^3, \ \Aa(x,y,z) = (x, y, 0)$. Kaj je $\im \Aa$ in $\ker \Aa$? Kaj je ekvivalenčni razredi v $\RR^3/_{\ker \Aa}$? Čemu je izomorfen $\RR^3/_{\ker \Aa}$? Kam slika izomorfizem navpične premice?
        \item \colorbox{orange!30}{\textbf{Posledica.}} Naj bo $V$ končnorazsežen vektorski prostor nad $\FF$. Naj bo $\Aa: \ V \to U$ linearna preslikava. Čemu je enaka $\dim (V/_{\ker \Aa})$?
        \begin{itemize}
            \item \colorbox{green!30}{\textbf{Dokaz.}} Uporabimo prejšnjo posledico in dimenzijsko enačbo.
        \end{itemize}
        \item \colorbox{orange!30}{\textbf{Posledica.}} Naj bo $V = V_1 \oplus V_2$. Kaj lahko povemo o $V/_{V_1}$ in $V/_{V_2}$? \textbf{Projekcija na $V_2$ vzdolž $V_1$}.
        \begin{itemize}
            \item \colorbox{green!30}{\textbf{Dokaz.}} Definiramo preslikavo $P: \ V \to V_2, \ P(v) = P(x_1 + x_2) = x_2$. Pokažemo, da je ta preslikava linearna ter velja $\im P = V_2, \ \ker P = V_1$. Uporabimo izrek in predpjejšnjo posledico.
        \end{itemize}
        \item \colorbox{orange!30}{\textbf{Posledica.}} Naj bo $V$ končnorazsežen vektorski prostor nad $\FF$ in $W \leq V$. Ali je $V/_W$ končnorazsežen? Čemu je enaka $\dim(V/_W)$?
        \begin{itemize}
            \item \colorbox{green!30}{\textbf{Dokaz.}} Uporabimo izrek o obstoju direktnega komplementa, prejšnjo posledico ter dimenzijsko enačbo.
        \end{itemize}
    \end{itemize}
    
    
    \newpage
    \item Prehod na novi bazi
    
    Naj bo $\Aa: V \to W$ linearna preslikava. Matrika $A$, ki pripada preslikave $\Aa$ ni odvisna samo od preslikave~$\Aa$, ampak tudi od baz $B_v$ in $B_w$. Recimo, da si izberimo še drugi dve bazi $B_{v'}$ in $B_{w'}$ in definiramo matriko $A' = \Aa_{B_{w'}, B_{v'}}$. Zanima nas, v kakšni zvezi sta $A$ in $A'$.
    \begin{itemize}
        \item \colorbox{purple!30}{\textbf{Definicija.}} Prehodna matrika med bazama $B_v$ in $B_{v'}$.
        \item \colorbox{yellow!30}{\emph{Opomba.}} Kako dobimo prehodno matriko $P$?
        \item \colorbox{blue!30}{\textbf{Trditev.}} Kateri preslikavi pripada $P$? Ali je $P$ obrnljiva? 
        \begin{itemize}
            \item \colorbox{green!30}{\textbf{Dokaz.}} Uporabimo zvezo med množenjem matrik in kompozitumom preslikav.
        \end{itemize}
        \item \colorbox{yellow!30}{\emph{Opomba.}} Naj bo $V = \FF^n$ in $B_v$ standardna baza. Kaj so stolpci prehodne matrike med bazama $B_v$ in $B_{v'}$?
        \item \colorbox{blue!30}{\textbf{Trditev.}} Naj bo $\Aa: V \to W$ linearna preslikava, $A = \Aa_{B_{w}, B_{v}}$ in $A' = \Aa_{B_{w'}, B_{v'}}$. V kakšni zvezi s prehodnimi matrikami sta $A$ in $A'$?
        \begin{itemize}
            \item \colorbox{green!30}{\textbf{Dokaz.}} Izračunamo $Q^{-1}AP$ z uporabo zveze med kompozitumom preslikav in množenjem matrik.
        \end{itemize}
        \item \colorbox{purple!30}{\textbf{Definicija.}} Ekvivalentni matriki. Oznaka.
        \item \colorbox{yellow!30}{\emph{Opomba.}} Ali sta matriki, ki pripada isti linearni preslikavi ekvivalentni?
        \item \colorbox{blue!30}{\textbf{Trditev.}} Ali je ekvivalentnost matrik ekvivalenčna relacija?
        \begin{itemize}
            \item \colorbox{green!30}{\textbf{Dokaz.}} Enostavno preverimo lastnosti.
        \end{itemize}
        \item \colorbox{blue!30}{\textbf{Trditev.}} Karakterizacija ekvivalentnosti matrik (obstoj linearne preslikave).
        \begin{itemize}
            \item \colorbox{green!30}{\textbf{Dokaz.}} $(\Leftarrow)$ Že vemo.
            
            $(\Rightarrow)$ Definiramo preslikavo $\Aa: \FF^n \to \FF^m$ s predpisom $\Aa x = Ax$. Vemo že, da tej preslikave glede na standardne baze prostorov $S_n$ in $S_m$ pripada matrika $A$. Ker $A \sim A', \ A' = Q^{-1} A P$. Definiramo $B_m = \mathcal{Q}(S_m)$ in $B_n = \mathcal{P}(S_n)$, kjer $\mathcal{Q}: \FF^m \to \FF^m, \ \mathcal{Q}(x) = Qx$ in $\mathcal{P}$ definirana podobno. Pokažemo, da je $B_m$ in $B_n$ bazi in izračunamo $\Aa_{B_m, B_n} = (\id \circ \Aa \circ \id)_{B_m, B_n}$.
        \end{itemize}
        \item \colorbox{purple!30}{\textbf{Definicija.}} Rang, jedro in slika matrike $A \in \FF^{m \times n}$. 
        \item \colorbox{blue!30}{\textbf{Trditev.}} Naj bo $\Aa: V \to W$ linearna preslikava in $A$ matrika te preslikave glede na poljubni bazi. Kakšna je zveza med $\rang \Aa$ in $\rang A$?
        \begin{itemize}
            \item \colorbox{green!30}{\textbf{Dokaz.}} Vemo, da diagram     
            \xymatrix{
                V \ar@{->}[r]^{\mathcal{A}} & W \\
                \mathbb{F}^n \ar@{->}[u]^{\phi_n} \ar@{->}[r]_{A} & \mathbb{F}^m \ar@{->}[u]_{\phi_m}
            }  komutira. S pomočjo njo pokažemo, da je $\im A = \phi^{-1}_m(\im \Aa)$, nato pokažemo, da $\dim \im A = \dim \im \Aa$.
        \end{itemize}
        
        \item \colorbox{orange!30}{\textbf{Posledica.}} V kakšni zvezi so rangi ekvivalentnih matrik?
        \begin{itemize}
            \item \colorbox{green!30}{\textbf{Dokaz.}} Karakterizacija ekvivalentnosti matrik + prejšnja trditev.
        \end{itemize}
        \item \colorbox{orange!30}{\textbf{Posledica.}} Kaj se zgodi z rangom matrike $A$, če jo pomnožimo (z leve ali z desne) z obrnljivo matriko?
        \begin{itemize}
            \item \colorbox{green!30}{\textbf{Dokaz.}} Pokažemo, da je $A \sim AP$ in $A \sim PA$.
        \end{itemize}
        \item \colorbox{blue!30}{\textbf{Trditev.}} Čemu je enaka slika matrike?
        \begin{itemize}
            \item \colorbox{green!30}{\textbf{Dokaz.}} Naj bo $A \in \FF^{m \times n}$ poljubna. Izračunamo $\im A$ s pomočjo slik baznih vektorjev prostora $\FF^n$.
        \end{itemize}
        \item \colorbox{blue!30}{\textbf{Izrek.}} Kakšne matrike je ekvivalentna vsaka matrika $A \in \FF^{m \times n}$? Neenakost za $\rang A$.
        \begin{itemize}
            \item \colorbox{green!30}{\textbf{Dokaz.}} Izberimo bazo za $\im A$, kjer $A: \FF^n \to \FF^m, \ A(x) = Ax$. Neenakost pokažemo s pomočjo dimenzijske enačbe ter definiciji slike.
            
            Dopolnimo bazo za $\im A$ do baze $B_m$ za $\FF^m$. V dokazu dimenzijske enačbe smo pokazali, da obstaja baza $B_n$ za $\FF^n$, da velja: $Av_i = w_i, \ i = 1, \ldots, r$ in $Av_i = 0, \ i > r$. Kakšna matrika je $A_{B_m, B_n}$?

            Predpostavimo, da $A \sim \begin{bmatrix}
                I_s & 0 \\ 0 & 0
            \end{bmatrix}$ in izračunamo $\rang \begin{bmatrix}
                I_s & 0 \\ 0 & 0
            \end{bmatrix}$.
        \end{itemize}        
        \item \colorbox{orange!30}{\textbf{Posledica.}} Karakterizacija ekvivalentnosti matrik iste velikosti (rang).
        \begin{itemize}
            \item \colorbox{green!30}{\textbf{Dokaz.}} $(\Rightarrow)$ Že vemo.
            
            $(\Leftarrow)$ $\sim$ je ekvivalenčna + trditev.
        \end{itemize}
        \item \colorbox{blue!30}{\textbf{Lema.}} Recimo, da je $P \in \Fnn$ obrnljiva. Kaj lahko povemo o $P^{T}$? Čemu je enako $(P^T)^{-1}$.
        \begin{itemize}
            \item \colorbox{green!30}{\textbf{Dokaz.}} Uporabimo transponiranje na enakosti $PP^{-1} = P^{-1}P = I$
        \end{itemize}
        \item \colorbox{orange!30}{\textbf{Posledica.}} V kakšni zvezi sta $\rang A$ in $\rang A^{T}$?
        \begin{itemize}
            \item \colorbox{green!30}{\textbf{Dokaz.}} Uporabimo transponiranje na enakosti $A = Q^{-1} \begin{bmatrix}
                I_r & 0 \\ 0 & 0
            \end{bmatrix} P$.
        \end{itemize}

        \newpage
        \item \colorbox{blue!30}{\textbf{Trditev.}} Naj bo $A \in \Fmn$. V kakšni zvezi so števila: največje možno število linearno neodvisnih stolpcev,  največje možno število linearno neodvisnih vrstic, $\rang A$?
        \begin{itemize}
            \item \colorbox{green!30}{\textbf{Dokaz.}} Naj bo matrika $A$ ima $r$ linearno neodvisnih stolpcev. S protislovjem pokažemo, da oni tvorijo bazo prostora $\im A$. Druga enakost sledi iz prejšnje posledice.
        \end{itemize}
    \end{itemize}
    
    \item[$\circ$] Izračun ranga.
    \begin{itemize}
        \item Permutacijska matrika $P_{p, q}$. Matrika $P_{\alpha, p}$. Matrika $P = I + \alpha E_{p, q}, \ p \neq q$.
        \item \colorbox{blue!30}{\textbf{Trditev.}} Ali so matriki $P_{p, q}$, $P_{\alpha, p}$ in $P = I + \alpha E_{p, q}, \ p \neq q$ obrnljivi?
        \begin{itemize}
            \item \colorbox{green!30}{\textbf{Dokaz.}} Inverz od $P_{p, q}$ je $P_{p, q}$. Inverz od $P_{\alpha, p}$ je $P_{\alpha^{-1}, p}$. 
            
            Inverz od $P = I + \alpha E_{p, q}, \ p \neq q$ je $P = I - \alpha E_{p, q}, \ p \neq q$.
        \end{itemize}
        \item \colorbox{blue!30}{\textbf{Trditev.}} Naj bo $A \in \Fmn$. Kaj se zgodi, če pomnožimo (z desne ali z leve) $A$ z $P_{p, q}$, $P_{\alpha, p}$ ali $P = I + \alpha E_{p, q}, \ p \neq q$?
        \begin{itemize}
            \item \colorbox{green!30}{\textbf{Dokaz.}} Pomnožimo z desne in izračunamo rezultat. Pri množenji z leve upoštevamo, da $AB = (AB)^{TT}$.
        \end{itemize}
        \item \colorbox{blue!30}{\textbf{Izrek.}} Ali lahko iz vsake matrike $A \in \Fmn$ pridelamo neko lepo matriko, da bi enostavno izračunali $\rang A$?
        \begin{itemize}
            \item \colorbox{green!30}{\textbf{Dokaz.}} Z indukcijo na $k$ dokažemo, da lahko dobimo neko matriko oblike $A_k = \begin{bmatrix}
                I_k & 0 \\ 0 & A_k'
            \end{bmatrix}$ za vsak $k \in \set{0, 1, \ldots, r}$.
        \end{itemize}
        \item \colorbox{yellow!30}{\emph{Primer.}} Izračunaj rang matrike $A = \begin{bmatrix}
            1 & 0 & 2 \\ -1 & 1 & -3 \\
            0 & 1 & -1 \\ 1 & 2 & 0
        \end{bmatrix}$.

        \item \colorbox{yellow!30}{\emph{Opomba.}} Kaj je običajno dovolj za računanje ranga?
    \end{itemize}

    \item[$\circ$] Reševanje sistema linearnih enačb
    
    Radi bi rešili sistem linaernih enačb
    \begin{align*}
        &a_{11}x_1 + a_{12}x_2 + \ldots + a_{1n}x_n = b_1, \\
        &a_{21}x_1 + a_{22}x_2 + \ldots + a_{2n}x_n = b_2, \\
        &\vdots \\
        &a_{m1}x_1 + a_{m2}x_2 + \ldots + a_{mn}x_n = b_m,
    \end{align*}
    kjer so $a_{ij} \in \FF$ in $b_i \in \FF$ dani skalarji, $\x{n}$ pa neznanki
    \begin{itemize}
        \item \colorbox{purple!30}{\textbf{Definicija.}} Matrika sistema. Vektor neznank. Vektor desni strani.
        \item \colorbox{yellow!30}{\emph{Opomba.}} Čemu je ekvivalenten sistem? Kakšnim enačbam bomo rekli sistem enačb?
        \item \colorbox{purple!30}{\textbf{Definicija.}} Homogen sistem. Nehomogen sistem. Protisloven sistem. Neprotisloven sistem.
        \item \colorbox{purple!30}{\textbf{Definicija.}} Razširjena matrika.
        \item \colorbox{blue!30}{\textbf{Kronecker-Capellijev izrek.}} Karakterizacija neprotislovnosti sistema. 
        \begin{itemize}
            \item \colorbox{green!30}{\textbf{Dokaz.}} Vemo: Rang matrike je maksimalno število linearno neodvisnih stolpcev ter slika matrike je linearna ogrinjača stolpcev. Po definiciji neprotislovnega sistema z ekvivalentnimi perehodami pokažemo, da trditev velja.
        \end{itemize}
        \item Ali je homogen sistem vedno rešljiv? Kaj so rešitve homogenega sistema? Kdaj je homogen sistem ima edino rešitev? Kaj je v splošnem rešitev homogenega sistema? Čemu je enaka dimenzija prostora rešitev?
        \item \colorbox{blue!30}{\textbf{Trditev.}} Recimo, da je nehomogen sistem neprotisloven. Čemu je potem enaka množica rešitev sistema? \textbf{Partikularna rešitev.}
        \begin{itemize}
            \item \colorbox{green!30}{\textbf{Dokaz.}} Pokažemo, da sta množici rešitev sistema $Ax = b$ in vektorjev oblike $v+u$, kjer je $v$ partikularna rešitev in $u \in \ker A$ enaki.
        \end{itemize}
    \end{itemize}
    \begin{itemize}
        \item \colorbox{blue!30}{\textbf{Gaussova eliminacija}.} Vektorska oblika rešitev. Kako rešujemo sistemi v praksi?
        \begin{itemize}
            \item \colorbox{green!30}{\textbf{Dokaz.}} Ali množenje razširjene matrike sistema z obrnljivo matriko z leve spremenijo rešitve?
        \end{itemize}
        \item \colorbox{yellow!30}{\emph{Primer.}} Reši sistem:
        \begin{align*}
            &x_1 + 2x_2 + 3x_3 + 4x_4 = -1, \\
            &8x_1 + 7x_2 + 6x_3 + 5x_4= 10, \\            
            &9x_1 + 10x_2 +  11x_3 + 12x_4 = 7.
        \end{align*}
        \item Reševanje več sistemov hkrati. Kako poiščemo inverz matrike $A \in \Fnn$?
        \item \colorbox{yellow!30}{\emph{Primer.}} Izračunaj $\begin{bmatrix}
            1 & 2 & -1 \\
            2 & 1 & 0 \\
            2 & -1 & 1
        \end{bmatrix}^{-1}$.
    \end{itemize}

    \item[$\circ$] Podobnost matrik 
    
    Naj bo $\Aa$ endomorfizem prostora $V$.  
    \begin{itemize}
        \item Običajno v $V$ izberemo eno bazo $B$ in napišemo matriko $\Aa_{B, B}$, in ne izberemo dveh različnih baz $B_1$ in $B_2$ in napišemo matriko $\Aa_{B_1, B_2}$. Zakaj to smiselno?
        \item Imejmo endomorfizem $\Aa: V \to V$ in bazi $B, B'$. Endomorfizmu priredimo matriki $A = \Aa_{B, B}$ in $A' = \Aa_{B', B'}$. V kakšni zvezi potem $A$ in $A'$?
        \item \colorbox{purple!30}{\textbf{Definicija.}} Podobnost matrik.
        \item \colorbox{yellow!30}{\emph{Opomba.}} Ali vsaki podobni matriki tudi ekvivalentni? Ali velja obrat?
        \item \colorbox{blue!30}{\textbf{Trditev.}} Ali je podobnost ekvivalenčna relacija?
        \begin{itemize}
            \item \colorbox{green!30}{\textbf{Dokaz.}} Kot pri ekvivalentnosti.
        \end{itemize}
        \item \colorbox{blue!30}{\textbf{Trditev.}} Karakterizacija podobnosti matrik z endomorfizmami.
        \begin{itemize}
            \item \colorbox{green!30}{\textbf{Dokaz.}} Kot pri ekvivalentnosti.
        \end{itemize}
    \end{itemize}
    Zanima nas, kakšna je najbolj enostavna matrika, ki pripada endomorfizmu $\Aa$ prostora $V$, torej kakšna je najbol enostavna matrika oblike $\Aa_{B, B}$, kjer je $B$ baza prostora $V$.
    \begin{itemize}
        \item Ali je vsaka matrika podobna matrike oblike $\begin{bmatrix}
            I_k & 0 \\ 0 & 0
        \end{bmatrix}$?
        \item Ali morda pa je matrika podobna neki diagonalni matriki?
        \item \colorbox{purple!30}{\textbf{Definicija.}} Kadar endomorfizem $\Aa \in L(V)$ se da diagonalizirati?
        \item \colorbox{yellow!30}{\emph{Primer.}} Ali se da diagonalizirati matrika $A = \begin{bmatrix}
            0 & 1 \\ 0 & 0
        \end{bmatrix}$?
        \begin{itemize}
            \item \colorbox{green!30}{\textbf{Dokaz.}} Recimo, da $A = P^{-1}DP$, za neko diagonalno matriko $D$ in obrnljivo matriko $P$.
        \end{itemize}
        \item Recimo, da se endomorfizem $\Aa: V \to V$ diagonalizira v bazi $B$ in naj bo $A = \Aa_{B, B}, \ B = \set{v_1, \ldots, v_n}$. Kam slika endomorfizem $\Aa$ vektorji iz baze $B$?
        \item \colorbox{purple!30}{\textbf{Definicija.}} Naj bo $\Aa \in L(V)$. Lastni vektor endomorfizma $\Aa$. Lastna vrednost endomorfizma $\Aa$ za lastni vektor $v$.
        \item \colorbox{blue!30}{\textbf{Trditev.}} Ali je lastna vrednost enolično določena z lastnim vektorjem?
        \begin{itemize}
            \item \colorbox{green!30}{\textbf{Dokaz.}} Enostavno.
        \end{itemize}
        \item \colorbox{blue!30}{\textbf{Trditev.}} Ali je lastni vektor enolično določen z lastno vrednostjo?
        \begin{itemize}
            \item \colorbox{green!30}{\textbf{Dokaz.}} Recimo, da je $\Aa v = \lambda v$. Kaj vemo o vektorju $\alpha v$?
        \end{itemize}
        \item \colorbox{blue!30}{\textbf{Trditev.}} Kako poiščemo lastni vrednosti endomorfizma $\Aa$? Kadar je $\lambda$ lastna vrednost endomorfizma $\Aa$?
        \begin{itemize}
            \item \colorbox{green!30}{\textbf{Dokaz.}} Pokažemo, da je $\lambda$ lastna vrednost endomorfizma $\Aa \Leftrightarrow \Aa - \lambda \id$ ni bijektivna.
        \end{itemize}
        \item \colorbox{blue!30}{\textbf{Trditev.}} Karakterizacija ni bijektivnosti preslikave $\Aa - \lambda \id$.
        \begin{itemize}
            \item \colorbox{green!30}{\textbf{Dokaz.}} Izomorfizem med prostorama endomorfizmov in matrik.
        \end{itemize}
    \end{itemize}
\end{enumerate}

\newpage
\



\newpage
\section{DETERMINANTE}
\begin{enumerate}
    \item $n$-linearne preslikave
    \begin{itemize}
        \item \colorbox{yellow!30}{\emph{Primer.}} Ali je skalarni produkt linearen v 1. in 2. faktorju? Kako pravimo take preslikave?
        \item \colorbox{purple!30}{\textbf{Definicija.}} $n$-linearna preslikava. $n$-linearen funkcional.
        \item \colorbox{purple!30}{\textbf{Definicija.}} Simetrična preslikava. Antisimetrična preslikava.
        \item \colorbox{yellow!30}{\emph{Primer.}} Ali so naslednji preslikavi $n$-linearne oz. $n$-linearni funkcionali. Ali so simetrični oz. antisimetrični?
        \begin{itemize}
            \item $f: \ \RR^3 \times \RR^3 \to \RR, \ f(\vec{x}, \vec{y}) = \vec{x} \cdot \vec{y}$.
            \item $f: \ \RR^3 \times \RR^3 \to \RR^3, \ f(\vec{x}, \vec{y}) = \vec{x} \times \vec{y}$.
            \item $f: \ \RR^3 \times \RR^3 \times \RR^3 \to \RR, \ f(\vec{x}, \vec{y}, \vec{z}) = (\vec{x} \times \vec{y}) \cdot \vec{z}$.
            \item $f: \ \FF^{n \times n} \times \FF^{n \times n} \to \FF^{n \times n}, \ f(A,B) = AB$.
        \end{itemize}
        \item \colorbox{yellow!30}{\emph{Primer.}} Kako računamo z $n$-linearnimi preslikavami? $f(\alpha_1 v_1 + \alpha_2 v_2, \beta_1 w_1 + \beta_2 w_2)$.
        \item \colorbox{blue!30}{\textbf{Trditev.}} 3 lastnosti antisimetričnih preslikav.
        \begin{itemize}
            \item \colorbox{green!30}{\textbf{Dokaz.}} (1) Definicija antisimetričnosti.
            
            (2) Definicija $n$-linearnosti in (1) točka.

            (3) Z indukcijo na število transpozicij v razcepu. Z pomočjo permutacije $\rho = \tau_1 \circ \pi$.
        \end{itemize}
    \end{itemize}
    \item[$\circ$] Tenzor $n$-linearne preslikave. \textcolor{red}{[Za študente, ki jih zanima več]}.
    \begin{itemize}
        \item \colorbox{blue!30}{\textbf{Trditev.}} Naj bo $f: \ V \times W \to U$ bilinearna preslikava. S čim je preslikava $f$ enolično določena?
        \begin{itemize}
            \item \colorbox{green!30}{\textbf{Dokaz.}} Izberimo $x \in V$ in $y \in W$ in izračunamo $f(x, y)$.
        \end{itemize}
        \item \colorbox{purple!30}{\textbf{Definicija.}} Tenzor reda $m \times n \times p$.
        \item \colorbox{yellow!30}{\emph{Opomba.}}  Kako bilinerarne preslikave priredimo tenzor reda $m \times n \times p$?
        \item \colorbox{yellow!30}{\emph{Primer.}} Naj bo $f: \ \FF^{2 \times 2} \times \FF^{2 \times 2} \to \FF^{2 \times 2}, \ f(A,B) = AB$.         
        Določi tenzor preslikave glede na bazo $\set{E_{11}, E_{12}, E_{21}, E_{22}}$.
        \item \colorbox{purple!30}{\textbf{Definicija.}} Seštevanje in množenje s skalarji tenzorjev fiksne velikosti $m \times n \times p$.
        \item \colorbox{purple!30}{\textbf{Definicija.}} Tenzorski produkt prostorov $\FF^m, \FF^n$ in $\FF^p$. Oznaka.
        \item \colorbox{yellow!30}{\emph{Opomba.}} Čemu je enaka $\dim \FF^m \otimes \FF^n \otimes \FF^p$? Kakšne vektorji sestavljajo bazo? Oznaka za tenzor, ki ima na $(i, j, k)$-tem mesu enico in drugod ničle.
        \item \colorbox{purple!30}{\textbf{Definicija.}} Tenzor reda $m_1 \times m_2 \times \ldots \times m_n \times p$.
        \item \colorbox{yellow!30}{\emph{Opomba.}} Recimo, da je $f: V_1 \times V_2 \times \ldots \times V_n \to U$ $n$-linearna preslikava in $\dim V_i = m_i$ za $i = 1, \ldots, n$ in $\dim U = p$. Kako preslikavi $f$ priredimo tenzor reda $m_1 \times m_2 \times \ldots \times m_n \times p$?
        \item \colorbox{purple!30}{\textbf{Definicija.}} Seštevanje in množenje s skalarji tenzorjev reda $m_1 \times m_2 \times \ldots \times m_n \times p$.
        \item \colorbox{yellow!30}{\emph{Opomba.}} Ali za ti operaciji tenzorji tvorijo vektorski prostor? Čemu je enaka njena dimenzija?
    \end{itemize}
    \item Definicija in lastnosti determinantov
    \begin{itemize}
        \item \colorbox{yellow!30}{\emph{Primer.}} Naj bo $2 \neq 0$ v $\FF$. Naj bo $\FF^{n \times n} \to \FF$ antisimetrična preslikava. Izračunaj $f(A)$.
        \item \colorbox{purple!30}{\textbf{Definicija.}} Determinanta.
        \item \colorbox{yellow!30}{\emph{Primer.}} Kaj če je $n=2$ ali $n=3$? Ali je to običajna determinanta?
        \item \colorbox{blue!30}{\textbf{Trditev.}} Naj bo $2 \neq 0$ v $\FF$. Naj bo $\FF^{n \times n} \to \FF$ antisimetrična preslikava. Čemu je enako $f(A)$?
        \begin{itemize}
            \item \colorbox{green!30}{\textbf{Dokaz.}} Izpeljava na začetku poglavja.
        \end{itemize}
        \item \colorbox{yellow!30}{\emph{Primer.}} Ali je mešani produkt determinanta? Kaj to pomeni o $3$-linearnih funkcionalih na $\RR^3$?
        \item \colorbox{blue!30}{\textbf{Trditev.}} Naj bo $A = [a_{ij}]_{i, j = 1}^n$. Ali je $\det A = \sum_{\rho \in S_n} s(\rho)a_{1, \rho(1)}a_{2, \rho(2)}\ldots a_{n, \rho(n)}$?
        \begin{itemize}
            \item \colorbox{green!30}{\textbf{Dokaz.}} Členi $a_{1, \rho(1)}a_{2, \rho(2)}\ldots a_{n, \rho(n)}$ in $a_{\rho^{-1}(1), 1} a_{\rho^{-1}(2),2}\ldots a_{\rho^{-1}(n), n}$ so isti, le v drugem vrstem redu.
        \end{itemize}
        \item \colorbox{orange!30}{\textbf{Posledica.}} Ali je $\det A = \det A^T$?
        \item \colorbox{blue!30}{\textbf{Trditev.}} Čemu je enaka determinanta zgornje (spodnje) trikotne matrike?
        \begin{itemize}
            \item \colorbox{green!30}{\textbf{Dokaz.}} Kadar člen v vsoti $\det A$ je enak $0$? 
        \end{itemize}
        \item \colorbox{orange!30}{\textbf{Posledica.}} Čemu je enaka determinanta diagonalne matrike?
        \item \colorbox{blue!30}{\textbf{Trditev.}} Čemu je enaka determinanta bločne zgornje (spodnje) trikotne matrike?
        \item \colorbox{orange!30}{\textbf{Posledica.}} Čemu je enaka determinanta bločno diagonalne matrike?
        \item \colorbox{blue!30}{\textbf{Trditev.}} Ali je determinanta $n$-linearen antisimetričen funkcional?
        \begin{itemize}
            \item \colorbox{green!30}{\textbf{Dokaz.}} Antisimetričnost: Dovolj pokazati za matriko $B$, ki jo dobimo tako, da v $A$ zamenajmo $1.$ in $2.$ stolpec. Za to izračunamo po definiciji $\det B$ s pomočjo $\tau = (1\ 2)$.
            
            $n$-linearnost: Fiksiramo stoplce $A^{(1)}, \ldots, A^{(i-1)}, A^{(i+1)}, \ldots, A^{(n)}$ in naj bo $A^{(i)} = \beta b + \gamma c$. Izračunamo determinanto.
        \end{itemize}

        \newpage
        \item \colorbox{orange!30}{\textbf{Posledica.}} Ali se determinanta spremeni, če nekemu stoplcu (ali vrstice) prištejemo večkratnik drugega stolpca (druge vrstice).
        \begin{itemize}
            \item \colorbox{green!30}{\textbf{Dokaz.}} Če $2 \neq 0$ to sledi iz lastnosti antisimetričnih funkcionalov.
            
            Če je $2 = 0$, to lahko dokažemo z računom, pri čemer vsoto $\det A$ razdelimo na vsoto po dosih permutacijah in vsoto po lihih permutacijah.
        \end{itemize}
        \item \colorbox{orange!30}{\textbf{Posledica.}} 3 lastnosti determinante na podlage trditve o $n$-linearnosti in antisimetričnosti.
        \item \colorbox{yellow!30}{\emph{Opomba.}} Kako lahko izračunamo determinanto?
        \item \colorbox{yellow!30}{\emph{Primer.}} Izračunaj $\begin{vmatrix}
            1 & 0 & -1 & 2 \\
            2 & 1 & -1 & 3 \\
            -1 & 2 & 0 & -1 \\
            3 & -1 & 1 & 0
        \end{vmatrix}$.
        \item \colorbox{orange!30}{\textbf{Posledica.}} Kadar je $n \times n$ matrika obrnljiva?
        \begin{itemize}
            \item \colorbox{green!30}{\textbf{Dokaz.}} Pri računanju ranga smo pokazalim da matriko $A$ lahko s elementarnimi operacijami na vrsticah in stolpcih prevedemo na neko lepo matriko $A_0$. Kakšna zvezna med $\det A$ in $\det A_0$?
        \end{itemize}
        \item \colorbox{purple!30}{\textbf{Definicija.}} Minor reda $k$. Glavni minor. Vodilni minor.
        \item \colorbox{yellow!30}{\emph{Opomba.}} Kaj je minor reda $1$? Kaj je minor reda $n$ matrike $A \in \FF^{n \times n}$?
        \item \colorbox{blue!30}{\textbf{Izrek.}} Kako so povezani minorji in rang matrike?
        \begin{itemize}
            \item \colorbox{green!30}{\textbf{Dokaz.}} Naj bo $\rang A = r$. Najprej s izbiro linearno neodvisnih stolpcev in vrstic konstruiramo neničeln minor reda $k$. Nato pa s protislovjem pokažemo, da so vsi minorji reda $k > r$ ničelni.
        \end{itemize}
    \end{itemize}
    \item Multiplikativnost determinante
    \begin{itemize}
        \item \colorbox{blue!30}{\textbf{Izrek.}} Ali je determinanta multiplikativen funkcional? Kaj to pomeni?
        \begin{itemize}
            \item \colorbox{green!30}{\textbf{Dokaz.}} Predpostavimo $2 \neq 0$. Naj bo $f: (\FF^n)^n \to \FF$ preslikava, definirama s predpisom 
            
            $f(v_1, \ldots, v_n) = \det [Av_1 \ Av_2 \ \ldots \ Av_n]$. Potem $f$ je $n$-linearen in antisimetričen funkcional na $\FF^{n \times n}$. Uporabimo znane lastnosti.
        \end{itemize}
        \item \colorbox{orange!30}{\textbf{Posledica.}} Ali velja $\det(AB) = \det(BA)$ za vse $A, B \in \FF^{n \times n}$?
        \item \colorbox{orange!30}{\textbf{Posledica.}} Naj bo $A \in \FF^{n \times n}$ obrnljiva. Čemu je enaka $\det A^{-1}$?
        \begin{itemize}
            \item \colorbox{green!30}{\textbf{Dokaz.}} Uporabimo $\det$ na enačbe $AA^{-1} = I$.
        \end{itemize}
        \item \colorbox{orange!30}{\textbf{Posledica.}} Ali podobni matriki imata enako determinanto?
        \begin{itemize}
            \item \colorbox{green!30}{\textbf{Dokaz.}} Uporabimo $\det$ na enačbe $A' = P^{-1}AP$.
        \end{itemize}
        \item \colorbox{purple!30}{\textbf{Definicija.}} Determinanta endomorfizma $\Aa \in L(V)$.
        \item \colorbox{yellow!30}{\emph{Opomba.}} Zakaj je definicija dobra?
    \end{itemize}
    \item Razvoj determinante
    \begin{itemize}
        \item \colorbox{purple!30}{\textbf{Definicija.}} Poddeterminanta matrike $A \in \FF^{n \times n}$. Prirejenka matrike $A$.
        \item \colorbox{yellow!30}{\emph{Primer.}} Določi prirejenko matrike $\begin{vmatrix}
            1 & 0 & -1 \\
            1 & 1 & 1 \\
            0 & -1 & 1
        \end{vmatrix}$.
        \item \colorbox{blue!30}{\textbf{Izrek.}} Razvij determinante po stolpcu. Razvoj determinante po vrstice.
        \begin{itemize}
            \item \colorbox{green!30}{\textbf{Dokaz.}} Zaradi enakosti $\det A = \det A^T$ lahko dokažemo samo za stolpce. Najprej izračunamo determinanto v posebnih primerah:
            
            (1) $A^{(j)} = e^j$.

            (2) $A^{(i)} = e^j$ za $i \neq j$.

            Nato z upoštevanjem $n$-linearnosti $\det$ izračunamo v spolšnem za $A^{(j)} = a_{1j}e_1 + a_{2j}e_2 + \ldots + a_{nj}e_n$.
        \end{itemize}
        \item \colorbox{yellow!30}{\emph{Primer.}} Z pomočjo razvoja determinante po vrstice ali stolpcu izračunaj $\begin{vmatrix}
            0 & -1 & 0 & 2 \\
            -1 & -2 & 3 & 1 \\
            0 & 1 & 0 & 0 \\
            2 & 3 & 1 & -1
        \end{vmatrix}$.
        \item \colorbox{blue!30}{\textbf{Izrek.}} Naj bo $A \in \FF^{n \times n}$. V kakšni zvezi sta matriki $A$ in $\widetilde{A}^T$? 
        \begin{itemize}
            \item \colorbox{green!30}{\textbf{Dokaz.}} Definicija produkta matrik. Razvoj determinante po ustreznemu stolpcu (vrstice) in lastnost determinante.
        \end{itemize}
        \item  \colorbox{orange!30}{\textbf{Posledica.}} Naj bo $A \in \FF^{n \times n}$ obrnljiva. Kako dobimo $A^{-1}$ z pomočjo determinante?
        \begin{itemize}
            \item \colorbox{green!30}{\textbf{Dokaz.}} Izračunamo.
        \end{itemize}
        \colorbox{yellow!30}{\emph{Primer.}} Določi $\begin{vmatrix}
            a & b \\ 
            c & d
        \end{vmatrix}^{-1}$.        
    \end{itemize}

    \newpage
    \item Uporaba pri reševanju sistem enačb
    
    Radi bi rešili sistem enačb $Ax=b$, kjer $A \in F^{n \times n}$ in $b \in \FF^n$.
    \begin{itemize}
        \item \colorbox{blue!30}{\textbf{Izrek.}} Cramerjeve formule.
        \begin{itemize}
            \item \colorbox{green!30}{\textbf{Dokaz.}} Napišemo po komponentah in izračunamo $x_j$ za vsak $j = 1, \ldots, n$. Z uporabo razvoja determinante po vrstice ali stolpcu.
        \end{itemize}
        \item\colorbox{yellow!30}{\emph{Primer.}} Reši sistem enačb:
        \begin{align*}
            &ax+by = c, \\
            &dx+dy = f.
        \end{align*}
    \end{itemize}  
\end{enumerate}

\newpage
\

\newpage
\section{STRUKTURA ENDOMORFIZMOV}

\textbf{Problem.} Naj bo $\mathcal{A}$ endomorfizem prostora $V$. Kakšno bazo $\mathcal{B}$ prostora $V$ moramo izbrati, da bo matrika $\mathcal{A}_{\mathcal{B},\mathcal{B}}$ čim bolj enostavna?

\begin{enumerate}

    \item Lastne vrednosti in lastni vektorji
    \begin{itemize}
        \item \colorbox{purple!30}{\textbf{Definicija.}} Kadar endomorfizem $\Aa \in L(V)$ se da diagonalizirati? \textcolor{red}{[Ponovitev]}
        \item \colorbox{purple!30}{\textbf{Definicija.}} Naj bo $\Aa \in L(V)$. Lastni vektor endomorfizma $\Aa$. Lastna vrednost endomorfizma $\Aa$ za lastni vektor $v$. \textcolor{red}{[Ponovitev]}
        \item \colorbox{yellow!30}{\emph{Opomba.}} Koliko lastnih vrednosti pripadajo lastnemu vektorju? Koliko lastnih vektorjev pripadajo lastni vrednosti? \textcolor{red}{[Ponovitev]}
        \item \colorbox{blue!30}{\textbf{Trditev.}} Ali je množica $\set{x \in V; \mathcal{A} x = \lambda x}$, kjer $\lambda$ lastna vrednost endomorfizma $\mathcal{A} \in L(V)$, vektorski podprostor prostora $V$?
        \begin{itemize}
            \item \colorbox{green!30}{\textbf{Dokaz.}} Pokažemo, da $Ax = \lambda x \Leftrightarrow x \in \ker (\Aa - \lambda \id)$.
        \end{itemize}
        \item \colorbox{purple!30}{\textbf{Definicija.}} Lastni podprostor endomorfizma $\mathcal{A}$ za lastno vrednost $\lambda$. Geometrijska večkratnost lastne vrednosti $\lambda$.
        \item \colorbox{yellow!30}{\emph{Opomba.}} Čemu je enak lastni podprostor endomorfizma $\mathcal{A}$ za lastno vrednost $\lambda?$ Kaj ga sestavljajo?
        \item \colorbox{blue!30}{\textbf{Trditev.}} Karakterizicija lastne vrednosti.
        \begin{itemize}
            \item \colorbox{green!30}{\textbf{Dokaz.}} Pokažemo, da $\lambda$ je lastna vrednost endomorfizma $\Aa \Leftrightarrow \det (\Aa_{B,B} - \lambda I) = 0$.
        \end{itemize}
        \item \colorbox{yellow!30}{\emph{Primer.}} Poišči lastne vrednosti in lastne vektorji matrike $\begin{bmatrix}
            -1 & -2 & -2 \\
            -2 & -1 & -2 \\
            2 & 2 & 3
        \end{bmatrix}$.
        \item \colorbox{blue!30}{\textbf{Trditev.}} Karakterizacija diagonalizacije endomorfizma (baza prostora $V$).
        \begin{itemize}
            \item \colorbox{green!30}{\textbf{Dokaz.}} Po definiciji matrike linearne preslikave, diagonalizaciji endomorfizma, lastnih vektorjev in lastnih vrednosti.
        \end{itemize}
        \item \colorbox{orange!30}{\textbf{Posledica.}} Karakterizacija diagonalizacije endomorfizma (število linearno neodvisnih lastnih vektorjev).
        \item \colorbox{yellow!30}{\emph{Opomba.}} Recimo, da se endomorfizem da diagonalizirati. Kaj so členi pripadajoče diagonalne matrike?
        \item \colorbox{yellow!30}{\emph{Primer.}} Naj bo $A \in \CC^{n \times n}$ matrika, ki jo identificiramo s preslikavo $\CC^n \to \CC^n, \ x \mapsto Ax$. \textbf{Kadar matrika se da diagonalizirati?}  Recimo, da je $A = PDP^{-1}$ za neko diagonalno matriko $D$ in neko obrnljivo matriko $P$. Kaj so diagonalni členi matrike $D$? Kaj je $P$?        
        \begin{itemize}
            \item \colorbox{green!30}{\textbf{Dokaz.}} Izračunamo $A = PDP^{-1} \Leftrightarrow AP = DP \Leftrightarrow \text{ za vsak } i = 1, \ldots, n \text{ velja: } APe_i = PDe_i$.
        \end{itemize}
        \item \colorbox{blue!30}{\textbf{Trditev.}} Ali so linearno neodvisni lastni vektorji, ki pripadajo različnim lastnim vrednostim?
        \begin{itemize}
            \item \colorbox{green!30}{\textbf{Dokaz.}} S protislovjem pokažemo, da je eden izmed lastnih vektorjev ničeln.
        \end{itemize}
        \item \colorbox{orange!30}{\textbf{Posledica.}} Kaj velja, če ima endomorfizem prostora $V$, za kateri velja $\dim V = n$, $n$ različnih lastnih vrednosti?
    \end{itemize}

    \item Karakteristični in minimalni polinom
    \begin{itemize}
        \item Naj bo $\Aa \in L(V)$ endomorfizem in $A = \Aa_{B, B}$ matrika tega enodmorfizma glede na neko bazo $B$ prostora $V$. 
        
        Ali je $\det (A - \lambda I)$ polinom v $\lambda$? Kakšne stopnje, kaj je vodilni koeficient in kaj je konstantni člen?
        
        \textbf{Kroneckerjev delta.}
        \item \colorbox{purple!30}{\textbf{Definicija.}} Karakteristični polinom matrike $A$. Oznaka.
        \item \colorbox{blue!30}{\textbf{Trditev.}} Ali podobne matrike imata enak karakteristični polinom?
        \begin{itemize}
            \item \colorbox{green!30}{\textbf{Dokaz.}} Naj bo $B = P^{-1}AP$. Izračunamo $\Delta_B(\lambda)$.
        \end{itemize}
        \item \colorbox{purple!30}{\textbf{Definicija.}} Karakteristični polinom endomorfizma $\Aa \in L(V)$. Ali je definicija smiselna?
        \item \colorbox{blue!30}{\textbf{Izrek.}} Kako povezane lastne vrednosti endomorfizma $\Aa \in L(V)$ in njegov karakteristini polinom?
        \item \colorbox{yellow!30}{\emph{Primer.}} $V = \RR^2, \ \Aa: \RR^2 \to \RR^2$ rotacija za $\pi/2$ v pozitivni smeri. Kaj je lastni vektorji? Kaj so ničle karakterističnega polinoma?
        \item[\textbf{!}] Od zdaj naprej in do konca poglavja o strukturi endomorfizmov bomo predpostavili, da je $\FF = \CC$. 
        
        Zakaj to smiselno?
        \item \colorbox{blue!30}{\textbf{Osnovni izrek algebre.}} \textcolor{red}{[Dokaz pri Analize 2]}
        \item \colorbox{orange!30}{\textbf{Posledica.}} Kakšne je oblike vsak nekonstanten polinom $p \in \CC[\lambda]$?
        \item \colorbox{orange!30}{\textbf{Posledica.}} Kadaj je $\alpha \in \CC$ lastna vrednost endomorfizma $\Aa \in L(V)$?
        \item \colorbox{yellow!30}{\emph{Opomba.}} Nad kakšnim poljem bomo obravnavali matrike?
        \item \colorbox{purple!30}{\textbf{Definicija.}} Lastne vrednosti matrike $A$. 
        \item \colorbox{yellow!30}{\emph{Opomba.}} Ali lastne vrednosti matrike je lastne vrednoti nekega endomorfizma?
        \item \colorbox{purple!30}{\textbf{Definicija.}} Algebraične večkratnosti lastnih vrednosti. Spekter matrike/endomorfizma. 
        Oznaka.  
        
        \newpage
        \item \colorbox{purple!30}{\textbf{Definicija.}} Kadar matrika $A$ se da diagonalizirati?
        \item \colorbox{yellow!30}{\emph{Opomba.}} Karakterizacija diagonalizacije matrike. (Ali mora biti podobna neke diagonalne?) Kaj je v tem primeru na diagonali matrike $D$? Kaj so stolpce matrike $P$?
        \item \colorbox{blue!30}{\textbf{Trditev.}} (1) Kakšna zveza med algebraično in geometrično večkratnostjo lastne vrednosti matrike $A$?
        
        (2) Karakterizacija diagonalizaciji matrike $A$.        
        \begin{itemize}
            \item \colorbox{green!30}{\textbf{Dokaz.}} (1) Naj bo $m = \dim(\ker(A- \alpha I))$ geometrična večkratnost lastne vrednosti $\alpha$. Dopolnimo bazo $\ker(A- \alpha I)$ do baze $B$ za $\CC^n$ in z pomočjo matrike $A_{B, B}$ izračunamo $\Delta_A(\lambda)$.
            
            (2) $(\Rightarrow)$ Naj bodo $\alpha_1, \ldots, \alpha_k$ vse različne lastne vrednosti matrike $A$, $n_1, \ldots, n_k$ algebraične večkratnosti in $m_1, \ldots, m_k$ geometrične večkratnosti. Ocenimo zgoraj in spodaj izraz $n_1 + \ldots + n_k$ (koliko linearno neodvisnih lastnih vektorjev ima matrika $A$, če se da diagonalizirati?).

            $(\Leftarrow)$ Za vsak $i$ izberimo bazo jedra $\ker(A - \alpha_i I): \set{v_{i,1}, v_{i, 2}, \ldots, v_{i, n_i}}$. Pokažemo, da so vektorji $v_{ij}$ tvorijo bazo prostora $\CC^n$.
        \end{itemize}
        \item \colorbox{yellow!30}{\emph{Opomba.}} Ali enako velja za endomorfizmi?
        \item \colorbox{purple!30}{\textbf{Definicija.}} Vrednost polinoma $p \in \CC[\lambda]$ v matrike $A \in F^{n \times n}$. 
        
        Vrednost polinoma $p \in \CC[\lambda]$ v endomorfizmu $\Aa \in L(V)$.
        \item \colorbox{purple!30}{\textbf{Definicija.}} Matrični polinom.
        \item \colorbox{blue!30}{\textbf{Izrek.}} Cayley-Hamiltonov izrek.
        \begin{itemize}
            \item \colorbox{green!30}{\textbf{Dokaz.}} Za vsak $\lambda \in \CC$ naj bo $B(\lambda) = \widetilde{A - \lambda I}$ prirejenka matrike $A - \lambda I$.
            
            Vemo: $B(\lambda)^T \cdot (A - \lambda I) = \det (A - \lambda I) \cdot I = \Delta_A(\lambda) \cdot I$ za vsak $\lambda \in \CC$. Z primejavo koeficientov matričnih polinomov pri posameznih potencah pridemo do rezultata.
        \end{itemize}
        \item \colorbox{orange!30}{\textbf{Posledica.}} Naj bo $A \in \CC^{n \times n}$ obrnljiva. Ali obstaja polinom $p$, da je $A^{-1} = p(A)$. Kakšne stopnje ta polinom?
        \begin{itemize}
            \item \colorbox{green!30}{\textbf{Dokaz.}} Z uporabo prejšnjega izreka dobimo $A^{-1}$.
        \end{itemize}
        \item \colorbox{purple!30}{\textbf{Definicija.}} Minimalni polinom matrike $A \in \CC^{n \times n}$.
        \item \colorbox{yellow!30}{\emph{Opomba.}} Ali je minimalni polinom obstaja? Ali je enoličen?
        \item \colorbox{blue!30}{\textbf{Trditev.}} Ali podobni matriki imata enak minimalni polinom?
        \begin{itemize}
            \item \colorbox{green!30}{\textbf{Dokaz.}} Naj bo $B = P^{-1}AP$ in $p(\lambda) = a_m\lambda^m + a_{m-1}\lambda^{m-1}+ \ldots + a_1 + \lambda a_0$ poljuben polinom. Z~izračunom $p(B)$ pokažemo, da je $p(A) = 0 \Leftrightarrow p(B) = 0$.
        \end{itemize}
        \item \colorbox{purple!30}{\textbf{Definicija.}} Minimalni polinom endomorfizma $\Aa \in L(V)$.
        \item \colorbox{yellow!30}{\emph{Opomba.}} Ali je definicja dobra? Zakaj?
        \item \colorbox{blue!30}{\textbf{Trditev.}} Opis minimalnega polinoma endomorfizma $\Aa \in L(V)$.
        \begin{itemize}
            \item \colorbox{green!30}{\textbf{Dokaz.}} Najprej pokažemo, da za poljuben polinom $p \in \CC[\lambda]$ velja: $\Phi(p(\Aa)) = p(\Phi(\Aa))$. Nato pokažemo, da ima minimalni polinom vodilni koeficient $1$, $\Aa$ za ničlo in, da minimalne možne stopnje med vsemi neničelnimi polinomi.
        \end{itemize}
        \item \colorbox{yellow!30}{\emph{Opomba.}} Ali vse, kar velja za minimalni polinomi matrik, velja tudi za minimalne polinome endomorfizmov, in obratno?
        \item \colorbox{blue!30}{\textbf{Izrek.}} Osnovni izrek o deljenju polinomov. \textbf{Kvocient} in \textbf{ostanek}. \textcolor{red}{[Dokaz v gimnaziji]}
        \item \colorbox{blue!30}{\textbf{Izrek.}} Naj bo $A \in \CC^{n \times n}$ in $p \in \CC[\lambda]$ tak polinom, da $p(A) = 0$. V kakšni zvezi sta $p$ in $m_A$?
        \begin{itemize}
            \item \colorbox{green!30}{\textbf{Dokaz.}} Po osnovnemu izreku o deljenju polinomov obstajata $k, r \in \CC[\lambda]$, da $p(\lambda) = k(\lambda) \cdot m_A(\lambda) + r(\lambda)$. Treba je dokazati, da $r \equiv 0$, zato poglejmo vrednost v matrike $A$.
        \end{itemize}
        \item \colorbox{orange!30}{\textbf{Posledica.}} Ali $m_A \, | \, \Delta_a$?
        \begin{itemize}
            \item \colorbox{green!30}{\textbf{Dokaz.}} Prejšnja trditev + Cayley-Hamiltonov izrek.
        \end{itemize}
        \item \colorbox{orange!30}{\textbf{Posledica.}} Ali vse ničle minimalnega polinoma so tudi ničle karakterističnega polinoma?
        \item \colorbox{blue!30}{\textbf{Izrek.}} Ali vse ničle karakterističnega polinoma so tudi ničle minimalnega polinoma?
        \begin{itemize}
            \item \colorbox{green!30}{\textbf{Dokaz.}} Naj bo $\alpha$ ničla $\Delta_A(\lambda)$. Po osnovnemu izreku o deljenju polinomov, obstajata $k, c \in \CC[\lambda]$, da $m_A(\lambda) = k(\lambda) \cdot (\lambda - \alpha) + c$. Radi bi pokazali, da je $c = 0$. Zato poglejmo vrednost polinomov v matriki $A$ in upoštevamo, da je $\alpha$ lastna vrednost matrike $A$.
        \end{itemize}
        \item \colorbox{orange!30}{\textbf{Posledica.}} Naj bo $A \in \CC^{n \times n}$ kakšne oblike je njen karakteristični in kakšne oblike je njen minimalni polinom?
        \item \colorbox{yellow!30}{\emph{Primer.}} Poišči minimalni polinom matrike $\begin{bmatrix}
            1 & 0 & 0 & 0 \\
            0 & 0 & 1 & 0 \\
            0 & 0 & 0 & 0 \\
            0 & 0 & 0 & 0
        \end{bmatrix}$.
    \end{itemize}

    \newpage
    \item Korenski podprostori
    
    V nadaljevanju bomo poiskali lepo matriko, ki pripada endomorfizmu, v primeru, ko se endomorfizem ne da diagonalizirati. Za začetek bomo pokazali, da endomorfizmu vedno pripada bločno diagonalna matrika, kjer ima vsak blok le eno lastno vrednost.
    \begin{itemize}
        \item \colorbox{purple!30}{\textbf{Definicija.}} Naj bo $\Aa \in L(V)$. Invarianten podprostor za endomorfizem $\Aa$.
        \item \colorbox{yellow!30}{\emph{Primer.}} Ugotovi, ali je podprostor invarianten.
        \begin{itemize}
            \item $\set{0}, V$ za poljuben endomorfizem $\Aa \in L(V)$.
            \item $\ker \Aa$ in $\im \Aa$ za poljuben endomorfizem $\Aa \in L(V)$.
            \item Lastni podprostori.
            \item Naj bo $V = \RR^3$ in $\Aa: \RR^3 \to \RR^3$ naj bo rotacija za $\frac{\pi}{2}$ okrog $z$-osi. Ali je $z$-os invarianten podprostor? Ali je ravnina $z=0$ invarianten podprostor? Ali je lastni podprostor?
        \end{itemize}
        \item \colorbox{yellow!30}{\emph{Opomba.}} Naj bo $\Aa \in L(V)$ in $U \leq V$. Ali zožitev $A|_U \in L(U, V)$? Kaj če je $U$ invarianten za $\Aa$?
        \item \colorbox{blue!30}{\textbf{Trditev.}} Naj bo $U \subseteq V$ invarianten podprostor za endomorfizem $\Aa \in L(V)$ in naj bo $B_U = \set{u_1, \ldots, u_m}$ baza prostora $U$. Dopolnimo jo do baze $B = \set{u_1, \ldots, u_m, v_1, \ldots v_k}$ prostora $V$. Kakšne oblike potem $A_{B, B}$?
        \begin{itemize}
            \item \colorbox{green!30}{\textbf{Dokaz.}} Napišemo matriko $\Aa_{B, B}$. Lahko si pomagamo z matriko zožitve $\Aa|_U$ glede na bazo $B_U$.
        \end{itemize}
        \item \colorbox{yellow!30}{\emph{Primer.}} Naj bo $V$ prostor polinomov stopnje največ $2$ in $\Aa \in L(V)$ odvajanje. $\Lin \set{1}$ je jedro, torej je invarianten podprostor. Baza jedra je $\set{1}$, ki jo dopolnimo do baze $B = \set{1, x, x^2}$ prostora $V$. Ali je matrika $\Aa_{B, B}$ res bločno zgornje trikotna? 
        \item \colorbox{yellow!30}{\emph{Primer.}} Naj bo $V = \RR^3$, $\Aa \in L(V)$ rotacija za $\frac{\pi}{2}$ okrog $z$-osi. netrivialna invariantna podprostora: $U_1 = z \text{-os}$, $U_2 = \text{ravnina } z = 0$. Naj bodo $B_1 = \set{(0, 0, 1)}, \ B_2 = \set{(1, 0, 0), (0, 1, 0)}$. Potem $B = B_1 \cup B_2$ baza za $\RR^3$. Določi $(\Aa|_{U_1})_{B_1, B_1}$, $(\Aa|_{U_2})_{B_2, B_2}$ in $\Aa_{B, B}$.
        \item \colorbox{blue!30}{\textbf{Trditev.}} Naj bo $\Aa \in L(V)$ in naj bo $V = V_1 \oplus V_2 \oplus \ldots \oplus V_k$, kjer so $V_1, \ldots, V_k$ invariantni za $\Aa$. Kakšne oblike potem $A_{B, B}$ glede na neko bazo $B$ prostora $V$? \textbf{Direktna vsota matrik}
        \begin{itemize}
            \item \colorbox{green!30}{\textbf{Dokaz.}} Napišemo matriko $\Aa_{B, B}$. Lahko si pomagamo z matrikami zožitev $\Aa|_{V_i}$ glede na bazo $B_i$.
        \end{itemize}
        \item[\textbf{!}] Do konca poglavja o korenskih podprostorov fiksiramo naslednje oznake:
        
        $V$ naj bo $n$-razsežen vektorski prostor nad $\CC$. $\Aa \in L(V)$ naj bo endomorfizem prostora $V$.

        $\lambda_1, \ldots, \lambda_2$ naj bodo vse različne lastne vrednosti endomorfizma $\Aa$.

        $\Delta_{\Aa}(\lambda) = (-1)^n(\lambda - \lambda_1)^{n_1}(\lambda - \lambda_2)^{n_2} \ldots (\lambda - \lambda_k)^{n_k}$.

        $m_{\Aa}(\lambda) = (\lambda - \lambda_1)^{m_1}(\lambda - \lambda_2)^{m_2} \ldots (\lambda - \lambda_k)^{m_k}$.
        \item \colorbox{purple!30}{\textbf{Definicija.}} Korenski podprostor za endomorfizem $\Aa$, ki pripada lastne vrednosti $\lambda_j$.
        \item \colorbox{yellow!30}{\emph{Opomba.}} Ali je korenski podprostor res vektorski podprostor $V$? Ali je korenski podprostor, ki pripada lastne vrednosti $\lambda_j$, vsebuje ustrezni lastni podprostor? Ali je vsak korenski podprostor netrivialen?
    \end{itemize}
    Radi bi uporabili prejšnjo trditev, kjer bomo za podprostore $V_j$ vzeli korenske podprostore.
    \begin{itemize}
        \item \colorbox{yellow!30}{\emph{Opomba.}} Ali polinomi v istem endomorfizmu komutirata?
        \item \colorbox{blue!30}{\textbf{Trditev.}} Ali je vsak korenski podprostor $W_j$ invarianten za endomorfizem $\Aa$?
        \begin{itemize}
            \item \colorbox{green!30}{\textbf{Dokaz.}} Pokažemo, da endomorfizmi $\Aa$ in $(\Aa - \lambda_j \id)^{m_j}$ komutirata (glavni argument: 
            
            $\Aa^k \circ \Aa^m = \Aa^m \circ \Aa^k$).
        \end{itemize}
        \item \colorbox{blue!30}{\textbf{Izrek.}} Naj bo $\FF$ poljubno polje in $d(\lambda)$ največji skupni delitelj polinomov $p_1(\lambda), \ldots, p_k(\lambda) \in \FF[\lambda]$. Kaj potem lahko zapišemo $d(\lambda)$? \textcolor{red}{[Dokaz pri Algebra 2]}
        \item \colorbox{blue!30}{\textbf{Izrek.}} Ali je prostor $V$ je direktna vsota korenskih podprostorov?
        \begin{itemize}
            \item \colorbox{green!30}{\textbf{Dokaz.}} Za vsak $i = 1, \ldots, k$, definiramo $p_j(\lambda) = \Pi_{i \neq j}(\lambda - \lambda_i)^{m_i} \in \CC[\lambda]$. Pokažemo, da obstajajo polinomi $q_1(\lambda), \ldots, q_k(\lambda)$, da velja: $p_1(\lambda)q_1(\lambda) + \ldots + p_k(\lambda)q_k(\lambda) = 1$ in vstavimo v ta enakost $\Aa$, dobimo enakost (*).
            
            Obstoj zapisa: Naj bo $x \in V$. Definiramo $x_j = p_j(\Aa)q_j(\Aa)x$. Izračunamo $x_1+x_2 + \ldots + x_k$ in pokažemo, da $x_j \in W_j$ za vsak $j = 1, \ldots, k$.

            Enoličnost: Recimo, da $x_1 + \ldots + x_k = y_1 + \ldots + y_k$. Definiramo $z_i = y_i - z_i$. Pokažemo, da je $z_i = 0$, zato izračunamo $p_j(\Aa)z_i, \ i \neq j$ in $p_j(\Aa)z_j$. Nato v enakost (*) vstavimo $z_i$.
        \end{itemize}
        \item \colorbox{orange!30}{\textbf{Posledica.}} Ali je v neki bazi prostora $V$ endomorfizmu $\Aa$ pripada bločno diagonalna matrika $[A_j]$, kjer je za vsak $j$, $A_j$ matrika, ki pripada zožitvi $\Aa|_{W_j}$ glede na neko bazo podprostra $W_j$? 
        \item \colorbox{blue!30}{\textbf{Izrek.}} Za vsak $j = 1, \ldots, k$ naj bo $\Aa_j = \Aa|_{W_j}$. Kaj potem $\Delta_{\Aa_j}(\lambda)$ in $m_{\Aa_j}(\lambda)$?
        \begin{itemize}
            \item \colorbox{green!30}{\textbf{Dokaz.}} Vemo, da $\Delta_{\Aa_j}(\lambda) = \Delta_{A_j}(\lambda)$. Izračunamo $\Delta_\Aa(\lambda)$ (*).
            
            Pokažemo, da je $A_j$ ničla polinoma $(\lambda - \lambda_j)^{m_j}$. Dobimo obliko karakteristinega in minimalnega polinoma.

            S pomočjo (*) pokažemo enakost za karakteristični polinom. 
            
            S pomočjo polinoma $p(\lambda) = (\lambda - \lambda_1)^{s_j} \ldots (\lambda - \lambda_k)^{s_k}$ pokažemo enakost za minimalni polinom.
        \end{itemize}
        \item \colorbox{orange!30}{\textbf{Posledica.}} Kakšne velikosti so matrike $A_j$?
        \newpage
        \item \colorbox{orange!30}{\textbf{Posledica.}} Kadar endomorfizem $A \in L(V)$ se da diagonalizirati (ničle polinoma $m_{\Aa}(\lambda)$)? 
        \begin{itemize}
            \item \colorbox{green!30}{\textbf{Dokaz.}} $(\Rightarrow)$ Naj bodo $A$ pripadajoča diagonalna matrika in $\lambda_1, \ldots, \lambda_k$ vse različne lastne vrednosti. Definiramo $p(\lambda) = (\lambda - \lambda_1) \ldots (\lambda - \lambda_k)$ in pokažemo, da $m_{\Aa}(\lambda) \, | \, p(\lambda)$ in $p(\lambda) \, | \, m_{\Aa}(\lambda)$.
            
            $(\Leftarrow)$ Naj bo $\Aa_j = \Aa|_{W_j}$. Pokažemo, da $A_j$ se da diagonalizirati za vsak $j = 1, \ldots, k$.
        \end{itemize}       
    \end{itemize}

    \item Jordanova kanonična forma
    
    Vsaka matrika podobna neki bločno diagonalni matriki, kjer ima vsak diagonalni blok le eno lastno vrednost. CIlj tega poglavja je najti čim enostavnejšo obliko teh diagonalnih blokov.

    \item[$\circ$] Enodmorfizmi z eno samo lastno vrednostjo
    
    $V$ naj bo $n$-razsežen vektorski prostor nad $\CC$ in $\Aa \in L(V)$ naj bo endomorfizem z eno samo lastno vrednostjo $\rho$. Potem $\Delta_{\Aa}(\lambda) = (-1)^n (\lambda - \rho)^n$. Minimalni polinom je oblike $m_{\Aa} (\lambda) = (\lambda - \rho)^r$ za nek $r$. 
    
    Velja: $(\Aa - \rho \id)^r = 0$ in $(\Aa - \rho \id)^{r-1} \neq 0$. Od $\Aa$ odštejemo $\rho \id$ in definiramo endomorfizem $\mathcal{N} = \Aa - \rho \id$. Velja: $\mathcal{N}^r = 0$ in $\mathcal{N}^{r-1} \neq 0$.

    \begin{itemize}
        \item \colorbox{purple!30}{\textbf{Definicija.}} Nilpotenten endomorfizem. Indeks nilpotentnosti.
        \item \colorbox{yellow!30}{\emph{Opomba.}} Recimo, da je $N$ matrika za $\mathcal{N}$ glede na neko bazo. Kaj potem $N - \rho I$?
        \item \colorbox{blue!30}{\textbf{Lema.}} Naj bo $\mathcal{N}$ nilpotenten endomorfizem in naj bo $V_j = \ker \mathcal{N}^j$ za vsak $j \in \NN_0$. 3 lastnosti $V_j$.
        \begin{itemize}
            \item \colorbox{green!30}{\textbf{Dokaz.}} Enostavno.
        \end{itemize}     
        \item \colorbox{blue!30}{\textbf{Lema.}} Naj bo $j \geq 2$ in naj bo $B = \set{v_1, \ldots, v_k} \subseteq V_j$ linearno neodvisna množica za katero velja 
        
        $(\Lin B) \cap V_{j-1} = \set{0}$. Kaj potem velja za množico $\mathcal{N}(B) = \set{\mathcal{N}v_1, \ldots, \mathcal{N}v_k}$?
        \begin{itemize}
            \item \colorbox{green!30}{\textbf{Dokaz.}} Enostavno z uporabo prejšnje leme.
        \end{itemize}   
        \item \colorbox{orange!30}{\textbf{Posledica.}} Ali je vsebovanost v predprejšnje leme stroga?
        \begin{itemize}
            \item \colorbox{green!30}{\textbf{Dokaz.}} Z obratno indukcijo pokažemo, da za vsak $j = 1, \ldots, r$ obstaja linearno neodvisna množica $B_j \subseteq V_j$, da je $(\Lin B_j) \cap V_{j-1} = \set{0}$.
        \end{itemize} 
    \end{itemize}

    \item[$\circ$] Konstrukcija Jordanove baze
    \begin{itemize}
        \item Naj bo $\mathcal{N} \in L(V)$ nilpotenten endomorfizem z indeksom nilpotentnosti $r$ in naj bo $V_j = \ker \mathcal{N}^j$ za vsak $j \geq 0$.  Konstrukcija Jordanove baze za nilpotentne endomorfizme. 
        \begin{itemize}
            \item \colorbox{green!30}{\textbf{Ideja.}} Za vsak $j = 1, \ldots, r$ najdemo $B_j \subseteq V_j$, kjer $B_j = \set{v_1^{j}, \ldots, v_{s_j}^j}$ taka linearno neodvisna podmnožica, da velja $U_j = \Lin B_j$ in $V_j = U_j \oplus V_{j-1}$. To lahko naredimo z obratno indukcijo z uporabo lem.
        \end{itemize} 
        \item \colorbox{purple!30}{\textbf{Definicija.}} Naj bo $\mathcal{N} \in L(V)$ nilpotenten endomorfizem Jordanova baza in Jordanova matrika enodmorfizma $\mathcal{N}$. Jordanova celica reda $t$. Oznaka.
        \item \colorbox{yellow!30}{\emph{Opomba.}} Kakšne oblike je $J(\mathcal{N})$? Ali je $J(\mathcal{N})$ neka direktna vsota? 
        \item \colorbox{yellow!30}{\emph{Opomba.}} Kaj lahko povemo o velikosti Jordanovih celic? Čemu je enaka velikost največje celice? Čemu je enako število celic?
        \item \colorbox{purple!30}{\textbf{Definicija.}} Naj bo $\Aa \in L(V)$ endomorfizem z eno samo lastno vrednostjo $\rho$. Jordanova baza in Jordanova matrika endomorfizma $\Aa$. Jordanova celica za lastno vrednost $\rho$.
        \item \colorbox{yellow!30}{\emph{Opomba.}} Čemu je enako število Jordanovih celic? Čemu je enaka velikost največje celice?
        \item \colorbox{purple!30}{\textbf{Definicija.}} Naj bo $\Aa \in L(V)$ enodmorfizem. Jordanova baza in Jordanova matrika endomorfizma $\Aa$.
        \item \colorbox{purple!30}{\textbf{Definicija.}} Jordanova baza in Jordanova matrika matrike $A \in \CC^{n \times n}$.
        \item \colorbox{yellow!30}{\emph{Opomba.}} Ali je $J(A)$ neka direktna vsota?
        \item \colorbox{yellow!30}{\emph{Opomba.}} Naj bo $\lambda_i$ neka lastna vrednost matrike $A$. Koliko krat pojavi $\lambda_i$ da diagonali? Čemu je enaka velikost največje celice $J_t(\lambda_i)$ za fiksen $i$? Čemu je enako število celic $J_t(\lambda_i)$ za fiksen $i$?
        \item \colorbox{yellow!30}{\emph{Opomba.}} Ali je Jordanova baza enolična? Ali je Jordanova matrika enolična? \textbf{Jordanova kanonična forma.}
        \item \colorbox{yellow!30}{\emph{Opomba.}} Čemu je enako število Jordanovih celic v $J(\Aa)$, ki pripada lastne vrednosti $\lambda_i$, velikosti vsaj $t \times t$? 
        \item \colorbox{yellow!30}{\emph{Primer.}} Določi Jordanovo formo in Jordanovo bazo matrike $\begin{bmatrix}
            0 & 0 & 0 & 1 & 1 & 1 \\
            1 & 0 & 0 & -1 & 0 & 0 \\
            -1 & 0 & 0 & -1 & -3 & -3 \\
            1 & 0 & 1 & 1 & 2 & 1 \\
            1 & 0 & 0 & 0 & 2 & 2 \\
            -1 & 0 & 0 & 0 & -1 & -1
        \end{bmatrix}$
    \end{itemize}

    \newpage
    \item Funkcije matrik 
    \begin{itemize}
        \item Naj bo $q$ polinom in $A \in \CC^{n \times n}$ matrika. Naj bo $J(A)$ Jordanova matrika matrike $A$ in $P$ prehodna matrika, da je $A = P J(A) P^{-1}$. Kaj potem $q(A)$? Ali je matrika $q(J(A))$ bločno diagonalna? Kaj potrebno vedeti, da bi znali izračunati $q(J(A))$ in $q(A)$?
        \item Kako zgleda polinom v Jordanovi celici $q(J_t(\rho))$?
        \begin{itemize}
            \item \colorbox{green!30}{\textbf{Ideja.}} Jordanovo celico $J_t(\rho) \in \CC^{t \times t}$ pišemo v obliki $J_t(\rho) = \rho I + N$, kjer je $N$ nilpotentna Jordanova celica velikosti $t \times t$. Polinom $q$ zapišemo v bazi $\set{1, \lambda - \rho, (\lambda - \rho)^2, \ldots, (\lambda - \rho)^k}$ in izračunamo koeficiente $a_0, a_1, \ldots, a_k$ (kot pri razvoju v $n$-ti Taylorjevi polinom).
        \end{itemize}         
        \item \colorbox{purple!30}{\textbf{Definicija.}} Analitična funkcija $f: D \to \CC$ v Jordanovi celici: $f(J_t(\rho))$.
        \item \colorbox{purple!30}{\textbf{Definicija.}} Analitična funkcija $f: D \to \CC$ v matriki: $f(A)$.
        \item \colorbox{yellow!30}{\emph{Opomba.}} Ali se ta definicija ujema z definicijo polinoma v matriki? 
        
        Ali za vsako (lepo) funkcijo $f$ obstaja polinom $q$ (odvisen of $f$ in od $A$), da je $f(A) = q(A)$?

        Jordanova baza ni enolična. Ali je definicija dobra?
        \item \colorbox{blue!30}{\textbf{Trditev.}} Naj bo $\mathcal{F}$ algebra vseh kompleksnih funkcij. ki jih mogoče v okolici spektra fiksne matrike $A$ razviti v Taylorjevo vrsto, in naj bo $\Phi: \ \mathcal{F} \to \CC^{n \times n}, \ \Phi(f) = f(A)$. Kaj potem velja za $\Phi$?
        \item \colorbox{yellow!30}{\emph{Primer.}} Naj bo $J = \begin{bmatrix}
            0 & 1 & 0 \\
            0 & 0 & 1 \\
            0 & 0 & 0
        \end{bmatrix}, \ f(z) = \sin z, \ g(z) = \cos z$. Izračunaj $f(J), \ g(J)$ in $f^2(J) + g^2(J)$.
        \item \colorbox{blue!30}{\textbf{Izrek.}} Izrek o preslikave spektra.
        \begin{itemize}
            \item \colorbox{green!30}{\textbf{Dokaz.}} Zaradi enakosti $f(A) = Pf(J(A))P^{-1}$ dovolj, da izrek dokažemo za vse Jordanove matrike. Kaj je $f(J(A))$?
        \end{itemize} 
    \end{itemize}    
\end{enumerate}

\newpage
\

\newpage
\section{PROSTORI S SKALARNIM PRODUKTOM}
Prevzemimo $\FF \in \set{\RR, \CC}$.
\begin{enumerate}
    \item Osnovni lastnosti
    \begin{itemize}
        \item \colorbox{purple!30}{\textbf{Definicija.}} Naj bo $V$ vektorski prostor nad $\FF$. Skalarni produkt na $V$. Oznaka.
        \item \colorbox{purple!30}{\textbf{Definicija.}} Vektorski prostor s skalarnim produktom. Evklidski prostor. Unitaren prostor.
        \item \colorbox{blue!30}{\textbf{Lema.}} Kaj velja v vektorskem prostoru s skalarnim produktom (3 lastnosti)?
        \begin{itemize}
            \item \colorbox{green!30}{\textbf{Dokaz.}} Poračunamo po definiciji.
        \end{itemize} 
        \item \colorbox{orange!30}{\textbf{Posledica.}} Ali je skalarni produkt v evklidkem prosotru bilinearen funkcional? Kaj pa v unitarnem? 
        \item \colorbox{yellow!30}{\emph{Primer.}}         
        \begin{itemize}
            \item \textbf{Stadnardni skalarni produkt na $\RR^n$}.
            \item \textbf{Stadnardni skalarni produkt na $\CC^n$}.
            \item Naj bo $V = C[a,b]$. Skalarni produkt definiramo s predpisom $\left\langle f,g \right\rangle = \int_{a}^{b}f(x)g(x) \, dx$. Kje potrebujemo zveznost?
        \end{itemize}
        \item \colorbox{purple!30}{\textbf{Definicija.}} Norma vektorja $x$.        
        \item \colorbox{blue!30}{\textbf{Izrek.}} Neenakost Cauchy, Schwarz, Bunjakovskega.
        \begin{itemize}
            \item \colorbox{green!30}{\textbf{Dokaz.}} Naj bo $\alpha, \beta \in \FF$. Izračunamo $\left\langle \alpha x + \beta y, \ \alpha x + \beta y \right\rangle$ in vstavimo $\alpha = ||y||^2, \ \beta = -\left\langle x, y\right\rangle$.
        \end{itemize} 
        \item \colorbox{yellow!30}{\emph{Primer.}}         
        \begin{itemize}
            \item CSB v $\FF^n$ s standardnim skalarnim produktom.
            \item CSB v $C[a,b]$ s $\left\langle f,g \right\rangle = \int_{a}^{b}f(x)g(x) \, dx$.
        \end{itemize}
        \item \colorbox{blue!30}{\textbf{Trditev.}} 3 lastnosti norme.
        \begin{itemize}
            \item \colorbox{green!30}{\textbf{Dokaz.}} Račun + CSB.
        \end{itemize} 
        \item \colorbox{purple!30}{\textbf{Definicija.}} Normiran prostor.
        \item \colorbox{yellow!30}{\emph{Opomba.}} Ali je vektorski prostor s skalarnim produktom normiran, če normo definiramo kot $||x|| = \sqrt{\left\langle x, x\right\rangle }$?
        \item \colorbox{blue!30}{\textbf{Trditev.}} Pitagorov izrek in paralelogramska enakost v prostoru s skalarnom produktom.
        \begin{itemize}
            \item \colorbox{green!30}{\textbf{Dokaz.}} Račun.
        \end{itemize}
        \item \colorbox{yellow!30}{\emph{Opomba.}} Kadar lahko v normiran prostor definiramo skalarni produkt, da bo $||x|| = \sqrt{\left\langle x, x\right\rangle }$?
        \item \colorbox{purple!30}{\textbf{Definicija.}} Razdalja.
        \item \colorbox{blue!30}{\textbf{Trditev.}} 3 lastnosti za razdaljo.
        \item \colorbox{purple!30}{\textbf{Definicija.}} Metrični prostor.
        \item \colorbox{purple!30}{\textbf{Definicija.}} Naj bo $V$ evklidski prostor s skalarnim produktom in $x, y \in V$ neničelna vektorja. Kot med vektorjama $x$ in $y$.
        \item \colorbox{yellow!30}{\emph{Opomba.}} Zakaj definicija dobra?
        \item \colorbox{purple!30}{\textbf{Definicija.}} Pravokotna (ortogonalna) vektorja $x, y \in V$.
        \item \colorbox{blue!30}{\textbf{Trditev.}} Recimo, da je $\left\langle x, y\right\rangle = 0$ za vse $y$ iz $V$. Kaj potem $x$?
        \item \colorbox{purple!30}{\textbf{Definicija.}} Ortogonalna množica.
        \item \colorbox{blue!30}{\textbf{Trditev.}} Kaj lahko povemo o ortogonalni množici, ki ne vsebuje $0$?
        \begin{itemize}
            \item \colorbox{green!30}{\textbf{Dokaz.}} Enostavno.
        \end{itemize}
        \item \colorbox{orange!30}{\textbf{Posledica.}} Recimo, da $\dim V = n$ in $M \subseteq V$ ortogonalna. Koliko potem lahko ima $M$ elementov?
        \item \colorbox{purple!30}{\textbf{Definicija.}} Ortonormirana množica.
        \item \colorbox{blue!30}{\textbf{Trditev.}} Naj bo $M$ ortogonalna množica in $0 \notin M$. Kako dobimo ortonormirano množico?
        \item \colorbox{orange!30}{\textbf{Posledica.}} Recimo, da $\dim V = n$ in $M \subseteq V$ ortonormirana. Koliko potem lahko ima $M$ elementov?
    \end{itemize}
    \item Ortogonalizacija
    \begin{itemize}
        \item Naj bo $x_1, \ x_2$ linearno neodvisni. Določi ortogonalno množico $\set{y_1, y_2}$, da bo $\Lin(\set{x_1, x_2}) = \Lin(\set{y_1, y_2})$. 
        
        \textbf{Pravokotna projekcija} vektorja $x_2$ na vektor $y_1$.
        \item \colorbox{blue!30}{\textbf{Izrek.}} Gram-Schmidtova ortogonalizacija.
        \begin{itemize}
            \item \colorbox{green!30}{\textbf{Dokaz.}} Z indukcijo na število linearno neodvisnih vektorjev $x_1, \ldots, x_k$.             
            Vektor $y_{k+1}$ dobimo na podoben način kot v primeru za $k=2$.
        \end{itemize}
        \item \colorbox{orange!30}{\textbf{Posledica.}} Naj bodo $x_1, \ldots, x_m$ linearno neodvisni. Ali obstaja ortonormirana množica $M$, 
        
        da $\Lin \set{x_1, \ldots, x_m} = \Lin M$?
        \item \colorbox{orange!30}{\textbf{Posledica.}} Ali vsak končnorazsežen vektorski prostor ima ONB?
        \item \colorbox{orange!30}{\textbf{Posledica.}}  Ali lahko vsako ortonormirano množico dopolnimo do ONB (ortonormirane baze)?
        \newpage
        \item \colorbox{blue!30}{\textbf{Lema.}} Naj bo $x, y \in V$ in $\set{v_1, \ldots, v_n}$ ONB prostora. Kako razvijemo $x$ po tej bazi? 
        
        Čemu je enak $\left\langle x, y \right\rangle$?
        \begin{itemize}
            \item \colorbox{green!30}{\textbf{Dokaz.}} (1) Razvijemo $x$ po ONB in skalarno pomnožimo enačbo z $v_j, \ j \in \set{1, \ldots, n}$.
            
            (2) Izračunamo.
        \end{itemize}
        \item \colorbox{purple!30}{\textbf{Definicija.}} Izomorfnost vektorskih prostorov s skalarnim produktom.
        \item \colorbox{blue!30}{\textbf{Izrek.}} Naj bo $V$ $n$-razsežen vektorski prostor s skalarnim produktom. Ali je potem izomorfen $\FF^n$ s standardnim skalarnim produktom?
        \begin{itemize}
            \item \colorbox{green!30}{\textbf{Dokaz.}} Izberimo ONB bazo in definiramo izomorfizem $F: V \to \FF^n$ kot v poglavju o Bazi in razsežnosti. Treba preveriti enakost skalarnih produktov (uporabimo lemo).
        \end{itemize}
        \item \colorbox{yellow!30}{\emph{Opomba.}} Ali je izomorfnost vektorskih prostorov s skalarnim produktom ekvivalenčna relacija?
        \item \colorbox{orange!30}{\textbf{Posledica.}} Karakterizacija izomorfnosti vektorskih prostorov s skalarnim produktom.
        \item \colorbox{purple!30}{\textbf{Definicija.}} Pravokotni podmnožici. Pravokotna vsota podprostorov. Oznaka.
        \item \colorbox{blue!30}{\textbf{Trditev.}} Ali je pravokotna vsota direktna?
        \begin{itemize}
            \item \colorbox{green!30}{\textbf{Dokaz.}} Treba pokazati, da vsak vektor iz $x$ lahko na enoličen način zapišemo kot linearno kombinacijo elementov iz podprostorov $V_1, V_2, \ldots, V_k$.
        \end{itemize}
        \item \colorbox{purple!30}{\textbf{Definicija.}} Ortogonalni (pravokotni) komplement množice $M \subseteq V$. Oznaka.
        \item \colorbox{yellow!30}{\emph{Opomba.}} Recimo, da $M \subseteq N$. Kaj lahko povemo o vsebovanosti $N^\perp$ in $M^\perp$? Ali je $M^\perp$ vektorski podprostor?
        \item \colorbox{blue!30}{\textbf{Trditev.}} Naj bo $\set{v_1, \ldots, v_n}$ ONB za $V$, $1 \leq m \leq n$, $V_1 =  \Lin \set{v_1, \ldots, v_m}$ in $V_2 = \Lin \set{v_m, \ldots, v_{m+1}}$. Ali je $V$ pravokotna vsota prostorov $V_1$ in $V_2$? Kako sta med sabo povezana prostora $V_1$ in $V_2$?
        \begin{itemize}
            \item \colorbox{green!30}{\textbf{Dokaz.}} (1) Definicija pravokotne vsote.
            
            (2) Razvijemo vektor $z \in V_1^\perp$ po bazi z uporabo leme.
        \end{itemize}
        \item \colorbox{orange!30}{\textbf{Posledica.}} Naj bo $W \leq V$. Kako zapišemo $V$ kot pravokotno vsoto? Čemu je enako $W^{\perp \perp}$?
        \item \colorbox{yellow!30}{\emph{Opomba.}} Čemu je enako $W^{\perp \perp}$, če $W$ ni vektorski podprostor?
        \item \colorbox{orange!30}{\textbf{Posledica.}} Čemu je enaka $\dim V$ (ustrezna pravokotna vsota)?
    \end{itemize}
\end{enumerate}

\newpage
\section{LINEARNI FUNKCIONALI}
\begin{enumerate}
    \item Dualna baza
    \begin{itemize}
        \item Naj bo $V$ $n$-razsežen vektorski prostor nad $\FF$. \textbf{Dualni prostor} prostora $V$. Ali sta prostora $V$ in dualni prostor izomorfna? Oznaka.
        \item \colorbox{purple!30}{\textbf{Definicija.}} Dualna baza.
        \item \colorbox{blue!30}{\textbf{Trditev.}} Ali je dualna baza obstaja? Ali je enolična? Ali je res baza prostora $V^*$?
        \begin{itemize}
            \item \colorbox{green!30}{\textbf{Dokaz.}} Enoličnost: Naj bo $x \in V$ poljuben. Razvijemo ga po bazi prostora $V$ in poglejmo, kam se preslika s preslikavo $\phi_j$.
            
            Obstoj: Predpis za preslikavo $\phi_j$ dovimo od enoličnosti.

            Baza: Dovolj, da pokažemo, da je funkcionali linearno neodvisni.            
        \end{itemize}
        \item \colorbox{blue!30}{\textbf{Trditev.}} Naj bo $B_V$ baza prostora $V$ in $B_{V^*}$ njej dualna baza. Naj bo $x \in V$ in $f \in V^*$. 
        
        Čemu je enako $f(x)$?
        \begin{itemize}
            \item \colorbox{green!30}{\textbf{Dokaz.}} Račun.            
        \end{itemize}
    \end{itemize}
    
    \item Dualna preslikava
    \begin{itemize}
        \item \colorbox{purple!30}{\textbf{Definicija.}} Dualna preslikava. Oznaka.
        \item \colorbox{blue!30}{\textbf{Trditev.}} Ali je dualna preslikava linearna?
        \begin{itemize}
            \item \colorbox{green!30}{\textbf{Dokaz.}} Račun.            
        \end{itemize}
        \item \colorbox{blue!30}{\textbf{Trditev.}} Naj bosta $\Aa, \Bb \in L(U, V)$ in $\mathcal{C} \in L(V, W)$. Naštej 3 lastnosti dualne preslikave.
        \begin{itemize}
            \item \colorbox{green!30}{\textbf{Dokaz.}} Račun.            
        \end{itemize}
        \item \colorbox{blue!30}{\textbf{Trditev.}} Naj preslikavi $\Aa\in L(V, W)$ pripada matrika $A$ glede na bazi $B_v$ in $B_w$. Katera matrika pripada preslikavi $\Aa^d \in L(W^*, V^*)$ glede na dualni bazi baz $B_v$ in $B_w$?
        \begin{itemize}
            \item \colorbox{green!30}{\textbf{Dokaz.}} Izračunamo $(\Aa^d(\psi_i))v_k$ na dva načina.           
        \end{itemize}
        \item \colorbox{yellow!30}{\emph{Opomba.}} Izpelji 3 lastnosti transponiranja.
    \end{itemize}

    \item Reprezentacija linearnih funkcionalov na prostoru s skalarnim produktom
    
    Naj bo $\FF \in \set{\RR, \CC}$ in naj bo $V$ vektorski prostor s skalarnim produktom nad $\FF$. 

    Naj bo $z \in V$ poljuben vektor. Potem je $\left\langle x, z \right\rangle \in \FF$ za vsak $x \in V$, zato lahko definiramo preslikavo 
    
    $\phi_z: \ V \to \FF, \ \phi_z(x) = \left\langle x, z \right\rangle$.
    \begin{itemize}
        \item \colorbox{blue!30}{\textbf{Lema.}} Ali je $\phi_z$ linearen funkcional za vsak $z \in V$?
        \begin{itemize}
            \item \colorbox{green!30}{\textbf{Dokaz.}} Račun.            
        \end{itemize}
    \end{itemize}    
    Definiramo preslikavo $\Phi: \ V \to V^*, \ \Phi(z) = \phi_z$. 
    \begin{itemize}
        \item  \colorbox{blue!30}{\textbf{Izrek.}} Ali je $\Phi$ poševni izomorfizem (aditiven, poševno homogen, injektiven in surjektiven)?
        \begin{itemize}
            \item \colorbox{green!30}{\textbf{Dokaz.}} Aditivnost in poševna homogenost: Račun.
            
            Injektivnost: Pokažemo, da ima trivialno jedro.

            Surjektivnost: Če je $\phi$ ničlen, vzemimo $z = 0$. Sicer, zapišemo $V$ kot $V = U \oplus U^\perp$, kjer je $U = \ker \phi$. Nato zapišemo $v \in V$ kot $v = u + \alpha w$, kjer $w \in U^\perp$, to enačbo skalarno pomnožimo z $w$, tako dobimo $\alpha$, nato uporabimo $\phi$ na tej enakosti.
        \end{itemize}
        \item \colorbox{orange!30}{\textbf{Posledica.}} Rieszov izrek o reprezentaciji linearnih funkcionalov.
    \end{itemize}
\end{enumerate}

\newpage
\

\newpage
\section{LINEARNE PRESLIKAVE NA PROSTORIH S SKALARNIM PRODUKTOM}

\begin{enumerate}
    \item Adjungirana preslikava
    
    Vsi vektorski prostori bodo nad $\FF \in \set{\RR, \CC}$ in bodo imeli skalarni produkt.
    \begin{itemize}
        \item \colorbox{purple!30}{\textbf{Definicija.}} Naj bo $\Aa: U \to V$ linearna preslikava. Adjungirana preslikava preslikave $\Aa$.
        \item \colorbox{blue!30}{\textbf{Lema.}} Ali je inverz poševnega izomorfizma poševno homogen?
        \begin{itemize}
            \item \colorbox{green!30}{\textbf{Dokaz.}} Račun.            
        \end{itemize}
        \item \colorbox{blue!30}{\textbf{Trditev.}} Ali je $\Aa^*$ linearna preslikava?
        \begin{itemize}
            \item \colorbox{green!30}{\textbf{Dokaz.}} Aditivnost: Kompozitum additivnih preslikav.            
            Homogenost: Račun.
        \end{itemize}
        \item \colorbox{blue!30}{\textbf{Lema.}} Recimo, da za linearni preslikavi $\Bb, \mathcal{C}: W \to Z$ velja $\left\langle \Bb w, z\right\rangle = \left\langle \mathcal{C}w, z \right\rangle$ za vsaka $w \in W$ in $z \in Z$. Kaj potem?
        \begin{itemize}
            \item \colorbox{green!30}{\textbf{Dokaz.}} Račun.            
        \end{itemize}
        \item \colorbox{blue!30}{\textbf{Izrek.}} Naj bo $\Aa: U \to V$ linearna preslikava. Ali je potem $\Aa^*$ enolična preslikava iz $V$ v $U$ z neko zanimivo lastnostjo?
        \begin{itemize}
            \item \colorbox{green!30}{\textbf{Dokaz.}} Enakost: Račun. Enoličnost: Prejšnja lema.           
        \end{itemize}
        \item \colorbox{orange!30}{\textbf{Posledica.}} Naj bo $\Aa, \Bb \in L(U, V)$, $\mathcal{C} \in L(W, U)$. 5 lastnosti adjungirane preslikave.
        \begin{itemize}
            \item \colorbox{green!30}{\textbf{Dokaz.}} 1. točko dokažemo z uporabo izreka in lastnosti skalarnega produkta, ostale pa s pomočjo 1.~točke in leme.        
        \end{itemize}
        \item \colorbox{blue!30}{\textbf{Izrek.}} Naj bo $\Aa \in L(U, V)$, naj bo $B_U$ ONB za $U$ in $B_V$ ONB za $V$. Naj bo $A = A_{B_V, B_U}$.
        
        (1) Čemu je enako $a_{ij}$?

        (2) Kakšna matrika pripada preslikavi $A^*$ glede na bazi $B_V$ in $B_U$?

        \textbf{Hermitska transponiranka matrike $A$}.
        \begin{itemize}
            \item \colorbox{green!30}{\textbf{Dokaz.}} (1) Definicija matrika linearne preslikave + skalarni produkt s ustreznim vektorjem.
            
            (2) Lastnost skalarnega produkta + 1. točka.
        \end{itemize}
        \item \colorbox{yellow!30}{\emph{Opomba.}} Matriko $A \in \FF^{m \times n}$ identificiramo z linearno preslikavo $A: \FF^n \to \FF^m$. Ali je $A^* = A^H$?
        \item \colorbox{orange!30}{\textbf{Posledica.}} Kakšna zveza med $\rang \Aa^*$ in $\rang \Aa$?
        \item \colorbox{blue!30}{\textbf{Trditev.}} Naj bo $\Aa \in L(U, V)$. Kako potem lahko zapišemo $U$ kot pravokotno vsoto? Čemu je enaka $\im \Aa^*$?
        \begin{itemize}
            \item \colorbox{green!30}{\textbf{Dokaz.}} Dovolj, da pokažemo $\im \Aa^* = (\ker \Aa)^\perp$. ($\subseteq$) Enostavno. 
            
            V drugo smer vzemimo $u \in (\im \Aa^*)^\perp$ in pokažemo, da je v $\ker \Aa$ ter uporabimo $^\perp$.
        \end{itemize}
        \item \colorbox{blue!30}{\textbf{Trditev.}} Naj bo $\Aa \in L(V)$ endomorfizem in $U \leq V$. Karakterizacija invariantnosti $U$.
        \begin{itemize}
            \item \colorbox{green!30}{\textbf{Dokaz.}} $(\Rightarrow)$ Vzemimo $x \in U^\perp$ in $y \in U$. Izračunamo $\left\langle x, \Aa y \right\rangle$.
            
            $(\Leftarrow)$ Uporabimo smer $(\Rightarrow)$.
        \end{itemize}
    \end{itemize}
    \item Normalni endomorfizmi
    \begin{itemize}
        \item \colorbox{purple!30}{\textbf{Definicija.}} Normalen endomorfizem. Normalna matrika.
        \item \colorbox{yellow!30}{\emph{Opomba.}} Ali je normalnemu endomorfizmu v ONB pripada normalna matrika?
        \item \colorbox{blue!30}{\textbf{Trditev.}} Zadosten pogoj normalnosti endomorfizma (ONB).
        \begin{itemize}
            \item \colorbox{green!30}{\textbf{Dokaz.}} Izberimo ONB prostora $B$, da matrika $A = \Aa_{B,B}$ diagonalna. Kakšna matrika pripada endomorfizmu $\Aa^*$?
        \end{itemize}
        \item \colorbox{blue!30}{\textbf{Trditev.}} Karakterizacija normalnosti endomorfizma s skalarnim produktom.
        \begin{itemize}
            \item \colorbox{green!30}{\textbf{Dokaz.}} Račun.
        \end{itemize}
        \item \colorbox{orange!30}{\textbf{Posledica.}} 6 lastnosti normalnih endomorfizmov.
        \begin{itemize}
            \item \colorbox{green!30}{\textbf{Dokaz.}} (1) Norma: Prejšnja trditev.            
            (2) Jedra: 1. točka.
            (3) Račun.
            
            (4) Lastne vrednosti: 2. in 3. točki.
            (5) Spekter: Očitno sledi iz 4. točki.
            
            (6) Pravokotnost vektorjev: Izračunamo $\lambda_1 \left\langle x_1, x_2 \right\rangle$.
        \end{itemize}
        \item \colorbox{yellow!30}{\emph{Opomba.}} Ali je 5. točka velja tudi, če $\Aa$ ni normalen?
        \item \colorbox{blue!30}{\textbf{Izrek.}} Naj bo $\Aa \in L(V)$ normalen endomorfizem.
        
        (1) Ali če je $\FF = \CC$, potem $\Aa$ se da diagonalizirati v ONB?

        (2) Zadosten pogoj za diagonalizacijo v ONB, če je $\FF = \RR$.
        \begin{itemize}
            \item \colorbox{green!30}{\textbf{Dokaz.}} Obe točki hkrati z indukcijo na $\dim V$.
            
            Najdemo en lastni vektor $v_1$ in pokažemo, da podprostor $v_1^\perp$ invarianten za $\Aa$ in uporabimo i.p.
        \end{itemize}
        \item \colorbox{orange!30}{\textbf{Posledica.}} Karakterizacija normalnosti matrike.
        \item \colorbox{blue!30}{\textbf{Izrek.}} Schurov izrek.
        \begin{itemize}
            \item \colorbox{green!30}{\textbf{Ideja dokaz.}} Poiščemo Jordanovo bazo in na njej uporabimo Gram-Scmidtov postopek.
        \end{itemize}
    \end{itemize}

    \newpage
    \item Sebi adjungirani endomorfizmi
    \begin{itemize}
        \item \colorbox{purple!30}{\textbf{Definicija.}} Sebi adjungiran endomorfizem.
        \item \colorbox{yellow!30}{\emph{Opomba.}} Ali so sebi adjungirane endomorfizmi normalni?
        \item \colorbox{purple!30}{\textbf{Definicija.}} Simetrična matrika. Hermitska matrika.
        \item \colorbox{yellow!30}{\emph{Opomba.}} Kakšne matrike pripadajo sebi adjungiranim endomorfizmam glede na ONB, 
        
        če je $F = \RR$ ali $F = \CC$?
        \item \colorbox{blue!30}{\textbf{Trditev.}} Karakterizacija sebi adjungiranosti endomorfizma s skalarnim produktom.
        \begin{itemize}
            \item \colorbox{green!30}{\textbf{Dokaz.}} Enostavni račun.
        \end{itemize}
        \item \colorbox{blue!30}{\textbf{Trditev.}} Zadosten pogoj, da bi bil sebi adjungiran endomorfizem ničeln.
        \begin{itemize}
            \item \colorbox{green!30}{\textbf{Dokaz.}} Izračunamo $\left\langle \Aa(x+y), x+y\right\rangle $ in vstavimo $y = \Aa x$.
        \end{itemize}
        \item \colorbox{blue!30}{\textbf{Izrek.}} Ničla karakterističnega polinoma sebi adjungiranega endomorfizma.
        \begin{itemize}
            \item \colorbox{green!30}{\textbf{Dokaz.}} 1. $\FF = \CC$: Če je $\Delta_\Aa(\alpha) = 0$, potem je $\alpha$ lastna vrednost $\Aa$. Uporabimo posledico s 6. točkami.
            
            2. $\FF = \RR$: Glede na neko ONB $\Aa$ priredimo simetrično matriko $A \in \RR^{n \times n}$. Gledamo na to matriko kot na endomorfizem kompleksnega prostora $\CC^n$.
        \end{itemize}
        \item \colorbox{orange!30}{\textbf{Posledica.}} Kaj velja za spekter sebi adjungiranega endomorfizma?
        \item \colorbox{blue!30}{\textbf{Izrek.}} Karakterizacija sebi adjungiranosti endomorfizma z ONB.
    \end{itemize}

    \item[$\circ$] Pozitivno (semi)definitni endomorfizmi
    \begin{itemize}
        \item \colorbox{purple!30}{\textbf{Definicija.}} Pozitivno semidefiniten, pozitivno definiten, negativno semidefiniten, negativno definiten sebi adjungiran endomorfizem $\Aa \in L(V)$
        \item \colorbox{blue!30}{\textbf{Izrek.}} [Lastne vrednosti] Naj bo $\Aa \in L(V)$ sebi adjungiran.
        
        (1) Karakterizacija pozitivne semidefinitnosti.
        (2) Karakterizacija pozitivne definitnosti.

        (3) Karakterizacija negativne semidefinitnosti.
        (4) Karakterizacija negativne definitnosti.
        \begin{itemize}
            \item \colorbox{green!30}{\textbf{Dokaz.}} (1) $(\Rightarrow)$ Enostavno.
            
            $(\Leftarrow)$ Sebi adjungiran endomorfizem se da diagonalizirati v ONB.
        \end{itemize}
        \item \colorbox{blue!30}{\textbf{Izrek.}} [Koeficienti karakterističnega polinoma] Naj bo $\Aa \in L(V)$ sebi adjungiran.
        
        (1) Karakterizacija pozitivne definitnosti.
        (2) Karakterizacija pozitivne semidefinitnosti.

        (3) Karakterizacija negativne definitnosti.
        (4) Karakterizacija negativne semidefinitnosti.
        \item \colorbox{blue!30}{\textbf{Izrek.}} Sylvestrov izrek.
        \item \colorbox{yellow!30}{\emph{Opomba.}} Kje je pomemben Sylvestrov izrek?
        \item \colorbox{yellow!30}{\emph{Primer.}} Ali je dovolj gledati samo vodilni minorji v 2. točki Sylvestrovega izreka? 
    \end{itemize}

    \item Unitarni endomorfizmi
    \begin{itemize}
        \item \colorbox{purple!30}{\textbf{Definicija.}} Unitarni endomorfizem.
        \item \colorbox{yellow!30}{\emph{Opomba.}} Ali so unitarni endomorfizmi normalni?
        \item \colorbox{purple!30}{\textbf{Definicija.}} Unitarna matrika $A \in \CC^{n \times n}$. Ortogonalna matrika $A \in \RR^{n \times n}$
        \item \colorbox{yellow!30}{\emph{Opomba.}} Kakšne matrike pripadajo unitarnim endomorfizmam?
        \item \colorbox{blue!30}{\textbf{Trditev.}} Karakterizacija unitarnosti endomorfizma (5 ekvivalenc).
        \item \colorbox{blue!30}{\textbf{Trditev.}} Karakterizacija unitarnosti endomorfizma (4 ekvivalnce, slike množic).
        \item \colorbox{orange!30}{\textbf{Posledica.}} Karakterizacija unitarnosti matrike (za $\FF = \CC$) oz. ortogonalnosti (za $\FF = \RR$).
        \begin{itemize}
            \item \colorbox{green!30}{\textbf{Dokaz.}} Za stolpce: Standardna baza je ortonormirana, stolpci matrike pa so slike standardne baze.
            
            Za vrstice: Pokažemo, da $\Aa$ je unitarna $\Leftrightarrow$ $\Aa^*$ je unitarna.
        \end{itemize}
        \item \colorbox{blue!30}{\textbf{Trditev.}} Ali je množica unitarnih endomorfizmov grupa za kompozitum?
        \begin{itemize}
            \item \colorbox{green!30}{\textbf{Dokaz.}} Preverimo aksiome.
        \end{itemize}
        \item \colorbox{blue!30}{\textbf{Trditev.}} Absolutna vrednost lastnih vrednosti unitarnega endomorfizma.
        \begin{itemize}
            \item \colorbox{green!30}{\textbf{Dokaz.}} Izračunamo $||\Aa x||^2$ na dva načina.
        \end{itemize}
        \item \colorbox{blue!30}{\textbf{Izrek.}} Naj bo $V$ unitaren prostor in $\Aa \in L(V)$. Karakterizacija unitarnosti $\Aa$ z diagonalizacijo.
        \item \colorbox{yellow!30}{\emph{Primer.}} Poišči vse realne $2 \times 2$ ortogonalne matrike.
        \item \colorbox{purple!30}{\textbf{Definicija.}} Unitarno podobni matriki $A, B \in \CC^{n \times n}$. Ortogonalno podobni matriki $A, B \in \RR^{n \times n}$.
        \item \colorbox{yellow!30}{\emph{Opomba.}} Ali sta unitarna in ortogonalna podobnost ekvivalenčni relaciji?
        \item \colorbox{yellow!30}{\emph{Opomba.}} Kadar sta unitarno/ortogonalni matriki podobni (endomorfizmi)?
        \item \colorbox{blue!30}{\textbf{Izrek.}} (1) Kakšne matrike je unitarno podobna vsaka matrika $A \in \CC^{n \times n}$?
        
        (2) Karakterizacija noramlnosti matrike $A \in \CC^{n \times n}$.

        (3) Kadar matrika $A \in \CC^{n \times n}$ hermitska?

        (4) Karakterizacija simetričnosti matrike $A \in \RR^{n \times n}$.

        (5) Karakterizacija unitarnosti matrike $A \in \CC^{n \times n}$.
    \end{itemize}
\end{enumerate}

\newpage
\section{KVADRATNI FUNKCIONALI}
Prevzemimo $\FF = \RR$. Na prostoru $\RR^n$ vzemimo standardni skalarni produkt.
\begin{enumerate}
    \item Bilinearni in kvadratni funkcionali
    \begin{itemize}
        \item \colorbox{blue!30}{\textbf{Lema.}} Kako zgleda bilinearna forma $\Bb: V \times W \to \RR$, če si izberimo neki bazi prostorov $V$ in $W$ in vektorji zapišemo po komponentah? Ali preslikava definirana s tem predpisom bilinearna?
        \item \colorbox{blue!30}{\textbf{Trditev.}} Povezava med bilinearnimi formami in skalarnim produktom.
        \item \colorbox{purple!30}{\textbf{Definicija.}} Kvadraten funkcional na $V$ (kvadratna forma).
        \item \colorbox{yellow!30}{\emph{Primer.}} Ali je preslikava $K: \RR \to \RR, \ K(x) = x^2$ kvadratna forma?
        \item \colorbox{blue!30}{\textbf{Trditev.}} Kadar je preslikava $K: V \to \RR$ kvadratna forma?
        \begin{itemize}
            \item \colorbox{green!30}{\textbf{Dokaz.}} Po definiciji.
        \end{itemize}
        \item \colorbox{orange!30}{\textbf{Posledica.}} Naj bo $V$ vektorski prostor nad $\RR$ z bazo $\set{e_1, \ldots, e_n}$. Kadar je preslikava $K: V \to \RR$ kvadratna forma?
        \item \colorbox{blue!30}{\textbf{Trditev.}} Kadar je preslikava $K: \RR^n \to \RR$ kvadratna forma?
        \item \colorbox{purple!30}{\textbf{Definicija.}} Kongruentni matriki.
        \item \colorbox{blue!30}{\textbf{Izrek.}} Sylvestrov izrek.
        \item \colorbox{orange!30}{\textbf{Posledica.}} Naj bo $K: \RR^n \to \RR$ kvadraten funkcional. Ali obstaja neka lepa baza?
    \end{itemize}

    \item Uporaba v geometriji
    \item[$\circ$] Krivulje 2. reda 
    \begin{itemize}
        \item Krivulja 2. reda.
        \item Kako enačbo krivulje 2. reda zapišemo z uporabo kvadratnih form?
        \item Obravnavaj rešitve enačbe za $d=e=0$.
        \item Obravnavaj splošen primer, ko $d, e$ nista nujno $0$.
    \end{itemize}
    \item[$\circ$] Ploskve 2. reda 
    \begin{itemize}
        \item Ploskev 2. reda.
        \item Postopek "risanja" ploskve.
    \end{itemize}
\end{enumerate}

\end{document}