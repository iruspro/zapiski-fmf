\section{Prostori in preslikave}
\subsection{Metrični prostori}

Naj bo $(M, d)$ metrični prostor. Za $S \subseteq M, \ S \neq \emptyset$ definiramo funkcijo razdalje $d(-, S): M \to \R$ s predpisom $d(x, S) := \inf \setb{d(x,y)}{y \in S}$. Velja
\begin{itemize}
    \item $d(-, S)$ je dobro definirana zvezna preslikava.
    \item $\overline{S} = d(-, S)^*(\set{0})$. $S \text{ je zaprta} \liff S = d(-, S)^*(\set{0})$.
\end{itemize}

\subsection{Topologija}
\begin{itemize}
    \item Naj bo $A+ B = \setb{\text{in}_1(a)}{a \in A} \cup \setb{\text{in}_2(b)}{b \in B}$. $\T_{A+B}$ je topologija porojena z bazo $\set{\img{{\text{in}_1}}(\T_A)} \cup \set{\img{{\text{in}_2}}(\T_B)}.$ Torej $\T_{A+B} = \setb{\img{{\text{in}_1}}(U) \cup \img{{\text{in}_2}}(V)}{U \in T_A \land V \in \T_B}$. Injekciji sta odprti in zaprti vložitvi.
    \item $\B_S = \setb{[a, b)}{a \in \R, b \in \R, a < b}$ generira $\T_S$, ki jo imenujemo \df{Sorgenfreyeva premica}. Oznaka: $\R_S$.
\end{itemize}

\subsection{Slike in praslike}
\begin{definicija}
    Naj bo $f: A \to B$ preslikava.
    \begin{itemize}
        \item \df{Praslika} podmnožice $S \in B$ je $f^*(S) := \setb{x \in A}{f(x) \in S}$.
        \item \df{Slika} podmnožice $T \in A$ je $f_*(T) := \setb{y \in B}{\some{x \in T} f(x) = y}$.
    \end{itemize}
\end{definicija}

\begin{trditev}
    Praslike ohranjajo preseke in unije.
\end{trditev}

\begin{trditev}
    Slike ohranjajo unije. Če je preslikava injektivna, potem slika ohranja tudi preseke.
\end{trditev}

\begin{trditev}
    Naj bo $f: A \to B$ preslikava. Za $S \subseteq B$ velja $\invimg{f}(S^c) = (\invimg{f}(S))^c$.
\end{trditev}

\begin{trditev}
    Naj bo $f: A \to B$ preslikava, $S \subseteq A, \ T \subseteq B$. Velja:
    \begin{itemize}
        \item $S \subseteq \invimg{f}(\img{f}(S))$.
        \item $\img{f}(\invimg{f}(T)) \subseteq T$.
    \end{itemize}
\end{trditev}

\subsection{Baze in predbaze}

\begin{definicija}
    Naj bo $(X, \T)$ prostor, $x \in X$. Družina $\B_x \subseteq \T$, $\all{B \in \B_x} x \in B$ je \df{lokalna baza okolic} točke~$x$, če za vsako odprto okolico $U \in \T$ točke $x$, obstaja $B \in \B_x$, da $x \in B \subseteq U$.
\end{definicija}

\begin{opomba}
    Običajno prevzamemo, da so množice iz $\B_x$ okolice točke $x$. S tem lahko poskusimo si predstaviti, kako zgleda prostor okoli vsake točke.
\end{opomba}

\begin{definicija}
    Naj bo $(X, \T)$ prostor. Družina $\B \subseteq \T$ je \df{baza} topologije $\T$, če lahko vse elemente $\T$ zapišemo kot unije elementov $\B$.
\end{definicija}

\begin{trditev}
    Naj bo $\B$ baza prostora $(X, \T)$, $\B'$ baza prostora $(X', \T')$ in $f: X \to X'$ poljubna funkcija. Velja:
    \begin{enumerate}
        \item $U \subseteq X \text{ je odprta} \liff \all{x \in U} \some{B \in \B} x \in B \subseteq U$.
        \item $f \text{ je zvezna} \liff \all{B' \in \B'} \invimg{f}(B') \in \T$.
        \item $f \text{ je odprta} \liff \all{B \in \B} \img{f}(B) \in \T'$.
    \end{enumerate}
\end{trditev}

\subsubsection{Topologija generirana z bazo}

\begin{trditev}
    Naj bo $\B$ družina podmnožic $X$, ki ustreza pogojema
    \begin{enumerate}
        \item Unija elementov $\B$ je cel $X$ (vzemimo točko $x \in X$ in najdemo neko bazno množico $B$, da je $x \in B$).
        \item Presek dveh baznih je unija baznih (vzemimo dve bazni okolici in točko v preseku, ter najdemo poljubno bazno okolico točke, ki še vedno v preseku)
    \end{enumerate} 
\end{trditev}
    
\begin{definicija}
    \df{Produktna topologija} $\T_{X \times X'}$ je topologija, ki jo kot baza generirana $\setb{U \times U'}{U \in \T, U' \in \T'}$.
\end{definicija}

\begin{opomba}
    Če sta $\B$ in $\B'$ bazi $\T$ in $\T'$, potem družina $\setb{U \times U'}{U \in \B, U' \in \B'}$ generira produktno topologijo.
\end{opomba}

\begin{trditev}
    Naj bo $(X \times Y, \T_{X \times Y})$ produktna. Projekciji $\pr_x: X \times Y \to X, \ \pr_y: X \times Y \to Y$ sta zvezni in odprti.
\end{trditev}

\subsubsection{Predbaza}

\begin{trditev}
    Naj bo $\mathcal{P}$ poljubna družina podmnožic $X$. Če je $\mathcal{P}$ pokritje $X$, potem je $\T$ topologija, ki jo kot baza generirajo končni preseki elementov $\mathcal{P}$.
    Pravimo, da je $\mathcal{P}$ \df{predbaza topologije $\T$}.
\end{trditev}

\newpage
\begin{trditev}
    Naj bosta $(X, \T_X), (Y, \T_Y)$ prostora. Naj bo $\mathcal{P}$ predbaza $\T_Y$. Velja:
    $$\text{Funkcija } f: X \to Y \text{ je zvezna} \liff \all{B \in \mathcal{P}} \invimg{f}(B) \in \T_X.$$
\end{trditev}

\textbf{\textcolor{red}{Pozor!}} Odprtost funkcije $f$ v splošnem ne moremo testirati na predbaze. Saj slike ne ohranjajo preseke.

\begin{trditev}
    Naj bodo $X, Y, Z$ prostori. Velja:
    $$\text{Funkcija } f:X \to Y \times Z, \ f = (f_Y, f_Z) \text{ je zvezna} \liff f_Y, f_Z \text{ sta zvezni}.$$
\end{trditev}

\subsubsection{Aksiomi števnosti}

\begin{definicija}[1. aksiom števnosti]
    Naj bo $(X, \T)$ prostor. Vsaka točka $x \in X$ ima števno bazo okolic.
\end{definicija}

\begin{definicija}[2. aksiom števnosti]
    Naj bo $(X, \T)$ prostor. Obstaja kaka števna baza za topologijo $\T$.
\end{definicija}

\begin{trditev}
    Naj bo prostor $(X, \T)$ $1$-števen. Velja:
    \begin{enumerate}
        \item Za vsako množico $A \subseteq X$ je $\overline{A} = L(A) = \setb{x}{x \text{ je limita zaporedja v } A}$.
    \end{enumerate}
\end{trditev}

\subsubsection{Separabilnost}
\begin{definicija}
    Podmnožica $A$ je \df{povsod gosta} v $X$, če seka vsako odprto množico $X$, ali ekvivalentno, če je $\overline{A} = X$.
\end{definicija}

\begin{definicija}
    Prostor $(X, \T)$ je \df{separabilen}, če v $X$ obstaja števna gosta podmnožica.
\end{definicija}

\begin{trditev}
    $2$-števnost implicira separabilnost.
\end{trditev}

\begin{izrek}
    Metrični prostor $(X, d)$ je $2$-števen natanko takrat, ko v njem obstaja števna povsod gosta podmnožica.
\end{izrek}

\begin{opomba}
    V metričnih prostorih je $2$-števnost ekvivalentna separabilnosti, slednjo pa je pogosto lažje dokazati.
\end{opomba}

\subsection{Podprostori}

Naj bo $(X, \T)$ prostor, $A \subseteq X$. Definiramo $\T_A := \setb{A \cap U}{U \in T}$. $\T_A$ topologija na $A$.

\begin{definicija}
    Topologiji $\T_A$ pravimo \df{inducirana} topologija na $A$.
    Prostor $(A, \T_A)$ je \df{podprostor} prostora $(X, \T)$.
\end{definicija}

\begin{trditev}
    Naj bo $(X, \T)$ prostor in $(A, \T_A)$ njegov podprostor.
    \begin{enumerate}
        \item Če je $\B$ neka baza topologije $\T$, potem je $\B_A := \setb{A \cap B}{B \in \B}$ baza topologije $\T_A$. Analogna trditev velja za predbaze.
        \item Množica $B \subseteq A$ je zaprta v topologiji $\T_A$, če in samo če je $F = A \cap F$ za neko množico $F$, ke je zaprta v topologiji~$\T$.
        \item Veljajo formule: $\Cl_A B = A \cap \Cl_X B$, $\Int_A B \supseteq A \cap \Int_X B$, $\Fr_A B \subseteq A \cap \Fr_X B$.
    \end{enumerate}    
\end{trditev}

\begin{trditev}
    Naj bo $(X, \T)$ prostor in $(A, \T_A)$ njegov podprostor.
    \begin{enumerate}
        \item Podmnožica odprtega podprostora odprta natanko tedaj, ko je odprta v celem prostoru.
        \item Podmnožica zaprtega podprostora zaprta natanko tedaj, ko je zaprta v celem prostoru.
    \end{enumerate}
\end{trditev}

\subsubsection{Dednost}
\begin{definicija}
    Topološka lastnost je \df{dedna}, če iz prevzetka, da $(X, \T)$ ima to lastnost sledi, da jo imajo tudi vsi podprostori.
\end{definicija}

\subsubsection{Odsekoma definirane funkcije}

\begin{definicija}
    Naj bo $\set{X_\lambda}$ pokritje $X$. Za družino $f_\lambda: X_\lambda \to Y$ rečemo, da je \df{usklajena}, če je ${f_\lambda}_{|X_\lambda \cap X_\mu} =~{f_\mu}_{|X_\lambda \cap X_\mu}$ za poljubna indeksa $\lambda, \mu$.
\end{definicija}

\begin{trditev}
    Vsaka usklajena družina določa funkcijo $f: X \to Y$, za katero je $f_{|X_\lambda} = f_\lambda$.
\end{trditev}

\begin{lema}
    Naj bo $\set{X_\lambda}$ odprto pokritje $X$. Tedaj je $A \subseteq X$ odprta natanko tedaj, ko je $X_\lambda \cap A$ odprta v $X_\lambda$ za vse~$\lambda$.
\end{lema}

\begin{definicija}
    Pokritje $\set{X_\lambda}$ za $X$ je \df{lokalno končno}, če za vsako točko $x \in X$ obstaja okolica, ki seka le končno mnogo različnih $X_\lambda$.
\end{definicija}

\begin{lema}
    Naj bo $\set{X_\lambda}$ zaprto pokritje $X$, ki je lokalno končno. Tedaj je $A \subseteq X$ zaprta natanko tedaj, ko je $X_\lambda \cap A$ zaprta v $X_\lambda$ za vse~$\lambda$.
\end{lema}

\newpage
\begin{izrek}
    Naj bo $\set{X_\lambda}$ pokritje za $X$, ki je bodisi odprto bodisi lokalno končno in zaprto. Tedaj vsaka usklajena družina preslikav $f_\lambda: X_\lambda \to Y$ enolično določa preslikavo $f: X \to Y$, za katero je $f_{|X_\lambda} = f_\lambda$.
\end{izrek}

\begin{posledica}
    Naj bo $\set{X_\lambda}$ pokritje za $X$, ki je bodisi odprto bodisi lokalno končno in zaprto. Tedaj je funkcija $f: X \to Y$ zvezna natanko tedaj, ko so zvezne vse zožitve $f_{|X_\lambda}$.
\end{posledica}

\subsubsection{Vložitve}
\begin{definicija}
    Preslikava $f: X \to Y$ je \df{vložitev}, če je $f$ homeomorfizem med $X$ in $\img{f}(X)$ (glede na od $Y$ podedovano topologijo).
\end{definicija}

\begin{opomba}
    Potreben primer za homeomorfizem je injektivnost preslikave $f$. 
\end{opomba}

\begin{trditev}
    Naj bo $f: X \to Y$ injektivna preslikava.
    \begin{itemize}
        \item Če je $f(X)$ odprt v $Y$, potem je $f$ vložitev natanko tedaj, ko je preslikava $f: X \to Y$ odprta.
        \item Če je $f(X)$ zaprt v $Y$, potem je $f$ vložitev natanko tedaj, ko je preslikava $f: X \to Y$ zaprta.
    \end{itemize}
\end{trditev}

\section{Topološki lastnosti}
\subsection{Ločljivost}
\begin{definicija}
    Za topologjo $\T$ na množici $X$ pravimo, da \df{loči} podmnožico $A \subseteq X$ od pomnožice $B \subseteq X$, če obstaja $U \in \T$, za katero je $A \subseteq U$ in $B \cap U = \emptyset$.
\end{definicija}

\begin{definicija}
    Za topologjo $\T$ na množici $X$ pravimo, da \df{ostro loči} podmnožici $A \subseteq X$ in $B \subseteq X$, če obstajata $U, V \in \T$, za kateri je $A \subseteq U, \ B \subseteq V$ in $U \cap V = \emptyset$.
\end{definicija}

\subsubsection{Hausdorffova in Frechetova lastnosti}
\begin{definicija}
    Za prostor $(X, \T)$ pravimo, da je \df{Hausdorffov}, če $\T$ ostro loči vsaki dve različni točki $X$.
\end{definicija}

\begin{trditev}
    Ekvivalentne so naslednje izjave:
    \begin{enumerate}
        \item $X$ je Hausdorffov.
        \item $\all{x \in X} \bigcap_{U \in \mathcal{U}} \overline{U} = \set{x}$, kjer je $\mathcal{U}$ družina vseh okolic $x$ (ekvivalentno $x \neq y \lthen \some{U \in \T} x \in U \land y \notin \overline{U}$).
        \item \df{Diagonala} $\Delta = \setb{(x,x) \in X \times X}{x \in X}$ je zaprt podprostor produkta $X \times X$.
    \end{enumerate}
\end{trditev}

\begin{izrek}
    Naj bo prostor $Y$ Hausdorffov. Velja:
    \begin{enumerate}
        \item Vsaka končna podmnožica $Y$ je zaprta. Posebej, točke so zaprte.
        \item Točka $y$ je stekališče množice $A \subseteq Y$ natanko tedaj, ko vsaka okolica $Y$ vsebuje neskončno točk iz $A$.
        \item Zaporedje v $Y$ ima največ eno limito.
        \item Množica točk ujemanja $\setb{x \in X}{f(x) = g(x)}$ je zaprta v $X$ za poljubni preslikavi $f, g: X \to Y$.
        \item Če se preslikavi $f, g: X \to Y$ ujemata na neki gosti podmnožici $X$, potem je $f = g$.
        \item Graf preslikave $f: X \to Y$ je zaprt podprostor produkta $X \times Y$.
    \end{enumerate}
\end{izrek}

\begin{izrek}
    Naj bo prostor $X$ $1$-števen, prostor $Y$ pa Hausdorffov. Potem je funkcija $f: X \to Y$ zvezna natanko takrat, ko za vsako konvergentno zaporedje $(x_n)$ v $X$ velja $\lim f(x_n) = f(\lim x_n)$.
\end{izrek}

\begin{definicija}
    Prostor $(X, \T)$ je \df{Frechetov}, če $\T$ vsako točko $X$ loči od vsake druge točke $X$.
\end{definicija}

\begin{trditev}
    Prostor $X$ je Frechetov natanko tedaj, ko so vse enojčki zaprte.
\end{trditev}

\subsubsection{Regularnost in normalnost}
Ostrejše zahteve za ločljivost dobimo, če točke nadomestimo z zaprtimi množicami.
\begin{definicija}
    Prostor $(X, \T)$ je \df{regularen} če je Frechetov in če $\T$ ostro loči točke od zaprtih množic.
\end{definicija}

\begin{definicija}
    Prostor $(X, \T)$ je \df{normalen}, če je Frechetov in če $\T$ ostro loči disjunktne zaprte množice.
\end{definicija}

\begin{trditev}
    Vsak metričen prostor je normalen.
\end{trditev}

\begin{trditev}
    Zaprt podprostor normalnega prostora je normalen. 
\end{trditev}

\begin{izrek}[Izrek Tihonova]
    Prostor, ki je regularen in $2$-števen je normalen.
\end{izrek}

\begin{opomba}
    Iz izreka sledi, da je poljuben podprostor normalnega $2$-števnega prostora normalen. Podobno je tudi produkt $2$-števnih normalnih prostorov normalen.
\end{opomba}

\subsubsection{Aksiomi ločljivosti}
\begin{itemize}
    \item [] \df{X je $T_0$}: Za točki $x, x' \in X$ obstaja okolica ene izmed točk $x, x'$, ki jo loči od druge točke.
    \item [] \df{X je $T_1$}: Za točki $x, x' \in X$ obstaja okolica $x$, ki jo loči od $x'$ in obstaja okolica točke $x'$, ki jo loči od~$x$.
    \item [] \df{X je $T_2$}: Za točki $x, x' \in X$ obstajata okolici, ki ostro ločita $x$ in $x'$.
    \item [] \df{X je $T_3$}: Za točko $x \in X$ in zaprto množico $A \subseteq X$, ki ne vsebuje $x$, obstajata okolici, ki ostro ločita $x$ in $A$.
    \item [] \df{X je $T_4$}: Za disjunktni zaprti množici $A, B \subseteq X$ obstajata okolici, ki ostro ločita $A$ in $B$.
\end{itemize}

\begin{opomba}
    $T_1$ je Frechetova lastnost, $T_2$ je Hausdorffova lastnost. Regularnost je $T_1 + T_3$, normalnost je $T_1 + T_4$.
\end{opomba}

\begin{trditev}
    Prostor $X$ ima lastnost $T_3$ natanko tedaj, ko za vsak $x \in X$ in vsako odprto okolico $U$ za $x$ obstaja taka odprta množica $V$, da velja $x \in V \subseteq \overline{V} \subseteq U$.
\end{trditev}

\begin{trditev}
    Prostor $X$ ima lastnost $T_4$ natanko tedaj, ko za vsako zaprto podmnožico $A \subseteq X$ in vsako odprto okolico $U$ za $A$ obstaja taka odprta množica $V$, da velja $A \subseteq V \subseteq \overline{V} \subseteq U$.
\end{trditev}

\subsection{Povezanost}

\begin{definicija}
    \df{Razcep} prostora $X$ je zapis $X$ kot disjunktne unije dveh nepraznih odprtih množic. Če prostor dopušča kakšen razcep, pravimo, da je \df{nepovezan}, v nasprotnem primeru pa pravimo, da je \df{povezan}.
\end{definicija}

\begin{trditev}
    Naslednje trditve so ekvivalentne:
    \begin{enumerate}
        \item Prostor $X$ je nepovezan.
        \item Prostor $X$ je disjunktna unija dveh nepraznih zaprtih podmnožic.
        \item Obstaja prava, neprazna, odprto-zaprta podmnožica $A \subseteq X$.
        \item Obstaja surjektivna preslikava $f: X \to \set{0,1}^\text{disk}$. 
    \end{enumerate}
\end{trditev}

\begin{izrek}
    Naj bo $X \subseteq \R$. Velja: $$X \text{ je povezan} \liff X \text{ je interval}.$$
\end{izrek}

\begin{izrek}
    Zvezna slika povezanega prostora je povezan prostor.
\end{izrek}

\begin{izrek}[Izrek o vmesni vrednosti]
    Naj bo $X$ povezan prostor. Če je funkcija $X \to \R$ zvezna, potem $\img{f}(X)$ je interval.
\end{izrek}

\begin{posledica}
    Naj bo $X$ povezan prostor. Če je funkcija $X \to \R$ zvezna in zaloga vrednosti $f$ vsebuje pozitivne in negativne vrednosti, potem $f$ ima ničlo.
\end{posledica}

\begin{izrek}
    Naj bo $X$ prostor. Zadostni pogoji za povezanost:
    \begin{itemize}
        \item Naj bo $\set{A_\lambda}_{\lambda \in \Lambda}$ družina povezanih podmnožic $X$ in $\bigcap A_\lambda \neq \emptyset$. Potem $\bigcup A_\lambda$ je povezan.
        \item Topološki produkt povezanih prostorov je povezan.
        \item Če za poljubna $a,b \in X$ obstaja pot od $a$ do $b$, kjer \df{pot v $X$} je preslikava $\gamma: [0,1] \to X, \gamma(0) = a, \gamma(1) = b$, potem $X$ je povezan.
        \item Naj bo $A \subseteq X$ povezan. Če za $B \subseteq X$ velja, da $A \subseteq B \subseteq \overline{A}$, potem $B$ je povezan.
    \end{itemize}
\end{izrek}

\begin{definicija}
    Prostor $X$ je \df{povezan s potmi}, če med poljubnima $a, b \in X$ obstaja pot v $X$ od $a$ do $b$.
\end{definicija}

Lastnosti povezanosti s potmi so podobne lastnostim povezanosti:
\begin{itemize}
    \item Povezanost s potmi je topološka lastnost.
    \item Zvezna slika povezanega s potmi prostora je povezana s potmi.
    \item Prostor, ki je povezan s potmi, je povezan.
    \item Če je $\set{A_\lambda}_{\lambda \in \Lambda}$ družina povezanih s potmi podmnožic $X$ in $\bigcap A_\lambda \neq \emptyset$. Potem $\bigcup A_\lambda$ je povezan s potmi.
    \item Zaprtje s potmi povezanega prostora v splošnem \textbf{ni} povezano s potmi.
\end{itemize}

\subsubsection{Komponente}
\begin{definicija}
    \df{Komponenta} $C(x)$ točke $x \in X$ je unija vseh povezanih podmnožic $X$, ki vsebujejo $x$.
\end{definicija}
\begin{trditev}
    Lastnosti komponent:
    \begin{itemize}
        \item $x \in C(x)$.
        \item $C(x)$ je povezana.
        \item $C(x)$ je maksimalna povezana podmnožica v $X$ izmed vseh povezanih podmnožic $X$, ki vsebujejo $x$.
        \item $C(x)$ je zaprta v $X$ (zaprtje je povezano).
        \item $\all{x, y \in X} C(x) \cap C(y) = \emptyset \lor C(x) = C(y)$.
    \end{itemize}
\end{trditev}

\newpage
\begin{izrek}
    Komponente prostora $X$ so maksimalne povezane podmnožice v $X$ in določajo particijo $X$ na disjunktne zaprte podmnožice. Za poljubno preslikava $f: X \to Y$ leži slika vsake komponente $X$ v celoti v neki komponenti $Y$.
\end{izrek}

\begin{definicija}
    Prostor $X$ je \df{lokalno povezan}, če ima bazo iz povezanih množic.
\end{definicija}

\begin{opomba}
    Prostor $X$ je \df{lokalno povezan}, če vsaka okolica vsake točke ima manjšo povezano okolico.
\end{opomba}

\begin{trditev}
    Prostor $X$ je lokalno povezan natanko takrat, ko so komponente vsake odprte podmnožice $X$ odprte. Posebej so komponente vsakega lokalno povezanega prostora odprte.
\end{trditev}

\df{Komponenta za povezanost potmi} točke $x \in X$ označimo z $\widetilde{C}(x)$ in definiramo kot unijo vseh povezanih s potmi podmnožic $X$, ki vsebujejo $x$.

\begin{opomba}
    \ 
    \begin{itemize}
        \item Potni komponente ravno maksimalne s potmi povezane podmnožice $X$.
        \item Potni komponente razdelijo $X$ na disjunktne podmnožice.
    \end{itemize}

    V splošnem potni komponente niso zaprte: varšavski lok ima dve komponenti za povezanost s potmi od katerih je le ena zaprta. Za $X$ pravimo, da je \df{lokalno povezan s potmi}, če ima bazo iz s potmi povezanih množic.
\end{opomba}

\begin{opomba}
    Prostor $X$ je \df{lokalno povezan s potmi}, če vsaka okolica vsake točke ima manjšo povezano s potmi okolico.
\end{opomba}

\begin{izrek}
    Če je prostor $X$ lokalno povezan s potmi, potem njegove komponente za povezanost sovpadajo s komponentami za povezanost s potmi.
\end{izrek}

\begin{posledica}
    Če je $X$ lokalno povezan s potmi, potem 
    $$X \text{ je povezan} \liff X \text{ je povezan s potmi}.$$
\end{posledica}

\begin{posledica}
    Odprte podmnožice v $\R^n$ so povezane natanko tadaj, ko so povezane s potmi.
\end{posledica}

\subsection{Kompaktnost}
\begin{definicija}
    Prostor $X$ je \df{kompakten}, če ima vsako odprto pokritje $X$ končno podpokritje.
\end{definicija}

\begin{trditev}
    Prostor $X$ je kompakten, če za vsako pokritje z množicami iz neke baze topologije obstaja končno podpokritje.
\end{trditev}

\begin{opomba}
    Naj bo $X$ prostor, $A \subseteq X$. Dovolj je dokazati, da za vsako pokritje $A$ z množicami, ki so odprte v $X$, obstaja končno podpokritje.
\end{opomba}

\begin{izrek}
    V kompaktnem prostoru ima vsaka neskončna množica stekališče.
\end{izrek}

\begin{izrek}[Multiplikativnost kompaktnosti]
    Če sta $X, Y$ kompaktna prostora, potem $X \times Y$ kompakten
\end{izrek}

\begin{posledica}
    Če so $X_1, \ldots, X_n$ kompaktni, potem $X_1 \times \ldots \times X_n$ kompakten.
\end{posledica}

\begin{izrek}[Bolzano-Weierstrass]
    Vsako omejeno zaporedje v $\R^n$ ima konvergentno podzaporedje.
\end{izrek}

\begin{trditev}
    Kompaktna podmnožica metričnega prostora je omejena.
\end{trditev}

\begin{trditev}
    Zaprt podprostor kompaktnega prostora je kompakten.
\end{trditev}

\begin{trditev}
    V Hausdorffovem prostoru topologija ostro loči kompakte od točk.
\end{trditev}

\begin{trditev}
    Če je $X$ Hausdorffov in $K \subseteq X$ kompakten, potem je $K$ zaprt v $X$.
\end{trditev}

\begin{posledica}
    Vsak kompakten Hausdorffov prostor je normalen.
\end{posledica}

\begin{izrek}[Heine-Borel-Lebesgue]
    Podprostor v $\R^n$ je kompakten natanko tedaj, ko je zaprt in omejen.
\end{izrek}

\begin{opomba}
    Če je \(M\) metrični prostor, potem so zaprte krogle kompaktne natanko tedaj, ko velja Heine-Borelov izrek za ta prostor.
\end{opomba}

\begin{trditev}[Reformulacija definiciji kompaktnosti na zaprte množice]
    Prostor $X$ je kompakten natanko tedaj, ko v vsaki družini zaprtih podmnožic s praznim presekom obstaja končna podmnožica, katere presek je prazen.
\end{trditev}

\begin{izrek}[Cantorjev izrek]
    Naj bo $X$ kompakten in $F_1 \supseteq F_2 \supseteq \ldots$ padajoče zaporedje zaprtih nepraznih podmnožic. Potem je $\bigcap F_\lambda \neq \emptyset$.
\end{izrek}

\newpage
\begin{izrek}
    Zvezna slika kompakta je kompakt.
\end{izrek}

\begin{posledica}
    Če je $X \subseteq \R^n$ kompakt, potem je vsaka preslikava $f: X \to \R$ omejena in zavzame minimum in maksimum.
\end{posledica}

\begin{izrek}[Lebesgueova lema]
    Za vsako odprto pokritje \(\mathcal{U}\) metričnega kompakta $X$ obstaja tako imenovano \df{Lebesgueovo število} $\lambda = \lambda (\mathcal{U})$ z lastnostjo, da vsaka krogla s polmerom manjšim od $\lambda$ leži v celoti v nekem elementu~$\mathcal{U}$.
\end{izrek}

\begin{izrek}
    Naj bosta $X$ in $Y$ metrična prostora. Če je $X$ kompakten, potem je vsaka preslikava $f: X \to Y$ enakomerno zvezna.
\end{izrek}

\begin{izrek}
    Naj bo $X$ kompakten, $Y$ pa Hausdorffov prostor.
    \begin{itemize}
        \item Vsaka preslikava $f: X \to Y$ je zaprta.
        \item Vsaka injektivna preslikava \(f:  X \to Y\) je vložitev.
        \item Vsaka bijektivna preslikava \(f:  X \to Y\) je homeomorfizem.
    \end{itemize}
\end{izrek}

\begin{definicija}
    Odprta podmnožica $U \subseteq X$ je \df{relativno kompaktna}, če je njeno zaprtje \(\overline{U}\) kompaktno.
\end{definicija}

\begin{definicija}
    Prostor $X$ je \df{lokalno kompakten}, če ima bazo iz relativno kompaktnih množic.
\end{definicija}

\begin{izrek}
    Hausdorffov prostor $X$, v katerem ima vsaka točka kakšno kompaktno okolico, je lokalno kompakten. Posebej, vsak kompakten prostor je lokalno kompakten.
\end{izrek}

\begin{primer}
    \(\Q\) ni lokalno kompakten.
\end{primer}

\begin{izrek}
    Vsak lokalno kompaktni Hausdorffov prostor je regularen.
\end{izrek}

\section{Prostori preslikav}

\subsection{Topologije na prostorih preslikav}
Če želimo govoriti o konvergenci zaporedij preslikav med prostoroma $X$ in $Y$, moramo najprej množico vseh preslikav \(C(X, Y)\) opremiti s primerno topologijo. Opisali bomo konstrukcijo, ki na enovit način posploši oba najpomembnejša primera, točkasto in enakomerno konvergenco. Za \(A \subseteq X\) in \(U \subseteq Y\) označimo z $\left\langle A, U\right\rangle$ množico vseh preslikav, ki slikajo $A$ v $U$, tj. 
$$\left\langle A, U \right\rangle = \setb{f \in C(X,Y)}{\img{f}(A) \subseteq U}.$$
Naj bo $\T$ topologija na $C(X, Y)$, ki jo kot predbaza generira družina \(\mathcal{P} = \setb{\left\langle \set{x}, U \right\rangle}{x \in X, \, U \text{ odprta v } Y}.\) Tipična bazična okolica v tej topologiji je presel $\left\langle x_1, U_1 \right\rangle \cap \ldots \cap \left\langle x_n, U_n \right\rangle$, ki si ga lahko predstavljamo kot družino vseh funkcij, ki gredo po točkah $x_1, \ldots, x_n$ skozi predpisane prehode $U_1, \ldots, U_n$.

\begin{trditev}
    Naj bo \((f_n)\) zaporedje preslikav v \(C(X, Y)\). Funkcije \(f_n\) konvergirajo po točkah proti neki funkciji $f$ natanko takrat, ko zaporedje \((f_n)\) konvergira proti $f$ v topologiji $\T$. Zaradi tega \(\T\) imenujemo \df{topologija konvergnce po točkah}.
\end{trditev}

\begin{definicija}
    \df{Kompaktno-odprta topologija} na $C(X,Y)$ je topologija, ki jo generira predbaza \[\mathcal{P'} = \setb{\left\langle K, U \right\rangle}{K \text{ kompakten v } X, \, U \text{ odprta v } Y}.\]
    Prostor zveznih funkcij, opremljen s to topologijo, označimo $\widehat{C}(X,Y)$. 
\end{definicija}

Bazo kompaktno-odprte topologije tvorijo preseki predbazičnih množic. Delo s temi preseli je včasih precej nepregledno, zato pri preslikavah v metrični prostor $Y$ raje kompaktno-odprto topologijo podamo z bazo, ki je posplošitev baze iz krogel v metričnih prostorih. Denimo torej, da je $(Y, d)$ metrični prostor, ter za izbrano preslikavo $f: X \to Y$, kompakt \(K \subseteq X\) in $\epsilon > 0$ vpeljimo  
$$\left\langle f, K, \epsilon \right\rangle = \setb{g \in C(X, Y)}{d(f(x), g(x)) < \epsilon \text{ za vse } x \in K}.$$

\begin{trditev}
    Naj bo $Y$ metrični prostor. Družina $\mathcal{B} = \setb{\left\langle f, K, \epsilon \right\rangle}{f \in C(X, Y), K \text{ kompakt v } X, \epsilon>0}$ je baza kompaktno odprte topologije na $C(X, Y)$.
\end{trditev}

\begin{opomba}
    \(\mathcal{B}\) generira kompaktno-odprto topologijo.
\end{opomba}

Če je $K \subseteq K'$ ali $\epsilon > \epsilon'$, je $\left\langle f, K', \epsilon' \right\rangle \subseteq \left\langle f, K, \epsilon \right\rangle$, zato po potrebi lahko za bazo kompaktno-odprte topologije vzamemo le množice \(\left\langle f, K, \epsilon \right\rangle\) za velike kompakte $K$ ali majhne $\epsilon$. Posebej če je $X$ kompakten, se lahko omejimo na množice \(\left\langle f, X, \epsilon \right\rangle\), kar so ravno $\epsilon$ krogle v supremum metriki. Vidimo torej, da se za kompakten $X$ in metričen $Y$ kompaktno-odprta topologija ujema s topologijo enakomerne konvergence. Če pa $X$ ni kompakten, je glede na to topologijo konvergenca enakomerna le, kadar se omejimo na kompaktne podmnožice (tako kot pri Taylorjevih vrstah), zato ji pravimo tudi \df{topologija enakomerne konvergence na kompaktih}.

\ 

Kodomeno $Y$ vedno lahko gledamo kot podprostor v $\widehat{C}(X,Y)$.
\begin{trditev}
    Preslikava $c: Y \to \widehat{C}(X,Y)$, ki vsakemu $y \in Y$ priredi konstantno preslikavo $c_y$, ki vse točke $X$ preslika v~$y$, je vložitev. Če je prostor $Y$ Hausdorffov, je vložitev zaprta.
\end{trditev}

\begin{trditev}
    Prostor $\widehat{C}(X,Y)$ je Hausdorffov natanko tedaj, ko je $Y$ Hausdorffov, in regularen natanko tedaj, ko je $Y$ regularen.
\end{trditev}

\subsection{Preslikave na normalnih prostorih}
Obstajajo primeri Hausdorffovih in celo regularnih prostorov, na katerih so edine realne preslikave konstante. Zakaj potem na evklidskih prostorih obstaja tako veliko zveznih preslikav? Morda je to metrika: v metričnem prostoru $(X,d)$ je za poljuben $x \in X$ funkcija $d(x,-): X \to \R, \ x' \mapsto d(x, x')$ zvezna in nekonstantna. Še več, vrednosti funkcije lahko delno definiramo vnaprej: za poljubna $A, B \subseteq X$ postavimo $f(x) := \frac{d(x, A)}{d(x, A) + d(x, B)}$. Funkcija $f$ je definirana in zvezna, če je le imenovalec neničeln, to je takrat, ko $x$ ni hkrati v zaprtju $A$ in v zaprtju $B$. Če je $\overline{A} \cap \overline{B} = \emptyset$, potem zgornja formula definira preslikavo $f: X \to [0,1]$, ki ima vrednost $0$ na množici $A$ in vrednost $1$ na množici $B$. Lahko rečemo, da smo s preslikavo $f$ ločili $A$ in $B$.

\begin{izrek}[Urisonova lema]
    Hausdorffov prostor $X$ je normalen natanko takrat, ko za poljubni disjunktni neprazni zaprti podmnožici $A, B \subseteq X$ obstaja preslikava $f :(X, A, B) \to ([0,1], 0, 1)$.
\end{izrek}

\begin{izrek}[Urisonov metrizacijski izrek]
    Vsak normalen, 2-števen prostor je metrizabilen.
\end{izrek}

\begin{posledica}
    V $2$-števnih prostorih je metrizabilnost ekvivalentna regularnosti.
\end{posledica}

\begin{izrek}[Tietzejev razširitveni izrek]
    Naj bo $A$ zaprt podprostor normalnega prostora $X$ in $J \subseteq \R$ poljuben interval. Tedaj vsako preslikavo $f: A \to J$ lahko razširimo do preslikave $F: X \to J$.
\end{izrek}



\section{Splošno}

\begin{center}
    \begin{tabular}{ |c| c |c| c| c| c | }
    \hline
     Lastnost & Top & Mul & Ded & Primeri & Proti primeri\\ \hline
     Metrizabilnost & + & + & + & m. pr., $\T_{\text{disk}}$ & $\T_{\text{triv}}$, $\T_\text{kk}$ \\ \hline
     $1$-števnost & + & + & + & m. pr., $\R_S$, $\T_{\text{evkl}}$ &  \\ \hline
     $2$-števnost & + & + & + & $\T_{\text{evkl}}$ & $(X^\text{neštevna}, \T_{\text{disk}})$, $\R_S$ \\ \hline
     Separabilnost (baza) & + & + &  & $(\R^n, \T_{\text{evkl}})$, $\R_S$ & \\ \hline  
     $T_0$ (baza) & + & + & + & m. pr., $\R_S$ & $\T_{\text{triv}}$ \\ \hline
     $T_1$ (baza) & + & + & + & m. pr., $\T_{\text{kk}}$, $\R_S$ & $\T_{\text{triv}}$ \\ \hline
     $T_2$ (baza) & + & + & + & m. pr., $\R_S$ & $\T_{\text{triv}}$, $(X^\text{neskončna}, \T_{\text{kk}})$ \\ \hline
     $T_3$ & + & + & + & m. pr., $\T_{\text{triv}}$, $\R_S$ &   \\ \hline
     $T_4$ & + & &  & m. pr., $\T_{\text{triv}}$, $\R_S$ &  \\ \hline
     Regularnost & + & + & + & m. pr. & $\T_{\text{triv}}$ \\ \hline
     Normalnost & + & &  & m. pr. & $\T_{\text{triv}}$ \\ \hline
     Povezanost & + & + & & & \\ \hline
     Povezanost s potmi & + & + & & & \\ \hline
     Kompaktnost & + & + & & & \\ \hline
    \end{tabular}
\end{center}