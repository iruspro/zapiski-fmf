\section{Prostori in preslikave}
\subsection{Topološki prostori}
Topologija poda pojem \df{bližine} brez sklicevanja na implicitno funkcjio razdalja.

\begin{definicija}
    Naj bo $X$ množica. \df{Topologija} na množici $X$ je družina $\T \subseteq P(X)$, ki zadošča naslednjim pogojem:
    \begin{enumerate}
        \item[(T0)] $\emptyset \in \T$, $X \in \T$.
        \item[(T1)] Poljubna unija elementov $\T$ je element $\T$.
        \item[(T2)] Poljuben končen presek elementov $\T$ je element $\T$.
    \end{enumerate}
    Elemente $\T$ razglasimo za \df{odprte množice} v $X$.
\end{definicija}

\begin{opomba}
    Aksiom (T2) zadošča preveriti za poljubne dve množice in uporabiti indukcijo.
\end{opomba}

\begin{definicija}
    \df{Topološki prostor} je množica $X$ z neko topologijo $\T$. Pišemo: $(X, \T)$.
\end{definicija}

\begin{definicija}
    Če za topologiji $\T$ in $\T'$ na $X$ velja $\T \subseteq \T'$, pravimo, da je $\T'$ \df{finejša} od $\T$ in da je $\T$ \df{grobejša} od $\T'$. 
\end{definicija}

\begin{primer}[Topologija iz metrike]
    Naj bo $(X, d)$ metrični prostor. Definiramo $\T_d = \set{\text{vse možne unije odprtih krogel}}$. $\T_d$~je topologija, ki je \df{porojena (inducirana)} z metriko $d$.
\end{primer}

\begin{primer}
    V $\R^n$ lahko podamo evklidsko metriko. Pripadajočemu topološkemu prostoru pravimo \df{$n$-razsežni evklidski prostor}.
\end{primer}

\begin{definicija}
    Topološki prostor je \df{metrizabilen}, če je porojen z neko metriko.
\end{definicija}

\begin{primer}[Trivialna topologija]
    Naj bo $X$ poljubna množica. Definiramo $\T = \set{\emptyset, X}$. Potem $\T$ je topologija, rečemo ji \df{trivialna topologija}.

    Trivialna topologija ni metrizabilna, če ima $X$ vsaj $2$ elementa, ker v metričnem prostoru z množico z vsaj $2$ elementoma vedno lahko najdemo disjunktne odprte krogle.
\end{primer}

\begin{primer}[Diskretna topologija]
    Naj bo $X$ poljubna množica. Definiramo $\T = P(X)$. Potem $\T$ je topologija, rečemo ji \df{diskretna topologija}.
    Je metrizabilna, ker inducirana z diskretno metriko.
\end{primer}

\begin{definicija}
    Naj bo $(X, \T)$ topološki prostor in $A \subseteq X$. \df{Notranjost} množice $A$ je največji element topolgije~$\T$, ki je vsebovan v $A$. Oznaka: $\Int A$.
\end{definicija}

\begin{trditev}
    $\Int A$ je unija vseh odprtih množic, ki so vsebovane v $A$, torej $\Int A = \bigcup \setb{U \in \T}{U \subseteq A}$.
\end{trditev}

\begin{trditev}
    $\Int A$ je množica vseh \df{notranjih točk} množice $A$, tj. $\setb{x \in A}{\some{U \in T} x \in U \subseteq A}$.
\end{trditev}

\begin{definicija}
    Naj bo $(X, \T)$ topološki prostor in $A \subseteq X$. Množica $A$ je \df{zaprta}, če je $A^c = X - A \in \T$.
\end{definicija}

\begin{opomba}
    Lahko topologijo vpeljemo tudi tako, da predpišemo, katere množice so zaprte.

    Denimo, da je dana družina $Z$ podmnožic $X$, za katero velja:
    \begin{enumerate}
        \item[(T0)] $\emptyset \in Z$, $X \in Z$.
        \item[(T1)] Poljuben presek elementov $Z$ je element $Z$.
        \item[(T2)] Poljubna končna unija elementov $Z$ je element $Z$.
    \end{enumerate}
    Potem komplementi množic iz $Z$ tvorijo topologijo na $X$ in $Z$ je ravno družina zaprtih množic v tej topologiji.
\end{opomba}

\begin{primer}
    Naj bo $X$ poljubna množica. Družina $\T = \setb{U \subseteq X}{X - U \text{ je končna}} \cup \set{\emptyset}$ je topologija na $X$. Tej topologiji rečemo \df{topologija končnih komplementov $\T_{kk}$}. Velja:

    \begin{itemize}
        \item Topologija končnih komplementov je najmanjša topologja v kateri vse točke zaprte. 
        \item Če je $X$ končna, potem $\T_{\text{kk}}=\T_{\text{disk}}$ na $X$.
    \end{itemize}    
\end{primer}

\begin{definicija}
    Naj bo $(X, \T)$ topološki prostor in $A \subseteq X$. \df{Zaprtje} množice $A$ je najmanjša zaprta množica v $X$, ki vsebuje $A$. Oznaka: $\Cl A = \overline{A}$.
\end{definicija}

\begin{trditev}
    $\Cl A$ je presek vseh zaprtih množic, ki vsebujejo $A$, torej $\Cl A = \bigcap \setb{F \in \T}{A \subseteq F}$.
\end{trditev}

\begin{trditev}
    $\Cl A$ je množica vseh točk, vsaka okolica katerih seka $A$, tj. $\Cl A = \setb{x \in X}{\all{U \in \T} x \in U \lthen U \cap A \neq \emptyset}$.
\end{trditev}

\begin{primer}
    Velja:
    \begin{itemize}
        \item $\overline{A \cup B} = \overline{A} \cup \overline{B}$. \textbf{Dokaz.} Definicija zaprtja.
        \item $\overline{A \cap B} \subseteq \overline{A} \cap \overline{B}$. \textbf{Dokaz.} Definicija zaprtja in $(\N, \T_{kk})$.
    \end{itemize}
\end{primer}

\begin{definicija}
    Naj bo $(X, \T)$ prostor in $A \subseteq X$. Točka $x \in X$ je \df{mejna} točka $A$, če vsaka okolica $x$ seka $A$ in~$A^c$.
\end{definicija}

\begin{definicija}
    Naj bo $(X, \T)$ prostor in $A \subseteq X$. \df{Meja} množice $A$ je množica vseh mejnih točk $A$.
\end{definicija}

\begin{trditev}
    Naj bo $(X, \T)$ topološki prostor in $A \subseteq X$. Meja množice $A$ je $\Fr A = \Cl A - \Int A$.
\end{trditev}

\begin{opomba}
    Meja $A$ je vedno zaprta množica, saj $\Fr A = \Cl A - \Int A = \Cl A \cap (\Int A)^c$.
\end{opomba}

\subsection{Zvezne preslikave}
\subsubsection{Slike in praslike}
\begin{definicija}
    Naj bo $f: A \to B$ preslikava.
    \begin{itemize}
        \item \df{Praslika} podmnožice $S \in B$ je $f^*(S) := \setb{x \in A}{f(x) \in S}$.
        \item \df{Slika} podmnožice $T \in A$ je $f_*(T) := \setb{y \in B}{\some{x \in T} f(x) = y}$.
    \end{itemize}
\end{definicija}

\begin{trditev}
    Naj bo $f: A \to B$ preslikava.
    \begin{itemize}
        \item Praslike so monotone: $S \subseteq T \subseteq B \lthen \invimg{f}(S) \subseteq \invimg{f}(T)$.
        \item Slike so monotone: $X \subseteq Y \subseteq A \lthen \img{f}(X) \subseteq \img{f}(Y)$.
    \end{itemize}
\end{trditev}

\begin{trditev}
    Praslike ohranjajo preseke in unije.
\end{trditev}

\begin{trditev}
    Naj bo $f: A \to B$ in $T: I \to P(A)$. Tedaj je
    $$\img{f}(\bigcup_{i \in I} T_i) = \bigcup_{i \in I} \img{f}(T_i) \quad \text{in} \quad \img{f}(\bigcap_{i \in I} T_i) \subseteq \bigcap_{i \in I} \img{f}(T_i).$$
    Enakost velja, če je $f$ injektivna.
\end{trditev}

\begin{trditev}
    Naj bo $f: A \to B$ preslikava. Za $S \subseteq B$ velja $\invimg{f}(S^c) = (\invimg{f}(S))^c$.
\end{trditev}

\begin{trditev}
    Naj bo $f: A \to B$ preslikava, $S \subseteq A, \ T \subseteq B$. Velja:
    \begin{itemize}
        \item $S \subseteq \invimg{f}(\img{f}(S))$.
        \item $\img{f}(\invimg{f}(T)) \subseteq T$.
    \end{itemize}
\end{trditev}

\subsubsection{Zvezne preslikave}
\begin{definicija}    
    Naj bosta $(X, \T_X)$ in $(Y, \T_Y)$ topološka prostora. Preslikava $f: (X, \T_X) \to (Y, \T_Y)$ je \df{zvezna}, če je praslika vsake odprte množice odprta, tj. $\all{V \in \T_Y} \invimg{f}(V) \in \T_X$.
\end{definicija}

\begin{primer}
    Primeri zveznih preslikav.
    \begin{itemize}
        \item Vse zvezne funkcije v smislu metričnih prostorov so zvezne kot funkcije med porojenimi topologijami.
        \item Naj bo $f: (X, \T_X) \to (Y, \T_Y)$. 
        \begin{itemize}
            \item Naj bo $\T_Y$ trivilna topologija, potem $f$ je vedno zvezna.
            \item Naj bo $\T_X$ diskretna topologija, potem $f$ je vedno zvezna.
        \end{itemize}
        \item Naj bosta $(X, \T)$ in $(X, \T')$ topološka prostora. Funkcija $\id: X \to X'$ je zvezna natanko tedaj, ko $\T' \subseteq \T$.
        \item Če je $f: (X, \T_X) \to (Y, \T_Y)$ konstanta, tj. $\some{y_0 \in Y} \all{x \in X} f(x) = y_0$, potem je $f$ zvezna.
        \item Naj bo $f: (\R, \T_{\text{kk}}) \to (\R, \T_{\text{evkl}})$. Potem konstante so edine zvezne funkcije.         
        \item Naj bosta $X, Y$ neskončni množici, $d$ metrika na $Y$. Naj bo $f: (X, \T_{\text{kk}}) \to (Y, \T_d)$. Potem 
        $$f \text{ je zvezna} \liff f \text{ je konstanta}.$$
    \end{itemize}
\end{primer}

\begin{trditev}
    Kompozitum preslikav je preslikava.
\end{trditev}

\begin{proof}
    Definicija zveznosti.
\end{proof}

\begin{trditev}
    Naj bosta $X, Y$ prostora. Ekvivalentne so izjave za $f: X \to Y$:
    \begin{enumerate}
        \item $f$ je zvezna.
        \item Praslika z $f$ vsake zaprte množice je zaprta.
        \item $\all{A \subseteq X} \img{f}(\overline{A}) \subseteq \overline{\img{f}(A)}$.
    \end{enumerate}
\end{trditev}

\begin{proof}
    $(1) \liff (2)$. $\invimg{f}(A^c) = (\invimg{f}(A))^c$.

    $(2) \liff (3)$. LMN: $A \subseteq \invimg{f}(f(A)), \ f(\invimg{f}(B)) \subseteq B$. Monotonost $\img{f}, \invimg{f}$. STOP:
    $\invimg{f}(B) \text{ je zaprta} \liff \invimg{f}(B) = \overline{\invimg{f}(B)}$.
\end{proof}

\newpage
\subsection{Homeomorfizmi}
\begin{definicija}
    Naj bo $f: (X, \T_X) \to (Y, \T_Y)$ funkcija. Funkcija $f$ je \df{homeomorfizem}, če:
    \begin{itemize}
        \item $f$ je bijekcija.
        \item $\img{f}$ je bijekcija med $\T_X$ in $\T_Y$.
    \end{itemize}
\end{definicija}

\begin{definicija}
    Če obstaja homeomorfizem $f: (X, \T_X) \to (Y, \T_Y)$, potem rečemo, da sta prostora $X$ in $Y$ \df{homeomorfna}. Oznaka: $X \approx  Y$.
\end{definicija}

\begin{opomba}
    Homeomorfizem je ekvivalenčna relacija. To pomeni, da lahko dokažemo, da sta dva prostora homeomorfna, če pokažemo, da sta vsak od njih homeomorfen nekemu drugemu.
\end{opomba}

\begin{definicija}
    Naj bosta $(X, \T_X)$ in $(Y, \T_Y)$ topološka prostora.
    \begin{itemize}
        \item Funkcija $f: (X, \T_X) \to (Y, \T_Y)$ je \df{odprta}, če je slika vsake odprte množice odprta.
        \item Funkcija $f$ je \df{zaprta}, če je slika vsake zaprte množice zaprta.
    \end{itemize}     
\end{definicija}

\begin{trditev}
    Naslednje izjave o funkciji $f: X \to Y$ so ekvivalentne:
    \begin{enumerate}
        \item $f: X \to Y$ je homeomorfizem.
        \item $f$ je bijektivna, $f$ in $f^{-1}$ sta zvezni.
        \item $f$ je bijektivna, zvezna in odprta.
        \item $f$ je bijektivna, zvezna in zaprta.
    \end{enumerate}
\end{trditev}

\begin{proof}
    Očitne implikacije.
\end{proof}

\begin{primer} 
    Ali sta prostora $[0, 1) \cup \set{2}$ in $[0,1]$ homeomorfna? Ali inverz zvezne bijekcije vedno zvezen?
\end{primer}

\begin{primer}
    Pokaži, da vsak interval (končen ali neskončen) homeomorfen enemu izmed $[0,1], \ [0, 1), \ (0,1)$.

    Pokaži, da intervali $[0,1], \ [0, 1), \ (0,1)$ niso paroma homeomorfni  (kompaktnost, povezanost).
\end{primer}

\begin{primer}
    Najboljša izbira za homeomorfizem $(-1, 1) \approx \R$ je $f: (-1, 1) \to \R, \ f(x) = \frac{x}{1 - |x|}$ in $g: \R \to (-1, 1), \ g(x) = \frac{x}{1+|x|}$.
\end{primer}

\begin{definicija} Definiramo:
    \begin{itemize}
        \item $B^n := \setb{(x_1, \ldots, x_n) \in \R^n}{||\vec{x}|| \leq 1}$ je \df{enotska $n$-krogla}.
        \item $\mathring{B}^n := \setb{(x_1, \ldots, x_n) \in \R^n}{||\vec{x}|| < 1}$ je \df{odprta enotska $n$-krogla}.
        \item $S^{n-1} := \setb{(x_1, \ldots, x_n) \in \R^n}{||\vec{x}|| = 1}$ je \df{enotska $(n-1)$-sfera}.
    \end{itemize}
\end{definicija}

\begin{primer}
    Kako lahko homeomorfizem med $(0, 1)$ in $\R$ posplošimo do homeomorfizma med odprto kroglo $\mathring{B}^n$ in $\R^n$?
\end{primer}

\begin{primer}
    Zakaj sfera $S^{n-1}$ v $\R^n$ topološko bolj podobna $\R^{n-1}$ kot $\R^n$? \df{Stereografska projekcija}.

    Ali je $S^{n-1}$ \df{lokalno homeomorfna} prostoru $\R^{n-1}$?
\end{primer}

\begin{definicija}
    Prostore, ki so lokalno homeomorfne kakemu evklidskemu prostoru, imenujemo \df{mnogoterosti}.
\end{definicija}

Oglejmo nekaj preslikav, ki jim v določenem smislu malo manjka do homeomorfizma.

\begin{primer}
    Ali je $f: [0, 2 \pi] \to S^1, \ f(t) = e^{it}$ zvezna in bijektivna? Ali je zaprta? Kaj to pove o $f^{-1}$?
\end{primer}

\begin{primer}
    Ali je projekcija $\pr: \R^2 \to \R, \ \pr(x,y) = x$ zaprta?
\end{primer}

Pri dokazovanju, da dva prostora nista homeomorfna, igrajo kljucno vlogo topološke lastnosti.

\begin{definicija}
    \df{Topološka lastnost} je katerakoli lastnost prostora, ki se ohranja pri homeomorfizmih.
\end{definicija}

\begin{primer}
    Ali je omejenost in polnost topološki lastnosti?
\end{primer}

\begin{primer}
    Ali je možno, da $\R \approx \R^2$ (povezanost)? Ali enak sklep deluje za $\R^3$ in $\R^2$?
\end{primer}

\newpage
\subsection{Baze in predbaze}
\subsubsection{Baza in lokalna baza}
Lažje bi shajali, če bi zadoščalo preveriti zveznost ali odprtost preslikave na neki manjši in bolj obvladljivi družini podmnožic.

\begin{definicija}
    Naj bo $(X, \T)$ prostor, $x \in X$. Družina $\B_x \subseteq \T$, $\all{B \in \B_x} x \in B$ je \df{lokalna baza okolic} točke $x$, če za vsako odprto okolico $U \in \T$ točke $x$, obstaja $B \in \B_x$, da $x \in B \subseteq U$.
\end{definicija}

\begin{opomba}
    Običajno prevzamemo, da so množice iz $\B_x$ okolice točke $x$. S tem lahko poskusimo si predstaviti, kako zgleda prostor okoli vsake točke.
\end{opomba}

\begin{primer}
    V metričnem prostoru $(X, d)$ je $\set{K(x,r); \ r \in \Q}$ lokalna baza pri $x$.
\end{primer}

\begin{definicija}
    Naj bo $(X, \T)$ prostor. Družina $\B \subseteq \T$ je \df{baza} topologije $\T$, če lahko vse elemente $\T$ zapišemo kot unije elementov $\B$.
\end{definicija}

\begin{primer}
    Primeri baz. Ali so lahko baze majhne?
    \begin{itemize}
        \item Naj bo $(X, d)$ metrični prostor. Krogle so baza metrične topologije $\T_d$. Tudi dovolj je, če vzamemo samo majhne krogle, npr. z radijem $\frac{1}{n}$.
        \item Če vzemimo $(X, \T_\text{disk})$, potem vsaka baza vsebuje vse enojčke.        
    \end{itemize}
\end{primer}

\begin{trditev}
    Naj bo $(X, \T)$ topološki prostor. Velja:
    \begin{itemize}
        \item  Če je $\mathcal{B}$ baza topologije $\T$, potem je $B_x = \setb{B \in \mathcal{B}}{x \in B}$ je lokalna baza okolic $x$.
        \item Obratno: $\mathcal{B} := \bigcup_{x \in X} \B_x$ je baza topologije $X$.
    \end{itemize}    
\end{trditev}

\begin{proof}
    Definicija baze in lokalne baze okolic.
\end{proof}

Pri računanju je klučno, da za bazo izberimo dovolj majhne in obvladljive okolice ter da potem na bazi preskusimo, ali veljajo določene lastnosti.

\begin{trditev}
    Naj bo $\B$ baza prostora $(X, \T)$, $\B'$ baza prostora $(X', \T')$ in $f: X \to X'$ poljubna funkcija. Velja:
    \begin{enumerate}
        \item $U \subseteq X \text{ je odprta} \liff \all{x \in U} \some{B \in \B} x \in B \subseteq U$.
        \item $f \text{ je zvezna} \liff \all{B' \in \B'} \invimg{f}(B') \in \T$.
        \item $f \text{ je odprta} \liff \all{B \in \B} \img{f}(B) \in \T'$.
    \end{enumerate}
\end{trditev}

\begin{proof}
    Slike in praslike ohranjajo unije.
\end{proof}

\begin{primer}
    Ali je $f: S^1 \to S^1 \subseteq \C \text{ (enotska kompleksna števila)}, \ f(z) = z^2$ odprta?
\end{primer}

\subsubsection{Topologija generirana z bazo}

Pojem baze nam ponuja alternativno pot za vpeljavo topologije na neki množici: izberemo družino podmnožic $\B$ v $X$ in za odprte razglasimo vse podmnožice $X$, ki so unije elementov $B$. Vprašanje je, ali tako definirane odprte množice zadoščajo pogojem za topologijo? Izskaže se, da družine $\B$ ne smemo izbrati povsem poljubno.

\begin{trditev}
    \label{trd:baza}
    Naj bo $\B$ družina podmnožic $X$, ki ustreza pogojema
    \begin{enumerate}
        \item Unija elementov $\B$ je cel $X$ (pravimo, da je $\B$ \df{pokritje} za $X$).
        \item Za vse $B_1, B_2 \in \B$, za vse $x \in B_1 \cap B_2$, obstaja tak $B_x \in \B$, da $x \in B_x \subseteq B_1 \cap B_2$.
    \end{enumerate} 
    Tedaj je družina vseh možnih unij elementov iz $\B$ topologija na $X$.     
    Rečemo, da je \df{topologija $\T$ generirana z bazo~$\B$}.
\end{trditev}

\begin{proof}
    Enostavno preverimo lastnosti.
\end{proof}

\begin{primer}
    Naj bosta $(X, \T), (X', \T')$ topologiji. Ali obstaja kak naraven način za vpeljavo topologije na $X \times X'$?
\end{primer}
    
\begin{definicija}
    \df{Produktna topologija} $\T_{X \times X'}$ je topologija, ki jo kot baza generirana družina $\setb{U \times U'}{U \in \T, U' \in \T'}$.
\end{definicija}

\begin{opomba}
    Če sta $\B$ in $\B'$ bazi topologij $\T$ in $\T'$, potem se hitro prepričamo, da tudi družina $\setb{U \times U'}{U \in \B, U' \in \B'}$ generira produtno topologijo.
\end{opomba}

\begin{primer}
    Kaj generira produktno topologijo na $\R^2 = \R \times \R$? Ali je dobljena topologija ekvivalentna evklidske?
\end{primer}

\begin{trditev}
    Naj bo $(X \times Y, \T_{X \times Y})$ produktna topologija. Projekciji $\pr_x: X \times Y \to X, \ \pr_y: X \times Y \to Y$ sta zvezni in odprti.
\end{trditev}

\begin{primer} 
    Naj bo $\pr_1: \R^2 \to \R$. Ali je $\pr_1$ zaprta (graf funkcije $f(x) = \frac{1}{x}$)?    
\end{primer}

\subsubsection{Predbaza}

Včasih imamo neko družino podmnožic $X$, recimo $\PP$, ki jih želimo razglasiti za odprte. Kaj je najmanjša topologija $\T$ na $X$, ki vsebuje $\mathcal{P}$?

\begin{trditev}
    Naj bo $\mathcal{P}$ poljubna družina podmnožic $X$. Če je $\mathcal{P}$ pokritje $X$, potem je $\T$ topologija, ki jo kot baza generirajo končni preseki elementov $\mathcal{P}$.
    Pravimo, da je $\mathcal{P}$ \df{predbaza topologije $\T$}.
\end{trditev}

\begin{proof}
    Družina vseh končnih presekov elementov $\mathcal{P}$ ustreza pogoju (2) za bazo.
\end{proof}

\begin{opomba}
    Družina $\T$ vseh množic, ki jih lahko zapišemo kot unije končnih presekov množic iz $\PP$ je najmanjša topologija na $X$, ki vsebuje vse množice iz $\PP$.
\end{opomba}

Pojem predbaze nam ponuja zelo učinkovit način za definicijo topologije, ki ustreza nekemu pogoju. Poleg tega lahko na predbazi testiramo tudi zveznost preslikave.

\begin{trditev}
    Naj bosta $(X, \T_X), (Y, \T_Y)$ prostora. Naj bo $\mathcal{P}$ predbaza $\T_Y$. Velja:
    $$\text{Funkcija } f: X \to Y \text{ je zvezna} \liff \all{B \in \mathcal{P}} \invimg{f}(B) \in \T_X.$$
\end{trditev}

\begin{proof}
    Enostavno.
\end{proof}


\textbf{\textcolor{red}{Pozor!}} Odprtost funkcije $f$ v splošnem ne moremo testirati na predbaze. Saj slike ne ohranjajo preseke.

\begin{primer}
    Kako lahko na produktu prostorov $X \times Y$ definiramo najmanjšo topologijo, ki ustreza pogoju, da sta projekciji $\pr_x: X \times Y \to X, \ \pr_y: X \times Y \to Y$ zvezni? Čemu je enaka ta topologija?
\end{primer}

\begin{primer}
    Kako lahko definiramo topologijo za poljubne produkte (poljubno mnogo faktorjev)?
\end{primer}

\begin{trditev}
    Naj bodo $X, Y, Z$ prostori. Velja:
    $$\text{Funkcija } f:X \to Y \times Z, \ f = (f_Y, f_Z) \text{ je zvezna} \liff f_Y, f_Z \text{ sta zvezni}.$$
\end{trditev}

\begin{proof}
    $(\lthen)$ Komponenti sta kompozitum zveznih funkcij.

    $(\Leftarrow)$ Poglejmo prasliko predbaznih množic.
\end{proof}

\subsubsection{Aksiomi števnosti}

Baze lahko uporabimo za neko grobo oceno velikosti topološkega prostora in bogatstva njegove topologie.

\begin{definicija}[1. aksiom števnosti]
    Naj bo $(X, \T)$ prostor. Vsaka točka $x \in X$ ima števno bazo okolic.

    Rečemo, da je prostor $(X, \T)$ \df{$1$-števen}. 
\end{definicija}

\begin{primer}
    Metrični prostori so $1$-števni.
\end{primer}

\begin{definicija}[2. aksiom števnosti]
    Naj bo $(X, \T)$ prostor. Obstaja kaka števna baza za topologijo $\T$.

    Rečemo, da je prostor $(X, \T)$ \df{$2$-števen}. 
\end{definicija}

\begin{primer}
    $\R$ je $2$-števen, ker za bazo lahko vzamemo družino vseh intervalov z racionalnimi krajšči. 
    
    Podobno je tudi $\R^n$ $2$-števen.
\end{primer}

\begin{opomba}
    Očitno, da $2$-števnost implicira $1$-števnost.
\end{opomba}

\begin{primer}
    Neštevna množica z diskretno topologijo je $1$-števna (metrizabilna), ni pa $2$-števna, ker vsaka baza mora vsebovati vsi enojci.
\end{primer}

\begin{trditev}
    Naj bo prostor $(X, \T)$ $1$-števen. Velja:
    \begin{enumerate}
        \item Za vsako množico $A \subseteq X$ je $\overline{A} = L(A) = \setb{x}{x \text{ je limita zaporedja v } A}$.
        \item Funkcija $f: (X, \T_X) \to (Y, \T_Y) \text{ je zvezna} \liff f(L(A)) \subseteq L(f(A))$.
    \end{enumerate}
\end{trditev}

\begin{proof}
    (1) Konstruiramo zaporedje s pomočjo števne baze okolic.

    (2) \textcolor{red}{TODO.}
\end{proof}

\newpage
\subsubsection{Separabilnost}
\begin{definicija}
    Podmnožica $A$ je \df{povsod gosta} v $X$, če seka vsako odprto množico $X$, ali ekvivalentno, če je $\overline{A} = X$.
\end{definicija}

\begin{definicija}
    Prostor $(X, \T)$ je \df{separabilen}, če v $X$ obstaja števna gosta podmnožica.
\end{definicija}

\begin{primer}
    $\Q$ v $\R$ ali $\Q^n$ v $\R^n$.
\end{primer}

\begin{trditev}
    $2$-števnost implicira separabilnost.
\end{trditev}

\begin{proof}
    Iz vsake bazične okolice izberimo po eni točki
\end{proof}

\begin{opomba}
    Ali separabilnost in $1$-števnost implicira $2$-števnost? Ne. \textcolor{red}{(?)}
\end{opomba}

\begin{izrek}
    Metrični prostor $(X, d)$ je $2$-števen natanko takrat, ko v njem obstaja števna povsod gosta podmnožica.
\end{izrek}

\begin{proof}
    \textcolor{red}{TODO}
\end{proof}

\begin{opomba}
    V metričnih prostorih je $2$-števnost ekvivalentna separabilnosti, slednjo pa je pogosto lažje dokazati (ali ovreči).
\end{opomba}

\subsection{Podprostori}
\subsubsection{Podprostori}
Poljubno podmnožico metričnega prostora lahko opremimo z metriko tako, da funkcijo razdalje preprosto zožimo na točke podmnožice. Tako dobimo metrični podprostor. Podobno ravnamo pri topoloških prostorih in vzamemo zožitve odprtih množic na dano podmnožico.

\,

Naj bo $(X, \T)$ prostor, $A \subseteq X$. Definiramo $\T_A := \setb{A \cap U}{U \in T}$.

\begin{trditev}
    $\T_A$ topologija na $A$.
\end{trditev}

\begin{definicija}
    Topologiji $\T_A$ pravimo \df{inducirana} (oz. \df{podedovana}) topologija na $A$.
    Prostor $(A, \T_A)$ je \df{podprostor} prostora $(X, \T)$.
\end{definicija}

\begin{primer}
    Podprostori.
    \begin{itemize}
        \item Evklidska topologija na $\R$ je inducirana z evklidsko topologijo na $\R^2$.
        \item Vzemimo $\N \subseteq (\R, \T_\text{evkl})$. Kakšna je inducirana topologija?
        \item Naj bo $d$ metrika na $X$. Pokaži, da $(\T_d)_A = \T_{(d_{|A})}$.        
        \item Naj bo $B \subseteq A \subseteq (X, \T)$. Pokaži, da $\T_B = (\T_A)_B$, tj. podprostor podprostora spet podprostor.
    \end{itemize}
\end{primer}

\begin{opomba}
    Pri delu s podprostori moramo upoštevati, da so topološki pojmi praviloma odvisni od tega, v katerem prostoru jih gledamo.
\end{opomba}

\begin{trditev}
    Naj bo $(X, \T)$ prostor in $(A, \T_A)$ njegov podprostor.
    \begin{enumerate}
        \item Če je $\B$ neka baza topologije $\T$, potem je $\B_A := \setb{A \cap B}{B \in \B}$ baza topologije $\T_A$. Analogna trditev velja za predbaze.
        \item Množica $B \subseteq A$ je zaprta v topologiji $\T_A$, če in samo če je $F = A \cap F$ za neko množico $F$, ke je zaprta v topologiji~$\T$.
        \item Veljajo formule:
        \begin{itemize}
            \item $\Cl_A B = A \cap \Cl_X B$.
            \item $\Int_A B \supseteq A \cap \Int_X B$.
            \item $\Fr_A B \subseteq A \cap \Fr_X B$.
        \end{itemize}
    \end{enumerate}
    
\end{trditev}

\begin{proof}
    \textcolor{red}{TODO}
\end{proof}

\begin{primer}
    Naj bo $X = \R$ in $A = [0,1)$. Izračunaj notranjost, zaprtje in mejo $A$ v $\R$ in v $[0, 1)$.
\end{primer}

Množica, ki je odprta v podprostoru, ni nujno odprta v celem prostoru: na primer množica $A$, ki ni odprta v $X$, je vendarle odprta v sami sebi. Te teževa izognemo, če je $A$ odprta v $X$. Torej

\begin{trditev}
    Naj bo $(X, \T)$ prostor in $(A, \T_A)$ njegov podprostor.
    \begin{enumerate}
        \item Podmnožica odprtega podprostora odprta natanko tedaj, ko je odprta v celem prostoru.
        \item Podmnožica zaprtega podprostora zaprta natanko tedaj, ko je zaprta v celem prostoru.
    \end{enumerate}
\end{trditev}

\begin{primer}
    Naj bo $(X, \T)$ prostor, $A \subseteq X$, $i: (A, \T_A) \hookrightarrow (X, \T)$ inkluzija.
    \begin{itemize}
        \item Ali je $i$ zvezna?
        \item Kaj lahko povemo o topologiji $\T_A$?
        \item Ali je zožitev zvezne funkcije $f: X \to (Y, \T')$ zvezna? Kaj velja za razšeritev?
    \end{itemize}
\end{primer}

\subsubsection{Dednost}
\begin{definicija}
    Topološka lastnost je \df{dedna}, če iz prevzetka, da $(X, \T)$ ima to lastnost sledi, da jo imajo tudi vsi podprostori.
\end{definicija}

\begin{primer}
    Pokaži, da
    \begin{itemize}
        \item Diskretnost in trivialnost topologije sta dedni.
        \item $1$-števnost in $2$-števnost sta dedni.
        \item Metrizabilnost je dedna.
        \item Separabilnost ni dedna.
    \end{itemize}
\end{primer}

\begin{opomba}
    Odprt podprostor separabilnega podprostora je separabilen.
\end{opomba}

\subsubsection{Odsekoma definirane funkcije}
Funkcije pogosto definiramo odsekoma, na primer: funkcijo na $\R$ lahko  podamo z različno formulo na intervalih $[n, n+1]$ za $n \in \Z$. Tako podana funkcija je zvezna, če je zvezna na vsakem intervalu posebej in če se sosednji definiciji ujemata v skupnem krajišču. Pri splošnih topoloških prostorih lahko podamo zvezne predpise na vseh množicah nekega pokritja in poskrbimo, da se definicije na presekih ujemajo.

\begin{definicija}
    Naj bo $\set{X_\lambda}$ pokritje $X$. Za družino $f_\lambda: X_\lambda \to Y$ rečemo, da je \df{usklajena}, če je ${f_\lambda}_{|X_\lambda \cap X_\mu} =~{f_\mu}_{|X_\lambda \cap X_\mu}$ za poljubna indeksa $\lambda, \mu$.
\end{definicija}

\begin{trditev}
    Vsaka usklajena družina določa funkcijo $f: X \to Y$, za katero je $f_{|X_\lambda} = f_\lambda$.
\end{trditev}

\begin{lema}
    Naj bo $\set{X_\lambda}$ odprto pokritje $X$. Tedaj je $A \subseteq X$ odprta natanko tedaj, ko je $X_\lambda \cap A$ odprta v $X_\lambda$ za vse~$\lambda$.
\end{lema}

\begin{proof}
    \textcolor{red}{TODO}
\end{proof}

\begin{definicija}
    Pokritje $\set{X_\lambda}$ za $X$ je \df{lokalno končno}, če za vsako točko $x \in X$ obstaja okolica, ki seka le končno mnogo različnih $X_\lambda$.
\end{definicija}

\begin{primer}
    Ugotovi, ali je pokritje lokalno končno:
    \begin{itemize}
        \item Končna pokritja.
        \item $X = \R$, pokritje z $\setb{[n, n+1]}{n \in \Z}$.
        \item $X = \R$, pokritje z $\setb{[n, \infty)}{n \in \Z}$.
        \item $X = \R$, pokritje z enojčki.
    \end{itemize}
\end{primer}

\begin{lema}
    Naj bo $\set{X_\lambda}$ zaprto pokritje $X$, ki je lokalno končno. Tedaj je $A \subseteq X$ zaprta natanko tedaj, ko je $X_\lambda \cap A$ zaprta v $X_\lambda$ za vse~$\lambda$.
\end{lema}

Formuliramo osnovni izrek o odsekoma definiranih preslikavih.
\begin{izrek}
    Naj bo $\set{X_\lambda}$ pokritje za $X$, ki je bodisi odprto bodisi lokalno končno in zaprto. Tedaj vsaka usklajena družina preslikav $f_\lambda: X_\lambda \to Y$ enolično določa preslikavo $f: X \to Y$, za katero je $f_{|X_\lambda} = f_\lambda$.
\end{izrek}

\begin{proof}
    Dovolj, da preverimo zveznost.
\end{proof}

\begin{posledica}
    Naj bo $\set{X_\lambda}$ pokritje za $X$, ki je bodisi odprto bodisi lokalno končno in zaprto. Tedaj je funkcija $f: X \to Y$ zvezna natanko tedaj, ko so zvezne vse zožitve $f_{|X_\lambda}$.
\end{posledica}

\subsubsection{Vložitve}
Podprostori znanih prostorov so naravni vir mnogih zanimivih topoloških prostorov. Pogosto pa si želimo tudi kak abstrakto ustvarjen prostor obravnavati kot podprorostor nekega znanega prostora in v ta namen moramo preveriti, ali se abstraktno definirana topologja ujema s podedavano topologijo.

\begin{primer}
    Naj bo $X = \set{x_0, x_1, x_2, \ldots}$ števna množica, opremljena z diskretno topologijo. Naj bo $f: \N_0 \to \R$ preslikava. Ali je topologija na $X$ ujema z topologijo, ki jo slika $f$ podeduje od $\R$, če
    \begin{itemize}
        \item $f(x_k) = k$.
        \item $g(x_k) = \begin{cases}
            0; &k = 0 \\ \frac{1}{k}; & k \neq 0
        \end{cases}.$
    \end{itemize}
\end{primer}

\begin{opomba}
    Primer pokaže, da če nek prostor lahko enačimo s podmnožico nekega drugega prostora, to še ne pomeni, da ga lahko gledamo kot podprostor.
\end{opomba}

\begin{definicija}
    Preslikava $f: X \to Y$ je \df{vložitev}, če je $f$ homeomorfizem med $X$ in $\img{f}(X)$ (glede na od $Y$ podedovano topologijo).
\end{definicija}

\begin{opomba}
    Potreben primer za homeomorfizem je injektivnost preslikave $f$. 
\end{opomba}

Preslikava $f: X \to \img{f}(X)$ mora biti odprta (in zaprta) preslikava glede na podedovano topologijo na $\img{f}(X)$, kar pa v splošnem ni v zvezi z odprtostjo (ali zaprtostjo) preslikave $f: X \to Y$. Pomembna izjema so primeri, ko je $\img{f}(X)$ odprt ali zaprt v $Y$.

\begin{trditev}
    Naj bo $f: X \to Y$ injektivna preslikava.
    \begin{itemize}
        \item Če je $f(X)$ odprt v $Y$, potem je $f$ vložitev natanko tedaj, ko je preslikava $f: X \to Y$ odprta.
        \item Če je $f(X)$ zaprt v $Y$, potem je $f$ vložitev natanko tedaj, ko je preslikava $f: X \to Y$ zaprta.
    \end{itemize}
\end{trditev}

\begin{primer}
    Ugotovi, ali je vložitev:
    \begin{itemize}
        \item Preslikava, ki odprti interval navije na "`osmico'" (zaprtost).
        \item Preslikava $g: (-\pi, \pi) \to \C$, $g(x) = e^{ix}$ (odprtost).
        \item Naj bo $A$ zaprta in imejena podmnožica v $\R^n$. Kmalu bomo pokazali, da je vsaka preslikava $f: A \to \R^m$ zaprta, torej je vsaka injektivna preslikava iz $A$ v $\R^m$ vložitev.
    \end{itemize}
\end{primer}