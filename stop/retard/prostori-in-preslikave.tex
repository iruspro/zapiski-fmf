\section{Prostori in preslikave}
\subsection{Topološki prostori}
\begin{definicija}
    Naj bo $X$ množica. \df{Topologija} na množici $X$ je družina $\T \subseteq P(X)$, ki zadošča naslednjim pogojem:
    \begin{enumerate}
        \item[(T0)] $\emptyset \in \T$, $X \in \T$.
        \item[(T1)] Poljubna unija elementov $\T$ je element $\T$.
        \item[(T2)] Poljuben končen presek elementov $\T$ je element $\T$.
    \end{enumerate}
    Elemente $\T$ razglasimo za \df{odprte množice} v $X$.
\end{definicija}

\begin{opomba}
    Aksiom (T2) zadošča preveriti za poljubne dve množice in uporabit indukcijo.
\end{opomba}

\begin{definicija}
    \df{Topološki prostor} je množica $X$ z neko topologijo $\T$. Pišemo: $(X, \T)$.
\end{definicija}

\begin{primer}[Topologija iz metrike]
    Naj bo $(X, d)$ metrični prostor. Definiramo $\T_d = \set{\text{vse možne unije odprtih krogel}}$. $\T_d$~je topologija, ki je \df{porojena (inducirana)} z metriko $d$.
\end{primer}

\begin{definicija}
    Topološki prostor je \df{metrizabilen}, če je porojen z neko metriko.
\end{definicija}

\begin{primer}[Trivialna topologija]
    Naj bo $X$ poljubna množica. Definiramo $\T = \set{\emptyset, X}$. $\T$ je topologija, rečemo ji \df{trivialna topologija}.

    Trivialna topologija ni metrizabilna, če ima $X$ vsaj $2$ elementa, ker v metričnem prostoru z množico z vsaj $2$ elementoma vedno lahko najdemo disjunktne odprte krogle.
\end{primer}

\begin{primer}[Diskretna topologija]
    Definiramo $\T = P(x)$. $\T$ je topologija, rečemo ji \df{diskretna topologija}.

    Je metrizabilna, ker inducirana z metriko $d(x, x') = 
    \begin{cases}
        0, &x = x' \\
        1, &x \neq x'    
    \end{cases}.$ Ker krogle s polmerom manj kot $1$ vsebujejo le središče, sklepamo, da so vse enoelementne množice odprte. Potem so pa vse podmnožice $X$ odprte, saj jih lahko predstavimo kot unije enoelementnih.
\end{primer}

\begin{opomba}
    Topologija poda pojem \df{bližine} na implicitni način z pomočjo okolic. 
\end{opomba}

\begin{definicija}
    Naj bo $(X, \T)$ topološki prostor in $A \subseteq X$. \df{Notranjost množice $A$} je največji element topolgije~$\T$, ki je vsebovan v $A$. Oznaka: $\Int A$.
\end{definicija}

\begin{opomba}
    Zakaj je definicija smiselna?
    \begin{itemize}        
        \item Pogoj za notranjo točko: $x \in U \subseteq A$, kjer $U \in \T$.
        \item $\Int A$ je unija vseh odprtih množic, ki so vsebovane v $A$, torej $\Int A = \bigcup \set{U \in \T; \ U \subseteq A}$. Sledi, da je $\Int A$ največja odprta podmnožica $A$.
    \end{itemize}
\end{opomba}

\begin{definicija}
    Naj bo $(X, \T)$ topološki prostor in $A \subseteq X$. Množica $A$ je \df{zaprta}, če je $A^c = X - A \in \T$.
\end{definicija}

\begin{opomba}
    Lahko topologijo vpeljemo tudi tako, da predpišemo, katere množice so zaprte.

    Denimo, da je dana družina $Z$ podmnožic $X$, za katero velja:
    \begin{enumerate}
        \item[(T0)] $\emptyset \in Z$, $X \in Z$.
        \item[(T1)] Poljuben presek elementov $Z$ je element $Z$.
        \item[(T2)] Poljubena končna unija elementov $Z$ je element $Z$.
    \end{enumerate}
    Potem komplementi množic iz $Z$ tvorijo topologijo na $X$ in $Z$ je ravno družina zaprtih množic v tej topologiji.
\end{opomba}

\begin{primer}
    Če zahtevamo, da so točke $X$ zaprte množice, potem so tudi končne podmnožice $X$ so zaprte. Torej družina
    $$\set{\text{končme podmnožice X}} \cup \set{X}$$
    zadošča zahtevam (T1) in (T2). Torej komplementi
    $$\T = \set{U \subset X; X - U \text{ končna}} \cup \set{\emptyset}$$
    so topologija na $X$. Tej topologiji rečemo \df{topologija končnih komplementov $\T_{kk}$}.

    Topologija končnih komplementov je najmanjša topologja v kateri vse točke zaprte. 
    
    Če je $X$ končna, potem $\T_{\text{kk}}=\T_{\text{disk}}$ na $X$.
\end{primer}

\begin{definicija}
    Naj bo $(X, \T)$ topološki prostor in $A \subseteq X$. \df{Zaprtje množice $A$} je presek vseh zaprtih množic, ki vsebujejo $A$. Torej zaprtje množice $A$ je najmanjša zaprta množica v $X$, ki vsebuje $A$. Oznaka: $\Cl A = \overline{A}$.
\end{definicija}

\begin{primer}
    Velja:
    \begin{itemize}
        \item $\overline{A \cup B} = \overline{A} \cup \overline{B}$. \df{Dokaz.} Definicija zaprtja.
        \item $\overline{A \cap B} \subseteq \overline{A} \cap \overline{B}$. \df{Dokaz.} Definicija zaprtja in $(\N, \T_{kk})$.
    \end{itemize}
\end{primer}

\begin{definicija}
    Naj bo $(X, \T)$ topološki prostor in $A \subseteq X$. \df{Meja množice $A$} je $\Fr A = \Cl A - \Int A$.
\end{definicija}

\begin{opomba}
    Meja $A$ je vedno zaprta množica, saj $\Fr A = \Cl A - \Int A = \Cl A \cap (\Int A)^c$.
\end{opomba}

\subsection{Zvezne preslikave}
\begin{definicija}    
    Naj bosta $(X, \T_X)$ in $(Y, \T_Y)$ topološka prostora. Preslikava $f: (X, \T_X) \to (Y, \T_Y)$ je \df{zvezna}, če je praslika vsake odprte množice odprta, tj. če iz $V \in \T_Y$ sledi $\invimg{f}(V) \in \T_X$.
\end{definicija}

\begin{primer}
    Primeri zveznih preslikav.
    \begin{enumerate}
        \item Vse zvezne funkcije v smislu metričnih prostorov so zvezne kot funkcije med porojenimi topologijami.
        \item Naj bo $f: (X, \T_X) \to (Y, \T_Y)$. 
        \begin{enumerate}
            \item Naj bo $\T_Y$ trivilna topologija, potem $f$ je vedno zvezna.
            \item Naj bo $\T_X$ diskretna topologija, potem $f$ je vedno zvezna.
        \end{enumerate}
        \item Naj bosta $(X, \T)$ in $(X, \T')$ topološka prostora. Funkcija $\id: X \to X'$ je zvezna natanko tedaj, ko $\T' \subseteq \T$.
        \item Če je $f: (X, \T_X) \to (Y, \T_Y)$ konstanta, tj. $\some{y_0 \in Y} \all{x \in X} f(x) = y_0$, potem je $f$ zvezna.
        \item Naj bo $f: (\R, \T_{\text{kk}}) \to (\R, \T_{\text{evkl}})$. Potem konstante so edine zvezne funkcije. 
        
        \df{Dokaz.} V $(X, \T_{\text{kk}})$ ni disjunktnih nepraznih odprtih množic, če je $X$ neskončna.
        
        \df{Splošneje.} Naj bosta $X, Y$ neskončni množici, $d$ metrika na $Y$. Naj bo $f: (X, \T_{\text{kk}}) \to (Y, \T_d)$. Potem 
        $$f \text{ je zvezna} \liff f \text{ je konstanta}.$$
    \end{enumerate}
\end{primer}

Uvedemo neke oznake in okrajšave:
\begin{itemize}
    \item Naj bosta $(X, \T_X), (Y, \T_Y)$ topološka prostota. Označimo z $C((X, \T_X), (Y, \T_Y))$ množico vseh zveznih preslikav $C(X, Y)$. Tudi $C(X) = C(X, \R)$.
    \item \df{Prostor $X$} je množica z neko topologijo.
    \item \df{Preslikava} je zvezna funkcija.
\end{itemize}

\begin{trditev}
    Kompozitun preslikav je preslikava.
\end{trditev}

\begin{proof}
    Definicija zveznosti.
\end{proof}

\begin{trditev}
    Naj bosta $X, Y$ prostora. Ekvivalentne so izjave za $f: X \to Y$:
    \begin{enumerate}
        \item $f$ je zvezna.
        \item Praslika z $f$ vsake zaprte množice je zaprta.
        \item $f(\overline{A}) \subseteq \overline{f(A)}$.
    \end{enumerate}
\end{trditev}

\begin{proof}
    $(1) \liff (2)$. $\invimg{f}(A^c) = (\invimg{f}(A))^c$.

    $(2) \liff (3)$. LMN: $A \subseteq \invimg{f}(f(A)), \ f(\invimg{f}(B)) \subseteq B$. Monotonost $\img{f}, \invimg{f}$. STOP:
    $\invimg{f}(B) \text{ je zaprta} \liff \invimg{f}(B) = \overline{\invimg{f}(B)}$.
\end{proof}

\subsection{Homeomorfizmi}
\begin{definicija}
    Naj bo $f: (X, \T_X) \to (Y, \T_Y)$ funkcija. Funkcija $f$ je \df{homeomorfizem}, če:
    \begin{itemize}
        \item $f$ je bijekcija.
        \item $\img{f}$ je bijekcija med $\T_X$ in $\T_Y$, tj. $\all{U \in \T_X} \img{f}(U) \in \T_Y \land \all{V \in \T_Y} \invimg{f}(V) \in \T_X$.
    \end{itemize}
\end{definicija}

\begin{opomba}
    Pogoj $\all{V \in \T_Y} \invimg{f}(V) \in \T_X$ je ravno zveznost funkcije $f$.
\end{opomba}

\begin{definicija}
    Če obstaja homeomorfizem $f: (X, \T_X) \to (Y, \T_Y)$, potem rečemo, da sta prostora $X$ in $Y$ \df{homeomorfna}. Oznaka: $X \approx  Y$.
\end{definicija}

\begin{opomba}
    Homeomorfizem je ekvivalenčna relacija. To pomeni, da lahko dokažemo, da sta dva prostora homeomorfna, če pokažemo, da sta vsak od njih homeomorfen nekemu drugemu.
\end{opomba}

\begin{definicija}
    Funkcija $f: (X, \T_X) \to (Y, \T_Y)$ je \df{odprta}, če je slika vsake odprte množice odprta. Funkcija $f$ je \df{zaprta}, če je slika vsake zaprte množice zaprta.
\end{definicija}

\begin{trditev}
    Naslednje izjave o funkciji $f: X \to Y$ so ekvivalentne:
    \begin{enumerate}
        \item $f: X \to Y$ je homeomorfizem.
        \item $f$ je bijektivna, $f$ in $f^{-1}$ sta zvezni.
        \item $f$ je bijektivna, zvezna in odprta.
        \item $f$ je bijektivna, zvezna in zaprta.
    \end{enumerate}
\end{trditev}

\begin{proof}
    Očitne implikacije.
\end{proof}

\begin{primer} 
    Ali sta prostora $[0, 1) \cup \set{2}$ in $[0,1]$ homeomorfna? Ali inverz zvezne bijekcije vedno zvezen?
\end{primer}

\newpage
\begin{trditev}
    Nekatere zvezne funkcije so avtomatično zaprte (oz. odprte):
    \begin{itemize}
        \item $f^{\text{zv}}: X^{\text{komp}} \to Y^{\text{metr}}$ je vedno zaprta.
        \item Projekcija $X \times Y \to X$ je vedno odprta.
        \item Preslikave $f: \R^n \to \R^n$, ki so gladke in imajo neničelni odvod, so vedno odprte.
    \end{itemize}
\end{trditev}

\begin{primer}
    Pokaži, da vsak interval (končen ali neskončen) homeomorfen enemu izmed $[0,1], \ [0, 1), \ (0,1)$.

    Pokaži, da intervali $[0,1], \ [0, 1), \ (0,1)$ niso paroma homeomorfni.
\end{primer}

\begin{definicija}
    \df{Topološka lastnost} je katerakoli lastnost prostora, ki se ohranja pri homeomorfizmih.
\end{definicija}

\begin{primer}
    Ali je kompaktnost/omejenost/polnost topološka lastnost?
\end{primer}

Upeljamo oznake:
\begin{itemize}
    \item $B^n := \set{\vec{x} \in \R^n; \ ||\vec{x}|| \leq 1}$ je \df{enotska $n$-krogla}.
    \item $\mathring{B}^n := \set{\vec{x} \in \R^n; \ ||\vec{x}|| < 1}$ je \df{odprta enotska $n$-krogla}.
    \item $S^{n-1} := \set{\vec{x} \in \R^n; \ ||\vec{x}|| = 1}$ je \df{enotska $(n-1)$-sfera}.
\end{itemize}

Homeomorfizem med $(0, 1)$ in $\R$ lahko posplošimo do homeomorfizma med odprto kroglo $\mathring{B}^n$ in $\R^n$. Navaden homeomorfizem je
$$f: \mathring{B}^n \to \R^n, \ f(\vec{x}) := \frac{\vec{x}}{1 - ||\vec{x}||}, \ f^{-1}(\vec{x}) := \frac{\vec{x}}{1 + ||\vec{x}||},$$
tj, raztegnimo vsak poltrak od $0$ do $\infty$.

Sfera v $\R^n$ topološko bolj podobna $\R^{n-1}$ kot $\R^n$.

Naj bo $N = (0, \ldots, 0, 1) \in \R^n$ severni tečaj sfere. Navaden homeomorfizem med $S^{n-1} - \set{N}$ in $\R^{n-1}$ je 

$$f: S^{n-1} - \set{N} \to \R^{n-1}, \ f(x_1, \ldots, x_n) = \frac{1}{1 - x_n}(x_1, \ldots, x_{n-1}),$$
tj. gledamo presek premic skozi točki $N$ in $T \in S^{n-1}$ z ravnino $\R^{n-1}$.

Njen inverz je dan z
$$\R^{n-1} \to S^{n-1} - \set{N},\  g(\vec{x}) = \left(\frac{2\vec{x}}{||\vec{x}||^2 + 1}, \frac{||\vec{x}||^2-1}{||\vec{x}||^2+1}\right).$$
Bijekcijo $f$ imenujemo \df{stereografska projekcija}.

Sledi, da $S^{n-1} - \set{N} \approx \R^{n-1}$. Jasno je, da bi enak rezultat dobili, če bi iz sfere izrezali katerokoli točko. Sklepamo, da ima vsaka točka $S^{n-1}$ okolico, ki je homeomorfna $\R^{n-1}$. Pravimo, da je $S^{n-1}$ \df{lokalno homeomorfna} prostoru $\R^{n-1}$.

\begin{definicija}
    Prostore, ki so lokalno homeomorfne kakemu evklidskemu prostoru, imenujemo \df{mnogoterosti}.
\end{definicija}

\subsection{Baze in predbaze}
\begin{definicija}
    Naj bo $(X, \T)$ prostor. Družina $B \subseteq \T$ je \df{baza topologije $\T$}, če lahko vse elemente $\T$ zapišemo kot unije elementov $B$.
\end{definicija}

\begin{primer}
    Primeri baz. Ali so lahko baze majhne?
    \begin{itemize}
        \item Naj bo $(X, d)$ metrični prostor. Krogle so baza metrične topologije $\T_d$. Še več: Dovolj je, če vzamemo samo majhne krogle, npr. z radijem $\frac{1}{n}$.
        \item Če vzemimo $(X, \T_\text{disk})$, potem vsaka baza vsebuje vse enojčke.
        \item Ali bi v $(\R^n, \T_\text{evkl})$ lahko za bazo vzeli le krogle s središči v $\Q^n$?
    \end{itemize}
\end{primer}

\begin{trditev}
    Naj bo $(X, \T_X)$ prostor, $B_x$ baza $\T_X$. Naj bo $(Y, \T_Y)$ prostor, $B_Y$ baza $\T_Y$. Velja:
    \begin{enumerate}
        \item $A \subseteq X \text{ je odprta} \liff \all{a \in A} \some{B \in B_x} a \in B \subseteq A$.
        \item Naj bo $f: X \to Y$ funkcija. Potem
        \begin{itemize}
            \item $f \text{ je zvezna} \liff \all{B \in B_Y} \invimg{f}(B) \in \T_X$.
            \item $f \text{ je odprta} \liff \all{B \in B_X} \img{f}(B) \in \T_Y$.
        \end{itemize}
    \end{enumerate}
\end{trditev}

\begin{proof}
    Slike in praslike ohranjajo unije.
\end{proof}

\begin{primer}
    Ali je $f: S^1 \to S^1 \subseteq \C \text{ (enotska kompleksna števila)}, \ f(z) = z^2$ odprta?
\end{primer}

\begin{definicija}
    Naj bo $X$ prostor, $x \in X$. Družina $B_X \subseteq \T$ je \df{lokalna baza okolic $X$}, če za vsako odprto okolico $U$, ki vsebuje $x$, obstaja $B \in B_X$, da $x \in B \subseteq U$. 
\end{definicija}

\begin{opomba}
    Običajno prevzamemo, da so $B_X$ okolice $x$.
\end{opomba}

\begin{trditev}
    Če je $\mathcal{B}$ baza topologije $\T$, potem je $B_x = \set{B \in \mathcal{B}; \ x \in B}$ je lokalna baza okolic $x$.

    Obratno: $\mathcal{B} := \bigcup_{x \in X} B_x$ je baza topologije $X$.
\end{trditev}

\begin{primer}
    V metričnem prostoru $(X, d)$ je $\set{K(x,r); \ r \in \Q}$ lokalna baza pri $x$.
\end{primer}

\newpage
Vzemimo neko družino $B$. Definiramo $\T = \set{\text{unije elementov } B}$. Ali je $\T$ topologija?

\begin{trditev}
    \label{trd:baza}
    Naj bo $\mathcal{B}$ družina podmnožic $X$. Definiramo $\T = \set{\text{unije elementov } \mathcal{B}}$. Potem $\T$ je topologija na $X$ natanko tedaj, ko
    \begin{enumerate}
        \item $\mathcal{B}$ je pokritje $X$,
        \item za vse $B, B' \in \mathcal{B}$, za vse $x \in B \cap B'$, obstaja $B'' \in \mathcal{B}$, da $x \in B'' \subseteq B \cap B'$.
    \end{enumerate} 
    Rečemo, da je \df{topologija $\T$ generirana z bazo $\mathcal{B}$}.
\end{trditev}

\begin{proof}
    Enostavno preverimo lastnosti.
\end{proof}

Naj bosta $(X, \T_X), (Y, \T_Y)$ topologiji. Radi bi definirali topologijo na množici $X \times Y$.

\begin{definicija}
    \df{Produktna topologija} $\T_{X \times Y}$ je topologija, ki je generirana z bazo $\set{U \times V; \ U \in \T_X, V \in \T_Y}$.
\end{definicija}

\begin{zgled} 
    Projekciji.
    \begin{itemize}
        \item Naj bo $(X \times Y, \T_{X \times Y})$ produktna topologija. Projekciji $\pr_x: X \times Y \to X, \ \pr_y$ sta zvezni in odprti.
        \item Naj bo $\pr_1: \R^2 \to \R$. Ali je $\pr_1$ zaprta?
    \end{itemize} 
\end{zgled}

Naj bo $\mathcal{P}$ poljubna družina podmnožic $X$. Kaj je najmanjša topologija $\T$ na $X$, ki vsebuje $\mathcal{P}$?

\begin{trditev}
    Naj bo $\mathcal{P}$ poljubna družina podmnožic $X$. Če je $\mathcal{P}$ pokritje $X$, potem je $\T$ topologija, ki jo kot baza generirajo končni preseki elementov $\mathcal{P}$.
    Pravimo, da je $\mathcal{P}$ \df{predbaza topologije $\T$}.
\end{trditev}

\begin{proof}
    Družina vseh končnih presekov elementov $\mathcal{P}$ ustreza pogoju (2) iz trditve \ref{trd:baza}
\end{proof}

\begin{primer}
    Produktna topologia na $X \times Y$ je najmanjša topologija za katero sta projekciji $\pr_X$ in $\pr_Y$ zvezni. 
    
    Množica~$\mathcal{P} = \set{\invimg{\pr_X}(U), \ U \in \T_X} \cup \set{\invimg{\pr_Y}(V), \ V \in \T_Y}$ je predbaza. 

    S pomočjo predbaze $\mathcal{P}$ lahko definiramo produktno topologijo za poljubno mnogo faktorjev.
\end{primer}

\begin{trditev}
    Naj bosta $(X, \T_X), (Y, \T_Y)$ prostora. Naj bo $\mathcal{P}$ predbaza $\T_Y$. Velja:
    $$\text{Funkcija } f: X \to Y \text{ je zvezna} \liff \all{B \in \mathcal{P}} \invimg{f}(B) \in \T_X.$$
\end{trditev}

\begin{proof}
    Enostavno.
\end{proof}

\textbf{\textcolor{red}{Pozor!}} Odprtost funkcije $f$ v splošnem ne moremo testirati na predbaze.

\begin{trditev}
    Naj bodo $X, Y, Z$ prostori. Velja:
    $$\text{Funkcija } f:X \to Y \times Z, \ f = (f_Y, f_Z) \text{ je zvezna} \liff f_Y, f_Z \text{ sta zvezni}.$$
\end{trditev}

\begin{proof}
    $(\lthen)$ Komponenti sta kompozitum zveznih funkcij.

    $(\Leftarrow)$ Poglejmo prasliko predbaznih množic.
\end{proof}

Baze lahko uporabimo za neko grobo oceno velikosti topološkega prostora in bogatstva njegove topologie.

\begin{aksiom}[1. aksiom števnosti]
    Naj bo $(X, \T)$ prostor. Vsaka točka $x \in X$ ima števno bazo okolic.

    Rečemo, da je prostor $(X, \T)$ \df{$1$-števen}. 
\end{aksiom}

\begin{aksiom}[2. aksiom števnosti]
    Naj bo $(X, \T)$ prostor. Obstaja kaka števna baza za topologijo $\T$.

    Rečemo, da je prostor $(X, \T)$ \df{$2$-števen}. 
\end{aksiom}

\begin{primer}
    $1$-števni in $2$-števni prostori. 
    \begin{itemize}
        \item Metrični prostori so $1$-števni. 
        \item Neštevna množica z diskretno topologijo je $1$-števna (metrizabilna), ni pa $2$-števna, ker vsaka baza mora vsebovati vsi enojci.
    \end{itemize}
\end{primer}

\begin{trditev}
    Očitno velja: $2\text{-števnost} \lthen 1\text{-števnost}$. Obrat ne velja. 
\end{trditev}

\begin{trditev}
    Naj bo prostor $(X, \T)$ $1$-števen. Velja:
    \begin{enumerate}
        \item Za vsako množico $A \subseteq X$ je $\Cl A = L(A) = \set{x; \ x \text{ je limita zaporedja v } A}$.
        \item Funkcija $f: (X, \T_X) \to (Y, \T_Y) \text{ je zvezna} \liff f(L(A)) \subseteq L(f(A))$.
    \end{enumerate}
\end{trditev}

\begin{proof}
    (1) Konstruiramo zaporedje s pomočjo števne baze okolic.

    (2) \textcolor{red}{TODO.}
\end{proof}

\begin{primer}
    $(\R, \T_{\text{evkl}})$ je $2$-števna.
\end{primer}

\newpage
\begin{definicija}
    Prostor $(X, \T)$ je \df{separabilen}, če v $X$ obstaja števna gosta podmnožica.
\end{definicija}

\begin{primer}
    $\Q$ v $\R$ ali $\Q^n$ v $\R^n$.
\end{primer}

Ali separabilnost in $1$-števnost implicira $2$-števnost? Ne.
Očitno: $2$-števnost implicira separabilnost.

\begin{trditev}
    V metričnih prostorih je separabilnost ekvivalentna $2$-števnosti.
\end{trditev}

\begin{proof}
    \textcolor{red}{TODO}
\end{proof}

\begin{primer}
    Opazujemo množico $C([0,1])$. 
    
    Weierstrassov izrek pravi, da vsako zvezno funkcijo na poljubnem zaprtem intervalu lahko enakomerno aproksimiramo z polinomi.
    Vsak polinom pa lahko enakomerno aproksimiramo z polinomi z racionalnimi koeficienti. Slida, da je množica polinomov z racionalnimi koeficienti je števna gosta v $C([0,1])$. Torej $C([0,1])$ je separabilen (in metričen) sledi, da je $2$-števen.
\end{primer}

\subsection{Podprostori}
Naj bo $(X, \T)$ prostor, $A \subseteq X$. Definiramo $\T_A := \setb{A \cap U}{U \in T}$. Trdimo, da je $\T_A$ topologija na $A$.

\begin{definicija}
    Topologija $\T_A$ je \df{inducirana} (oz. \df{podedovana}) topologija na $A$.
    Prostor $(A, \T_A)$ je \df{podprostor} prostora $(X, \T)$.
\end{definicija}

\begin{primer}
    Podprostori.
    \begin{itemize}
        \item Evklidska topologija na $\R$ je inducirana z evklidsko topologijo na $\R^2$.
        \item Naj bo $d$ metrika na $X$. Velja:
        $$\xymatrix{
            & (X, d) \ar@{->}[ld] \ar@{->}[rd] &  &  \\
           (X, \mathcal{T}_d) \ar@{->}[d] &  & (A, d|_A) \ar@{->}[d] &  \\
           (A, (\mathcal{T}_d)_A) & = & (A, \mathcal{T}_{(d_{|A})}) & \text{(odprte krogle so iste)}
           }$$
        \item Naj bo $B \subseteq A \subseteq (X, \T)$. Velja: $\T_B = (\T_A)_B$ (podprostor podprostora spet podprostor).
        \item Opazujemo $(\R, \T_{\text{evkl}})$. Vzemimo $\N \subseteq \R$. Inducirana topologija je diskretna.
    \end{itemize}
\end{primer}

\begin{trditev}
    Zaprte množice v $\T_A$ so preseki $A$ z zaprtimi podmnožicamo v $X$.
\end{trditev}

\begin{proof}
    \textcolor{red}{TODO}
\end{proof}

Velja:
\begin{itemize}
    \item Če je $B$ baza $\T$, potem $B_A := \setb{A \cap U}{U \in B}$ je baza $\T_A$.
    \item Če je $(X, \T)$ je $1$-števen/$2$-števen, potem $(A, \T_A)$ je $1$-števen/$2$-števen.
\end{itemize}

\begin{definicija}
    Topološka lastnost je \df{dedna}, če iz prevzetka, da $(X, \T)$ ima to lastnost sledi, da jo imajo tudi vsi podprostori.
\end{definicija}

\begin{primer}
    Dedni lastnosti.
    \begin{itemize}
        \item Diskretnost in trivialnost topologije sta dedni.
        \item $1$-števnost in $2$-števnost sta dedni.
        \item Metrizabilnost je dedsna.
        \item Separabilnost ni dedna.
        
        Naj bo $X$ neštevna množica, $a \in X$. Definiramo $\T := \setb{U \subseteq X}{a \in U} \cup \set{\emptyset}$. Topologija, ki je inducirana na $X - \set{a}$ je diskretna, ki ni separabilna.
        
        Velja: odprt podprostor separabilnega podprostora je separabilen. Očitno $\set{a}$ je gosta v $X$. 
    \end{itemize}
\end{primer}