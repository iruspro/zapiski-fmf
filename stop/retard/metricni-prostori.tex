\section*{Metrični prostori}
\begin{definicija}
    \df{Metrični prostor} je množica $X$ skupaj z preslikavo $d: X \times X \to \R$, za katero velja:
    \begin{itemize}
        \item $d(x, x') = 0 \liff x=x'$,
        \item $d(x, x') = d(x', x)$,
        \item $d(x, x'') \leq d(x, x') + d(x', x'')$.
    \end{itemize}
\end{definicija}

\begin{definicija}
    Naj bo $(X, d)$ metrični prostor. 
    \begin{itemize}
        \item \df{Odprta krogla} s središčem v $a$ in polmerom $r$ je množica $K(a, r) = \setb{x \in X}{d(a,x) < r}$.
        \item \df{Zaprta krogla} s središčem v $a$ in polmerom $r$ je množica $K(a, r) = \setb{x \in X}{d(a,x) \leq r}$. 
        \item \df{Okolica} točke $a$ je vsaka taka množica, ki vsebuje odprto kroglo $K(a, r)$ za nek $r>0$.
    \end{itemize}    
\end{definicija}

\begin{definicija}
    Naj bo $(X, d)$ metrični prostor in $A \subset X$.
    \begin{itemize}
        \item Točka $x \in X$ je \df{notranja} točka množice $A$, če obstaja $r>0$, da $K(a,r) \subset A$.
        \item Točka $a \in X$ je \df{zunanja} točka množice $A$, če obstaja $r>0$, da $K(a,r) \cap A = \emptyset$.
        \item Točka $a \in X$ je \df{robna} točka množice $A$, če vsaka njena okolica seka $A$ in $A^c$.
    \end{itemize}
\end{definicija}

\begin{definicija}
    Naj bo $(X, d)$ metrični prostor in $A \subset X$.
    \begin{itemize}
        \item Množico $\Int A$ vseh notranjih točk množice $A$ imenujemo \df{notranjost} od $A$.
        \item Množico $\Cl A$ vseh točk, za katere za vsak $r>0$, krogla $K(a, r)$ seka $A$, imenujemo \df{zaprtje} množice $A$.
        \item Množico $\partial A$ vseh robnih točk množice $A$ imenujemo \df{meja} množice $A$.     
    \end{itemize}
\end{definicija}

\begin{trditev}
    Naj bo $(X, d)$ metrični prostor in $A \subset X$. Velja:
    \begin{itemize}
        \item $\partial A = \Cl A \setminus \Int A$.
        \item $\Cl A = \Int A \cup \partial A$.
        \item $\Int A = \Cl A \setminus \partial A$.
    \end{itemize}
\end{trditev}

\begin{definicija}    
    Naj bo $(X, d)$ metrični prostor in $A \subset X$.
    \begin{itemize}
        \item Množica $A$ je \df{odprta} v metričnem prostoru $X$, če je vsaka njena točka notranja.
        \item Množica $A$ je \df{zaprta} v metričnem prostoru $X$, če vsebuje vse svoje robne točke.
    \end{itemize}
\end{definicija}

\begin{trditev}
    Naj bo $(X, d)$ metrični prostor in $A \subset X$, potem 

    $$A \text{ je zaprta } \liff A^c \text{ je odprta}.$$
\end{trditev}

\begin{izrek}
    Naj bo $U$ družina vseh odprtih množic metričnega prostora $(X, d)$. Potem
    \begin{itemize}
        \item $X \in U$, $\emptyset \in U$.
        \item Če je $A_\lambda \in U$ za vsak $\lambda \in \Lambda$, potem $ \bigcup_{\lambda \in \Lambda} A_\lambda \in U$.
        \item Če je $n \in \N$ in $A_j \in U$ za vsak $j = 1, 2, \ldots, n$, potem $ \bigcap_{j=1}^n A_j \in U$.
    \end{itemize}
\end{izrek}

\begin{trditev}
    Vsaka odprta krogla je odprta množica in vsaka zaprta krogla je zaprta množica.
\end{trditev}

\begin{trditev}
    Naj bo $(X, d)$ metrični prostor in $A \subset X$, potem 

    $$A \text{ je odprta } \liff A \text{ lahko predstavimo kot unijo odprtih krogel}.$$
\end{trditev}

\begin{definicija}
    Naj bo $(X, d)$ metrični prostor in $A \subset X$. Točka $a \in X$ je \df{stekališče množice} $A$, če vsaka okolica točke~$a$ vsebuje neskončno mnogo točk iz množice $A$.
\end{definicija}

\begin{trditev}
    Naj bo $(X, d)$ metrični prostor in $A \subset X$. Potem 
    $$A \text{ je zaprta } \liff A \text{ vsebuje vsa svoja stekališča}.$$
\end{trditev}

\newpage
\subsection*{Zaporedja v metričnih prostorih}
\begin{definicija}
    Naj bo $(X, d)$ metrični prostor. \df{Zaporedje} v metričnem prostoru $X$ je preslikava $\N \to X$.
\end{definicija}

\begin{definicija}
    Naj bo $(X, d)$ metrični prostor. Pravimo, da zaporedje $(a_n)_n$ v $X$ \df{konvergira proti $a \in X$}, če

    $$\all{\epsilon > 0} \some{n_0 \in \N} \all{n \in \N} n \geq n_0 \lthen d(a_n, a) < \epsilon.$$
    V tem primeru $a$ imenujemo \df{limita} zaporedja.
\end{definicija}

\begin{definicija}
    Naj bo $(X, d)$ metrični prostor. Pravimo, da zaporedje $(a_n)_n$ v $X$ \df{izpolnjuje Cauchyjev pogoj}, če

    $$\all{\epsilon > 0} \some{n_0 \in \N} \all{n, m \in \N} n, m \geq n_0 \lthen d(a_n, a_m) < \epsilon.$$
\end{definicija}

\begin{izrek}
    Vsako konvergentno zaporedje v metričnem prostoru $(X, d)$ izpolnjuje Cauchyjev pogoj.
\end{izrek}

\begin{definicija}
    Pravimo, da je metrični prostor $(X, d)$ \df{poln}, ce je vsako Cauchyjevo zaporedje iz $X$ tudi konvergentno v~$X$.
\end{definicija}

\begin{izrek}
    Naj bo $C[a,b]$ z običajno (supremum) metriko. Tedaj 
    $$(f_n)_n \text{ v } C[a,b] \text{ konvergira proti } f \in C[a,b] \liff (f_n)_n \text{ enakomerno konvergira proti } f \text{ na } [a, b].$$
\end{izrek}

\begin{izrek}
    Metrični prostor $(C[a,b], d_\infty)$ je poln metrični prostor.
\end{izrek}

\subsection*{Preslikave med metričnimi prostori}
Naj bosta $(X, d)$ in $(X', d')$ metrična prostora. Naj bo $D \subset X, \ D \neq \emptyset$. Obravnamo preslikave $f: D \to X'$.

\begin{definicija}
    Preslikava $f: D \to X'$ je \df{zvezna v točki} $a \in X$, če
    $$\all{\epsilon > 0} \some{\delta > 0} \all{x \in D} d(x, a) < \delta \lthen d'(f(x), f(a)) < \epsilon.$$
\end{definicija}

\begin{izrek}
    Preslikava $f: D \to X'$ je zvezna v točki $a \in D$ natanko tedaj, ko za vsako zaporedje $(x_n)_n$ v $D$, ki konvergira proti $a \in D$, zaporedje $(f(x_n))_n$ v $X'$ konvergira proti $f(a) \in X'$. 
\end{izrek}

\begin{definicija}
    Pravimo, da je preslikava $f: D \to X'$ je \df{zvezna}, če je zvezna v vsaki točki iz $D$.
\end{definicija}

\begin{definicija}
    Preslikava $f: D \to X'$ je \df{enakomerno zvezna}, če
    $$\all{\epsilon > 0} \some{\delta > 0} \all{x, x' \in D} d(x, x') < \delta \lthen d'(f(x), f(x')) < \epsilon.$$
\end{definicija}

\begin{definicija}
    Preslikava $f: D \to X'$ je \df{$C$-lipschitzova}, če
    $$\all{x, x' \in D} d'(f(x), f(x')) \leq Cd(x, x').$$
\end{definicija}

\begin{trditev}
    Za preslikavo $f: D \to X'$ velja:

    $$f \text{ je $C$-lipschitzova } \lthen f \text{ je enakomerno zvezna } \lthen f \text{ je zvezna}.$$
\end{trditev}

\begin{izrek}
    Dana je preslikava $f: D \to X'$. Preslikava $f$ je zvezna natanko tedaj, ko praslika vsake odprte množice v~$X'$ je odprta v $D$.
\end{izrek}

Izskaže se da konvergenco in zveznost (ključna pojma analize) lahko opredelimo, kakor hitro vemo, katere množice so odprte.