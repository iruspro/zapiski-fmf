\documentclass[a4paper,oneside,8pt,landscape]{extarticle}

\usepackage[utf8]{inputenc}
\usepackage[T1]{fontenc}
\usepackage[slovene]{babel}
\usepackage{lmodern}

\usepackage{xcolor}
\usepackage{bbold}
\usepackage{amsmath}
\usepackage{amssymb}

% environments
\usepackage{amsthm}

\theoremstyle{definition}{
    \newtheorem{definicija}{Definicija}[section]
}

\theoremstyle{plain} {
    \newtheorem{izrek}[definicija]{Izrek}
    \newtheorem{trditev}[definicija]{Trditev}
    \newtheorem{posledica}[definicija]{Posledica}
    \newtheorem{lema}[definicija]{Lema}
    \newtheorem{aksiom}[definicija]{Aksiom}
}

\theoremstyle{remark}{    
    \newtheorem{opomba}{Opomba}
    \newtheorem{primer}{Primer}
    \newtheorem{zgled}{Zgled}
}
\usepackage[
  paper=a4paper,
  top=0.7cm,
  bottom=0.7cm,
  left=0.7cm,
  right=0.7cm,
  textwidth=10cm,
  textheight=18cm,
]{geometry}

\usepackage{enumitem}
\setlist[itemize]{topsep=-10pt, partopsep=0pt, itemsep=0pt, parsep=0pt, left=0pt}
\setlist[enumerate]{topsep=-10pt, partopsep=0pt, itemsep=0pt, parsep=0pt, left=0pt}

% lists with less vertical space
\newenvironment{itemize*}{\vspace{-6pt}\begin{itemize}\setlength{\itemsep}{0pt}\setlength{\parskip}{2pt}}{\end{itemize}}
\newenvironment{enumerate*}{\vspace{-6pt}\begin{enumerate}\setlength{\itemsep}{0pt}\setlength{\parskip}{2pt}}{\end{enumerate}}
\newenvironment{description*}{\vspace{-6pt}\begin{description}\setlength{\itemsep}{0pt}\setlength{\parskip}{2pt}}
{\end{description}}

\usepackage{multicol}
\setlength{\columnseprule}{1pt}
\def\columnseprulecolor{\color{black}}

\pagestyle{empty}              % vse strani prazne
\setlength{\parindent}{0pt}    % zamik vsakega odstavka
\setlength{\parskip}{10pt}     % prazen prostor po odstavku
% \setlength{\overfullrule}{30pt}  % oznaci predlogo vrstico z veliko črnine

\usepackage{titlesec} % Отступ от заголовков
\titlespacing*{\section}{0px}{0px}{-5px} 
\titlespacing*{\subsection}{0px}{0px}{-5px}

% default sets
\newcommand{\N}{\mathbb{N}}
\newcommand{\Z}{\mathbb{Z}}
\newcommand{\Q}{\mathbb{Q}}
\newcommand{\R}{\mathbb{R}}
\newcommand{\C}{\mathbb{C}}
\newcommand{\F}{\mathbb{F}}
\newcommand{\HH}{\mathbb{H}}

% logic
\newcommand{\all}[1]{\forall #1 \,.\,}
\newcommand{\some}[1]{\exists #1 \,.\,}
\newcommand{\exactlyone}[1]{\exists! #1 \,.\,}
\newcommand{\lthen}{\implies}
\newcommand{\liff}{\iff}

% sets
\newcommand{\set}[1]{\left\{#1\right\}}
\newcommand{\setb}[2]{\set{#1 \,|\, #2}}

% mappings
\newcommand{\img}[1]{#1_{*}}
\newcommand{\invimg}[1]{#1^{*}}









\DeclareMathOperator{\lin}{Lin}
\DeclareMathOperator{\rang}{rang}

\DeclareMathOperator{\sgn}{sgn}  % sign
\DeclareMathOperator{\id}{id}  % identity func
\DeclareMathOperator{\im}{im}  % slika

\DeclareMathOperator{\Int}{Int}  % interior
\DeclareMathOperator{\Cl}{Cl}  % closure

\DeclareMathOperator{\FV}{FV}  % Fourierjeva vrsta

\DeclareMathOperator{\eval}{eval}
% math symbols
\newcommand{\wt}[1]{\widetilde{#1}}

% Greece letters
\let\oldphi\phi
\let\oldtheta\theta
\newcommand{\eps}{\varepsilon}
\renewcommand{\phi}{\varphi}
\renewcommand{\theta}{\vartheta}

% style
\newcommand{\ds}{\displaystyle}
\newcommand{\mc}[1]{\mathcal{#1}}

% other
\newcommand{\todo}[1]{\textcolor{red}{TODO: #1}}


\newcommand{\gen}[1]{\left\langle {#1} \right\rangle}

% math operators
\DeclareMathOperator{\ch}{char}
\DeclareMathOperator{\red}{red}

\begin{document}

\begin{multicols*}{3}

\section*{Cela števila}
\textbf{Trd.} \(\all{m \in \Z} \all{n \in \N} \some{q, r \in \Z} m = qn +r \land 0 \leq r < n\).\\
% TODO: Vsaj eno ni nič
\textbf{Trd.} \(\all{m, n \in \Z} \some{\gcd(m,n)} \land \some{x, y \in \Z} \gcd(m ,n) = mx + ny\).\\
\textbf{Trd.} \(\all{m, n \in \Z} \gcd(m, n) = 1 \liff \some{x, y \in \Z} 1 = mx + ny\).

\section{Uvod v teorijo grup}
\textbf{Lagrange.} Naj bo \(G\) končna grupa in \(H \leq G\): \(|G| = [G:H] |H|\).
\textbf{\textcolor{red}{Grupa permutacij}}
\begin{itemize}
    \item Zapis s transpoziciji: \((i_1 i_2 \ldots i_n) = (i_1 i_n)(i_1 i_{n-1}) \ldots (i_1 i_3)(i_1 i_2)\)
    \item Inverz \(k\)-cikla: \((i_1 i_2 \ldots i_k)^{-1}  = (i_k i_{k-1} \ldots i_2 i_1)\)
    \item Konjugiranje: \(\pi \in S_n \Rightarrow \pi (i_1 i_2 \ldots i_k) \pi^{-1} = (\pi(i_1) \pi(i_2) \ldots \pi(i_k))\)
    \item \todo{\(A_n\), s čim je generirana}?
\end{itemize}
%
%%%%%%%%%%%%%
%%%%%%%%%%%%%
%
\textbf{\textcolor{red}{Diedrska grupa \(D_{2n}\)}}
\begin{itemize}
    \item \(z^kr = r^{-k}z = r^{n - k}z\)
    \item \(r^kz\) so zrcaljenja, \((r^kz)^2 = 1\)
\end{itemize}
%
%%%%%%%%%%%%%
%%%%%%%%%%%%%
%
\textbf{\textcolor{red}{Podgrupe}}
\begin{itemize}
    \item \(H, K \leq G \lthen |HK| = \frac{|H||K|}{|H \cap K|}\).
    \item Diagonalna podgrupa \(\triangle = \setb{(x, x)}{x \in G} \leq G \times G\) 
\end{itemize}
%
%%%%%%%%%%%%%
%%%%%%%%%%%%%
%
\textbf{\textcolor{red}{Ciklične grupe}}
\begin{itemize}
    \item Vsaka podgrupa ciklične grupe je ciklična
    \item Podgrupe v \(\Z\) so oblike \(n\Z, n \in \N\)
    \item Podgrupe v \(\Z_n\) so \(\Z_d\), kjer \(d\, |\, n\)
    \item \(G = \gen{a}, |G| < \infty \lthen G = \gen{a^k} \liff \gcd (k, n) = 1\)
    \item \(k \in Z_n \lthen \red k = \frac{n}{\gcd(n, k)}\)
    \item Konjugiranje ohranja red elementa
\end{itemize}
%
%%%%%%%%%%%%%
%%%%%%%%%%%%%
%
\textbf{\textcolor{red}{Generatorji grup}}\\
Naj želimo določiti \(\gen{A}\). Oglejmo množico \(\mc{A}\) vseh možnih produktov in inverzov (elementov, ki morajo biti v \(\gen{A}\)) ter pokažemo, da je podgrupa. Nato iz minimalnosti \(\gen{A}\) sledi enakost.\\
%
%%%%%%%%%%%%%
%%%%%%%%%%%%%
%
\textbf{\textcolor{red}{Splošno}}\\
Naj bo \(f: X \to X\) preslikava. Velja:
\begin{itemize}
    \item \(f\) ima levi inverz: \(g \circ f = \id \liff f\) injektivna. Če \(f\) tudi ni surjektivna, potem ima več levih inverzov.
    \item \(f\) ima desni inverz: \(f \circ h = \id \liff f\) surjektivna. Če \(f\) tudi ni injektivna, potem ima več desnih inverzov.        
\end{itemize}

\section{Uvod v teorijo kolobarjev}
\textbf{Brucevo sanje.} Naj bo \(F\) polje, \(\ch F = p\). Tedaj \((x+y)^p = x^p + y^p\).
\begin{itemize}
    \item Kolobar \(K\) je Boolov, če \(\all{x \in K}x^2 = x\). Boolov kolobar je komutativen in ima karakteristiko 2. 
    \item Kolobar \(\Z\) ni algebra nad nobenim poljem.
    \item Naj bo \(A\) končno-razsežna algebra, \(a \in A \setminus \set{0}\). Tedaj
    \begin{itemize}
        \item \((\some{b \in A \setminus \set{0}} a b = 0 \lor ba = 0) \sqcup  (\some{a^{-1}} a^{-1}a=aa^{-1} = 1)\).
        \item \(\some{b \in A} ab = 1 \lor ba = 1 \lthen a^{-1} = b\).
        \item Če je \(A\) obseg, je vsaka podalgebra podobseg.
    \end{itemize}
\end{itemize}
%
%%%%%%%%%%%%%
%%%%%%%%%%%%%
%
\textbf{\textcolor{red}{Algebra kvaternionov \(\HH\)}}
\begin{itemize}
    \item \(i^2=j^2=k^2=ijk=-1\)
    \item \(Q = \set{\pm i, \pm j, \pm k, \pm 1}\) je kvaternionska grupa.
    \item \(Z(\HH) = \R, \ Z(Q) = \set{-1, 1}\).
    \item \(\all{h \in \HH} \some{\alpha, \beta \in \R} h^2 + \alpha h + \beta = 0\), kjer \(-\alpha = h + \overline{h}\) in \(\beta = h \overline{h}\).
\end{itemize}
%
%%%%%%%%%%%%%
%%%%%%%%%%%%%
%
\textbf{\textcolor{red}{Kolobar \(Z_n\)}}
% TODO: \phi je Eulerjeva funkcija
\begin{itemize}
    \item Kolobar \(\Z\) ima 2 obrnljivih elementa: \(1\) in \(-1\)
    \item V \(\Z_n\) element \(k \in \Z_n\) je obrnljiv natanko tedaj, ko \(\gcd (k, n) = 1\).
    \item \(|\Z_n^*| = \phi(n)\). Če je \(p\) praštevilo, potem \(|\Z_p| = p - 1\).
\end{itemize}
%
%%%%%%%%%%%%%
%%%%%%%%%%%%%
%
\

\textbf{\textcolor{red}{Generatorji kolobarjev}}\\
Naj želimo določiti \(\gen{A}\). Postopamo kot pri grupah (vse možne vsote, nasprotni elementi ter produkti). Opazimo tudi, da \(\mc{A}\) vedno vsebuje \(1\). 
%
%%%%%%%%%%%%%
%%%%%%%%%%%%%
%
\section{Homomorfizmi}
\begin{itemize}
    \item Homomorfizem \(\varphi: \Z \to G, \ \varphi(1) = a\) obstaja za vsak \(a \in G\). Homomorfizem \(\varphi: \Z^n \to G, \ \varphi(1) = a\) natanko tedaj, ko \(a^n = 1\).
    \item Naj bo \(\varphi: G \to G'\) homomrfizem grup in naj ima element \(a \in G\) končen red. Tedaj \(\red \varphi(a) \, | \, \red a\). Če je \(\varphi\) vložitev, potem reda sta enaka.
\end{itemize}
%
%%%%%%%%%%%%%
%%%%%%%%%%%%%
%
\section{Kvocientne strukture}
\textbf{Izr.} Naj bo \(K\) komutativen kolobar, \(M \triangleleft K\).\\ Tedaj je \(M\) maksimalen \(\liff K/_M\) polje.\\
%
%%%%%%%%%%%%%
%%%%%%%%%%%%%
%
\textbf{\textcolor{red}{Kvocientne grupe}}
\begin{itemize}
    \item \(\gen{r}\) je edinka v \(D_{2n}\) za \(n \geq 3\).
    \item Če je \(G/_{Z(G)}\) ciklična, potem je \(G\) Abelova.
\end{itemize}
%
%%%%%%%%%%%%%
%%%%%%%%%%%%%
%
\textbf{\textcolor{red}{1.\ izrek o izomorfizmu}}
\begin{itemize}
    \item To, da je podgrupa \(N \triangleleft G\) edinka v \(G\) lahko dokažemo tako, da najdemo ustrezni homomorfizem \(\varphi\), za kateri \(\ker \varphi = N\).
\end{itemize}
%
%%%%%%%%%%%%%
%%%%%%%%%%%%%
%
\textbf{\textcolor{red}{Kvocientni kolobarji}}
\begin{itemize}
    \item Za vsak kolobar \(K\) velja, da \(\all{a \in K} aK = \setb{ak}{k \in K} = Ka\) je ideal. To je \textbf{glavni ideal} v \(K\), generiran z \(a\).
    \item Enostavnost kolobarja \(K\) uporabimo/dokažemo tako, da predpostavimo, da podan ideal ni trivialen, torej mora biti enak \(K\).
    \item Kolobar \(M_n(D)\) je enostaven, če je \(D\) obseg.
    \item Center enostavnega kolobarja je polje. Komutativen kolobar je enostaven natanko tedaj, ko je polje.
    \item Naj bosta \(K_1\) in \(K_2\) kolobarja. Tedaj vsak ideal direktnega produkta \(K_1 \times K_2\) je oblike \(I_1 \times I_2\), kjer je \(I_1\) ideal v \(K_1\) ter \(I_2\) ideal v \(K_2\).
\end{itemize}
%
%%%%%%%%%%%%%
%%%%%%%%%%%%%
%
\section{Klasifikacija končnih grup}
\textbf{Def.} \textbf{Komutator} elementov $a,b\in G$ je $[a,b]:=aba^{-1}b^{-1}$.\\
\textbf{Def.} Naj bo $G$ grupa, potem je $T(G) = \{g\in G \ | \ \text{red}(g) < \infty \}$ \textbf{torzijska podgrupa} $G$. Če je $T(G) = \{0\}$, pravimo, da je $G$ \textbf{brez torzije}.
\textbf{Izr.} Če \(\gcd(n, m) = 1\), potem \(\Z_n \oplus \Z_m \approx \Z_{nm}\).
%
%%%%%%%%%%%%%
%%%%%%%%%%%%%
%
\begin{itemize}
    \item $G$ $p$-grupa, $H$ $q$-grupa, $p\neq q: G,H$ ciklični $\iff G\oplus H$ ciklična.
\end{itemize}
\textbf{\textcolor{red}{Vse grupe do izomorfizma natančno}}\\
Naj treba poiskati vse grupe reda \(n\) do izomorfizma natančno. Zapišemo \(n = p_1^{k_1} \cdot \ldots \cdot p_n^{k_n}\). Nato zapišemo vse grupe moči \(p_i^{k_i}\): to so grupe oblike \(\Z_{l_1} \oplus \ldots \oplus Z_{l_j}\), kjer \(l_1 + \ldots + l_j = p_i^{k_i}\) razčlenitev števila \(p_i^{k_i}\).
%
%%%%%%%%%%%%%
%%%%%%%%%%%%%
%
\section{Delovanje grup}
Naj $G$ deluje na $X$.\\
\textbf{Def.} \textbf{Orbita} elementa $x\in X$ je $G\cdot x := \{g\cdot x \ | \ g\in G\}$.\\
\textbf{Def.} \textbf{Stabilizator} elementa $x$ je $G_x := \{g\in G \ | \ g\cdot x = x\}$.\\
\textbf{Def.} \textbf{Množica fiksnih točk} $g\in G$ je $X^g := \{x\in X \ | \ g \cdot x= x\}$.\\
\textbf{Def.} \textbf{Fiksne točke delovanja} je množica $X^G := \bigcap_{g\in G}X^g$.\\
\textbf{Def.} \textbf{Konjugiranostni razred} \(x \in G\) je $\text{Raz}(x) := \{axa^{-1} \ | \ a\in G\}$.\\ Konjugiranostni razred je orbita pri delovanju \(G\) na \(G\) s konjugiranjem.
\begin{itemize}
    \item Konjugiranostni razred \(x\) je \(\set{x} \liff {x \in Z(G)}\).
\end{itemize}
\textbf{O orbite in stabilizatorju.} Potem za $\forall x\in X$ velja $|G\cdot x| = [G:G_x]$ in če $G$ končna $|G| = |G\cdot x| \cdot |G_x|$.
%
%%%%%%%%%%%%%
%%%%%%%%%%%%%
%
\section{Izreki Sylowa}
\textbf{Def.} Naj bo $H\leq G$, množici $N(H) := \{a\in G\ | \ aHa^{-1} = H\}$ pravimo \textbf{normalizator} $H$.\\
\textbf{Def.} $H\leq G$ je $p$-\textbf{podgrupa Sylowa}, če je $|H|=p^k \land p^{k+1} \nmid |G|$.\\ Z $n_p$  ozn. $\# p$-podgrup Sylowa grupe $G$.\\
\textbf{Sylow.} Naj praštevilo $p$ deli red končne grupe $G$:
\begin{itemize}
    \item $p^k \mid |G| \implies $ $G$ vsebuje vsaj eno $p$-podgrupo reda $p^k$.
    \item $\forall p$-podgrupa $G$ je vsebovani v kaki $p$-podgrupi Sylowa.
    \item $\forall p$-podgrupi Sylowa sta konjugirani.
    \item $\# p$-podgrup Sylowa grupe $G$ deli $|G|$.
    \item $\# p$-podgrup Sylowa grupe $G$ je $pm+1$ za nek $m\geq 0$.
\end{itemize}
\textbf{Trd.} \(n_p = 1 \liff\) \(p\)-podgrupa Sylowa je edinka.\\
\textbf{Def.} Grupa $G$ je \textbf{enostavna}, če sta njeni edini edinki $\{1\}$ in $G$.
\begin{itemize}
    \item To, da grupa ni enostavna lahko dokažemo tako, da najdemo edino \(p\)-podgrupo Sylowa.
    \item Opazujemo tudi moč preseka in produkta dveh podgrup.
\end{itemize}
%
%%%%%%%%%%%%%
%%%%%%%%%%%%%
%
\section{Kolobar polinomov}
Naj bo $F$ polje.\\
%
%%%%%%%%%%%%%
%%%%%%%%%%%%%
%
\textbf{\textcolor{red}{Nerazcepnost}}\\
\textbf{Trd.} Naj bo $p(x)\in F[x]$, $\text{deg}(p)>0$:
\begin{itemize}
    \item $\text{deg}(p)=1\implies$ $p(x)$ nerazcepen.
    \item $\text{deg}(p)\geq2$ in $p(x)$ nerazcepen $\implies$ nima ničle v $F$.
    \item $\text{deg}(p)\in\{2,3\}\implies (p(x)\text{ nerazcepen } \iff $ nima ničle v $F$).
\end{itemize}
\textbf{Izr.} Naj bo $f(x)\in \mathbb{Z}[x]$ tak, da ga ne moremo zapisati kot produkt dveh nekonstantnih polinomov v $\mathbb{Z}[x]$, potem je $f(x)$ nerazcepen tudi nad $\mathbb{Q}[x]$.\\
\textbf{Eisenstein.} Naj bo $f(x) = a_nx^n+\cdots + a_1 x + a_0$ in $\exists p\in\mathbb{P}$, da $p\mid a_i$ za $i<n$, $p\nmid a_n$ in $p \nmid a_0^2$. Potem je $f(x)$ nerazcepen nad $\mathbb{Q}[x]$.
\textbf{Trd.} $f(x) = x^n+1$ nerazcepen nad $\mathbb{Q} \iff n = 2^k, k\geq 1$.\\
\textbf{Trd.} Če je polinom \(a_nx^n + \ldots + a_1x + a_0\) razcepen nad \(\Q\), kjer so \(a_0, \ldots, a_n \in \Z\), potem je \(a_nx^n + \ldots + a_1x + a_0\) razcepen nad \(\Z_p\), kjer \(p \in \mathbb{P},\ p \not | a_n\), koeficienti pa po modulu \(p\).\\
\textbf{Trd.} $a,b,c$ liha $\implies ax^4 + bx + c$ nerazcepen nad $\mathbb{Q}$.\\
\textbf{Trd.} Naj bodo $a_1,\dots, a_n\in \mathbb{Z}$ različna števila, potem sta polinoma $(x-a_1)\cdots (x-a_n)-1$ in $(x-a_1)^2\cdots (x-a_n)^2+1$ nerazcepna nad $\mathbb{Q}$.\\
\textbf{Trd.} $x^p-x+1$ je nerazcepen in separabilen nad $\mathbb{Z}_p$.
\begin{itemize}
    \item Lahko pogledamo \(f(x+1)\).
    \item Nad \(\Z_2\) je \(x^2+x+1\) edini nerazcepni polinom stopnje \(2\). Ostali polinomi pa so \(x^2,\ x^2+1, x^2+x\).
    \item Uporabimo Brucevo sanje.
\end{itemize}
%
%%%%%%%%%%%%%
%%%%%%%%%%%%%
%
\textbf{\textcolor{red}{Razširitve polj}}\\
Naj bo $K/ F$ razširitev polj.\\
\textbf{Def.} $a\in K$ \textbf{algebraičen} nad $F$, če $\some{p(x)\in F[x]} p(a)=0$. Če je $p(x)$ moničen in minimalne stopnje, pravimo da je $m_a(x):=p(x)$ \textbf{minimalni polinom} za $a$ nad $F$ in $a$ stopnje algebraičnosti $\deg(m_a(x))$ nad $F$. Sicer je \(a\) \textbf{transcendentalen} nad $F$.\\
%
\textbf{Izr.} Naj bo $a\in K$ algebraičen nad $F$ in $p(x)\neq 0\in F[x]\,.\,p(a)=0$ moničen. NTSE:
\begin{enumerate}
    \item $p(x) \text{ minimalen polinom za a}$.
    \item $p(x) \text{ nerazcepen}$.
    \item $\all{q(x)\in F[x]} q(a)=0\implies p(x)\vert q(x)$.
\end{enumerate}
\textbf{Def.} Razširitev $K/F$ je \textbf{končna}, če je $K$ končno razsežen vektorski prostor nad $F$ in pišemo $[K:F]:=\dim_F(K)$.\\
\textbf{Izr.} Naj bosta razširitvi $L/K$ in $K/F$ končni razširitvi.\\ Tedaj velja $[L:F] = [L:K]\cdot [K:F]$.\\
%
\textbf{Trd.} Vsaka končna razširitev je algebraična.\\
%
\textbf{Def.} Razširitev $K/F$ je \textbf{primitivna}, če $\exists a\in K\,.\, K = F(a)$. Elementu $a$ pravimo \textbf{primitivni element} $K$.\\
%
\textbf{Izr.} Naj bo $K/F$ razširitev in $a\in K$ algebraičen nad $F$ stopnje $n$. Potem je $F(a) = F[a] = \{\alpha_0 + \alpha_1a+\dots + \alpha_{n-1}a^{n-1} \ | \ \alpha_i \in F\}$ končna razširitev $F$ in $[F(a):F] = n$. Torej, \(a_0, \ldots, a_{n-1}\) je baza prostora.\\
%
\textbf{Trd.} Naj bosta $a,b$ alg.\ nad $F$ in $\gcd([F(a):F], [F(b):F]) = 1$. Tedaj $[F(a,b):F] = [F(a):F]\cdot [F(b):F]$.\\
%
\textbf{Trd.} Naj bo $[E:F]=p\in\mathbb{P}$, potem je $\forall a\in E\backslash F$ algebraičen stopnje $p$ nad $F$.\\
%
\textbf{Trd.} $F(a^k, a^l) = F(a^d)$ za $d=\gcd(k,l)$.\\
%
\textbf{Trd.} Naj bosta $a_1,\dots,a_n$ algebraični nad $F$.\\ Tedaj $[F(a_1,\dots,a_n):F]\leq [F(a_1):F]\cdot \ldots \cdot [F(a_n):F]$.\\
%
\textbf{Trd.} Ničle $f(x)$ so v poljubni razširitvi $F$ enostavne $\iff f(x)$ in $f'(x)$ tuja.\\
%
\textbf{Trd.} Naj bo $E/F$ razširitev in $\ch(F)=0$, $a\in E$ je $k$-kratna ničla $f(x)\in F[x] \iff f(a) = f'(a) =\dots = f^{(k)}(a) =0$ in $f^{(k+1)}(a)\neq 0$.
%
\begin{itemize}
    \item Stopnja primitivni razširitvi \([F(a):F]\) je enaka stopnje minimalnega polinoma \(a\) nad \(F\).
    \item Naj bo \(E \subseteq F,\ a \in F\). Tedaj \(E(a) = E \liff [E(a):E] = 1\).
    \item Stopnjo razširitve določimo bodisi s pomočjo minimalnega polinoma bodisi s pomočjo verigi razširitev.
    \item Lahko dokažemo, da je \(F(a, b) = F(a+b)\).
\end{itemize}
%
%%%%%%%%%%%%%
%%%%%%%%%%%%%
%
\textbf{\textcolor{red}{Razpadna polja}}\\
\textbf{Def.} Naj bo $K/F$ razširitev in $f(x)\in F[x]$. Pravimo da $f(x)$ \textbf{razpade} nad $K$, če je enak produktu linearnih polinomov v $K[x]$. Če $\nexists$ pravo podpolje $K$, v katerem $f(x)$ razpade, pravimo da je $K$ \textbf{razpadno polje} $f(x)$ nad $F$.
\begin{itemize}
    \item Razpadno polje dobimo tako, da vzemimo vsa ničla polinoma in tvorimo \(F(x_1, \ldots, x_n)\). Ponavadi je treba dokazati enakost z drugim poljem. To naredimo z levo in desno vsebovanostjo.
\end{itemize}

\end{multicols*}

\end{document}
