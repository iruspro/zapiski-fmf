\section{Kolobarji polinomov}

Gledamo polinome nad poljem \(F\), torej kolobar \(F[X]\).
\begin{enumerate}
    \item Kolobarji polinomov
    \begin{itemize}
        \item Zapis polinoma stopnje \(n\). Stopnja ničelnega polinoma.
        \item Čemu je enaka stopnja produkta polinomov?
        \item \textbf{Lema.} Ali ima kolobar \(F[X]\) delitelji niča? Kaj so njegove obrnljive elemente?
        \item \textcolor{blue}{(*)} \textbf{Izrek.} Osnovni izrek o deljenju polinomov.
        \item \textbf{Posledica.} Kaj lahko povemo o vsakem idealu v kolobarju \(F[X]\)?
        \item Ničla polinoma.
        \item \textbf{Opomba.} Ali lahko v splošnem identificiramo polinomi s polinomskimi funkciji?
        \item \textbf{Trditev.} Karakterizacija ničle polinoma.
        \item \textbf{Posledica.} Koliko ničel lahko ima neničeln polinom?
        \begin{proof}
            Z indukcijo na stopnjo polinoma.
        \end{proof}
    \end{itemize}

    \item Nerazcepni polinomi
    
    Naj bo \(F\) polje.
    \begin{itemize}
        \item \textcolor{blue}{(*)} \textbf{Definicija.} Nerazcepni polinom nad \(F\).
        \item \textbf{Zgled.} Kaj so nerazcepni polinomi v \(\C[X]\)? Kaj so v \(\R[X]\)?
        \item \textbf{Trditev.} Naj bo \(p(X) \in F[X]\) stopnje vsaj \(1\).
        \begin{itemize}
            \item Kaj če stopnja \(p(X)\) enaka 1?
            \item Kaj če stopnja vsaj 2 ter \(p(X)\) nerazcepen? (ničle)
            \item Kdaj je polinom stopnje 2 ali 3 nerazcepen?
        \end{itemize}
    \end{itemize}

    Od tod dalje gledamo kolobar \(\Q[X]\). Če polinom \(p(X) \in \Q[X]\) pomnožimo s skupnim imenovalcem koeficientov dobimo polinom v \(\Z[X]\).
    \begin{itemize}
        \item \textbf{Definicija.} Primitiven polinom.
        \item \textcolor{blue}{(*)} \textbf{Trditev.} Gaussova lema.
        \item \textcolor{blue}{(*)} \textbf{Izrek.} Zadosten pogoj, da je polinom \(p(X) \in \Q[X]\) nerazcepen.
        \item \textcolor{red}{(*)} \textbf{Izrek.} Eisensteinov kriterij.
        \item \textbf{Zgled.} Ciklotomični polinomi.
    \end{itemize}

    \item Razširitve polj
    \begin{itemize}
        \item \textbf{Definicija.} Razširitev polja.
        \item \textbf{Zgled.} Razširitev \(\Q\) z \(\R\). \(\C\) kot razširitev. Polje \(\Q(\sqrt{2})\).
        \item \textbf{Definicija.} Algebraičen element. Transcendenten element.
        \item \textbf{Zgled.} Ali sta \(\pi\) in \(\sqrt{2}\) algebraična nad \(\Q\)?
        \item \textbf{Definicija.} Minimalni polinom elementa.
        \item \textbf{Zgled.} Ali minimalni polinom, če obstaja, enolično določen?
        \item \textbf{Trditev.} 2 karakterizacije minimalnosti polinoma.
        \item \textbf{Definicija.} Stopnja elementa.
        \item \textbf{Zgled.} \
        \begin{itemize}
            \item Kaj so algebraični elementi stopnje \(1\) v \(K/F\)?
            \item Kakšno stopnjo ima \(\sqrt{2}\) v \(\Q(\sqrt{2})/\Q\)?
            \item Kakšno stopnjo imajo elementi \(\C\) nad \(\R\)?
            \item Ali je \(\sqrt{2} + \sqrt{3}\) algebraičen v \(\R/\Q\)?
        \end{itemize}
    \end{itemize}

    \newpage
    \item Končne razširitve polj 
    
    Naj bo \(K/F\) razširitev polj.
    \begin{itemize}
        \item Ali lahko gledamo \(K\) kot vektorski prostor nad \(F\)?
        \item \textbf{Definicija.} Kdaj rečemo, da je razširitev končna? Stopnja razširitve.
        \item \textbf{Zgled.} Ali je \(\C/\R\) končna? Ali je \(\R/\Q\) končna?
        \item \textcolor{red}{(*)} \textbf{Trditev.} Zveza stopenj razširitev \(F \subseteq L \subseteq K\).
        \item \textbf{Definicija.} Algebraična razširitev. Transcendentna razširitev.
        \item \textcolor{blue}{(*)} \textbf{Trditev.} Kaj lahko povemo o končni razširitvi?
        \item Naj bo \(K/F\) razširitev polj, \(a \in K\). Podkolobar v \(K\) generiran z \(F\) in \(a\). Podpolje v \(K\), generirano z \(F\) in \(a\). Podobno za \(n\) elementov.
        \item \textbf{Definicija.} Primitivna razširitev.
        \item \textcolor{red}{(*)} \textbf{Izrek.} Naj bo \(a \in K\) algebraičen. Zveza med \(K[x]\) in \(K(x)\). Stopnja primitivne razširitve.
        \item \textbf{Posledica.} Verzija izreka za \(n\) elementov.
        \item \textbf{Zgled.}
        \begin{itemize}
            \item Določi stopnjo \(\Q(\sqrt[n]{p})/\Q\), kjer je \(p\) praštevilo. 
            \item Naj bo \(a \in K\). Oglejmo si \(\eval: F[X] \to F[a]\). Kaj če je \(a\) transcendenten? Kaj če je \(a\) algebraičen?
        \end{itemize}
        \item \textcolor{blue}{(*)} \textbf{Izrek.} Podpolje algebraičnih elementov razširitve.
        \item \textcolor{blue}{(*)} \textbf{Zgled.} Algebraična razširitev \(\Q\), ki ni končna.
    \end{itemize}

    \item Konstrukcije z ravnilom in šestilom
    
    \begin{itemize}
        \item \textbf{Problemi.} Podvojitev kocke. Kvadratura kroga. Trisekcija kota. 
        \item Problemi v jeziku koordinat.
        \item Kako lahko dobimo nove točke?
        \item \textcolor{red}{(*)} \textbf{Izrek.} Konstruktibilna števila.
        \item \textcolor{red}{(*)} \textbf{Posledica.} Ali so problemi rešljivi?
    \end{itemize}

    \item Razpadna polja polinomov
    \begin{itemize}
        \item \textcolor{blue}{(*)} \textbf{Trditev.} Naj bo \(f(X) \in F[X]\) nekonstanten polinom. Ali lahko najdemo vsaj eno ničlo?
        \item \textbf{Posledica.} Naj bo \(f(X) \in F[X]\) nekonstanten polinom. Ali lahko najdemo vse ničle?
        \item \textbf{Definicija.} Kdaj pravimo, da polinom razpade nad poljem? Razpadno polje polinoma.
        \item \textbf{Opomba.} Ali razpadno polje vedno obstaja? Ali je razpadno polje polinoma končna razširitev?
    \end{itemize}

    Koliko razpadnih polj ima vsak polinom?
    \begin{itemize}
        \item \textbf{Lema.} \textcolor{red}{(*)}
        \item \textbf{Izrek.} \textcolor{red}{(*)}
        \item \textcolor{blue}{(*)} \textbf{Posledica.} Kaj lahko povemo o razpadnih poljih nekonstantnega polinoma \(f(x) \in F[X]\)? 
    \end{itemize}

    \item Normalne razširitve polj
    \begin{itemize}
        \item \textbf{Definicija.} Normalna razširitev.
        \item \textcolor{blue}{(*)} \textbf{Izrek.} Karakterizacija končnih normalnih razširitev.
        \item \textbf{Zgled.} Normalne in ni normalne razširitve.
    \end{itemize}

    \newpage
    \item Algebraično zaprtje polja
    \begin{itemize}
        \item \textcolor{blue}{(*)} \textbf{Definicija.} Algebraično zaprto polje.
        \item \textbf{Lema.} Naj bodo \(F \subseteq L \subseteq K\) polja. Recimo, da je \(L/F\) algebraična in \(a \in K\) algebraičen nad \(L\). Ali je algebraičen tudi nad \(F\)?
        \item \textcolor{blue}{(*)} \textbf{Definicija.} Algebraično zaprtje polja \(F\).
        \item \textbf{Opomba.} Ali je vsako polje vsebovano v nekem algebraično zaprtem polju? Ali je algebraično zaprtje vedno obstaja? Ali je enolično?
        \item \textbf{Zgled.} Algebraično zaprtje \(\R\). Ali je \(\overline{\Q} = \C\)?
        \item \textcolor{blue}{(*)} \textbf{Izrek.} Kako dobimo \(\overline{F}\), če je polje \(F\) vsebovano v nekem algebraično zaprtem polju \(A\)?
        \item \textcolor{blue}{(*)} \textbf{Zgled.} Algebraično zaprtje \(\Q\).
    \end{itemize}

    \item Končna polja
    
    Naj bo \(K\) končno polje, \(\ch K = p \in \mathbb{P}\).
    \begin{itemize}
        \item Ali je \(\Z_p\) podpolje v \(K\)? Razširitev \(K/\Z_p\).
        \item \textcolor{red}{(*)} \textbf{Trditev.} Moč \(K\).
        \item \textcolor{red}{(*)} \textbf{Lema.} Kaj če je \(K\) polje moči \(p^n\)?
        \item \textcolor{blue}{(*)} \textbf{Lema.}  Kaj če je \(L\) razpadno polje polinoma \(X^{p^n}-X\) nad \(\Z_p\)?
        \item \textbf{Posledica.} Naj bo \(p \in \mathbb{P}\) in \(n \in \N\). Koliko polj moči \(p^n\) obstaja?
        \item \textbf{Trditev.} Kaj lahko povemo o multiplikativni grupi končnega polja?
    \end{itemize}

    \item Separabilne razširitve
    \begin{itemize}
        \item \textbf{Definicija.} Separabilen polinom.
        \item \textbf{Definicija.} Separabilna razširitev.
        \item \textcolor{red}{(*)} \textbf{Izrek.} Kaj lahko povemo o ničlah nerazcepnih polinomov nad poljem z karakteristiko 0?
        \item \textcolor{blue}{(*)} \textbf{Posledica.} Kaj lahko povemo o razširitvah polja z karakteristiko 0?
        \item \textbf{Zgled.} Ali je \(\Q(\sqrt{2}, \sqrt{3}) / \Q\) separabilna? Razširitev, ki ni separabilna.
        \item \textbf{Definicija.} Enostavna razširitev.
        \item \textcolor{blue}{(*)} \textbf{Izrek.} O primitivnem elementu.
        \item \textbf{Definicija.} Perfektno polje.
        \item \textbf{Zgled.} Kaj so gotovo perfektna polja?
        \item \textbf{Trditev.} Ali so končna polja perfektna?
        \item \textcolor{blue}{(*)} \textbf{Definicija.} Galoisjeve razširitve.
        \item \textbf{Zgled.} Recimo, da je \(\ch F = 0\) in \(K\) razpadno polje polinoma \(f(x) \in F[X]\). Ali je \(K/F\) Galoisjeva? Ali je vsaka končna razširitev \(\Z_p\) Galoisjeva?
        \item \textbf{Lema.} Naj bodo \(F \subseteq L \subseteq K\) polja.
        \begin{itemize}
            \item Kaj če je \(K/F\) končna?
            \item Kaj če je \(K/F\) normalna?
            \item Kaj če je \(K/F\) separabilna?
        \end{itemize}

        \item \textbf{Opomba.} Kaj lahko povemo o druge razširitve v verigi? 
    \end{itemize}

\end{enumerate}