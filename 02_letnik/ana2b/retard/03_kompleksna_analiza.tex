\section{Kompleksna analiza}
\subsection{Kompleksna števila}
\begin{definicija}
    Kompleksna števila so urejeni pari realnih števil \((a,b) \in \R \times \R\).
\end{definicija}

V \(\R \times \R\) v vpeljemo operacije 
\begin{align*}
    (a, b) + (c, d) &= (a + c, b + d) \\
    (a, b) \cdot (c, d) &= (ac - bd, ad + bc)
\end{align*}
S tema operacijama \(\R^2\) postane komutativni obseg \(\C\). Pri tem je \(\R \subseteq \C\), če \(a \in \R\) identificiramo z \((a, 0)\), kar inducira običajni operaciji v \(\R\).

\begin{definicija}
    Element \((0,1)\) označimo z \(i\) in imenujemo \textbf{imaginarna enota.}
\end{definicija}

Velja: \[i^2 = -1.\]

Vsako kompleksno število \(z = (a, b)\) lahko predstavimo v obliki 
\[z = a + i b = (a, 0) + (0, 1)(b, 0).\]

Če je \(z = a +ib, \ a, b \in \R\), imenujemo \(a\) \textbf{realni del} kompleksnega števil