\section{Klasifikacija končnih Abelovih grup}
\begin{enumerate}
    \item Direktni produkt

    Naj bo \(G\) grupa.
    \begin{itemize}
        \item \textbf{Definicija.} Direktni notranji produkt edink \(N_1, \ldots, N_s\).
        \item \textbf{Zgled.} Zapis produkta grup kot produkt edink.
        \item \textbf{Lema.} Karakterizacija kdaj je \(G\) notranji direktni produkt edink \(N_1, \ldots, N_s\).
        \item \textbf{Definicija.} Komutator elementov \(x, y \in G\).
        \item \textbf{Opomba.} Kaj in zakaj meri komutator?
        \item \textbf{Lema.} Recimo, da \(M, N \triangleleft G\) ini \(M \cup N = \set{1}\). Kaj lahko povemo o elementih \(M\) in \(N\)?
        \item \textbf{Izrek.} Kaj če \(G\) notranji direktni produkt edink \(N_1, \ldots, N_s\). \todo{*}
        \item \textbf{Zgled.} \
        \begin{itemize}
            \item Ali zapis grupe \(G\) kot notranji direktni produkt vedno obstaja?
            \item Zapiši \(D_4\) kot notranji direktni produkt pravih edink. Čemu je izomorfna \(D_4\)?
            \item Zapiši \(\GL_n(\R)\), kjer je \(n\) liho število, kot notranji direktni produkt \(\SL_n(\R)\) in grupe skalarnih matrik. Čemu je enak center grupe \(\GL_n(\R)\)?
        \end{itemize}
        \item \textbf{Opomba.} Neskončni notranji produkt. Ali izrek še vedno drži?
        \item \textbf{Definicija.} Naj bo \(G\) Abelova. Direktna vsota edink \(N_1, \ldots, N_s\).
    \end{itemize}

    \item Klasifikacija končnih grup
    
    Naj bo \(G\) končna Abelova grupa z operacijo seštevanja.
    \begin{itemize}
        \item \textbf{Lema.} Recimo, da je \(|G| = m \cdot n\), kjer sta \(m, n\) tuji. Kako lahko zapišemo \(G\) kot direktno vsoto?
        \item \textbf{Zgled.} Dokaži: če sta \(m, n\) tuji, potem \(\Z_m \oplus \Z_n \approx \Z_{mn}\). Ali je \(\Z_2 \oplus \Z_2 \approx \Z_4\)?
        \item \textbf{Posledica.} Kako lahko zapišemo vsako grupo moči \(n\)?
        \item \textbf{Definicija.} \(p\)-grupa.
        \item \textbf{Opomba.} Ali je vsaka končna Abelova grupa direktna vsota \(p_i\)-grup?
        \item \textbf{Lema.} Kdaj je \(p\)-grupa ciklična?
        \item \textbf{Lema.} Ali lahko vsako \(p\)-grupo zapišemo kot vsoto ciklične podgrupe in neke druge podgrupe?
        \item \textbf{Posledica.} Ali vsako \(p\)-grupo lahko zapišemo kot direktno vsoto cikličnih grup? Ali vsako grupo lahko zapišemo kot direktno vsoto cikličnih grup?
        \item \textbf{Opomba.} Kako vidimo, ali dva razcepa Abelovih grup na direktni vsoti cikličnih \(p_i\)-grup prestavljata isto grupo do izomorfizma natančno?
        \item \textbf{Izrek.} Kdaj sta končni Abelovi \(p\)-grupi izomorfni?
        \item \textbf{Povzetek.} Čemu je izomorfna vsaka končna Abelova grupa? \todo{*}
        \item \textbf{Zgled.} Poišči vse Abelove grupe moči \(432\).
    \end{itemize}

    \item Klasifikacija končno generiranih Abelovih grup
    
    Naj bo \(G\) končno generirana Abelova grupa.
    \begin{itemize}
        \item \textbf{Izrek.} Čemu je izomorfna grupa \(G\)? Torzijska podgrupa. Kdaj pravimo, da je \(G\) brez torzije? \todo{*}
        \item \textbf{Opomba.} Kaj je potenca \(n\) v izomorfizmu iz prejšnjega izreka?
        \item \textbf{Trditev.} Ali lahko vsako končno generirano Abelovo grupo zapišemo kot direktno vsoto končne Abelove grupe in neke druge?
        \item \textbf{Opomba.} Ali iz tega, da je \(G\) Abelova in ima vsak element končen red sledi, da je \(G\) končna?
    \end{itemize}

\item Delovanja grup
    
    Naj bo \(G\) grupa in \(X\) neprazna množica.
    \begin{itemize}
        \item \textbf{Definicija.} Kadar pravimo, da \(G\) deluje na \(X\)? Delovanje.
        \item \textbf{Opomba.} Ali pri vektorskih prostorih polje deluje na vektorji? Ali iz 1.\ pogoja sledi 2.\ pogoj? Levo in desno delovanje. Kako iz levega delovanja pridemo do desnega?
        \item \textbf{Zgled.} Delovanje porodi homomorfizem \(G \to \Sym X\) in obratno.
        \item \textbf{Definicija.} Jedro delovanja. Zvesto delovanje. Kdaj pravimo, da se \(G\) vloži v \(\Sym X\)?
        \item \textbf{Zgled.} \
        \begin{itemize}
            \item Trivialno delovanje.
            \item Levo množenje. Cayleyjev izrek. Levo regularno delovanje.
            \item Delovanje grupe \(G\) na množico \(G\) z konjugiranjem.
            \item Naj bo \(H \leq G\). Delovanje \(G\) na \(G/_H\) s predpisom \(g \cdot hH = (gh)H\).
            \item Naj \(G\) deluje na množice \(X\). Naj bo \(Y\) neprazna množica. Delovanje \(G\) na množice \(Y^X\) s predpisom \(g \cdot f = x \mapsto f(g^{-1} \cdot x)\).
            \item Naj bo \(G\) deluje na \(X\) in na \(Y\). Naj bo \(Y\) neprazna množica. Delovanje \(G\) na množice \(Y^X\) s predpisom \(g \cdot f = g * f(g^{-1} \cdot x)\)
            \item Naj bo \(V\) vektorski prostor nad \(\F\). Delovanje grupe avtomorfizmov na množico vektorjev. 
            \item Naj bo \(K\) komutativen kolobar. Gledamo \(K[x_1, x_2, \ldots, x_n]\). Delovanje \(S_n\) na \(K[x_1, x_2, \ldots, x_n]\) s permutacijo spremenljivk.
        \end{itemize}        
    \end{itemize}

    \item Orbite, stabilizatorje in fiksne točke delovanj
    
    Naj grupa \(G\) deluje na množice \(X\).
    \begin{itemize}
        \item \textbf{Definicija.} Orbita elementa \(x \in X\). Stabilizator elementa \(x \in X\). Množica fiksnih točk elementa \(g \in G\). Fiksne točke delovanja.
        \item \textbf{Lema.} Čemu je enak \(x \in X\), če \(g \cdot x = y\)?
        \item \textbf{Trditev.} Ali je \(G_x \leq G\)?
        \item \textbf{Trditev.} Ekvivalenčna relacija na \(X\), ki jo porodi delovanje. Kaj so ekvivalenčni razredi?
        \item \textbf{Posledica.} Kaj lahko povemo o orbitah? Prostor orbit.
        \item \textbf{Definicija.} Tranzitivno delovanje.
        \item \textbf{Zgled.} Določi orbite, stabilizatorji ter fiksne točki delovanj:
        \begin{itemize}
            \item Naj bo \(G\) deluje na \(G\) z levim množenjem. Ali je tranzitivno?
            \item Naj bo \(G\) deluje na \(G\) s konjugiranjem. Konjugirani razred elementa \(x \in G\).
            \item Naj bo \(H \leq G\). \(G\) deluje na \(G/_H\).
            \item Naj bo \(S_n\) deluje na \(K[x_1, \ldots, x_n]\) [le fiksne točke]. Simetrični polinomi.
        \end{itemize}
        \item \textbf{Izrek.} Izrek o orbiti in stabilizatorju. \todo{*}
        \begin{proof}
            Dovolj dokazati bijekcijo med \(G \cdot x\) in \(G/_{G_x}\).
        \end{proof}
        \item \textbf{Izrek.} Recimo, da \(G\) deluje na končni množici \(X\). Kako lahko zapišemo moč \(X\)?
        \item \textbf{Posledica.} Naj bo \(G\) končna \(p\)-grupa, ki deluje na končni množici \(X\). Kakšna je zvezna med \(|X|\) in \(|X^G|\)?
        \newpage
        \item \textbf{Lema.} Burnsideova lema (število orbit).
        \begin{proof}
            Izračunamo moč množice \(A = \setb{(g, x) \in G \times X}{g \cdot x = x}\).
        \end{proof}
        \item \textbf{Zgled.} Naj barvamo oglišča kvadrata z \(n\) barvami, pri tem med samo identificiramo barvanja, če je eno rotacije druge. Koliko barvanj obstaja?
    \end{itemize}

    \item Razredna formula in Cauchyjev izrek
    \begin{itemize}
        \item \textbf{Posledica.} Razredna formula.
        \begin{proof}
            Splošna formula + delovanje s konjugiranjem.
        \end{proof}
        \item \textbf{Posledica.} Ali lahko ima \(p\)-grupa trivialen center?
        \item \textbf{Posledica.} Kaj lahko povemo o grupi moči \(p^2\), kjer je \(p\) praštevilo?
        \item \textbf{Izrek.} Cauchyjev izrek. \todo{*}
        \begin{proof}
            Z indukcijo po \(|G|\). Uporabimo razredno formulo. \(p\) lahko deli \(|Z(G)|\) ali ne.
        \end{proof}
    \end{itemize}

    \item Izrek Sylowa
    
    Lagrangeev izrek za končne grupe pove, da moč vsake podgrupe deli moč grupe. Kaj pa obrat? Ali za vsak delitelj moči grupe lahko najdemo podgrupo dane moči?
    
    \ 

    Naj bo \(G\) končna grupa ter \(H \leq G\).
    \begin{itemize}
        \item \textbf{Definicija.} \(p\)-podgrupa Sylowa.
        \item \textbf{Izrek.} Izrek Sylowa. \todo{*}
        \begin{proof}
            \todo{}
        \end{proof}
        \item \textbf{Opomba.} Kdaj je \(n_p = 1\)? Povezava z edinki.
        \item \textbf{Zgled.} Naj bosta \(p, q\) različni praštevili ter \(p < q\). Kaj lahko povemo o grupah moči \(p \cdot q\)?
    \end{itemize}

    \item Končne enostavne grupe
    \begin{itemize}
        \item \textbf{Definicija.} Enostavna grupa.
        \item \textbf{Zgled.} kdaj je končna Abelova grupa enostavna?
        \item \textbf{Zgled.} Ali je \(A_3\) enostavna? Kaj pa \(A_4\)?
        \item \textbf{Izrek.} Kaj lahko povemo o enostavnosti \(A_n\) za \(n \geq 5\)?
        \item \textbf{Opomba.} Kako lahko klasificiramo končne enostavne grupe?
        \item \textbf{Opomba.} Zakaj so enostavne grupe dobre? Kompozicijska vrsta grupe.
    \end{itemize}

    \item Rešljive grupe
    \begin{itemize}
        \item \textbf{Definicija.} Rešljiva grupa.
        \item \textbf{Zgled.} Ali so rešljive:
        \begin{itemize}
            \item Abelove grupe;
            \item \(A_4\) ter \(S_4\);
            \item nekomutativna enostavna grupa \(G\), \(A_n\) za \(n \geq 5\).
        \end{itemize}
        \item \textbf{Trditev.} Kaj lahko povemo o podgrupah rešljivih grup? Kaj lahko povemo o faktorske grupe rešljive grupe?
        \item \textbf{Trditev.} Zadosten pogoj, da je \(G\) rešljiva.
        \item \textbf{Opomba.} Ali so vse grupe lihe moči rešljive? Ali je vsaka končna \(p\)-grupa rešljiva?
        \begin{proof}
            \(Z(G) \neq \set{1}\). Z indukcijo po \(|G|\).
        \end{proof}
    \end{itemize}
\end{enumerate}