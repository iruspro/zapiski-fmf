\section{Delovanja grup}
\begin{enumerate}
    \item Delovanja grup
    
    Naj bo \(G\) grupa in \(X\) neprazna množica.
    \begin{itemize}
        \item \textbf{Definicija.} Kadar pravimo, da \(G\) deluje na \(X\)? Delovanje.
        \item \textbf{Opomba.} Ali pri vektorskih prostorih polje deluje na vektorji? Ali iz 1.\ pogoja sledi 2.\ pogoj? Levo in desno delovanje. Kako iz levega delovanja pridemo do desnega?
        \item \textbf{Zgled.} Delovanje porodi homomorfizem \(G \to \Sym X\) in obratno.
        \item \textbf{Definicija.} Jedro delovanja. Zvesto delovanje. Kdaj pravimo, da se \(G\) vloži v \(\Sym X\)?
        \item \textbf{Zgled.} \
        \begin{itemize}
            \item Trivialno delovanje.
            \item Levo množenje. Cayleyjev izrek. Levo regularno delovanje.
            \item Delovanje grupe \(G\) na množico \(G\) z konjugiranjem.
            \item Naj bo \(H \leq G\). Delovanje \(G\) na \(G/_H\) s predpisom \(g \cdot hH = (gh)H\).
            \item Naj \(G\) deluje na množice \(X\). Naj bo \(Y\) neprazna množica. Delovanje \(G\) na množice \(Y^X\) s predpisom \(g \cdot f = x \mapsto f(g^{-1} \cdot x)\).
            \item Naj bo \(G\) deluje na \(X\) in na \(Y\). Naj bo \(Y\) neprazna množica. Delovanje \(G\) na množice \(Y^X\) s predpisom \(g \cdot f = g * f(g^{-1} \cdot x)\)
            \item Naj bo \(V\) vektorski prostor nad \(\F\). Delovanje grupe avtomorfizmov na množico vektorjev. 
            \item Naj bo \(K\) komutativen kolobar. Gledamo \(K[x_1, x_2, \ldots, x_n]\). Delovanje \(S_n\) na \(K[x_1, x_2, \ldots, x_n]\) s permutacijo spremenljivk.
        \end{itemize}        
    \end{itemize}

    \item Orbite, stabilizatorje in fiksne točke delovanj
    
    Naj grupa \(G\) deluje na množice \(X\).
    \begin{itemize}
        \item \textbf{Definicija.} Orbita elementa \(x \in X\). Stabilizator elementa \(x \in X\). Množica fiksnih točk elementa \(g \in G\). Fiksne točke delovanja.
        \item \textbf{Lema.} Čemu je enak \(x \in X\), če \(g \cdot x = y\)?
        \item \textbf{Trditev.} Ali je \(G_x \leq G\)?
        \item \textbf{Trditev.} Ekvivalenčna relacija na \(X\), ki jo porodi delovanje. Kaj so ekvivalenčni razredi?
        \item \textbf{Posledica.} Kaj lahko povemo o orbitah? Prostor orbit.
        \item \textbf{Definicija.} Tranzitivno delovanje.
        \item \textbf{Zgled.} Določi orbite, stabilizatorji ter fiksne točki delovanj:
        \begin{itemize}
            \item Naj bo \(G\) deluje na \(G\) z levim množenjem. Ali je tranzitivno?
            \item Naj bo \(G\) deluje na \(G\) s konjugiranjem. Konjugirani razred elementa \(x \in G\).
            \item Naj bo \(H \leq G\). \(G\) deluje na \(G/_H\).
            \item Naj bo \(S_n\) deluje na \(K[x_1, \ldots, x_n]\) [le fiksne točke]. Simetrični polinomi.
        \end{itemize}
    \end{itemize}
\end{enumerate}