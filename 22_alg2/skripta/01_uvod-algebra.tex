\section{Uvod v teorijo grup}

\begin{enumerate}
    \item Osnovni pojmi teoriji grup
    \begin{itemize}
        \item \textbf{Definicija.} Binarna operacija na množice $S$. Kadar pravimo, da je operacija asociativna. Kadar pravimo, da je operacija komutativna?       
        \item \textbf{Definicija.} Polgrupa.
        \item \textbf{Definicija.} Nevtralni element.
        \item \textbf{Trditev.} Ali če v množici \(S\) obstaja enota za operacijo \(*\), potem je ena sama?
        \item \textbf{Definicija.} Monoid.
        \item \textbf{Definicija.} Levi inverz. Desni inverz. Inverz. 
        \item \textbf{Definicija.} Obrnljiv element.
        \item \textbf{Trditev.} Kaj če v monoidu ima element \(x\) levi in desni inverz?
        \item \textbf{Posledica.} Koliko inverzov lahko ima obrnljiv element v monoidu?
        \item \textbf{Posledica.} Kaj če je \(x\) obrnljiv element monoida in \(xy = 1\)?
        \item \textbf{Trditev.} Obrnljivost produkta obrnljivih elementov.
        \item \textbf{Definicija.} Grupa. Abelova grupa.
        \item \textbf{Definicija.} Multiplikativni in aditivni zapis operacije. Kdaj jih uporabljamo?
        \item \textbf{Trditev.} Računanje z potenci v grupi. Pravilo krajšanja v grupi.
        \item \textbf{Zgled.} Primeri številskih grup. Simetrična grupa množice \(X\). Grupa permutacij.
        \item \textbf{Zgled.} Grupa simetrij kvadrata. Diedrska grupa \(D_{2n}\) moči \(2n\).
        \item \textbf{Zgled.} Kako iz monoida dobimo grupo? Splošna linearna grupa \(\GL_n(\F)\).
        \item \textbf{Zgled.} Direktni produkt grup. 
    \end{itemize}   
    
    \item Grupa permutacij \(S_n\)
    \begin{itemize}
        \item \textbf{Izrek.} Kako lahko zapišemo vsako permutacijo?
        \item \textbf{Definicija.} Transpozicija.
        \item \textbf{Trditev.} Kako lahko zapišemo vsako permutacijo z pomočjo transpozicij? Koliko je transpozicij v tem zapisu?
        \item \textbf{Definicija.} Soda permutacija. Liha permutacija. Znak permutacije.
        \item \textbf{Trditev.} Znak produkta permutacij.
    \end{itemize}

    \item Podgrupe
    \begin{itemize}
        \item \textbf{Definicija.} Podgrupa.
        \item \textbf{Opomba.} Kaj sta vedno podgrupi grupe \(G\)? Ali je enota vedno vsebovana v podgrupi? Ali se enota deduje pri monoidih?
        \item \textbf{Trditev.} Dve karakterizaciji podgrupe.
        \item \textbf{Posledica.} Karakterizacija podgrupe končne grupe \(G\).
        \item \textbf{Zgled.} \ 
        \begin{itemize}
            \item \textcolor{red}{(*)}  Kakšne so oblike vse prave podgrupe grupe \(\Z\)?
            \item Specialna linearna grupa \(\SL_n(\F)\). Grupa ortogonalnih matrik \(\text{O}_n(\F)\). Specialna grupa ortogonalnih matrik \(\text{SO}_n(\F)\).
        \end{itemize}
        \item \textbf{Trditev.} Ali je presek podgrup grupe \(G\) podgrupa grupe \(G\)?
        \item \textbf{Definicija.} Produkt podgrup.
        \item \textbf{Zgled.} Ali je produkt podgrup vedno podgrupa?
        \item \textbf{Trditev.} Zadosten pogoj, da je produkt podgrup podgrupa.
        \item \textbf{Zgled.} Konjugiranje podgrupe \(H \leq G\) z elementom \(a \in G\). Ali je konjugiranje podgrupa?
        \newpage
        \item \textbf{Zgled.} Center \(Z(G)\) grupe \(G\). Centralizator \(C_a(G)\) elementa \(a \in G\). Ali sta podgrupi?
        \item \textbf{Zgled.} Krožna grupa \(\mathbb{T}\). \(n\)-ti koreni enote \(\textbf{U}_n\). Ali sta podgrupi \(\C^*\)?
        \item \textbf{Zgled.} Alternirajoča grupa \(A_n\).
    \end{itemize}

    \item Odseki podgrup in Lagrangeev izrek
    
    Naj bo \(G\) grupa in \(H \leq G\).
    \begin{itemize}
        \item Relacija \(\sim\) na \(G\), ki porodi leve odseke.
        \item \textbf{Trditev.} Ali je relacija \(\sim\) ekvivalenčna?
        \item \textbf{Definicija.} Ekvivalenčni razred elementa \(a \in G\).
        \item \textbf{Definicija.} Ekvivalenčne razredi po relaciji \(\sim\). Levi odseki \(G\) po podgrupe \(H\).
        \item \textbf{Opomba.} Z kakšno ekvivalenčno relacijo dobimo desne odseke?
        \item \textbf{Definicija.} Kvocientna množica glede na relacijo \(\sim\).
        \item \textbf{Opomba.} Kaj tvorijo ekvivalenčni razredi glede na množico \(G\)?
        \item \textbf{Opomba.} Ali je \(G/H\) vedno grupa? Kadar sta dva odseka enaka? Ali je \(G/H\) končna, če je \(G\) končna?
        \item \textbf{Definicija.} Indeks podgrupe \(H\).
        \item \textcolor{red}{(*)} \textbf{Izrek.} Lagrangeev izrek.
        \item \textbf{Posledica.} Ključni pomen izreka.
        \item \textbf{Opomba.} Kako lahko definiramo operacijo na \(G/H\), če je \(G\) Abelova?
        \item \textbf{Trditev.} Ali je s prej definirano operacijo \(G/H\) Abelova grupa?
        \item \textbf{Zgled.} Grupa ostankov po modulu \(n\). Ali za vsako naravno število \(n\) obstaja grupa moči \(n\)?
    \end{itemize}

    \item Generatorji grup. Ciklične grupe
    
    Naj bo \(G\) grupa ter \(X \subseteq G\).
    \begin{itemize}
        \item \textbf{Definicija.} Podgrupa, generirana z množico \(X\).
        \item \textbf{Opomba.} Ali je \(\gen{X}\) vedno obstaja?
        \item \textbf{Definicija.} Grupa, generirana z množico \(X\). Generatorji grupe. Končno generirana grupa. Ciklična grupa.
        \item \textbf{Trditev.} Kako zgledajo elementi \(\gen{X}\)?
        \item \textbf{Posledica.} Kako zgledajo elementi \(\gen{x}\)?
        \item \textbf{Zgled.} Generatorji grup \(\Z\) in \(\Z_n\). 
        \item \textbf{Zgled.} S čim sta generirani grupi \(D_{2n}\) in \(S_n\)? Ali je \(A_n\) generirana z 3-cikli?
        \item \textbf{Zgled.} Ali je grupa \(\textbf{U}_n\) ciklična? Kaj pa \(D_4\)?
        \item \textbf{Zgled.} Ali je \(\Q^*\) končno generirana?
        \item \textcolor{blue}{(*)}  \textbf{Definicija.} Red elementa.
        \item \textbf{Zgled.} Kateri elementi v grupi imajo red \(1\)? Kakšen red imajo transpozicije v grupi \(S_n\)?
        \item \textcolor{blue}{(*)} \textbf{Trditev.} Karakterizacija reda elementa.
        \item \textcolor{blue}{(*)} \textbf{Posledica.} Kdaj je končna grupa \(G\) ciklična?
        \item \textcolor{blue}{(*)} \textbf{Posledica.} Kaj lahko povemo o redu elementa \(a\) v končni grupi? Kaj če je \(|G|\) praštevilo?
    \end{itemize}
\end{enumerate}

\newpage
\subsection*{Rezultati vaj}
\begin{enumerate}
    \item Monoidi     
    \begin{itemize}
        \item (naloga 2.21) Ali je v končnem monoidu levi inverz avtomatično tudi desni inverz? Kakšno obliko ima? 
        \item (naloga 2.22) Ali je element monoida obrnljiv, če obrnljiva neka njegova potenca? 
    \end{itemize}

    \item Grupe
    \begin{itemize}
        \item (naloga 3.10) Ali je polgrupa z deljenjem grupa?
        \item (naloga 3.9) Zadostni pogoj, da je grupa Abelova.
    \end{itemize}
    \item Grupa permutacij
    \begin{itemize}
        \item Kako zapišemo permutacijo kot produkt transpozicij?
        \item (naloga 3.13) Kako dobimo inverz $k$-cikla?
        \item (naloga 3.19) Konjugiranje cikla.
        \item (naloga 3.20) Kadar pravimo, da permutaciji $\pi, \pi' \in S_n$ imata enako zgradbo disjunktnih ciklov?
        \item (naloga 3.21) Kako sta povezana komutativnost in konjugiranje?
        \item (naloga 3.103) S čim je generirana grupa $S_n$?
    \end{itemize}

    \item Diedrska grupa
    \begin{itemize}
        \item (naloga 3.22) Grupa $D_\infty$.
    \end{itemize}

    \item Podgrupe
    \begin{itemize}
        \item (naloga 3.31) Diagonalna podgrupa.
        \item (naloga 3.60) Naj bosta $H, G \leq G$, $H, G$ končni. Čemu je enaka $|HK|$?
    \end{itemize}

    \item Ciklične grupe
    \begin{itemize}
        \item (naloga 3.71) Kadar je $\Z_n$ vsebuje podgrupo reda $k$? Ali je ta podgrupa enolična?
        \item (naloga 3.72) Kaj lahko povemo o vsake podgrupe ciklične grupe?
        \item (naloga 3.81) Naj bo $k \in \Z_n$. Čemu je enak $\red (k)$? Kadar je $\left\langle k \right\rangle  = \Z_n$?
        \item (naloga 3.85) Ali je konjugiranje ohranja red elementa?
    \end{itemize}
\end{enumerate}
