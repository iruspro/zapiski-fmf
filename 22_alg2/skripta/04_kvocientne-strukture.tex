\section{Kvocientne strukture}

\begin{enumerate}
    \item Kvocientne grupe
    
    Naj bo \(G\) grupa in \(H \leq G\). Kdaj lahko na množici \(G/_H\) vpeljemo operacijo z predpisom 
    \[(aH) \cdot (bH) = (ab)H?\]
    \begin{itemize}
        \item \textbf{Zgled.} Kdaj ne moremo vpeljati tako operacijo?
        \item \textbf{Definicija.} Podgrupa edinka v \(G\).
        \item \textbf{Zgled.} Kaj so vedno edinki v \(G\)? Enostavne grupe. Kaj so edinki v Abelovih grupih? Nekomutativna grupa, kjer je vsaka podgrupa edinka. Edinki v \(S_3\).
        \item \textbf{Trditev.} 4 karakterizacije edink. 
        \item \textbf{Trditev.} Zadosten pogoj, da je grupa edinka (indeks podgrupe).
        \begin{proof}
            Karakterizacija \(aH = Ha\).
        \end{proof}
        \item \textbf{Zgled.} Ali je \(A_n \triangleleft S_n\)? Ali je \(\gen{r} \triangleleft D_{2n}\)?
        \item \textbf{Trditev.} Recimo, da \(H \leq G\) in \(N \triangleleft G\). Kaj lahko povemo o produktu podgrup? Kaj če tudi \(H \triangleleft G\)?
        \begin{proof}
            Definicija podgrupe ednike.
        \end{proof}
        \item \textbf{Izrek.} Kvocientna grupa. Epimorfizem \(\pi\) grup \(G\) in \(G/_N\). Jedro \(\ker \pi\).
        \item \textbf{Izrek.} 1.\ izrek o izomorfizmu.
        \item \textbf{Opomba.} Kaj so edinke (jedra)? Kanonični epimorfizem. Diagram.
        \item \textbf{Izrek.} 2.\ izrek o izomorfizmu.
        \item \textbf{Izrek.} 3.\ izrek o izomorfizmu.
    \end{itemize}


\end{enumerate}

