\section{Kvocientne strukture}

\begin{enumerate}
    \item Kvocientne grupe
    
    Naj bo \(G\) grupa in \(H \leq G\). Kdaj lahko na množici \(G/_H\) vpeljemo operacijo z predpisom 
    \[(aH) \cdot (bH) = (ab)H?\]
    \begin{itemize}
        \item \textcolor{blue}{(*)} \textbf{Zgled.} Kdaj ne moremo vpeljati tako operacijo?
        \item \textcolor{blue}{(*)} \textbf{Definicija.} Podgrupa edinka v \(G\).
        \item \textbf{Zgled.} Kaj so vedno edinki v \(G\)? Enostavne grupe. Center grupe. Kaj so edinki v Abelovih grupih? Nekomutativna grupa, kjer je vsaka podgrupa edinka. Edinki v \(S_3\).
        \item \textcolor{blue}{(*)} \textbf{Trditev.} 3 karakterizacije edink. 
        \item \textcolor{blue}{(*)} \textbf{Trditev.} Zadosten pogoj, da je grupa edinka (indeks podgrupe).
        \begin{proof}
            Karakterizacija \(aH = Ha\).
        \end{proof}
        \item \textbf{Zgled.} Ali je \(A_n \triangleleft S_n\)? Ali je \(\gen{r} \triangleleft D_{2n}\)?
        \item \textbf{Trditev.} Recimo, da \(H \leq G\) in \(N \triangleleft G\). Kaj lahko povemo o produktu podgrup? Kaj če tudi \(H \triangleleft G\)? Presek edink.
        \begin{proof}
            Karakterizacija \(aH = Ha\) in definicija.
        \end{proof}
        \item \textcolor{blue}{(*)} \textbf{Izrek.} Kvocientna grupa. Epimorfizem \(\pi\) grup \(G\) in \(G/_N\). Jedro \(\ker \pi\).
        \item \textcolor{red}{(*)} \textbf{Izrek.} 1.\ izrek o izomorfizmu.
        \item \textbf{Opomba.} Kaj so edinke (jedra)? Kanonični epimorfizem. Diagram.
        \item \textcolor{blue}{(*)} \textbf{Izrek.} 2.\ izrek o izomorfizmu.
        \begin{proof}
            Ideja: 1.\ izrek o izomorfizmu.
        \end{proof}
        \item \textcolor{blue}{(*)} \textbf{Izrek.} 3.\ izrek o izomorfizmu.
        \begin{proof}
            Ideja: 1.\ izrek o izomorfizmu.
        \end{proof}
        \item \textbf{Lema.} Naj bo \(\varphi: G \to H\) homomorfizem grup, \(K \subseteq G, \ L \subseteq H\).
        \begin{itemize}
            \item Zadosten pogoj, da je \(\img{\varphi}(K) \leq H\);
            \item Zadosten pogoj, da je \(\img{\varphi}(K) \triangleleft H\);
            \item Zadosten pogoj, da je \(\invimg{\varphi}(L) \leq G\);
            \item Zadosten pogoj, da je \(\invimg{\varphi}(L) \triangleleft G\).
        \end{itemize}
        \item \textcolor{blue}{(*)} \textbf{Izrek.} Korespondenčni izrek.
    \end{itemize}

    \item Uporaba izrekov
    \begin{itemize}
        \item \textcolor{blue}{(*)} \textbf{Trditev.} Opis cikličnih grup do izomorfizma natančno.
        \item \textcolor{blue}{(*)} \textbf{Trditev.} Opis podgrup v \(\Z_n\).
        \item \textcolor{blue}{(*)} \textbf{Trditev.} Naj bo \(G\) netrivialna grupa. Kdaj nima \(G\) pravih netrivialnih podgrup?
        \item \textbf{Lema.} Naj bo \(G\) grupa, \(N \triangleleft G\) in \(a \in G\). Kaj lahko povemo o redu elementa \(aN\), če red elementa \(a\) enak \(n \in \N\)?
        \item \textcolor{red}{(*)} \textbf{Izrek.} Cauchyjev izrek za Abelove grupe.
        \begin{proof}
            Indukcija po \(n = |G|\).
        \end{proof}
        \item \textbf{Zgled.} Čemu so izomorfne grupe \(S_n/_{A_n}\) in \(G/_{Z(G)}\)? Ali so kvocienti dobro definirani?
        \item \textbf{Zgled.} Naj bosta \(G_1, G_2\) grupi. "`Navadni'" edinki v \(G_1 \times G_2\). Kaj pa kvocienta?
    \end{itemize}

    \newpage
    \item Kvocientni kolobarji
    
    Naj bo \(K\) kolobar ter \((I, +) \leq (K, +)\). Radi bi na \(K/_I\) vpeljali množenje z predpisom
    \[(a + I) \cdot (b + I) = ab + I.\]
    \begin{itemize}
        \item \textcolor{blue}{(*)} \textbf{Definicija.} Ideal. Levi (desni) ideal.
        \item \textbf{Zgled.} Kaj so vedno ideali v \(K\)? Enostavni kolobarji. \(aK\) in \(Ka\) kot ideali. Glavni ideal. Ideali v \(\Z\).
        \item \textbf{Zgled.} Desni ideal, ki ni levi v \(\R^{2 \times 2}\). Levi ideal, ki ni desni v \(\R^{2 \times 2}\). Ali je \(\R^{n \times n}\) enostaven?
        \item \textbf{Opomba.} Ideali v algebri.
        \item \textbf{Trditev.} Kvocientni kolobar.
        \item \textcolor{blue}{(*)} \textbf{Trditev.} Kaj če (levi/desni) ideal vsebuje obrnljiv element?
        \begin{proof}
            \(a \in I\) obrnljiv, poten \(\all{x \in K} x = xa^{-1}a\).
        \end{proof}
        \item \textbf{Trditev.} Presek idealov. Produkt idealov. Vsota idealov.
        \item \textbf{Definicija.} Glavni ideal.
        \item \textcolor{blue}{(*)} \textbf{Izrek.} 1.\ izrek o izomorfizmu.
        \item \textbf{Opomba.} Kaj so ideali (jedra)? Kanonični epimorfizem. Diagram.
        \item \textbf{Izrek.} 2.\ izrek o izomorfizmu.
        \item \textbf{Izrek.} 3.\ izrek o izomorfizmu.
        \item \textbf{Izrek.} Korespondenčni izrek.
        \item \textcolor{blue}{(*)} \textbf{Definicija.} Maksimalen ideal.
        \item \textcolor{red}{(*)} \textbf{Izrek.} Karakterizacija maksimalnih idealov. \todo{*}
        \begin{proof}
            \((\Rightarrow)\) Naj bo \(a + M \in K/_M \setminus \set{0}\). Oglejmo si ideal \(M + aK\).

            \((\Leftarrow)\) Vzemimo strogo večji od \(M\) ideal.
        \end{proof}
        \item \textbf{Opomba.} Zakaj potrebujemo predpostavko o komutativnosti?
        \item \textcolor{blue}{(*)} \textbf{Izrek.} Ali je vsak pravi ideal vsebovan v nekem maksimalnem idealu? \textcolor{red}{(*)}
    \end{itemize}
\end{enumerate}

\newpage
\subsection*{Rezultati z vaj}
\begin{enumerate}
    \item Kvocientne grupe
    
    \begin{itemize}
        \item (naloga 6.11) Ali lahko kvocient po \(Z(G)\) nekomutativne grupe \(G\) cikličen?
        \item (naloga 6.66) Ali iz pogoja \(N \triangleleft G, \ N \neq \set{1}\) sledi, da \(G \not \approx G/_N\)?
    \end{itemize}

    \item Kvocientne kolobarji
    \begin{itemize}
        \item (naloga 6.27) Ali je kolobar \(M_n(D)\) enostaven, če je \(D\) obseg?
        \item (naloga 6.29) Kdaj je komutativen kolobar \(K\) enostaven?
        \item (naloga 6.32) Kakšne oblike so ideali v direktnem produktu kolobarjev?
    \end{itemize}
\end{enumerate}

