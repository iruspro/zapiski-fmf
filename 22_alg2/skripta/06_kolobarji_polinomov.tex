\section{Kolobarji polinomov}

Gledamo polinome nad poljem \(F\), torej kolobar \(F[X]\).
\begin{enumerate}
    \item Kolobarji polinomov
    \begin{itemize}
        \item Zapis polinoma stopnje \(n\). Stopnja ničelnega polinoma.
        \item Čemu je enaka stopnja produkta polinomov?
        \item \textbf{Lema.} Ali ima kolobar \(F[X]\) delitelji niča? Kaj so njegove obrnljive elemente?
        \item \textbf{Izrek.} Osnovni izrek o deljenju polinomov.
        \item \textbf{Posledica.} Kaj lahko povemo o vsakem idealu v kolobarju \(F[X]\)?
        \item Ničla polinoma.
        \item \textbf{Opomba.} Ali lahko v splošnem identificiramo polinomi s polinomskimi funkciji?
        \item \textbf{Trditev.} Karakterizacija ničle polinoma.
        \item \textbf{Posledica.} Koliko ničel lahko ima neničeln polinom?
    \end{itemize}

    \item Nerazcepni polinomi
    
    Naj bo \(F\) polje.
    \begin{itemize}
        \item \textbf{Definicija.} Nerazcepni polinom nad \(F\).
        \item \textbf{Zgled.} Kaj so nerazcepni polinomi v \(\C[X]\)? Kaj so v \(\R[X]\)?
        \item \textbf{Trditev.} Naj bo \(p(X) \in F[X]\) stopnje vsaj \(1\).
        \begin{itemize}
            \item Kaj če stopnja \(p(X)\) enaka 1?
            \item Kaj če stopnja vsaj 2 ter \(p(X)\) nerazcepen? (ničle)
            \item Kdaj je polinom stopnje 2 ali 3 nerazcepen?
        \end{itemize}
    \end{itemize}

    Od tod dalje gledamo kolobar \(\Q[X]\). Če polinom \(p(X) \in \Q[X]\) pomnožimo s skupnim imenovalcem koeficientov dobimo polinom v \(\Z[X]\).
    \begin{itemize}
        \item \textbf{Definicija.} Primitiven polinom.
        \item \textbf{Trditev.} Gaussova lema.
        \item \textbf{Izrek.} Zadosten pogoj, da je polinom \(p(X) \in \Q[X]\) nerazcepen. (kolobar \(\Z[X]\))
        \item \textbf{Izrek.} Eisensteinov kriterij.
        \item \textbf{Zgled.} Ciklotomični polinomi.
    \end{itemize}

    \item Razširitve polj
    \begin{itemize}
        \item \textbf{Definicija.} Razširitev polja.
        \item \textbf{Zgled.} Razširitev \(\Q\) z \(\R\). \(\C\) kot razširitev. Polje \(\Q(\sqrt{2})\).
        \item \textbf{Definicija.} Algebraičen element. Transcendenten element.
        \item \textbf{Zgled.} Ali sta \(\pi\) in \(\sqrt{2}\) algebraična nad \(\Q\)?
        \item \textbf{Definicija.} Minimalni polinom elementa.
        \item \textbf{Zgled.} Ali minimalni polinom, če obstaja, enolično določen?
        \item \textbf{Trditev.} 2 karakterizacije minimalnosti polinoma.
        \item \textbf{Definicija.} Stopnja elementa.
        \item \textbf{Zgled.} \
        \begin{itemize}
            \item Kaj so algebraični elementi stopnje \(1\) v \(K/F\)?
            \item Kakšno stopnjo ima \(\sqrt{2}\) v \(\Q(\sqrt{2})/\Q\)?
            \item Kakšno stopnjo imajo elementi \(\C\) nad \(\R\)?
            \item Ali je \(\sqrt{2} + \sqrt{3}\) algebraičen v \(\R/\Q\)?
        \end{itemize}
    \end{itemize}

    \item Končne razširitve polj 
    
    Naj bo \(K/F\) razširitev polj.
    \begin{itemize}
        \item Ali lahko gledamo \(K\) kot vektorski prostor nad \(F\)?
        \item \textbf{Definicija.} Kdaj rečemo, da je razširitev končna? Stopnja razširitve.
        \item \textbf{Zgled.} Ali je \(\C/\R\) končna? Ali je \(\R/\Q\) končna?
        \item \textbf{Trditev.} Zveza stopenj razširitev \(F \subseteq L \subseteq K\).
        \item \textbf{Definicija.} Algebraična razširitev. Transcendentna razširitev.
        \item \textbf{Trditev.} Kaj lahko povemo o končni razširitvi?
        \item Naj bo \(K/F\) razširitev polj, \(a \in K\). Podkolobar v \(K\) generiran z \(F\) in \(a\). Podpolje v \(K\), generirano z \(F\) in \(a\). Podobno za \(n\) elementov/
        \item \textbf{Definicija.} Primitivna razširitev.
        \item \textbf{Izrek.} Naj bo \(a \in K\) algebraičen. Zveza med \(K[x]\) in \(K(x)\). Stopnja primitivne razširitve.
        \item \textbf{Posledica.} Verzija izreka za \(n\) elementov.
        \item \textbf{Zgled.}
        \begin{itemize}
            \item Določi stopnjo \(\Q(\sqrt[n]{p})/\Q\), kjer je \(p\) praštevilo. 
            \item Naj bo \(a \in K\). Oglejmo si \(\eval: F[X] \to F[a]\). Kaj če je \(a\) transcendenten? Kaj če je \(a\) algebraičen?
        \end{itemize}
        \item \textbf{Izrek.} Podpolje algebraičnih elementov razširitve.
        \item \textbf{Zgled.} Algebraična razširitev \(\Q\), ki ni končna.
    \end{itemize}

    \item Konstrukcije z ravnilom in šestilom
    
    \todo{Po izpitu}

    \item Razpadna polja polinomov
    \begin{itemize}
        \item \textbf{Trditev.} Naj bo \(f(X) \in F[X]\) nekonstanten polinom. Ali lahko najdemo vsaj eno ničlo?
        \item \textbf{Posledica.} Naj bo \(f(X) \in F[X]\) nekonstanten polinom. Ali lahko najdemo vse ničle?
        \item \textbf{Definicija.} Kdaj pravimo, da polinom razpade nad poljem? Razpadno polje polinoma.
        \item \textbf{Opomba.} Ali razpadno polje vedno obstaja? Ali je razpadno polje polinoma končna razširitev?
    \end{itemize}

    Koliko razpadnih polj ima vsak polinom?
    \begin{itemize}
        \item 
    \end{itemize}

\end{enumerate}