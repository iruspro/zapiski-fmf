\section{Kolobarji polinomov}

Gledamo polinome nad poljem \(F\), torej kolobar \(F[X]\).
\begin{enumerate}
    \item Kolobarji polinomov
    \begin{itemize}
        \item Zapis polinoma stopnje \(n\). Stopnja ničelnega polinoma.
        \item Čemu je enaka stopnja produkta polinomov?
        \item \textbf{Lema.} Ali ima kolobar \(F[X]\) delitelji niča? Kaj so njegove obrnljive elemente?
        \item \textbf{Izrek.} Osnovni izrek o deljenju polinomov.
        \item \textbf{Posledica.} Kaj lahko povemo o vsakem idealu v kolobarju \(F[X]\)?
        \item Ničla polinoma.
        \item \textbf{Opomba.} Ali lahko v splošnem identificiramo polinomi s polinomskimi funkciji?
        \item \textbf{Trditev.} Karakterizacija ničle polinoma.
        \item \textbf{Posledica.} Koliko ničel lahko ima neničeln polinom?
    \end{itemize}

    \item Nerazcepni polinomi
    
    Naj bo \(F\) polje.
    \begin{itemize}
        \item \textbf{Definicija.} Nerazcepni polinom nad \(F\).
        \item \textbf{Zgled.} Kaj so nerazcepni polinomi v \(\C[X]\)? Kaj so v \(\R[X]\)?
        \item \textbf{Trditev.} Naj bo \(p(X) \in F[X]\) stopnje vsaj \(1\).
        \begin{itemize}
            \item Kaj če stopnja \(p(X)\) enaka 1?
            \item Kaj če stopnja vsaj 2 ter \(p(X)\) nerazcepen? (ničle)
            \item Kdaj je polinom stopnje 2 ali 3 nerazcepen?
        \end{itemize}
    \end{itemize}

    Od tod dalje gledamo kolobar \(\Q[X]\). Če polinom \(p(X) \in \Q[X]\) pomnožimo s skupnim imenovalcem koeficientov dobimo polinom v \(\Z[X]\).
    \begin{itemize}
        \item \textbf{Definicija.} Primitiven polinom.
        \item \textbf{Trditev.} Gaussova lema.
        \item \textbf{Izrek.} Zadosten pogoj, da je polinom \(p(X) \in \Q[X]\) nerazcepen. (kolobar \(\Z[X]\))
        \item \textbf{Izrek.} Eisensteinov kriterij.
    \end{itemize}
\end{enumerate}