\section{Cela števila}

\begin{enumerate}
    \item Osnovni izrek o deljenju celih števil
    \begin{itemize}
        \item Načelo dobre urejenosti v $\N$.
        \item Načeli dobre urejenosti v $\Z$.
        \item \textbf{Izrek.} Osnovni izrek o deljenju celih števil. Ostanek.
    \end{itemize}

    \item Največji skupni delitelj
    \begin{itemize}
        \item \textbf{Definicija.} Kadar pravimo, da celo število $k \neq 0$ deli celo število $m$? Zapis.
        \item \textbf{Definicija.} Delitelj. Število $m$ deljivo s številom $k$.
        \item \textbf{Definicija.} Skupni delitelj. Največji skupni delitelj.
        \item \textbf{Izrek.} Obstoj največjega skupnega delitelja. Kako lahko ga zapišemo?
        \item \textbf{Definicija.} Tuji števili.
        \item \textbf{Posledica.} Kadar sta števili $m$ in $n$ tuji?
    \end{itemize}

    \item Osnovni izrek aritmetike
    \begin{itemize}
        \item \textbf{Definicija.} Praštevila.
        \item \textbf{Lema.} Evklidova lema.
        \item \textbf{Izrek.} Osnovni izrek aritmetike.
        \item \textbf{Izrek.} Ali je praštevil neskončno?
    \end{itemize}
\end{enumerate}