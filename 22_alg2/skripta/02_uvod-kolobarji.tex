\section{Uvod v teorijo kolobarjev}

\begin{enumerate}
    \item Uvod v teorijo kolobarjev
    \begin{itemize}
        \item \textbf{Definicija.} Kolobar. Enica kolobarja. Komutativen kolobar.
        \item \textbf{Zgled.} Številski kolobarji. Kolobar matrik. Kolobar \(\R^X\), kjer \(X \subseteq \R\).
        \item \textbf{Definicija.} Levi/desni delitelj niča. Delitelj niča. Idempotent. Nilpotent.
        \item \textbf{Opomba.} Kako so idempotenti in nilpotenti povezani z delitelji niča?
        \item \textbf{Opomba.} Ali v kolobarjih brez delitelja niča velja pravilo krajšanja?
        \item \textbf{Zgled.} Delitelji niča v $\R^{2 \times 2}$. Idempotenti v poljubnem kolobarju. Nilpotenti v \(\R^{n \times n}\).
        \item \textbf{Definicija.} Cel kolobar.
        \item \textbf{Zgled.} Ali je \((\Z, +, \cdot)\) cel kolobar?
        \item \textbf{Definicija.} Obseg. Polje.
        \item \textbf{Zgled.} Številski polja.
        \item \textbf{Trditev.} Ali lahko obrnljiv element kolobarja delitelj niča?
        \item \textbf{Posledica.} Ali so lahko v obsegu delitelji niča? 
        \item \textbf{Definicija.} Algebra nad poljem \(\F\).
    \end{itemize}

    \item Primeri kolobarjev in algeber
    \begin{itemize}
        \item Kolobar (algebra) kvadratnih matrik. Algebra endomorfizmov.
        \item Algebra realnih funkcij.
        \item Polinomi:
        \begin{itemize}
            \item \textbf{Definicija.} Polinom s koeficienti iz kolobarja \(K\).
            \item Seštevanje in množenje v \(K[X]\).
            \item Polinomi več spremenljivk. Kolobar formalnih potenčnih vrst.
            \item \textbf{Trditev.} Ali je \(K[X]\) komutativen, če je \(K\) komutativen? Ali je isto velja, če je \(K\) brez deliteljev niča ali \(K\) cel?
        \end{itemize}
        \item Polje ulomkov celega kolobarja \(K\):
        \begin{itemize}
            \item Ekvivalenčna relacija na \(P = K \times (K \setminus \set{0})\).
            \item Množenje in seštevanje na \(P/_\sim\).
            \item \textbf{Trditev.} Ali je \((P/_\sim, +, \cdot)\) polje?
            \item \textbf{Zgled.} Polje ulomkov kolobarja \(\Z\).
            \item Kako lahko \(K\) vložimo v \(P/_\sim\)?
        \end{itemize}
        \item \textbf{Trditev.} Potreben pogoj, da je algebra nad \(\R\) obseg.
        \item Algebra kvaternionov:
        \begin{itemize}
            \item Baza prostora kvaternionov.
            \item Definicija množenja v \(\HH\).
            \item \textbf{Definicija.} Kvaternioni. Konjugiran kvaternion.
            \item \textbf{Trditev.} Ali je \(\HH\) obseg? Ali je algebra?
            \item \textbf{Definicija.} Kvaternionska algebra \(\HH\). Kvaternionska grupa \(Q\).
        \end{itemize}
        \item \textbf{Zgled.} Ali je direktni produkt polj lahko polje?
    \end{itemize}

    \item Podkolobarji, podalgebre, podpolja
    \begin{itemize}
        \item \textbf{Definicija.} Podkolobar. Podalgebra. Podpolje.
        \item \textbf{Zgled.} Zakaj moramo zahtevati, da podkolobar vsebuje enico?
        \item \textbf{Definicija.} Razšeritev polja.
        \item \textbf{Trditev.} Karakterizacija podkolobarja.
        \item \textbf{Trditev.} Karakterizacija podalgebre.
        \item \textbf{Trditev.} Karakterizacija podpolja.
        \item \textbf{Zgled.} Številski primeri podkolobarjev. Odnos med celi kolobarji in njihovim poljem ulomkov.
        \item \textbf{Zgled.} Podkolbar Gaussovih celih števil \(\Z[i]\).
        \item \textbf{Zgled.} Podalgebra zgornje trikotnih matrik v \(\R^{n \times n}\). Podalgebra zveznih funkcij v \(\R^X\), kjer \(X \subseteq \R\).
        \item \textbf{Zgled.} Center kolobarja.
        \item \textbf{Zgled.} Podalgebra konvergentnih zaporedij.
    \end{itemize}

    \item Kolobar ostankov in karakteristika kolobarja
    \begin{itemize}
        \item Definicija množenja v \(\Z_n\). Ali je dobra?
        \item \textbf{Trditev.} Ali je $(\Z_n, +, \cdot)$ komutativen kolobar?
        \item \textbf{Definicija.} Karakteristika kolobarja.
        \item \textbf{Zgled.} Določi \(\ch \Z\) ter \(\ch \Z_n\).
        \item \textbf{Trditev.} Naj bo \(K\) kolobar s karakteristiko \(n > 0\).
        \begin{itemize}
            \item Čemu je enako \(n \cdot x\) za vsak \(x \in K\)?
            \item Kdaj je \(m \cdot 1 = 0\)?
            \item Kaj če je \(K\) neničeln kolobar in nima deliteljev niča?
        \end{itemize}
        \item \textbf{Lema.} Ali je končen cel kolobar vedno polje?
        \item \textbf{Opomba.} Ali lema še vedno drži brez predpostavki o komutativnosti? Ali so vsi končni obsegi komutativni?
        \item \textbf{Trditev.} Kdaj je \(\Z_n\) polje?
        \item \textbf{Zgled.} Karakteristika kolobarja matrik \(M_k(\Z_n)\), kolobarja polinomov \(\Z_n[X]\), polja racionalnih funkcij \(\Z_p(X)\).
        \item \textbf{Izrek.} Mali Fermatov izrek. \todo{*}
    \end{itemize}

    \item Generatorji kolobarjev, algeber, polj
    \begin{itemize}
        \item \textbf{Definicija.} Podkolobar (podalgebra, podpolje) generiran z množico \(X\).
        \item \textbf{Trditev.} Kako zgledajo elementi v podkolobarju (podalgebre, podpolju), ki je generiran z množico \(X\)?
        \item \textbf{Zgled.} \
        \begin{itemize}
            \item Kaj je podkolobar kolobarja $\C$, generiran z $1$?
            \item Kaj je podpolje kolobarja $\C$, generirano z $1$?
            \item Kaj je podkolobar kolobarja $\C$, generiran z $i$?
            \item Kaj je podpolje kolobarja $\C$, generirano z $i$?
            \item Kaj je podkolobar kolobarja $\R[X]$, generiran z $X$?
            \item S čim je generirana realna algebra $\R[X]$?
            \item S čim je generirana algebra $M_2(\R)$? Čemu je enaka $\dim M_2(\R)$.
            \item Kaj je podkolobar kolobarja $M_2(\R)$, generiran z $E_{12}$ in $E_{21}$?
        \end{itemize}
    \end{itemize}
\end{enumerate}

\newpage
\subsection*{Rezultati z vaj}
\begin{enumerate}
    \item Kolobarji, obsegi, polja
    \begin{itemize}
        \item (naloga 4.3) Kako iz kolobarja brez enote lahko naredimo kolobar z enoto?
        \item (nalogi 4.10-4.11) \emph{Boolov kolobar.} Primer Boolova kolobarja.
    \end{itemize}

    \item Algebre
    \begin{itemize}
        \item (naloga 4.27) Ali je $\Z$ lahko algebra nad kakim poljem?
        \item (naloga 4.30) Naj bo $A$ končnorazsežna algebra.    
        \begin{itemize}
            \item Kaj velja za vsak $a \in A \setminus \set{0}$?
            \item Kaj če ima $a \in A$ levi ali desni inverz?
            \item Recimo, da je $A$ tudi obseg. Kaj lahko povemo o vsaki podalgebri?
        \end{itemize}
        \item Algebra kvaternionov.
        \begin{itemize}
            \item (naloga 4.52) Čemu je enak $Z(\HH)$? Čemu je enak $Z(Q)$?
            \item (naloga 4.56) Kaj lahko povemo o enačbi $h^2 + \alpha h + \beta = 0$ za vsak $h \in \HH$?
        \end{itemize}

        \item Kolobar $\Z_n$.
        \begin{itemize}
            \item Kadar je $k \in Z_n$ obrnljiv?
            \item Koliko je obrnljivih elementov v $\Z$? Koliko v $\Z_n$? Kaj če je $n$ praštevilo?
        \end{itemize}
    \end{itemize}
\end{enumerate}