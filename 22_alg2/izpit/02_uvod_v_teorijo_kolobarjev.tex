\section{Uvod v teorijo kolobarjev}
\begin{itemize}
    \item Kolobar \(K\) je Boolov, če \(\all{x \in K}x^2 = x\). Boolov kolobar je komutativen in ima karakteristiko 2. 
    \item Kolobar \(\Z\) ni algebra nad nobenim poljem.
    \item Naj bo \(A\) končno-razsežna algebra. Tedaj 
    \begin{itemize}
        \item \(\all{a \in A \setminus \set{0}} (\some{b \in A \setminus \set{0}} a b = 0 \lor ba = 0) \lor (\some{a^{-1}} a^{-1}a=aa^{-1} = 1)\).
        \item \(\all{a \in A} \some{b \in A} ab = 1 \lor ba = 1 \lthen a^{-1} = b\).
        \item Če je \(A\) obseg, je vsaka podalgebra podobseg.
    \end{itemize}
\end{itemize}

\subsection{Algebra kvaternionov}
\begin{itemize}
    \item \(i^2=j^2=k^2=ijk=-1\)
    \item \(Z(\HH) = \R, \ Z(Q) = \set{-1, 1}\).
    \item \(\all{h \in \HH} \some{\alpha, \beta \in \R} h^2 + \alpha h + \beta = 0\), kjer \(-\alpha = h + \overline{h}\) in \(\beta = h \overline{h}\).
\end{itemize}

\subsection{Kolobar \(Z_n\)}
\begin{itemize}
    \item Kolobar \(Z\) ima 2 obrnljivih elementa: \(1\) in \(-1\)
    \item V \(\Z_n\) element \(k \in \Z_n\) je obrnljiv natanko tedaj, ko \(\gcd (k, n) = 1\).
    \item \(|\Z_n^*| = \phi(n)\), kjer je \(\phi\) Eulerjeva funkcija. Če he \(p\) praštevilo, potem \(|\Z_p| = p - 1\).
\end{itemize}

\subsection{Generatorji}
\begin{itemize}
    \item Poglejmo kaj mora vsebovati kolobar (vedno vsebuje enoto), ki je generiran z neko množico \(A\), ter pokažemo, da je dobljena množica podkolobar.
\end{itemize}