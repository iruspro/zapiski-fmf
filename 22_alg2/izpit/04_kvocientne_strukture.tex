\section{Kvocientne strukture}
\subsection{Kvocientne grupe}
\begin{itemize}
    \item \(\gen{r}\) je edinka v \(D_{2n}\) za \(n \geq 3\).
    \item Če je \(G/_{Z(G)}\) ciklična, potem je \(G\) Abelova.
\end{itemize}

\subsection{1.\ izrek o izomorfizmu}
\begin{itemize}
    \item To, da je podgrupa \(N \triangleleft G\) edinka v \(G\) lahko dokažemo tako, da najdemo ustrezni homomorfizem \(\varphi\), za kateri \(\ker \varphi = N\).
\end{itemize}

\subsection{Kvocientni kolobarji}
\begin{itemize}
    \item Za vsak kolobar \(K\) velja, da \(\all{a \in K} aK = \setb{ak}{k \in K} = Ka\) je ideal. 
    \item Enostavnost kolobarja \(K\) uporabimo/dokažemo tako, da predpostavimo, da podan ideal ni trivialen, torej mora biti enak \(K\).
    \item Kolobar \(M_n(D)\) je enostaven, če je \(D\) obseg.
    \item Center enostavnega kolobarja je polje. Komutativen kolobar je enostaven natanko tedaj, ko je polje.
    \item Naj bosta \(K_1\) in \(K_2\) kolobarja. Tedaj vsak ideal direktnega produkta \(K_1 \times K_2\) je oblike \(I_1 \times I_2\), kjer je \(I_1\) ideal v \(K_1\) ter \(I_2\) ideal v \(K_2\).
    \item Z \((g(X))\) označujemo glavni ideal kolobarja polinomov \(F[X]\), generiran s polinomom \(g(X) \in F[X]\), torej 
    \[
        (g(X)) = \setb{g(x)f(x)}{f(x) \in F[X]}.
    \]
\end{itemize}
