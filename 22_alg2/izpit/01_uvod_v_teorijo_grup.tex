\section{Uvod v teorijo grup}
\subsection{Grupa permutacij}
\begin{itemize}
    \item Zapis s transpoziciji: \((i_1 i_2 \ldots i_n) = (i_1 i_n)(i_1 I_{n-1}) \ldots (i_1 i_3)(i_1 i_2)\)
    \item Inverz \(k\)-cikla: \((i_1 i_2 \ldots i_k)^{-1}  = (i_k i_{k-1} \ldots i_2 i_1)\)
    \item Konjugiranje: \(\pi \in S_n \lthen \pi (i_1 i_2 \ldots i_k) \pi^{-1} = (\pi(i_1) \pi(i_2) \ldots \pi(i_k))\)
    \item Generatorji:
    \begin{itemize}
        \item \(S_n = \gen{(1 2), (1 3), (1n)} = \gen{(12)(23)\ldots(n-1,n)}= \gen{(12), (12\ldots n)}\)
    \end{itemize}
\end{itemize}

\subsection{Diedrska grupa \(D_{2n}\)}
\begin{itemize}
    \item \(z^kr = r^{-k}z = r^{n - k}z\)
    \item \(r^kz\) so zrcaljenja, \((r^kz)^2 = 1\)
\end{itemize}

\subsection{Podgrupe}
\begin{itemize}
    \item \(H, K \leq G \lthen |HK| = \frac{|H||K|}{|H \cap K|}\).
\end{itemize}

\subsection{Ciklične grupe}
\begin{itemize}
    \item Vsaka podgrupa ciklične grupe je ciklična
    \item Podgrupe v \(\Z\) so oblike \(n\Z, n \in \N\)
    \item Podgrupe v \(\Z_n\) so \(\Z_d\), kjer \(d\, |\, n\)
    \item \(G = \gen{a}, |G| < \infty \lthen G = \gen{a^k} \liff \gcd (k, n) = 1\)
    \item \(k \in Z_n \lthen \red k = \frac{n}{\gcd(n, k)}\)
    \item Konjugiranje ohranja red elementa
\end{itemize}

\subsection{Generatorji grup}
\begin{itemize}
    \item Oglejmo vsi možni produkti in poiščemo izomorfizem.
\end{itemize}

\subsection{Splošno}
\begin{itemize}
    \item \(f: X \to X\) preslikava. Velja:
    \begin{itemize}
        \item \(f\) ima levi inverz: \(g \circ f = \id\) natanko tedaj, ko je \(f\) injektivna. Če \(f\) tudi ni surjektivna, potem ima več levih inverzov.
        \item \(f\) ima desni inverz: \(f \circ h = \id\) natanko tedaj, ko je \(f\) surjektivna. Če \(f\) tudi ni surjektivna, potem ima več desnih inverzov.
    \end{itemize}
\end{itemize}

\section{Uvod v teorijo kolobarjev}
\begin{itemize}
    \item Kolobar \(K\) je Boolov, če \(\all{x \in K}x^2 = x\). Boolov kolobar je komutativen in ima karakteristiko 2. 
\end{itemize}

\subsection{Algebra kvaternionov}
\begin{itemize}
    \item \(i^2=j^2=k^2=ijk=-1\)
\end{itemize}



\section{homomorfizem}
\begin{itemize}
    \item Matrike so obrnljive? Morda to je \(\HH\)?
\end{itemize}

\section{Splošno}
\subsection{Matrike}
\begin{itemize}
    \item Naj bo \(A \in M_n(\R), \ \rang A = 1\). Tedaj \(\some{\lambda \in \R} A^2 = \lambda A\). Tako matriko lahko zapišemo tudi v obliki: stolpec krat vrstica.
\end{itemize}
