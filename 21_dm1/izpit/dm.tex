\section{Kombinatorika}
\subsection{Formalne potenčne vrste (Rodovne funkcije)}
\begin{definicija}
    Naj bo $a: \N \to \C$ zaporedje. Vrsta $\sum_{n \geq 0}a_n x^n$ je \emph{formalna potenčna vrsta} zaporedja $a_n$.
\end{definicija}

\begin{definicija}
    Naj bo $(a_n)_n, \, (b_n)_n \in \C^{\N_0}$.
    \begin{itemize}
        \item $\sum_{n \geq 0}a_n x^n + \sum_{n \geq 0}b_n x^n = \sum_{n \geq 0}(a_n + b_n) x^n$.
        \item $\lambda \cdot \sum_{n \geq 0}a_n x^n = \sum_{n \geq 0}(\lambda a_n) x^n$.
        \item $(\sum_{n \geq 0}a_n x^n) \cdot (\sum_{n \geq 0}b_n x^n) = \sum_{n \geq 0}c_n x^n$, kjer $c_n = \sum_{k=0}^{n}a_kb_{n-k}$.
    \end{itemize}
\end{definicija}

\begin{definicija}
    F.\ p.\ v.\ $A(x)$ je \emph{obrnljiva}, če obstaja f.\ p.\ v.\ $B(x)$, da velja $A(x) \cdot B(x) = (1, 0, 0, \ldots)$.
\end{definicija}

\begin{trditev}
    Formalna potenčna vrsta $A(x)$ je obrnljiva natanko tedaj, ko $a_0 \neq 0$
\end{trditev}

\begin{definicija}
    Če je $(a_n)_n$ zaporedje, ki je rešitev nekega kombinatoričnega problema, potem $\sum_{n \geq 0}a_n x^n$ pravimo \emph{rodovna funkcija}. Pišemo: $G(x) = \sum_{n \geq 0}a_n x^n$.
\end{definicija}

\begin{definicija}
    Naj bo $A(x) = \sum_{n \geq 0}a_n x^n$ f.\ p.\ v. \emph{Odvod} f.\ p.\ v.\ je $A'(x) =  \sum_{n \geq 1}na_n x^{n-1} =  \sum_{n \geq 0}(n+1)a_n x^n$.
\end{definicija}

\begin{trditev}
    $(A(x) \cdot B(x))' = A'(x)B(x) + A(x)B'(x)$.
\end{trditev}

\subsubsection*{Reševanje rekurzij z rodovnimi funkcijami}
\paragraph{Koraki splošnega reševanja nekega kombinatoričnega problema}
\begin{enumerate}
    \item Rešitev problema zapišemo z rekurzivno enačbo.
    \item Zapišemo rodovno funkcijio zaporedja z pomočjo rekurzivne zveze.
    \item Z algebro nad rodovnimi funkcijami, rodovno funkcijo razvijemo v vrsto.
    \item Iz razvoja preberemo rešitev našega začetnega problema.
\end{enumerate}

\subsection{Catalanova števila}
Imejmo produkt $n$ števil $x_1x_2 \ldots x_n$. Na koliko načinov lahko izračunamo ta produkt, če po vrsti zmnožimo dve zaporedni števili in jih nadomestimo s produktom?

\begin{definicija}
    \emph{Catalanovo število $C_n$} je število načinov, s katerimi lahko izračunamo produkt $x_1x_2 \ldots x_n$.
\end{definicija}

\begin{opomba}
    Velja rekurzivna zveza: $a_n = \sum_{k=0}^{n-1}a_ka_{n-k-1}$.
\end{opomba}

\begin{trditev}
    $C_n = \frac{1}{n+1} \binom{2n}{n}$.
\end{trditev}

\begin{proof}
    $G(x) = \frac{1-\sqrt{1-4x}}{2}$, ker $G(0)= 0$.
\end{proof}

\section{Teorija grafov}
\subsection{Osnovni pojmi}
\begin{definicija}
    \emph{Graf} G je urejen par $(V(G), E(G))$, kjer je $V(G)$ množica \emph{vozlišč} grafa $G$ in $E(G)$ množica \emph{povezav} grafa~$G$, kjer je $E(G) \subseteq \binom{V(G)}{2}$.
\end{definicija}

\begin{definicija}
    Naj bo $G$ graf, $u \in V(G)$:
    \begin{itemize}
        \item $N_G(u) = \setb{v \in V(G)}{uv \in E(G)}$ je \emph{soseščina} vozlišča $u$.
        \item $\deg_G(u) = |N_G(u)|$ je \emph{stopnja} vozlišča $u$. Označimo z $\delta(G) = \min_{v \in V(G)}\deg_G(v)$.
    \end{itemize}
\end{definicija}

\begin{definicija}
    Graf $G$ je \emph{regularen}, če imajo vsa vozlišča isto stopnjo.
\end{definicija}

\begin{primer}
    Grafe lahko predstavimo tudi z matrikami. Možnosti:
    \begin{enumerate}
        \item \emph{Matrika sosednosti.} Za graf $G$ z vozlišči $v_1, \ldots, v_n$ je matrika sosednosti matrika $A(G) \in \R^{n \times n}$, definirana z $$a_{ij} = \begin{cases}
            1, &v_i \sim v_j, \\
            0, &v_i \nsim v_j.
        \end{cases}.$$
        \item \emph{Incidenčna matrika}. Če ima graf $G$  vozlišča $v_1, \ldots, v_n$ in povezave $e_1, \ldots, e_m$, je to matrika $B(G) \in \R^{n \times m}$, podana s predpisom $$b_{ij} = \begin{cases}
            1, &v_i \in e_j, \\
            0, &v_i \notin e_j.
        \end{cases}.$$
    \end{enumerate}
\end{primer}

\subsection{Lema o rokovanju}
\begin{lema}[Lema o rokovanju]
    Naj bo $G$ graf. Velja: $$\sum_{u \in V(G)} \deg_G(u) = 2 \cdot |E(G)|.$$
\end{lema}

\begin{proof}
    Incidenčna matrika in načelo dvojnega preštevanja.
\end{proof}

\begin{posledica}
    Število vozlišč lihe stopnje danega grafa je sodo.
\end{posledica}

\subsection{Podgrafi}
\begin{definicija}
    Naj bosta $G$ in $H$ grafa. Če velja $V(H) \subseteq V(G)$ in $E(H) \subseteq E(G)$, tedaj rečemo, da je $H$ \emph{podgraf} grafa~$G$, in pišemo $H \leq G$. Pri tem:
    \begin{itemize}
        \item Pograf $H$ je \emph{vpet} podgraf, če je $V(H) = V(G)$ (odstranimo nekaj povezav).
        \item Podgraf $H$ je \emph{porojen} (oz. \emph{induciran}), če velja:  $\all{u, v \in V(H)} u \sim_G v \lthen u \sim_H v$ (odstranimo nekaj vozlišč).
    \end{itemize}
\end{definicija}

\subsection{Nekatere družine grafov}
\paragraph{Poti.} Grafu z množico vozlišč $\Z_n$ in množico povezav $E(P_n) = \setb{i(i+1)}{i \in \set{0,1, \ldots, n-2}}$ rečemo \emph{pot dolžine~$n$}. Oznaka: $P_n$.

\paragraph{Cikli.} Za $n \in \N, n \geq 3$, s $C_n$ označimo graf z množico vozlišč $\Z_n$ in množico povezav $E(C_n) = \setb{i(i+1)}{i \in \Z_n}$. Grafu $C_n$ rečemo \emph{cikel dolžine $n$}.

\paragraph{Polni dvodelni grafi.} Naj bosta $A$ in $B$ poljubni disjunktni neprazni množici. Grafu $K_{A, B}$ z množico vozlišč $A \cup B$ in množico povezav $E(K_{A,B}) = \setb{uv}{u \in A, v \in B}$ rečemo \emph{polni dvodelni graf} na paru množic $A, B$. Oznaka: $K_{A, B}$ ali $K_{m, n}$, če $A = [m], \ B = [n]$.

\paragraph{Hiperkocke.} Za $d \in \N$ naj $\Z_2^d$ predstavlja množico vseh urejenih $d$-teric ničel in enic. Na $\Z_2^d$ lahko gledamo kot na Abelovo grupo. Za $i \in {1, 2, \ldots, d}$ naj bo $e_i$ tisti element množice $\Z_2^d$, ki ima na $i$-tem mesu enico, povsod grugod pa ničlo. Pišimo $S = \set{e_1, e_2, \ldots, e_d}$. \emph{Hiperkocka razsežnosti $d$} je graf z množico vozlišč $\Z_2^d$ in z dvema vozliščema $u, v \in \Z_2^d$ sosednjima, če in samo če je $u+ v \in S$, tj. $u$ in $v$ razlikujeta v natanko eni komponenti. Oznaka: $Q_n$. Velja:
\begin{itemize}
    \item $|V(Q_n)| = 2^d$.
    \item $|E(Q_n)| = d \cdot 2 ^{d-1}$.
\end{itemize}

\paragraph{Posplošeni Petersenovi grafi.} Naj bosta $n, k \in \N, \ n \geq 3, \ 2k < n$. Naj bosta $U = \set{u_0, u_1, \ldots, u_{n-1}}$ in $V~=~\set{v_0, v_1, \ldots, v_{n-1}}$ dve disjunktni $n$-elementni množici. \emph{Posplošeni Petersonov graf $P_{n,k}$} je graf z množico vozlišč $V(P_{n,k}) = U \cup V$ in množico povezav enako $E(P_{n,k}) = E_1 \cup E_2 \cup E_3$, kjer je 
$$E_1 = \setb{u_iu_{i+1}}{i \in \Z_n}, \quad E_2 = \setb{v_iv_{i+k}}{i \in Z_n}, \quad E_3 = \setb{u_iv_i}{i \in \Z_n}.$$

\subsection{Sprehodi, poti in cikli}
\begin{definicija}
    \emph{Sprehod} v grafu $G$ je zaporedje vozlišč grafa $G$: $v_0v_1\ldots v_k$, tako da je $v_iv_{i+1} \in E(G)$ za $i \in [k-1]$. Rečemo, da je to $v_0v_k$-sprehod. Sprehod je \emph{enostaven}, če so vse njegove povezave paroma različne.
\end{definicija}

\begin{definicija}
    \emph{Pot} v grafu $G$ je sprehod s samimi različnimi vozlišči.
\end{definicija}

\begin{definicija}
    Če ima sprehod v grafu $G$ isti začetek in konec pravimo, da je sprehod \emph{sklenjen}. \emph{Enostaven sklenjen sprehod} je sklenjen sprehod, ki je enostaven. \emph{Cikel} v grafu $G$ je sklenjen sprehod z samimi različnimi vozlišči.
\end{definicija}

\begin{lema}
    Če v grafu $G$ obstaja $uv$-sprehod, obstaja tudi $uv$-pot.
\end{lema}

\begin{proof}
    Cilj: paroma različna vozlišča.
\end{proof}

\begin{lema}
    Če v grafu $G$ obstajata dve različi $uv$-poti, potem ima $G$ vsaj en cikel.
\end{lema}

\begin{proof}
    Skupna vozlišča.
\end{proof}

\begin{lema}
    Če v grafu $G$ obstaja sklenjen sprehod lihe dolžine, obstaja v $G$ tudi cikel lihe dolžine.
\end{lema}

\begin{proof}
    Indukcija na dolžino sprehoda.
\end{proof}

\subsection{Povezane komponente, razdalja in premer}
\begin{definicija}
    Graf $G$ je \emph{povezan}, če za vsak par vozlišč $u, v \in V(G)$ obstaja $uv$-sprehod (ekvivalentno: $uv$-pot).
\end{definicija}

\begin{definicija}
    Maksimalni podgrafi grafa, ki so povezani, so \emph{komponente} grafa. 
\end{definicija}
Označimo z $\Omega(G)$ število komponent grafa $G$.

\begin{definicija}
    Naj bo $G$ povezan graf, $u, v \in V(G)$. \emph{Razdalja $d_G(u,v)$} med $u$ in $v$ je število povezav na najkrajši $uv$-poti.
\end{definicija}

\begin{opomba}
    Naj bo $G$ povezan graf. Velja:
    \begin{itemize}
        \item $d_G(u,v) = 0 \liff u = v$.
        \item $d_G(u,v) = 1 \liff uv \in E(G)$.
    \end{itemize}
\end{opomba}

\begin{trditev}
    Naj bo $G$ povezan graf. Potem $(V(G), d_G)$ je metrični prostor.
\end{trditev}

\begin{definicija}
    Naj bo $G$ povezan graf. \emph{Ekscentričnost $\ecc_G(u)$} vozlišča $u$ je $\max_{v \in V(G)} d_g(u,v)$.
\end{definicija}

\begin{definicija}
    Naj bo $G$ povezan graf. \emph{Premer $\diam G$} grafa $G$ je $\max_{u \in V(G)} \ecc_G(u) = \max_{u,v \in V(G)} d_G(u,v)$.
\end{definicija}

\begin{definicija}
    Naj bo $G$ povezan graf. \emph{Polmer $\rad(G)$} grafa $G$ je $\min_{u \in V(G)} \ecc_G(u)$
\end{definicija}

\begin{primer}
    Velja:
    \begin{itemize}
        \item $\diam(P_{5,2}) = \rad(P_{5,2}) = 2$.
        \item $\diam(P_n) = n-1, \ \rad(P_n) = \lfloor \frac{n}{2} \rfloor$.
    \end{itemize} 
\end{primer}

\subsection{Dvodelnost}
\begin{definicija}
    Graf $G$ je \emph{dvodelen}, če obstaja razdelitev $V(G) = V_1 \cup V_2$, tako da velja: $$uv \in E(G) \lthen u \in V_1 \land v \in V_2.$$
\end{definicija}

\begin{primer}
    Dvodelnost grafa lahko raziščemo z barvanjo vozlišč z dvema barvama.
\end{primer}

\begin{izrek}
    Graf $G$ je dvodelen natanko tedaj, ko ne vsebuje lihih ciklov.
\end{izrek}

\begin{proof}
    $(\lthen)$ Imamo težavo z barvanjo cikla.
    
    $(\Leftarrow)$ Izberimo $x \in V(G)$ in razdelimo $V(G)$ na vozlišča, ki so na sode razdalje od $x$ in, ki so na lihe razdalje od $x$. Pokažemo, da znotraj posameznih množic ni povezav.
\end{proof}

\subsection{Morfizmi grafov}
\subsubsection*{Homomorfizmi grafov}
\begin{definicija}
    Naj bosta $G, H$ grafa. Preslikava $f: V(G) \to V(H)$ je \emph{homomorfizem}, če velja: $$uv \in E(G) \lthen f(u)f(v) \in E(H).$$
    Injektivni homomorfizem je \emph{vložitev} grafa $G$ v graf $H$.
\end{definicija}

\begin{opomba}
    Če je $G$ dvodelen, potem obstaja homomorfizem $f: G \to K_2$.
\end{opomba}

\begin{definicija}
    Vložitev $f: V(G) \to V(H)$ grafa $G$ v graf $H$ je \emph{izometrična}, če velja:
    $$\all{u, v \in V(G)}d_G(u,v)=d_H(f(u), f(v)).$$
\end{definicija}

\begin{primer}
    Preslikava $\varphi: V(C_4) \to V(Q_3)$ definarana s 
    \begin{center}
        \begin{tabular}{ c | c c c c }
         $x$ & 0 & 1 & 2 & 3 \\ \hline
         $\varphi(x)$ & 000 & 100 & 110 & 010  
        \end{tabular}
    \end{center}
    je homomorfizem.
\end{primer}

\subsubsection*{Izomorfizmi grafov}
\begin{definicija}
    Naj bosta $G, H$ grafa. Preslikava $f: V(G) \to V(H)$ je \emph{izomorfizem}, če
    \begin{enumerate}
        \item $f$ je bijekcija;
        \item $f$ je homomorfizem;
        \item $f^{-1}$ je homomorfizem.
    \end{enumerate}
\end{definicija}

\begin{opomba}
    Pogoj 2. pravi, da $f$ slika povezave v povezave. Pogoj 3. pravi, da $f$ slika nepovezave v nepovezave.
\end{opomba}

\begin{definicija}
    Grafa $G$ in $H$ sta \emph{izomorfna}, če obstaja izomorfizem med njima. Oznaka: $G \cong H$.
\end{definicija}

\begin{opomba}
    Izomorfnost je algoritmično zahtevna naloga. Če za dva grafa domnevamo, da sta izomorfna, potem to dokažemo tako, da med njima poiščemo izomorfizem. Če za dva grafa domnevamo, da nista izomorfna, tedaj to najlažje dokažemo tako, da poiščemo neko številsko karakteristiko grafov, v kateri se razlikujeta, pri izomorfizmu pa se ohranja: število vozlišč, število povezav, ožina, število podgrafov izbrane vrste (trikotniki, cikli dolžine 4), urejeno zaporedje snopenj vozlišč. Lahko si pomagamo z lastnosti grafov: povezanost, dvodelnost.
\end{opomba}

\subsubsection*{Grupa avtomorfizmov grafa}
\begin{definicija}
    \emph{Avtomorfizem} grafa $G$ je izomorfizem $f: G \to G$.
\end{definicija}

Definiramo $\Aut (G) = \setb{\varphi}{\varphi \text{ je avtomorfizem grafa } G}$. Dodamo še komponiranje in dobimo \emph{grupo avtomorfizmov grafa $G$}.

\begin{opomba}
    Velja:
    \begin{itemize}
        \item Vsaka končna grupa je grupa avtomorfizmov nekega grafa.
        \item $\Aut(K_n) \cong S_n$, $\Aut(P_n) \cong \Z_2$, $\Aut(C_n) \cong D_{2n}$.
        \item Skoraj vsi grafi imajo trivialno grupo $\Aut(G)$.
    \end{itemize}
\end{opomba}

\subsection{Operacije z grafi}
\subsubsection*{Komplementarni graf}
\begin{definicija}
    \emph{Komplement $\overline{G}$} grafa $G$ je graf z isto množico vozlišč kot $G$ in z dvema vozliščema sosednjima v $\overline{G}$, če in samo če nista sosednji v grafu $G$.
\end{definicija}

\begin{trditev}
    Če graf $G$ ni povezan, potem je $\diam(\overline{G}) \leq 2$.
\end{trditev}

\begin{posledica}
    Za poljuben graf $G$ je vsaj eden od grafov $G$ in $\overline{G}$ povezan.
\end{posledica}

\subsubsection*{Odstranevanje vozlišč in povezav}
Naj bo $G$ graf.
\begin{itemize}
    \item Naj bo $v \in V(G)$. $G - v$ je graf brez vozlišča $v$ in povezav z $v$.
    \item Naj bo $e \in E(G)$. $G - e$ graf brez povezave $e$.
    \item Naj bo $X \subseteq V(G)$. $G - X$ je graf brez vseh vozlišč iz $X$.
    \item Naj bo $F \subseteq E(G)$. $G-F$ je graf brez vseh povezav iz $F$.
\end{itemize}

\begin{opomba}
    Vsak podgraf grafa $G$ je enak grafu $(G-X)-F$, kjer je $X$ neka množica vozlišč in $F$ neka množica povezav grafa $G$.
\end{opomba}

\subsubsection*{Skrčitev povezav in minorji}
Naj bo $e \in E(G)$. Z $G/e$ označimo graf, ki ga dobimo iz grafa $G$ tako, da identificiramo krajišči povezave $e$ in odstranimo zanko ter morebitne vzporedne povezave. Pri tem rečemo, da smo $G / e$ dobili s \emph{skrčitvijo} povezave $e$.

\begin{definicija}
    Graf $H$ je \emph{minor} grafa $G$, če ga lahko dobimo iz nekega podgrafa grafa $G$ s skrčitvijo nekaj povezav.
\end{definicija}

\begin{primer}
    $K_5$ je minor Petersenovega grafa.
\end{primer}

\begin{opomba}
    Graf $H$ je minor grafa $G$ natanko tedaj, ko $H$ lahko dobimo z zaporedjem operacij: odstranevanje vozlišča, odstranevanje povezave, skčitev povezave v poljubnem vrstem redu.
\end{opomba}

\subsubsection*{Subdivizije povezav in glajenje vozlišč}
\begin{definicija}
    \emph{Subdivizija} povezave $e \in E(G)$ je zamenjava povezave s potjo dolžine $2$. Oznaka: $G^+(e)$.
\end{definicija}
\begin{definicija}
    Graf $H$ je \emph{subdivizija} grafa $G$, če $H$ lahko dobimo iz $G$ z zaporedjem subdivizij povezav.
\end{definicija}
\begin{opomba}
    Graf $H$ je subdivizija grafa $G$, če vsako povezavo v $G$ nadomestimo z neko potjo in vse te poti so paroma disjunktne po povezavih.
\end{opomba}
\begin{definicija}
    Grafa $G$ in $H$ sta \emph{homeomorfna}, če obstaja graf $X$, tako da $G$ in $H$ subdiviziji grafa $X$.
\end{definicija}

\begin{definicija}
    Obratna operacija od subdivizije povezave, je \emph{glajenje} vozlišča $w \in V(G)$ stopnje $2$. Oznaka: $G^-(w)$.
\end{definicija}

\begin{opomba}
    Grafa sta homeomorfna, če z glajenjem pridemo do istega grafa.
\end{opomba}

\subsubsection*{Kartezični produkt}
\begin{definicija}
    \emph{Kartezični produkt} grafov $G_1 = (V_1, E_1)$ in $G_2 = (V_2, E_2)$ je graf $G = G_1 \square G_2$ z množico vozlišč $V(G) = V_1 \times V_2$ in relacijo sosednosi $\sim$, določeno s predpisom
    $$(u_1, v_1) \sim_G (u_2, v_2) \liff (u_1 \sim_{G_1} u_2 \land v_1 = v_2) \lor (u_1 = u_2 \land v_1 \sim_{G_2} v_2).$$
\end{definicija}

\begin{trditev}
    Naj bo $G_1, \ G_2$ in $G_3$ poljubni grafi in naj bo $K_1$ polni graf z enim vozliščem. Tedaj velja naslednje:
    \begin{itemize}
        \item $G_1 \square G_2 \cong G_2 \square G_1$.
        \item $(G_1 \square G_2) \square G_3 \cong G_1 (\square G_2 \square G_3)$.
        \item $G_1 \square K_1 \cong G_1$.
    \end{itemize}
\end{trditev}

Lastnost asociativnosti nam omogoča, da definiramo pojem \emph{kartezične potence grafa}. Za graf $G$ in $n \in \N$ naj bo $$G^{\square n} = \underbrace{G \square G \square \ldots \square G}_\text{$n$ kopij}.$$

\begin{primer}
    $K_2^{\square n} = Q_n$.
\end{primer}

\subsection{Prerezna vozlišča in $k$-povezanost}
\begin{definicija}
    Naj bo $G$ graf.
    \begin{itemize}
        \item Vozlišče $u \in V(G)$ je \emph{prerezno}, če $\Omega(G-u) > \Omega(G)$.
        \item Povezava $f \in E(G)$ je \emph{prerezna} ali \emph{most}, če $\Omega(G-f) > \Omega(G)$.
        \item Množica vozlišč $S \subseteq V(G)$ je \emph{prerez}, če $\Omega(G-S) > \Omega(G)$.
        \item Množica povezav $F \subseteq E(G)$, če \emph{povezavni prerez}, če $\Omega(G-F) > \Omega(G)$.
    \end{itemize}
\end{definicija}

\begin{definicija}
    Graf $G$ je \emph{$k$-povezan}, če ima vsaj $k+1$ vozlišč in nima prerezov moči strogo manj kot $k$. \emph{Povezanost} grafa~$G$ je največji $k$, za katerega je $G$ $k$-povezan. Oznaka: $K(G)$.
\end{definicija}

\begin{primer}
    Povezanost nekaterih grafov:
    \begin{itemize}
        \item $K(K_n) = n-1$.
        \item $K(C_n) = 2, \ n \geq 3$.
        \item $K(Q_n) = n$.
    \end{itemize}
\end{primer}

\begin{opomba}
    Naj bo $G$ graf, potem $K(G) \leq \delta(G)$.
\end{opomba}

\begin{definicija}
    Naj bo $G$ graf, $u,v \in V(G)$. $uv$-poti $P$ in $Q$ sta \emph{notranji-disjunktni}, če je $V(P) \cap V(Q) = \set{u,v}$.
\end{definicija}

\begin{izrek}[Whitney]
    Graf $G$ je $2$-povezan natanko tedaj, ko za vsak par vozlišč $u, v \in V(G), \ u \neq v$ obstajata dve notranji-disjunktni $uv$-poti.
\end{izrek}

\begin{proof}
    $(\Leftarrow)$ Pokažemo, da $G$ nima prereza moči 1.

    $(\lthen)$ Naj bo $u,v \in V(G)$. Obstoj dokažimo z indukcijo po $d_G(u,v)$.
\end{proof}

\subsection{Drevesa}
\begin{definicija}
    \emph{Gozd} je graf brez ciklov. \emph{Drevo} je povezan gozd. \emph{List} je vozlišče stopnje $1$.
\end{definicija}

\begin{lema}
    Naj bo $T$ drevo, $|V(T)| > 2$. Potem $T$ premore list.
\end{lema}

\begin{proof}
    Če vozlišče ni list, potem ima vsaj enga novega soseda.
\end{proof}

\begin{lema}
    Naj bo $T$ drevo, potem $|E(T)| = |V(T)| - 1$.
\end{lema}

\begin{proof}
    Indukcija po $|V(T)|$.
\end{proof}

\begin{lema}
    Naj bo $G$ povezan graf in $e \in E(G)$ leži na nekem ciklu, potem je $G-e$ povezan graf.
\end{lema}

\begin{proof}
    Definicija povezanosti.
\end{proof}

\begin{lema}
    Naj bo $G$ povezan graf, potem je $|E(G)| \geq |V(G)| -1$.
\end{lema}

\begin{opomba}
    Drevesa so povezani grafi z najmanjšim številom povezav.
\end{opomba}

\begin{izrek}
    Za graf $G$ so ekvivalentne trditve:
    \begin{enumerate}
        \item $G$ je drevo.
        \item Za vsak par vozlišč v $G$ obstaja enoličen pot med njima.
        \item $G$ je povezan in vsaka povezava je most.
        \item $G$ je povezan in $|E(G)| = |V(G)|-1$.
    \end{enumerate}
\end{izrek}

\subsection{Vpeta drevesa}
\begin{definicija}
    \emph{Vpeto drevo} grafa je vpet podgraf, ki je drevo.
\end{definicija}

\begin{trditev}
    Graf je povezan natanko tedaj, ko vsebuje vpeto drevo.
\end{trditev}

Označimo z $\tau(G)$ število vpetih dreves grafa $G$.
\begin{primer}
    $\tau(C_n) = n$.
\end{primer}

\begin{opomba}
    Sedaj obravnamo skrčitev povezave $G/e$ tako, da vzporednih povezav ne odstranimo.
\end{opomba}

\begin{trditev}
    Naj bo $e \in E(G)$, potem $$\tau(G) = \tau(G-e) + \tau (G/e).$$
\end{trditev}

\begin{proof}
    Naj bo $T$ vpeto drevo v $G$. Imamo 2 možnosti: $e \notin T$ in $e \in T$.
\end{proof}

\begin{definicija}
    Naj bo $G$ graf. \emph{Laplaceova matrika $L(G)$} grafa $G$ z množico vozlišč $V(G) = \set{v_1, v_2, \ldots, v_n}$ je $n \times n$ matrika, za katero velja: $$L(G)_{ij} = \begin{cases}
        \deg(v_i); &i=j \\ - \text{število povezav med $v_i$ in $v_j$}; &i \neq j
    \end{cases}.$$
\end{definicija}

\begin{izrek}
    Število vpetih dreves grafa $G$ je enako determinante matrike, ki jo dobimo iz $L(G)$ tako, da izbrišemo vrstico in stolpec nekega vozlišča.
\end{izrek}

\begin{izrek}[Cayley]
    $\tau(K_n) = n^{n-2}$
\end{izrek}

\subsection{Eulerjevi grafi}
\begin{definicija}
    Sprehod v grafu je \emph{Eulerjev}, če je enostaven in prehodi vse povezave. Sklenjen Eulerjev sprehod je \emph{Eulerjev obhod}. Graf je \emph{Eulerjev}, če premore Eulerjev obhod.
\end{definicija}

\begin{izrek}
    Graf je Eulerjev natanko tedaj, ko je povezan in so vsa njegova vozlišča sode stopnje.
\end{izrek}

\begin{proof}
    $(\lthen)$ Opazujemo poljuben Eulerjev obhod.

    $(\Leftarrow)$ Indukcija po $|E(G)|$. Baza: Cikli.
\end{proof}

\begin{izrek}
    Povezan graf premore Eulerjev sprehod natanko tedaj, ko ima kvečjemu dve vozlišči lihe stopnje.
\end{izrek}

\paragraph{Fleuryjev algoritem} Kako poiščemo Eulerjev obhod v poljubnem Eulerjevem grafu?

\begin{enumerate}
    \item Začnemo v poljubnem vozlišču.
    \item Za seboj brišemo povezave, ki smo jih prehodili.
    \item Na vsakem koraku izberimo poljubno povezavo, pri tem pa pazimo le na to, da na most gremo le v premeru, če ni druge možnosti.
\end{enumerate}

\subsection{Hamiltonovi grafi}
\begin{definicija}
    \emph{Hamiltonov cikel} grafa je cikel, ki vsebuje vsa vozlišča grafa. \emph{Hamiltonov pot} je pot d grafu, ki vsebuje vsa vozlišča. Graf je \emph{Hamiltonov}, če premore Hamiltonov cikel.
\end{definicija}

\begin{izrek}
    Naj bo $G$ Hamiltonov graf in $X \subseteq V(G)$. Tedaj je $$\Omega(G-X) \leq |X|.$$
\end{izrek}

\begin{proof}
    Narišemo Hamiltonov cikel.
\end{proof}

\begin{primer}
    $K_{n,n}$ je Hamiltonov natanko tedaj, ko $n=m$.
\end{primer}

\begin{izrek}[Ore]
    Če je $G$ povezan graf in za vsak par nesosednih vozlišč $u, v \in V(G)$ velja $$\deg(u)+\deg(v) \geq |V(G)|,$$ potem je $G$ Hamiltonov.
\end{izrek}

\begin{proof}
    Metoda najmanjšega proti primera: med vsemi protiprimeri izberimo takega, ki ima najmanj vozlišč, med vsemi taki, pa izberimo protiprimer, ki ima največ povezav. Poiščemo v tem grafu Hamiltonov cikel.
\end{proof}

\begin{izrek}[Dirac]
    Naj bo $G$ graf. Če $|V(G)| \geq 3$ in $\delta(G) \geq \frac{|V(G)|}{2}$, potem je graf $G$ Hamiltonov.
\end{izrek}

\subsection{Ravninski grafi}
\begin{definicija}
    Graf $G$ je \emph{ravninski}, če ga lahko narišemo v ravnini tako, da se povezave ne križajo. Ravninski graf skupaj z ustrezni rusbi v ravnini, je graf \emph{vložen v ravnino}.
\end{definicija}

\begin{primer}
    $K_{2,3}$ je ravninski, $K_{3,3}$ pa ni ravninski.
\end{primer}

\begin{definicija}
    Naj bo $G$ graf vložen v ravnino. Med seboj ločeni povezani območji imenujemo \emph{lica}. Označimo z $F(G)$ množico vseh lic grafa.
\end{definicija}

\begin{opomba}
    Graf lahko vložimo v ravnino natanko tedaj, ko ga lahko vložimo v sfero (obratno implikacijo naredimo z stereografsko projekcijo). Torej imamo še \emph{zunanje lice}, ki je neomejeno v ravnini in omejeno na sfere.
\end{opomba}

\begin{definicija}
    Vsako lice $f \in F(G)$ je omejeno s sklenjenim sprehodom v grafu $G$, ki mu rečemo \emph{rob} lica $f$. Rob lica $f$ označimo z $\partial f$, njegovo dolžino (tj. število povezav na robu lica) pa z $l(f)$.
\end{definicija}

\begin{opomba}
    Vsaka povezava grafa $G$ tipično leži na robu dveh lic, enega, ki leži na enem bregu povezave, in drugega, ki leži na drugem bregu. Izjemoma pa se lahko primeri, da ti dve lici sovpadata. V tem primeru pri računanju $l(f)$ take povezave štejemo dvakrat.
\end{opomba}

\begin{trditev}[Lema o rokovanju za ravninske grafe]
    Če je $G$ graf vložen v ravnino, potem je $$2 \cdot |E(G)| = \sum_{f \in F(G)} l(f).$$
\end{trditev}

\begin{definicija}
    Naj bo $G$ graf, ki premore cikle. Potem je njegova \emph{ožina} $g(G)$ je dolžina najkrajšega cikla v $G$.
\end{definicija}

\begin{trditev}
    Če je $G$ graf vložen v ravnino in ima vsaj en cikel, potem 
    $$|E(G)| \geq \frac{g(G)}{2}|F(G)|$$
\end{trditev}

\begin{proof}
    Ožina grafa $G$ je omejena z dolžino vsakega lica.
\end{proof}

\begin{izrek}[Eulerjeva formula]
    Če je $G$ vložen v ravnino, potem $$|V(G)| - |E(G)|+|F(G)| = 1 + \Omega (G).$$
\end{izrek}

\begin{proof}
    Naj bo $\Omega(G) = 1$. Dokazujemo za dani $|V(G)|$ z indukcijo po $|E(G)|$.

    Če je $\Omega(G) > 1$ uporabimo formulo na vsake komponente posebej.
\end{proof}

\begin{posledica}
    Naj bo $G$ povezan graf vložen v ravnino. Potem $$|V(G)| - |E(G)|+|F(G)| = 2.$$
\end{posledica}

\begin{trditev}
    Naj bo $G$ povezan graf vložen v ravnino in ima vsaj en cikel. Potem
    $$|E(G)| \leq \frac{g(G)}{g(G)-2}(|V(G)| - 2)$$
\end{trditev}

\begin{proof}
    Uporabimo Eulerjevo formulo in oceno za $|E(G)|$.
\end{proof}

\begin{opomba}
    Velja: $g(G) \geq 3$
\end{opomba}

\begin{posledica}
    Naj bo $G$ povezan graf vložen v ravnino in ima vsaj en cikel. Potem $$|E(G)| \leq 3|V(G)| - 6.$$
\end{posledica}

\begin{posledica}
     Naj bo $G$ povezan graf vložen v ravnino, ki nima trikotnikov in ima vsaj en cikel. Potem $$|E(G)| \leq 2|V(G)| - 4.$$
\end{posledica}

\begin{opomba}
    1.\ in 2.\ posledica veljata tudi za drevesi z vsaj 3 vozlišči.
\end{opomba}

\begin{primer}
    Ni ravninski grafi:
    \begin{itemize}
        \item $K_5$ ni ravninski (ima preveč povezav po 1.\ posledici).
        \item $K_{3,3}$ ni ravninski (ima preveč povezav po 2.\ posledici).
    \end{itemize}
\end{primer}

\begin{izrek}[Kuratowski]
    Graf $G$ je ravninski natanko tedaj, kadar ne vsebuje podgrafa, ki je subdivizija $K_5$ ali subdivizija~$K_{3,3}$.
\end{izrek}

\begin{proof}
    $(\lthen)$ Graf je ravninski natanko tedaj, ko je vsaka njegova subdivizija je ravninska.
\end{proof}

\begin{primer}
    Ni ravninski grafi:
    \begin{itemize}
        \item Petersenov graf (ima subdivizijo $K_{3,3}$).
        \item $Q_4 = (K_2 \square K_2) \square (K_2 \square K_2)$ niso ravninski.
    \end{itemize}
\end{primer}

\begin{izrek}[Wagner]
    Graf $G$ je ravninski natanko tedaj, ki niti $K_5$ niti $K_{3,3}$ nista njegova minorja.
\end{izrek}

\begin{primer}
    Tudi z Wagnerjevem izrekom lahko dokažemo, da Petersonov graf ni ravninski ($K_5$ je minor). 
\end{primer}