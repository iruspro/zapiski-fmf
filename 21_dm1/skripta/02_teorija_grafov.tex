\section{Teorija grafov}

\begin{enumerate}
    \item Osnovni pojmi
    \begin{itemize}
        \item \colorbox{purple!30}{\textbf{Definicija.}} Graf. Množica vozlišč, množica povezav.
        \item \colorbox{yellow!30}{\emph{Opomba.}} Moč \(V(G)\). Krajišči povezave. Sosedni vozlišči.
        \item \colorbox{purple!30}{\textbf{Definicija.}} Soseščina vozlišča. Stopnja vozlišča. Minimalna in maksimalna stopnja v grafu.
        \item \colorbox{purple!30}{\textbf{Definicija.}} Regularen graf. \(r\)-regularen graf.
        \item \colorbox{yellow!30}{\emph{Primer.}} Petersonov graf. Ali je regularen?
        \item \colorbox{purple!30}{\textbf{Definicija.}} Matrika sosednosti. Incidenčna matrika.
    \end{itemize}

    \item Lema o rokovanju
    \begin{itemize}
        \item \colorbox{blue!30}{\textbf{Lema.}} Lema o rokovanju.
        \item \colorbox{orange!30}{\textbf{Posledica.}} Koliko lahko ima graf vozlišč lihe stopnje?
    \end{itemize}

    Izpitna vprašanja:
    \begin{itemize}
        \item Kaj je stopnja vozlišča grafa in kaj pravi lema o rokovanju? Kako dokažemo to lemo?
    \end{itemize}    

    \item Podgrafi
    \begin{itemize}
        \item \colorbox{purple!30}{\textbf{Definicija.}} Podgraf. Vpet podgraf. Induciran podgraf.
    \end{itemize}

    \item Nekatere družine grafov
    \begin{itemize}
        \item Polni grafi. Koliko ima povezav?
        \item Poti.
        \item Cikli.
        \item Polni dvodelni grafi.
        \item Hiperkocke. Koliko ima vozlišč in koliko ima povezav?
        \item Posplošeni Petersenovi grafi.
    \end{itemize}

    \item Sprehodi, poti in cikli
    \begin{itemize}
        \item \colorbox{purple!30}{\textbf{Definicija.}} Sprehod. Enostaven sprehod. Pot.
        \item \colorbox{purple!30}{\textbf{Definicija.}} Sklenjen sprehod. Enostaven sklenjen sprehod. Cikel.
        \item \colorbox{blue!30}{\textbf{Lema.}} Kaj če ima graf \(uv\)-sprehod?
        \item \colorbox{blue!30}{\textbf{Lema.}} Kaj če ima graf dve različini \(uv\)-poti?
        \item \colorbox{blue!30}{\textbf{Lema.}} Kaj če graf ima sklenjen sprehod lihe dolžine?
    \end{itemize}
    Izpitna vprašanja:
    \begin{itemize}
        \item  Pojasnite sprehod, sklenjen sprehod, pot v grafu, cikel v grafu. Pokažite, da vsak graf, ki vsebuje sklenjen sprehod lihe dolžine, vsebuje tudi cikel lihe dolžine.
    \end{itemize}  
    
    \item Povezane komponente, razdalja in premer
    \begin{itemize}
        \item \colorbox{purple!30}{\textbf{Definicija.}} Povezan graf. Komponente grafa. Število komponent grafa.
        \item \colorbox{purple!30}{\textbf{Definicija.}} Razdalja med vozliščimi \(u\) in \(v\).
        \item \colorbox{blue!30}{\textbf{Trditev.}} Kakšno strukturo ima \((V(G), d_G)\), če je \(G\) povezan?
        \item \colorbox{purple!30}{\textbf{Definicija.}} Ekscentričnost vozlišča. Premer grafa. Polmer grafa.
        \item \colorbox{yellow!30}{\emph{Primer.}} Določi \(\diam(P_{5,2})\) in \(\rad(P_{5,2})\) ter \(\diam(P_n)\) in \(\rad(P_n)\).
    \end{itemize}

    \item Inačice kocepta "`graf"'
    \begin{itemize}
        \item Enostavni grafi. Vzporedne povezave in zanke.
        \item Utežni grafi, omrežje, utežno omrežje.
        \item Usmerjeni grafi.
        \item Hipergrafi.
    \end{itemize}

    \item Dvodelnost
    \begin{itemize}
        \item \colorbox{purple!30}{\textbf{Definicija.}} Dvodelen graf.
        \item \colorbox{yellow!30}{\emph{Opomba.}} Kako lahko raziščemo dvodelnost grafa?
        \item \colorbox{blue!30}{\textbf{Izrek.}} Karakterizacija dvodelnosti.
    \end{itemize}

    Izpitna vprašanja:
    \begin{itemize}
        \item Kaj so dvodelni grafi? Kako jih karakteriziramo? Kako dokažemo to karakterizacijo?
    \end{itemize}  

    \newpage
    \item Morfizmi grafov
    \begin{itemize}
        \item \colorbox{purple!30}{\textbf{Definicija.}} Homomorfizem grafov. Vložitev.
        \item \colorbox{yellow!30}{\emph{Opomba.}} Kaj če je \(G\) dvodelen?
        \item \colorbox{purple!30}{\textbf{Definicija.}} Izometrična vložitev.
        \item \colorbox{purple!30}{\textbf{Definicija.}} Izomorfizem grafov. Izomorfna grafa.
        \item \colorbox{yellow!30}{\emph{Opomba.}} Ali je izomorfnost ekvivalenčna relacija?
        \item \colorbox{purple!30}{\textbf{Definicija.}} Avtomorfizem grafa. Grupa avtomorfizmov grafa.
    \end{itemize}

    Izpitna vprašanja:
    \begin{itemize}
        \item Kaj je homomorfizem grafov, izomorfizem grafov in avtomorfizem grafa? Kaj je to \(\Aut(G)\)? Kakšno algebrsko strukturo ima?
    \end{itemize}  

    \item Operaciji z grafi
    \begin{itemize}
        \item \colorbox{purple!30}{\textbf{Definicija.}} Komplement grafa.
        \item \colorbox{blue!30}{\textbf{Trditev.}} Recimo, da \(G\) ni povezan. Kako lahko ocenimo \(\diam (\overline{G})\)?
        \item \colorbox{orange!30}{\textbf{Posledica.}} Kaj lahko povemo o povezanosti \(G\) in \(\overline{G}\)?
        \item \colorbox{purple!30}{\textbf{Definicija.}} Odstranevanje vozlišč in povezav.
        \item \colorbox{yellow!30}{\emph{Opomba.}} Kako dobimo podgraf grafa \(G\)?
        \item \colorbox{purple!30}{\textbf{Definicija.}} Skrčitev povezave. Minor.
        \item \colorbox{yellow!30}{\emph{Primer.}} Ali je \(K_5\) minor \(P_{5,2}\)?
        \item \colorbox{yellow!30}{\emph{Opomba.}} Kako dobimo minor grafa \(G\)?
        \item \colorbox{purple!30}{\textbf{Definicija.}} Subdivizija povezave. Subdivizija. Glajenje vozlišča.
        \item \colorbox{purple!30}{\textbf{Definicija.}} Homeomorfna grafa.
        \item \colorbox{purple!30}{\textbf{Definicija.}} Kartezični produkt grafov.
        \item \colorbox{blue!30}{\textbf{Trditev.}} Komutativnost, asociativnost kartezičnega produkta. Enota. 
        \item \colorbox{yellow!30}{\emph{Opomba.}} Kartezična potenca grafa.
        \item \colorbox{yellow!30}{\emph{Opomba.}} Čemu je enak \(Q_n\)?
    \end{itemize}

    Izpitna vprašanja:
    \begin{itemize}
        \item Kaj pomeni, da je graf \(H\) minor grafa \(G\)? Kdaj sta dva grafa homeomorfna? Pojasnite operacijo kartezičnega produkta grafov.
    \end{itemize}  

    \item Prerezna vozlišča in \(k\)-povezanost
    \begin{itemize}
        \item \colorbox{purple!30}{\textbf{Definicija.}} Prerezno vozlišče. Most. Prerez. Povezavni prerez.
        \item \colorbox{purple!30}{\textbf{Definicija.}} \(k\)-povezan graf. Povezanost grafa.
        \item \colorbox{yellow!30}{\emph{Primer.}} Določi \(K(K_n), \ K(C_n), \ K(Q_n)\).
        \item \colorbox{yellow!30}{\emph{Opomba.}} Kako lahko ocenimo \(K(G)\)?
        \item \colorbox{purple!30}{\textbf{Definicija.}} Notranji-disjunktni \(uv\)-poti.
        \item \colorbox{blue!30}{\textbf{Izrek.}} Karakterizija \(2\)-povezanosti (Whitney).
        \item \colorbox{blue!30}{\textbf{Izrek.}} Mengerjev izrek.
    \end{itemize}

    Izpitna vprašanja:
    \begin{itemize}
        \item Kaj so to prerezna vozlišča in prerezne povezave grafa? Kdaj je graf \(k\)-povezan in kaj je to povezanost grafa?
        \item Pojasnite Whitney-ev izrek, ki karakterizira \(2\)-povezane grafe. Skicirajte dokaz tega izreka. Zapišite Mengerjev izrek.
    \end{itemize}  
    
    \item Drevesa
    \begin{itemize}
        \item \colorbox{purple!30}{\textbf{Definicija.}} Gozd. Drevo. List.
        \item \colorbox{blue!30}{\textbf{Lema.}} Ali drevo vendo premore list?
        \item \colorbox{blue!30}{\textbf{Lema.}} Število povezav v drevesu.
        \item \colorbox{blue!30}{\textbf{Lema.}} Naj bo \(G\) povezan graf in \(e \in E(G)\) leži na nekem ciklu. Kaj lahko povemo o \(G - e\)?
        \item \colorbox{blue!30}{\textbf{Lema.}} Kako lahko ocenimo število povezav v povezanem grafu?
        \item \colorbox{yellow!30}{\emph{Opomba.}} Kaj so drevesa z vidika števila povezav?
        \item \colorbox{blue!30}{\textbf{Izrek.}} 3 karakterizaciji drevesa.
    \end{itemize}

    Izpitna vprašanja:
    \begin{itemize}
        \item Kaj je drevo in kaj je gozd? Katere karakterizacije dreves poznate?
    \end{itemize}  

    \newpage
    \item Vpeta drevesa
    \begin{itemize}
        \item \colorbox{purple!30}{\textbf{Definicija.}} Vpeto drevo. Število vpetih dreves grafa.
        \item \colorbox{yellow!30}{\emph{Primer.}} Določi \(\tau (C_n)\).
        \item \colorbox{blue!30}{\textbf{Trditev.}} Karakterizacija povezanosti grafa.
        \item \colorbox{blue!30}{\textbf{Trditev.}} Rekurzivna formula za število vpetih dreves.
        \item \colorbox{purple!30}{\textbf{Definicija.}} Laplaceova matrika.
        \item \colorbox{blue!30}{\textbf{Izrek.}} Kako izračunamo število vpetih dreves s pomočjo Laplaceove matrike?
        \item \colorbox{blue!30}{\textbf{Izrek.}} Določi \(\tau(K_n)\) (Cayley).
    \end{itemize}

    Izpitna vprašanja:
    \begin{itemize}
        \item Kaj je vpeto drevo grafa? Kateri grafi premorejo vpeta drevesa? Kako lahko rekurzivno določimo število vpetih dreves povezanega grafa?
    \end{itemize}      

    \item Eulerjevi in Hamiltonovi grafi
    \begin{itemize}
        \item \colorbox{purple!30}{\textbf{Definicija.}} Eulerjev sprehod. Eulerjev obhod. Eulerjev graf.
        \item \colorbox{blue!30}{\textbf{Izrek.}} Karakterizacija Eulerjevih grafov.
        \item \colorbox{blue!30}{\textbf{Izrek.}} Kadar povezan graf premore Eulerjev obhod?
        \item Fleuryjev algoritem za poisk Eulerjeva obhoda.
        \item \colorbox{purple!30}{\textbf{Definicija.}} Hamiltonov cikel. Hamiltonov pot. Hamiltonov graf.
        \item \colorbox{blue!30}{\textbf{Izrek.}} Potreben pogoj, da je graf Hamiltonov.
        \item \colorbox{yellow!30}{\emph{Primer.}} Kadar je \(K_{n,m}\) Hamiltonov?
        \item \colorbox{blue!30}{\textbf{Izrek.}} Orejev izrek.
        \item \colorbox{blue!30}{\textbf{Izrek.}} Diracov izrek.
    \end{itemize}

    Izpitna vprašanja:
    \begin{itemize}
        \item Kaj pomeni, da je graf eulerjev? Kako karakteriziramo eulerjeve grafe? Skicirajte dokaz slednjega rezultata.
        \item Kdaj je graf hamiltonov? Navedite in pojasnite potrebni pogoj z razpadom grafa za obstoj hamiltonovega cikla v grafu.
        \item Navedite Orejev zadostni pogoj za obstoj hamiltonovega cikla v grafu. Skicirajte dokaz tega izreka.
    \end{itemize}  

    \item Ravninski grafi
    \begin{itemize}
        \item \colorbox{purple!30}{\textbf{Definicija.}} Ravninski graf. Graf vložen v ravnino.
        \item \colorbox{yellow!30}{\emph{Primer.}} Ali je \(K_{2,3}\) ravninski? Kaj pa \(K_{3,3}\)?
        \item \colorbox{purple!30}{\textbf{Definicija.}} Lica grafa. Množica vseh lic grafa.
        \item \colorbox{yellow!30}{\emph{Opomba.}} Ali je isto vložiti graf v ravnino in v sfero? Kaj to pomeni za lica?
        \item \colorbox{purple!30}{\textbf{Definicija.}} Rob lica. Dolžina roba. 
        \item \colorbox{yellow!30}{\emph{Opomba.}} Ali je vsaka povezava leži na rovu dveh lic? Kaj pa če na robu le enega?
        \item \colorbox{blue!30}{\textbf{Trditev.}} Lema o rokovanju za ravninske grafe.
        \item \colorbox{purple!30}{\textbf{Definicija.}} Ožina grafa.
        \item \colorbox{blue!30}{\textbf{Trditev.}} Ocena navzdol števila povezav, za graf, ki ima vsaj en cikel in vložen v ravnino.
        \item \colorbox{blue!30}{\textbf{Izrek.}} Eulerjeva formula.
        \item \colorbox{orange!30}{\textbf{Posledica.}} Eulerjeva formula za povezane grafe.
        \item \colorbox{blue!30}{\textbf{Trditev.}} Ocena navzgor števila povezav, za graf, ki ima vsaj en cikel in vložen v ravnino.
        \item \colorbox{orange!30}{\textbf{Posledica.}} Ocena, če upoštevamo, da je \(g(G) \geq 3\).
        \item \colorbox{orange!30}{\textbf{Posledica.}} Ocena za graf, ki nima trikotnikov.
        \item \colorbox{yellow!30}{\emph{Primer.}} Pokaži, da nista ravninska \(K_5\) in \(K_{3,3}\).
        \item \colorbox{blue!30}{\textbf{Izrek.}} Izrek Kuratowskega.
        \item \colorbox{yellow!30}{\emph{Primer.}} Ali sta ravninska: Petersenov graf, \(Q_4\).
        \item \colorbox{blue!30}{\textbf{Izrek.}} Wagnerjev izrek.
    \end{itemize}

    Izpitna vprašanja:
    \begin{itemize}
        \item  Kaj so ravninski grafi? Kaj so lica ravninske vložitve grafa in čemu je enaka vsota dolžin vseh lic ravninske vložitve grafa? Kako lahko omejimo število povezav ravninskega grafa s pomočjo njegove ožine?
        \item Kaj pravi Eulerjeva formula za ravninske grafe? Skicirajte njen dokaz. Katere posledice Eulerjeve formule poznate?
    \end{itemize}  
\end{enumerate}