\section{Riemannov integral}
\begin{enumerate}
    \item Darbouxev integral
    \begin{itemize}
        \item \textbf{Definicija.} Kvader. Delitev kvadra.
        \item \textbf{Definicija.} Finejša delitev.
        \item \textbf{Definicija.} Spodnja in zgornja Darbouxoevi vsoti.
        \item \textbf{Lema.} Kako so povezane \(s(D), \ s(D'), \ S(D), \ S(D')\), kjer je \(D \leq D'\)?
        \item \textbf{Posledica.} Kakšna je povezava med \(s(D)\) in \(S(D)\)?
        \item \textbf{Definicija.} Kadar rečemo, da je funkcija \(f\) integrabilna po Darbouxju na kvadru?
        \item Dvojni in trojni integral.
    \end{itemize}

    \item Riemannov integral
    \begin{itemize}
        \item \textbf{Definicija.} Riemannova vsota.
        \item \textbf{Definicija.} Kadar rečemo, da je funkcija \(f\) integrabilna po Riemannu na kvadru?
        \item \textbf{Opomba.} Ali je Riemannovo integrabilna funkcija omejena?
        \item \textcolor{red}{(*)} \textbf{Izrek.} Povezava med \(I_D\) in \(I_R\).
        \item \textbf{Trditev.} Kaj lahko povemo o zvezni funkciji na kvadru \(K\)?
        \item \textbf{Lema.} Dovolj majhne delitve. 
    \end{itemize}

    \item Osnovne lastnosti Riemannovega integrala
    \begin{itemize}
        \item \textbf{Trditev.} Struktura množice integrabilnih funkcij na kvadru \(K\).
        \item \textbf{Trditev.} Monotonost integrala.
        \item \textbf{Trditev.} Trikotniška neenakost.
    \end{itemize}

    \item Fubinijev izrek
    \begin{itemize}
        \item \textcolor{red}{(*)} \textbf{Izrek.} Fubinijev izrek.
        \item \textbf{Posledica.} Kako računamo \(n\)-terni integral zvezne funkcije?
        \item \textbf{Posledica.} Kako računamo dvojni integral zvezne funkcije?
        \item \textbf{Posledica.} Kako računamo trojni integral zvezne funkcije?
    \end{itemize}

    \item Riemannov integral na omejeni množicah
    \begin{itemize}
        \item \textbf{Definicija.} Kadar pravimo, da je omejena funkcija integrabilna na omejeni množici \(A\)?
        \item \textbf{Trditev.} Struktura množice integrabilnih funkcij na omejeni množici \(A\).
        \item \textbf{Definicija.} Karakteristična funkcija množice \(A\).
        \item \textbf{Definicija.} Kadar pravimo, da ima omejena množica \(A\) prostornino?
        \item \textbf{Opomba.} Kaj lahko povemo o integrabilnosti konstantnih funkcij na množice c prostornino?
        \item \textbf{Trditev.} Zadostni in potrebni pogoj, da ima omejena množica \(A\) prostornino.
    \end{itemize}

    \item Lastnosti omejenih množic s prostornino \(0\)
    \begin{itemize}
        \item \textbf{Trditev.} Zadostni in potrebni pogoj, da ima omejena množica \(A\) prostornino \(0\).
        \item \textbf{Trditev.} Čemu je enaka prostornina končne unije množic s prostornino \(0\)?
        \item \textbf{Trditev.} Čemu je enaka prostornina  grafa integrabilne na kvadru \(K\) funkciji?
        \item \textbf{Trditev.} Recimo, da \(A \subseteq \R^n\) ima prostornino, \(A \subseteq K\), kjer je \(K\) kvader. Kaj lahko povemo o prostornine množice \(K \setminus A\)?
        \item \textbf{Trditev.} Čemu je enak integral po množice s prostornino \(0\)?
        \item \textbf{Posledica.} Kaj lahko povemo o integralih funkcij, ki se razlikujeta na množice s prostornino \(0\)?
        \item \textbf{Definicija.} Kadar pravimo, da ima množica \(A\) mero \(0\)?
        \item \textbf{Posledica.} Čemu je enaka mera množice s prostornino \(0\)?
        \item \textbf{Trditev.} Čemu je enaka mera števne unije množic s mero \(0\)?
        \item \textbf{Posledica.} Čemu je enaka mera vsake števne množice v \(R^n\)?
        \item \textbf{Trditev.} Čemu je enaka mera grafa zvezne funkcije?
        \item \textbf{Trditev.} Kaj lahko povemo o prostornine in mere kompaktne množice?
        \item \textbf{Definicija.} Kadar pravimo, da je neka lastnost velja skoraj povsod?
        \item  \textbf{Trditev.} Kaj lahko povemo o funkciji \(f\) na kvadru \(K\), če \(\int_K f(x) \, dx = 0\)?
        \item \textbf{Posledica.} Recimo, da \(\all{x \in K} f(x) \leq g(x)\) in \(\int_K f(x) \, dx = \int_K g(x) \, dx\). Kaj lahko povemo o funkcijah \(f\) in \(g\)?
        \item \textbf{Lema.} 
        \item \textbf{Trditev.} Kaj lahko povemo o integralu integrabilne funkcije \(f\) na omejeni množici \(A\) z mero \(0\)?
    \end{itemize}

    \item Lebesguev izrek
    \begin{itemize}
        \item \textcolor{blue}{(*)} \textbf{Izrek.} Lebesguev izrek.
        \item \textbf{Posledica.} 
        \item \textbf{Posledica.} Kadar je funkcija \(f\) integrabilna na omejeni množici \(A\) s prostornino?
    \end{itemize}

    \item Osnovni lastnosti integrala po omejenih množicah.
    \begin{itemize}
        \item \textbf{Trditev.} Struktura množice integrabilnih funkcij na omejeni množici \(A\).
        \item \textbf{Trditev.} Monotonost integrala.
        \item \textbf{Trditev.} Trikotniška neenakost.
        \item \textbf{Trditev.} Kaj če \(\all{x \in A} m \leq f(x) \leq M\) in \(A\) ima prostornino?
        \item \textbf{Trditev.} Kaj če je \(A\) kompaktna povezana množica s prostornino in \(f\) zvezna?
        \item \textbf{Trditev.} Integral na uniji množic.
        \item \textcolor{red}{(*)} \textbf{Izrek.} Fubinijev izrek.
        \item \textbf{Trditev.} Recimo, da ima \(A \subseteq \R^n\) prostornino. Kaj lahko povemo o prostornine \(\Int A\) in \(\Cl A\)?
        \item \textbf{Trditev.} Recimo, da ima \(A \subseteq \R^n\) prostornino in \(f: \Cl A \to \R\) omejena in integrabilna na \(A\). Kaj lahko povemo o integralu na \(\Int A\) in \(\Cl A\)?
    \end{itemize}
\end{enumerate}