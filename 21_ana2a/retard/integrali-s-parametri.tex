\section{Integrali s parametri}
Naj bo $f:[a,b]_x \times [c,d]_y \to \R$ funkcija. Gledamo funkcijo $\ds F(y) = \int_{a}^{b}f(x,y) \, dx$, kjer $y \in [c,d]$ je \emph{parameter}.

Zanima nas v kakšni so povezavi lastnosti funkcije $f$ in funkcije $F$.

\begin{zgled}
    Izračunaj $\ds F(y) = \int_{0}^{\pi} \sin (xy) \,dx$. Ali je $F(y)$ zvezna? Kaj je $D_F$?
\end{zgled}

\begin{zgled}
    \emph{Eulerjeva funkcija gama} je $\ds \Gamma (s) = \int_{0}^{\infty} x^{s-1} e^{-x} \, dx$.
    \begin{itemize}
        \item Določi $D_\Gamma$.
        \item Kakšen predznak ima $\Gamma$ na $D_\Gamma$?
        \item Določi osnovno rekurzivno relacijo za $\Gamma$.
        \item Kakšna povezava med fakulteto in $\Gamma$?
        \item Kako bi lahko definirali $\Gamma$ za negativne vrednosti? Za katere lahko?
    \end{itemize}
\end{zgled}

\begin{definicija}
    Podmnožica $D \subseteq \R^n$ je \emph{lokalno kompaktna}, če
    $$\all{a \in D} \some{r \in \R \, . \, r > 0} D \cap \overline{K(a,r)} \text{ kompaktna množica}.$$
\end{definicija}

\begin{zgled}
    Primeri lokalno kompaktnih množic.
    \begin{itemize}
        \item Vsaka zaprta in vsaka odprta množica v $\R^n$ je lokalno kompaktna.
        \item $D \subseteq \R^2, \ D = K(0,1) \cup \set{(1,0)}$ ni lokalno kompaktna.
    \end{itemize}
\end{zgled}

\begin{trditev}
    Recimo, da velja
    \begin{enumerate}
        \item $D \subseteq \R^n$ lokalno kompaktna podmnožica;
        \item $I$ zaprt interval na $\R$;
        \item funkcija $f: I_x \times D_y$ zvezna.
    \end{enumerate}
    Tedaj je funkcija $\displaystyle F(u, v, y) = \int_{u}^{v}f(x,y) \, dx$, kjer so $(u, v, y) \in I \times I \times D$, zvezna na $I \times I \times D$.
\end{trditev}

\begin{proof}
    Dokazujemo zveznost v točki $(u_0, v_0, y_0) \in I \times I \times D$. Ocenimo razliko $|F(u,v,y) - F(u_0, v_0, y_0)|$.
    \begin{itemize}
        \item  Kaj vemo o funkciji $f$ na nekem kompaktu? \qedhere
    \end{itemize}
\end{proof}

\begin{posledica}
    Recimo, da velja
    \begin{enumerate}
        \item $D \subseteq \R^n$ lokalno kompaktna podmnožica;
        \item $I = [a,b]$;
        \item funkcija $f: I_x \times D_y$ zvezna.
    \end{enumerate}
    Tedaj je funkcija $\ds F(y) = \int_{a}^{b}f(x,y) \, dx$, zvezna na $D$.
\end{posledica}

\subsection{Odvajanje integralov s parametri}
\begin{trditev}
    Recimo, da velja
    \begin{enumerate}
        \item funkcija $f:[a,b]_x \times (c,d)_y \to \R$ zvezna;
        \item $\all{(x,y) \in [a,b] \times (c,d)} f \text{ parcialno odvedljiva po } y$;
        \item funkcija $\podv{f}{y}(x,y)$ zvezna na $[a,b] \times (c,d)$.
    \end{enumerate}
    Tedaj je 
    \begin{enumerate}
        \item $\ds F(y) = \int_{a}^{b}f(x,y) \, dx$ odvedljiva funkcija na $(c,d)$.
        \item $\ds F'(y) = \frac{dF}{dy}(y) = \frac{d}{dy} \int_{a}^{b}f(x,y) \, dx = \int_{a}^{b} \podv{f}{y}(x,y) \, dx$, tj. lahko zamenjamo vrstni red odvajanja.
    \end{enumerate}    
\end{trditev}

\begin{proof}
    Dokazujemo, da je $F$ odvedljiva v točki $y \in (c,d)$. Ocenimo razliko $\ds \left|\frac{F(y+h)-F(y)}{h} - \int_{a}^{b} \podv{f}{y}(x,y) \,dx \right|$
    \begin{itemize}
        \item Lagrangeev izrek.
        \item Ustrezni kompakti. \qedhere
    \end{itemize}
\end{proof}

\newpage
\begin{posledica}
    Recimo, da velja
    \begin{enumerate}
        \item funkcija $f:[a,b]_x \times (c,d)_y \to \R$ zvezna;
        \item $\all{(x,y) \in [a,b] \times (c,d)} f \text{ parcialno odvedljiva po } y$;
        \item funkcija $\podv{f}{y}(x,y)$ zvezna na $[a,b] \times (c,d)$.
        \item funkciji $\alpha, \beta: (c,d) \to [a,b]$ zvezno odvedljivi.
    \end{enumerate}
    Tedaj $F'(y) = \ds \frac{d}{dy} \int_{\alpha(y)}^{\beta(y)}f(x,y) \, dx = \int_{\alpha(y)}^{\beta(y)} \podv{f}{y}(x,y) \, dx + \beta'(y)f(\beta(y), y) - \alpha'(y)f(\alpha(y), y)$.
\end{posledica}

\begin{proof}
    $\ds F(u,v,y) = \int_{u}^{v}f(x,y) \, dx \lthen \int_{\alpha(y)}^{\beta(y)} f(x,y)  \,dx = F(\alpha(y), \beta(y), y)$. Torej treba izračunati odvod funkcije treh spremenljivk. 
    \begin{itemize}
        \item Osnovni izrek analize. \qedhere
    \end{itemize}
\end{proof}

\begin{posledica}
    Recimo, da velja
    \begin{enumerate}
        \item podmnožica $D\subseteq \R^n$ odprta;
        \item funkcija $f: [a,b]_x \times D_y \to \R$ zvezna;
        \item $\all{(x,y) \in [a,b] \times D} \all{j \in [n]} f \text{ parcialno odvedljiva po } y_j$;
        \item $\all{j \in [n]} \podv{f}{y_j}(x,y) \text{ so zvezne funkcije na } [a,b] \times D$.
    \end{enumerate}
    Tedaj je 
    \begin{enumerate}
        \item $\ds F(y) = \int_{a}^{b} f(x,y) \, dx $ funkcija razreda $C^1$ na $D$.
        \item $\ds \podv{F}{y_j}(y) = \int_{a}^{b} \podv{f}{y_j}(x,y) \, dx$.
    \end{enumerate}    
\end{posledica}

\begin{zgled}
    S pomočjo integrala s parametrom $\ds F(a) = \int_{0}^{1} \frac{x^a-1}{\ln x} \,dx $ izračunaj $\ds \int_{0}^{1} \frac{x-1}{\ln x} \,dx $.
\end{zgled}

\subsection{Integral integrala s parametrom}
\begin{izrek}
    Recimo, da velja
    \begin{enumerate}
        \item funkcija \(f:[a,b] \times [c,d] \to \R\) zvezna.
    \end{enumerate}
    Tedaj je 
    $$\int_{c}^{d} \left(\int_{a}^{b} f(x,y) \,dx \right) \,dy =  \int_{a}^{b} \left(\int_{c}^{d} f(x,y) \,dy \right) \,dx.$$
\end{izrek}

\begin{definicija}
    Integrali tipa $\ds \int_{c}^{d} \left(\int_{a}^{b} f(x,y) \,dx \right) \,dy$ imenujemo \emph{dvakratni integrali}. 
\end{definicija}

\begin{proof}
    Definiramo $\ds \Psi(y) = \int_{c}^{y} \left(\int_{a}^{b} f(x,s) \,dx \right) \,ds $ in $\ds \Phi(y) = \int_{a}^{b} \left(\int_{c}^{y} f(x,s) \,ds \right) \,dx$. Dololj, da dokažemo:
    \begin{itemize}
        \item $\Psi$ in $\Phi$ se ujemata v eni točki.
        \item $\Psi' = \Phi'$.
    \end{itemize}
    Pomagamo si s osnovnim izrekom analize.
\end{proof}

\begin{zgled}
    Izračunaj $\ds \int_{0}^{1} \left(\int_{0}^{2} (x + y^2)  \,dx \right)  \,dy $ na dva načina.
\end{zgled}

\subsection{Posplošeni integrali s parametri}
Naj bo $Y$ neka množica, $a \in \R$, $f: [a, \infty)_x \times Y_y \to \R$ funkcija. Standardni predpostavki:
\begin{itemize}
    \item Funkcija $f$ za vsak $y \in Y$ zvezna, tj. $x \mapsto f(x,y)$ zvezna na $[a, \infty)$ za vsak \(y \in Y\).
    \item Za vsak \(y \in Y\) obstaja integral \( \ds F(y)= \int_{a}^{\infty} f(x,y) \,dx \)
\end{itemize}

\begin{opomba}
    Integral \(\ds F(y) = \int_{a}^{\infty} f(x,y) \, dx\) obstaja po definiciji, če obstaja $\ds \lim_{b \to \infty} \int_{a}^{b} f(x,y) dx$. Ta limita obstaja natanko tedaj, ko $\ds \lim_{b \to \infty} \int_{b}^{\infty} f(x,y) \, dx = 0$, kar je ravno konvergenca po točkah. \textcolor{red}{?}
\end{opomba}

\begin{definicija}
    Integral \(\ds F(y) = \int_{a}^{\infty} f(x,y) \, dx\) \emph{konvergira enakomerno} na \(Y\), če 
    $$\all{\epsilon > 0} \some{b_0 \geq a} \all{b \geq b_0} \all{y \in Y} \left|\int_{b}^{\infty} f(x,y) \, dx \right| < \epsilon.$$
\end{definicija}

\begin{zgled}
    Izračunaj \(\ds F(y) = \int_{0}^{\infty} y e^{-xy} \, dx\) za \(y \in [0, \infty)\). Ali je konvergenca enakomerna na \([c, \infty), \ c > 0\)? Ali je konvergenca enakomerna na $(0, \infty)$?
\end{zgled}

\begin{opomba}
    Recimo, da \(\ds F(y) = \int_{a}^{\infty} f(x,y) \, dx\) konvergira enakomerno. Kaj to pomeni? Naj bo $F_b(y) = \int_{a}^{\infty} f(x,y) \, dx$. Potem funkcijsko zaporedje $F_b(y)$ konvergira enakomerno proti $F(y)$ na $Y$.
\end{opomba}

\begin{trditev}
    Recimo, da velja
    \begin{itemize}
        \item podmnožica \(Y \subseteq \R^n\) lokalno kompaktna;
        \item funkcija \(f :[a, \infty) \times Y \to \R\) zvezna;
        \item integral s parametri \(\ds F(y) = \int_{a}^{\infty} f(x,y) \, dx\) konvergira enakomerno na \(Y\).
    \end{itemize}
    Tedaj je \(F\) zvezna na \(Y\).
\end{trditev}

\begin{proof}
    Enakomerna limita zveznih funkcij.
\end{proof}

\begin{opomba}
    Zveznost (in odvedljivost) sta lokalni lastnosti (zvezna (oz. odvedljiva) v vsaki točki), tj. $f$ je zvezna na $Y$, če je zvezna v vsaki točki \(y \in Y\) (tudi, če je zvezna v okolici vsake točke \(y \in Y\)). Zato v prejšnji trditvi je za zveznost \(F\) na \(Y\) dovolj zahtevati, da je integral \emph{lokalno enakomerno konvergira}, tj
    $$\all{y\in Y} \some{r>0} F \text{ enakomerno konvergira na } Y \cap K(y,r).$$
\end{opomba}

\begin{trditev}[Test enakomerne konvergence]
    Recimo, da velja
    \begin{enumerate}
        \item funkcija \(f:[a, \infty) \times Y \to \R\) zvezna za vsak $y \in Y$;
        \item obstaja taka zvezna funkcija \(g:[a, \infty) \to \R\), da za vsak $(x,y) \in [a, \infty) \times Y$ velja $|f(x,y)| \leq g(x)$;
        \item obstaja integral \(\ds \int_{a}^{\infty} g(x) \, dx\).
    \end{enumerate}
    Tedaj integral \(\ds F(y) = \int_{a}^{\infty} f(x,y) \, dx\) konvergira enakomerno na \(Y\).
\end{trditev}

\begin{proof}
    Cauchyjev kriterij za konvergenco integralov.
\end{proof}

\begin{zgled}
    Obravnavaj lokalno enakomerno konvergenco funkcij
    \begin{itemize}
        \item \(\ds s \mapsto \int_{1}^{\infty} x^{s-1} e^{-x} \, dx\).
        \item \(\ds s \mapsto \int_{0}^{1} x^{s-1} e^{-x} \, dx\).
    \end{itemize}

    Vpeljava nove spremenljivke $x = t^N$ \textcolor{red}{?}
\end{zgled}

\begin{trditev}
    Recimo, da velja
    \begin{enumerate}
        \item funkcija \(f:[a, \infty)_x \times [c,d]_y \to \R\) zvezna;
        \item integral \(\ds F(y) = \int_{a}^{\infty} f(x,y) \, dx\) konvergira enakomerno na \([c,d]\).
    \end{enumerate}
    Tedaj \[\int_{c}^{d}\left(\int_{a}^{\infty} f(x,y) \, dx\right) \, dy = \int_{a}^{\infty}\left(\int_{c}^{d} f(x,y) \, dy\right) \, dx.\]
\end{trditev}

\begin{proof}
    Račun.
\end{proof}

Kadar je \(\ds \int_{c}^{\infty}\left(\int_{a}^{\infty} f(x,y) \, dx\right) \, dy = \int_{a}^{\infty}\left(\int_{c}^{\infty} f(x,y) \, dy\right) \, dx\)?

\begin{opomba}
    Podobno vprašanje: Kadar je \(\sum_{i=1}^{\infty}\sum_{j=1}^{\infty}a_{ij} = \sum_{j=1}^{\infty}\sum_{i=1}^{\infty}a_{ij}\)?
\end{opomba}
\begin{trditev}
    Recimo, da velja
    \begin{enumerate}
        \item funkcija \(f:[a, \infty)_x \times [c,\infty)_y \to [0, \infty)\) nenegativna in zvezna;
        \item integral \(\ds F(y) = \int_{a}^{\infty} f(x,y) \, dx\) konvergira lokalno enakomerno na \([c, \infty)\) in
        
        integral \(\ds G(x) = \int_{c}^{\infty} f(x,y) \, dy\) konvergira lokalno enakomerno na \([a, \infty)\) (imamo zveznost $F$ in $G$).
    \end{enumerate}
    Tedaj 
    \[\int_{c}^{\infty}\left(\int_{a}^{\infty} f(x,y) \, dx\right) \, dy = \int_{a}^{\infty}\left(\int_{c}^{\infty} f(x,y) \, dy\right) \, dx.\]
    Torej ali sta oba enaka \(\infty\), ali pa sta oba končna in enaka.
\end{trditev}

\begin{proof}
    Ocenimo navzgor \(\ds \int_{a}^{b} G(x) \, dx\) in \(\ds \int_{c}^{d} F(y) \, dy\).
\end{proof}

\begin{trditev}
    Recimo, da velja
    \begin{enumerate}
        \item funckija \(f:[a, \infty) \times [c,\infty) \to \R\) zvezna;
        \item integral \(\ds F(y) = \int_{a}^{\infty} |f(x,y)| \, dx\) konvergira lokalno enakomerno na \([c, \infty)\) in
        
        integral \(\ds G(x) = \int_{c}^{\infty} |f(x,y)| \, dy\) konvergira lokalno enakomerno na \([a, \infty)\);
        \item Ali \(\ds \int_{c}^{\infty}\left(\int_{a}^{\infty} |f(x,y)| \, dx\right) \, dy\) končen ali \(\ds \int_{a}^{\infty}\left(\int_{c}^{\infty} |f(x,y)| \, dy\right) \, dx\) končen.
    \end{enumerate}
    Tedaj je 
    \[\int_{c}^{\infty}\left(\int_{a}^{\infty} |f(x,y)| \, dx\right) \, dy = \int_{a}^{\infty}\left(\int_{c}^{\infty} |f(x,y)| \, dy\right) \, dx.\]
\end{trditev}

\begin{trditev}[Odvod posplošenega integrala s parametri]
    Recimo, da velja
    \begin{enumerate}
        \item funkcija \(f:[a, \infty) \times (c,d) \to \R\) zvezna;
        \item integral \(\ds F(y) = \int_{a}^{\infty} f(x,y) \, dx\) konvergira na \((c,d)\);
        \item $\all{(x,y) \in [a,b] \times (c,d)} f \text{ parcialno odvedljiva po } y$;
        \item funkcija $\podv{f}{y}(x,y)$ zvezna na $[a,b] \times (c,d)$;
        \item integral \(\ds y \mapsto \int_{a}^{\infty} \podv{f}{y}(x,y) \, dx\) konvergira lokalno enakomerno na \((c,d)\).
    \end{enumerate}
    Tedaj je 
    \begin{enumerate}
        \item $\ds F(y) = \int_{a}^{\infty}f(x,y) \, dx$ zvezno odvedljiva funkcija na $(c,d)$;
        \item $\ds F'(y) = \frac{dF}{dy}(y) = \frac{d}{dy} \int_{a}^{\infty}f(x,y) \, dx = \int_{a}^{\infty} \podv{f}{y}(x,y) \, dx$.
    \end{enumerate}    
\end{trditev}

\begin{proof}
    \textcolor{red}{TODO}
\end{proof}

\begin{zgled}
    Naj bo \(0<c<d\). Izračunaj \(\ds \int_{0}^{\infty} \frac{e^{-cx} - e^{-dx}}{x}\).
\end{zgled}

\begin{trditev}
    Recimo, da velja
    \begin{enumerate}
        \item podmnožica \(D \subseteq \R^n\) odprta; 
        \item funkcija \(f:[a,\infty) \times D \to \R\) zvezna;
        \item za vsak \((z,y) \in [a,\infty) \times \) obstajajo \(\podv{f}{y_j}\) in so zvezni;
        \item za vsak \(y \in D\) obstaja \(\ds F(y) = \int_{a}^{\infty} f(x,y) \, dx\);
        \item za vsak \(j \in \set{1, \ldots, n}\) integral \(\ds F(y) = \int_{a}^{\infty} \podv{f}{y_j} \, dx\) konvergira lokalno enakomerno na \(D\).
    \end{enumerate}
    Tedaj je
    \begin{enumerate}
        \item $\ds F(y) = \int_{a}^{\infty}f(x,y) \, dx$ zvezno odvedljiva funkcija na $D$;
        \item $\ds F'(y) = \frac{\partial F}{\partial y_j}(y) = \frac{\partial}{\partial y_j} \int_{a}^{\infty}f(x,y) \, dx = \int_{a}^{\infty} \podv{f}{y_j}(x,y) \, dx$ za vse \(j \in \set{1, \ldots, n}\).
    \end{enumerate}    
\end{trditev}

\begin{zgled}
   Opazujemo integral \(\ds \int_{0}^{\infty} \frac{\sin x}{x} \, dx\). Velja:
   \begin{itemize}
    \item \(\ds \frac{2}{\pi} \int_{0}^{\infty} \frac{\sin (ax)}{x} \, dx = \sgn (a)\).
    \item \(\ds \frac{\sin x}{x}\) je nihanje z padajočo amplitudo, kar je podobno alternirajoče harmonične vrste.
    \item Integral \(\ds \int_{0}^{\infty} \frac{|\sin x|}{x} \, dx\) ne obstaja.
    \item \(\ds \int_{0}^{\infty} \frac{\sin x}{x} \, dx = \frac{\pi}{2}\). Dokaz \textcolor{red}{TODO}
   \end{itemize}
\end{zgled}

\newpage
\subsection{Eulerjeva funkcija gama}
\begin{definicija}
    Funkcija $\ds \Gamma (s) = \int_{0}^{\infty} x^{s-1}e^{-x} \, dx$ je \emph{Eulerjeva funkcija gama}.
\end{definicija}

\begin{trditev}
    Lastnosti Eulerjeve funkcije gama:
    \begin{itemize}
        \item $D_\Gamma = (0, \infty)$.
        \item \(\Gamma(s+1) = s \Gamma(s)\). Če je \(n \in \N\), potem \(\Gamma(n+1) = n!\).
        \item \(\Gamma(1) = 1\).
        \item \(\Gamma  \in C((0, \infty))\).
        \item \(\Gamma (s) = \frac{\Gamma (s+1)}{s}, \ s > 0\). Če je \(s \approx 0\), potem \(\Gamma(s) \approx \frac{1}{s}\).
        \item \(\Gamma \in C^\infty ((0, \infty))\).
        \item \(\Gamma(s)>0\).
        \item \(\Gamma\) je konveksna funkcija na \((0, \infty)\). Tudi \(\ln \Gamma\) konveksna funkcija na \((0, \infty)\).
    \end{itemize}
\end{trditev}

\begin{proof}
    \textcolor{red}{TODO}
\end{proof}

\begin{opomba}
    O konveksnosti. \textcolor{red}{TODO}
\end{opomba}

\begin{zgled}
    Naj bo \(a>0\). S pomočjo Eulerjeve funkcije gama izračunaj \(\ds \int_{0}^{\infty} e ^{-ax^2} \, dx\).
\end{zgled}

\begin{zgled}
    Naj bo \(a \in \R, \ \sigma > 0\). S pomočjo prejšnjega zgleda izračunaj \(\ds \int_{-\infty}^{\infty} \exp \left(\frac{-(x-a)^2}{2\sigma^2}\right) \, dx\).
\end{zgled}

\begin{izrek}
    Eulerjeva funkcija \(\Gamma\) je natanko določena z lastnostmi:
    \begin{enumerate}
        \item \(\Gamma(1) = 1\);
        \item \(\Gamma(s+1) = s \Gamma(s)\);
        \item \(\Gamma(s)>0\) in \(\Gamma\) je zvezna na \((0, \infty)\);
        \item \(\ln \Gamma\) je konveksna.
    \end{enumerate}
\end{izrek}

\subsection{Eulerjeva funkcija beta}
\begin{definicija}
    Funkcija \(\ds B(p,q) = \int_{0}^{1}x^{p-1}(1-x)^{q-1} \, dx\) je \emph{Eulerjeva funkcija beta}.
\end{definicija}

\begin{trditev}
    Lastnosti Eulerjeve funkcije beta:
    \begin{itemize}
        \item \(D_B = (0, \infty) \times (0, \infty)\).
        \item \(B(p,q) = B(q,p)\).
        \item \(\ds \frac{1}{2} B(\frac{\alpha + 1}{2}, \frac{\beta + 1}{2}) = \int_{0}^{\frac{\pi}{2}} \sin^\alpha t \cos^\beta t \, dt\) za \(\alpha, \beta > -1\).
    \end{itemize}
\end{trditev}

\begin{trditev}
    \(\ds B(p,q) = \int_{0}^{\infty} \frac{t^{p-1}}{(1+t)^{p+q}} \, dt\).
\end{trditev}

\begin{proof}
    V \(B(p,q)\) vpeljamo \(t = \frac{x}{1-x}\).
\end{proof}

\begin{posledica}
    \(\ds B(p, 1-p) \int_{0}^{\infty} \frac{t^{p-1}}{1+t} \, dt\) za \(0<p<1\).
\end{posledica}

\begin{posledica}
    \(B(\frac{1}{2}, \frac{1}{2}) = \pi\).
\end{posledica}

\begin{proof}
    Račun.
\end{proof}

\begin{opomba}
    Za \(p \in (0,1)\) velja: \[B(p, 1-p) = \frac{\pi}{\sin (p \pi)}.\]
\end{opomba}

\begin{izrek}[Osnovna povezava med \(B\) in \(\Gamma\)]
    \[B(p,q) = \frac{\Gamma(p) \Gamma(q)}{\Gamma(p+q)}.\]
\end{izrek}

\begin{proof}
    \textcolor{red}{TODO}
\end{proof}

\begin{posledica}
    \(\Gamma(\frac{1}{2}) = \sqrt{\pi}\).
\end{posledica}

\begin{proof}
    Račun z pomočjo osnovne povezave med \(B\) in \(\Gamma\).
\end{proof}

\begin{primer}
    Izračunaj \(\Gamma(\frac{7}{2})\).
\end{primer}

\begin{primer}
    S pomočjo Eulerjeve funkcije beta izračunaj \(\ds \int_{0}^{\frac{\pi}{2}} \sin^8x \cos^6x \, dx\).
\end{primer}

\newpage
\begin{izrek}[Stirlingova formula]
    \[\lim_{s \to \infty} \frac{\Gamma(s+1)}{s^s \, e^{-s} \, \sqrt{2 \pi s}} = 1.\]
\end{izrek}

\begin{proof}
    \textcolor{red}{TODO}
\end{proof}

\begin{posledica}
    \(\ds \lim_{n \to \infty} \frac{n!}{n^n \, e ^{-n} \, \sqrt{2 \pi n}} = 1\), tj. \(n! \approx \sqrt{2 \pi n} (\frac{n}{e})^n\).
\end{posledica}

\begin{primer}
    Izračunaj \( \lim_{n \to \infty} \frac{(2n)!}{n! \, n^n \, 2^n}\).
\end{primer}