\section{Dodatek}
\subsection{Banachovo skrčitveno načelo}
\begin{definicija}
    Naj bo \((M, d)\) metrični prostor. Preslikava \(f: M \to M\) je \textbf{skrčitev}, če 
    \[
        \some{q \in [0, 1)} \all{x, y \in M} d(f(x), f(y)) \leq qd(x, y)
    \]
\end{definicija}

\begin{izrek}[Banachovo skrčitveno načelo]
    \label{izr:banachovo-skrcitveno-nacelo}
    Naj bo 
    \begin{itemize}
        \item prostor \((M, d)\) je poln metrični prostor;
        \item preslikava \(f: M \to M\) je skrčitev.
    \end{itemize}
    Tedaj ima preslikava \(f\) natanko eno negibno točko, tj.
    \[\exactlyone{a \in M} f(a) = a\]
    ter za vsak \(x \in M\) zaporedje 
    \[
        x, f(x), f(f(x)), \ldots
    \]
    konvergira proti točki \(a\).
\end{izrek}