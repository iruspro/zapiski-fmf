\section{Energija}
\begin{enumerate}
    \item Kinetična energija točkastega telesa
    \begin{itemize}
        \item \textbf{Izrek.} O mehanske kinetične energije. Kinetična energija.
        \item \textbf{Definicija.} Delo.
        \item \textbf{Opomba.} Koliko dela opravimo pri nošenju vrečki?
        \item \textbf{Zgled.} Koliko je \(1\, \text{J}\)? Zveza med kilokaloriji in Jouli.
    \end{itemize}

    \item Sistem točkastih teles
    \begin{itemize}
        \item Kinetična energija težišča. Psevdodelo.
        \item \textbf{Opomba.} Kdaj je psevdodelo enako delu?
    \end{itemize}

    \item Potencialna energija
    \begin{itemize}
        \item Delo sile teže.
        \item Potencialna energija. 
        \item \textbf{Eksperiment.} \todo{}
    \end{itemize}

    \item Prožnostna energija
    \begin{itemize}
        \item Delo prožnostne sile.
        \item Izrek o mehanske energije.
        \item \textbf{Zgled.} Izračunaj končno hitrost telesa, ki pada navpično navzdol z višine \(h\) z začetno hitrostjo \(v_0 = 0 \ \text{m/s}\).
    \end{itemize}

    \item Moč
    \begin{itemize}
        \item \textbf{Definicija.} Moč.
    \end{itemize}
\end{enumerate}

\newpage
\subsection*{Izpitna vprašanja}
\begin{enumerate}
    \item Energija
    \begin{itemize}
        \item Zapiši izraz za delo, ki ga opravi neka sila, in natančno pojasni posamezne člene.
        \item Na grobo oceni kinetično energijo teniške žogice z maso \(60\) g in hitrostjo \(20\)~m/s, in energijsko vrednost bombona mase 3 g (oboje v SI enotah). Bombon naj bo ves iz sladkorja, ta pa ima kalorično vrednost 400 kcal na 100 g.
        \item Zapiši enačbo za kinetično in potencialno energijo točkastega delca.
        \item Mož nadmorske višine \(h_0\) na nadmorsko višino \(h_1 > h_0\) enkrat pride po strmi 2 km dolgi poti, drugič pa po položnejši 3 km dolgi poti. Kateri predznak ima v tem primeru delo sile teže? Ali je mož obakrat opravil isto delo?
    \end{itemize}
\end{enumerate}