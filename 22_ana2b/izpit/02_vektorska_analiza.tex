\section{Vektorska analiza}
\subsection{Gradient, divergenca in rotor. Potencial.}
\begin{itemize}
    \item \(\rot \grad u = 0\) in \(\divr \rot \vec{f} = 0\).
    \item Vektorsko polje \(\vec{f}\) na zvezdastem območju je potencialno natanko tedaj, ko \(\rot \vec{f} = 0\).
    \item Naj bo \(f(\vec{r}) = \frac{1}{|\vec{r} - \vec{a}|}\) skalarno polje. Velja:
    \[
        \grad f = - \frac{\vec{r} - \vec{a}}{|\vec{r} - \vec{a}|^3} \quad \text{ter} \quad \divr (\grad f) = 0.
    \]
    Torej je \(\grad f\) \textbf{solenoidalno} polje, tj.\ \(\divr \vec{f} = 0\).
    \item \(\rot (\vec{r} \times \vec{a}) = - 2 \vec{a}\); \(\rot (\vec{a} \times \vec{r}) = 2 \vec{a}\).
    \item Vektorsko polje \(\vec{f}\) je \textbf{potencialno}, če obstaja skalarno polje \(u\), da \(\vec{f} = \grad u\). Potencial polja dobimo z unijo členov po integraciji parcialnih odvodov.
    \item Laplaceov operator: \(\triangle u = \divr \grad u\).
\end{itemize}

\subsection{Krivuljni integral skalarnega polja}
Naj bo \(K\) krivulja z regularno parametrizacijo \(\vec{r}: [a,b] \to \R^3, \ \vec{r} = (x, y, z)\). Tedaj 
\[
    ds = |\dot{\vec{r}}(t)| dt = \sqrt{\dot{x}^2 + \dot{y}^2 + \dot{z}^2}.
\]
Naj bo \(u: D \subseteq \R^3 \to \R\) skalarno polje. Definiramo 
\[
    \int_{K} u \, ds = \int_{a}^{b} u(\vec{r}(t)) |\dot{\vec{r}}(t)| dt.
\]

\subsection{Ploskovni integral skalarnega polja}
Naj bo \(S\) ploskev z regularno paramaterizacijo \(\vec{r}: \triangle \subseteq \R^2 \to \R^3\).  Tedaj
\[
    dS = \norm{\vec{r}_u \times \vec{r}_v} dudv = \sqrt{\norm{\vec{r}_u}^2 \cdot \norm{\vec{r}_v}^2 - \scp{r_u}{r_v}^2} \, dudv = \sqrt{EG-F^2} \, dudv.
\]

Naj bo \(\mu: D \subseteq \R^3 \to \R\) skalarno polje. Definiramo 
\[
    \int_{S} \mu \, dS = \int_{\triangle} \mu(\vec{r}(u, v)) |\vec{r}_u \times \vec{r}_v| \, dudv.
\]

\begin{itemize}
    \item Če je \(S \subseteq \R \times \R \times \set{a}\), potem 
    \[
        \int_{S} \mu \, dS = \int_{\triangle} \mu(x,y,a) \, dxdy,
    \]
    kjer je \(\triangle\) projekcija v \(xy\)-ravnino. Podobno za poljubno permutacijo koordinat.
\end{itemize}

\subsection{Krivuljni integral vektorskega polja}
Naj bo \(K\) krivulja z regularno parametrizacijo \(\vec{r}: [a,b] \to \R^3, \ \vec{r} = (x, y, z)\). 

Naj bo \(\vec{f}: D \subseteq \R^3 \to \R^3\) vektorsko polje. Definiramo 
\[
    \int_{K} \vec{f} \cdot \, d\vec{r} = \int_{a}^{b} (\vec{f}(\vec{r}(t)) \cdot \dot{\vec{r}}(t)) \, dt.
\]

\begin{itemize}
    \item Parametrizacija krivulje določa tudi njeno orientacijo.
    \item \textbf{Cirkulacija} je integral vektorskega polja vzdolž sklenjene krivulje.
    \item Integral potencialnega polja je enak vrednosti potenciala v končni točki minus vrednosti potenciala v začetni točki.
\end{itemize}

\subsection{Ploskovni integral vektorskega polja}
Naj bo \(S\) ploskev z regularno paramaterizacijo \(\vec{r}: \triangle \subseteq \R^2 \to \R^3\).

Naj bo \(\vec{f}: D \subseteq \R^3 \to \R^3\) vektorsko polje. Definiramo
\[
    \int_{S} \vec{f} \cdot d \vec{S} = \int_{S} (\vec{f} \cdot \vec{n})\, dS,
\]
kjer je \(\vec{n}\) enotska normala. Orientacija ploskve je potem določna z smerjo normale. Za izračun uporabljamo formulo
\[
    \int_{S} \vec{f} \cdot d \vec{S} = \int_{\triangle} \vec{f}(\vec{r}(u, v)) \cdot (\vec{r}_u \times \vec{r}_v) \, dudv,
\]
pri čemer smer \(\vec{r}_u \times \vec{r}_v\) se mora ujemati s predpisano orientacijo.
\begin{itemize}
    \item To je tudi \textbf{pretok} polja \(\vec{f}\) skozi ploskev \(S\).
    \item Ravno ploskev lahko parametriziramo v obliki \(\vec{n} \cdot dS\).
\end{itemize}

\subsection{Integralski izreki}
Naj bo \(\vec{f}: D \subseteq \R^3 \to \R^3\) vektorsko polje.

\begin{izrek}[Gaussov izrek]
    Naj bo \(\Omega^\text{odp} \subseteq D\) omejena množica, katere rob \(\partial \Omega\) je sestavljen iz končnega števila gladkih ploskev. Rob \(\partial \Omega\) orientiramo tako, da izberimo zunanjo normalo. Tedaj
    \[
        \iint_{\partial \Omega} \vec{f} \cdot d\vec{S} = \iiint_{\Omega} \divr \vec{f} \, dV.
    \]
\end{izrek}

\begin{izrek}[Stokesov izrek]
    Naj bo \(\Sigma \subseteq D\) odsekoma gladka orientirana omejena ploskev, katere rob \(\partial \Sigma\) je sestavljen iz končnega števila gladkih krivulj. Rob \(\partial \Sigma\) orientiramo skladno s \(\Sigma\): Če hodimo po \(\partial S\) v smeri predpisane orientacije in je \(S\) na naši levi strani, glava določa normalo \(\vec{n}\). Tedaj
    \[
        \int_{\partial \Sigma} \vec{f} \cdot d\vec{r} = \iint_{S} \rot \vec{f} \cdot d \vec{S}.
    \]
\end{izrek}

\begin{izrek}[Greenova formula] 
    Naj bo \(D^\text{odp} \subseteq \R^2\) omejena množica, katere rob \(\partial D\) je sestavljen iz končnega števila gladkih krivulj. Rob \(\partial D\) orientiramo skladno s \(D\). 
    
    Naj bosta \(X, Y: D \to \R\) gladki funkciji. Tedaj
    \[
        \int_{\partial D} X \, dx + Y \, dy = \iint_D (Y_x - X_y) \, dxdy.
    \]
\end{izrek}

\subsection{Splošno}
\begin{itemize}
    \item Pri izračunu gradienta, divergence, rotorja itn.\ vektorji zapišemo v kartezičnih koordinatih.
    \item Pri parametrizaciji lahko si pomagamo s sferični, cilindrični itn.\ koordinati.
    \item Problematične točke lahko izoliramo s krogli.
\end{itemize}

