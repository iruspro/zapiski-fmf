\section{Vektorska analiza}
\subsection{Krivuljni integral skalarnega polja}
Naj bo \(K\) krivulja z regularno parametrizacijo \(\vec{r}: [a,b] \to \R^3, \ \vec{r} = (x, y, z)\). Tedaj 
\[
    ds = |\dot{\vec{r}}(t)| dt = \sqrt{\dot{x}^2 + \dot{y}^2 + \dot{z}^2}.
\]
Naj bo \(u: D \subseteq \R^3 \to \R\) skalarno polje. Definiramo 
\[
    \int_{K} u \, ds = \int_{a}^{b} u(\vec{r}(t)) |\dot{\vec{r}}(t)| dt.
\]

\subsection{Ploskovni integral skalarnega polja}
Naj bo \(S\) ploskev z regularno paramaterizacijo \(\vec{r}: \triangle = [a, b] \times [c,d] \to \R^3\).  Tedaj
\[
    dS = \sqrt{EG-F^2} \, dudv
\]
Naj bo \(\mu: D \subseteq \R^3 \to \R\) skalarno polje. Definiramo 
\[
    \int_{S} \mu \, dS = \int_{\triangle} \mu(\vec{r}(u, v)) |\vec{r}_u \times \vec{r}_v| \, dudv.
\]

\begin{itemize}
    \item Če je \(S \subseteq \R \times \R \times \set{a}\), potem 
    \[
        \int_{S} \mu \, dS = \int_{\triangle} \mu(x,y,a) \, dxdy,
    \]
    kjer je \(\triangle\) projekcija v \(xy\)-ravnino. Podobno za poljubno permutacijo koordinat.
\end{itemize}

\subsection{Krivuljni integral vektorskega polja}
Naj bo \(K\) krivulja z regularno parametrizacijo \(\vec{r}: [a,b] \to \R^3, \ \vec{r} = (x, y, z)\). 

Naj bo \(\vec{f}: D \subseteq \R^3 \to \R^3\) vektorsko polje. Definiramo 
\[
    \int_{K} \vec{f} \cdot \, d\vec{r} = \int_{a}^{b} (\vec{f}(\vec{r}(t)) \cdot \dot{\vec{r}}(t)) \, dt.
\]

\begin{itemize}
    \item Parametrizacija krivulje določa tudi njeno orientacijo.
    \item \textbf{Cirkulacija} je integral vektorskega polja vzdolž sklenjene krivulje.
\end{itemize}

\subsection{Ploskovni integral vektorskega polja}
Naj bo \(S\) ploskev z regularno paramaterizacijo \(\vec{r}: \triangle = [a, b] \times [c,d] \to \R^3\).

Naj bo \(\vec{f}: D \subseteq \R^3 \to \R^3\) vektorsko polje. Definiramo
\[
    \int_{S} \vec{f} \cdot d \vec{S} = \int_{S} (\vec{f} \cdot \vec{n})\, dS,
\]
kjer je \(\vec{n}\) enotska normala. Orientacija ploskve je potem določna z smerjo normale. Za izračun uporabljamo formulo
\[
    \int_{S} \vec{f} \cdot d \vec{S} = \int_{\triangle} \vec{f}(\vec{r}(u, v)) \cdot (\vec{r}_u \times \vec{r}_v) \, dudv,
\]
pri čemer smer \(\vec{r}_u \times \vec{r}_v\) se mora ujemati s predpisano orientacijo.
\begin{itemize}
    \item \todo{Naloga na strani 8: parametrizacija sfere}.
\end{itemize}

\subsection{Integralski izreki}
Naj bo \(\vec{f}: D \subseteq \R^3 \to \R^3\) vektorsko polje.
\begin{izrek}[Stokesov izrek]
    Naj bo \(\partial S\) sklenjena krivulja. Tedaj
    \[
        \int_{\partial S} \vec{f} \cdot d\vec{r} = \iint_{S} \rot \vec{f} \cdot d \vec{S},
    \]
    pri čemer orientacije za \(\partial S\) in \(S\) morajo biti usklajeni: Če hodimo po \(\partial S\) v smeri predpisane orientacije in je \(S\) na naši levi strani, glava določa normalo \(\vec{N}\).
\end{izrek}

\begin{izrek}[Gaussov izrek]
    Naj bo \(\partial D\) sklenjena ploskev. Tedaj
    \[
        \int_{\partial D} \vec{f} \cdot d\vec{S} = \int_{D} \divr \vec{f} \, dV
    \]
\end{izrek}

\subsection{Splošno}
\begin{itemize}
    \item Naj bo \(f(\vec{r}) = \frac{1}{|\vec{r} - \vec{a}|}\) skalarno polje. \(\divr (\grad f) = 0\). Torej je \(\grad f = - \frac{\vec{r} - \vec{a}}{|\vec{r} - \vec{a}|^3}\) solenoidalno polje.
    \item \(\rot (\vec{r} \times \vec{a}) = - 2 \vec{a}\); \(\rot (\vec{a} \times \vec{r}) = 2 \vec{a}\).
    \item Potencial polja dobimo z unijo členov po integraciji parcialnih odvodov.
    \item Pri parametrizaciji lahko si pomagamo s sferični, cilindrični itn.\ koordinati.
\end{itemize}

