\section{Fourierjeva analiza}

Naj bo funkcija \(f: [-\pi, \pi]\) nezvezna v končno mnogo točkah, kjer obstajata levi in desni odvod, vmes pa je med tema točkama odvedljiva. Tedaj 
\[
    \text{FV}(f)(x) = a_0 + \sum_{n=1}^{\infty}(a_n \cos(nx) + b_n \sin(nx)),
\]
kjer \(\ds a_0 = \frac{1}{2\pi} \int_{-\pi}^{\pi} f(x) \, dx\) ter \(\ds a_n = \frac{1}{\pi} \int_{-\pi}^{\pi} f(x)  \cos(nx) \, dx\) in \(\ds b_n = \frac{1}{\pi} \int_{-\pi}^{\pi} f(x)  \sin(nx) \, dx\).

Za vsak \(x \in [-\pi, \pi]\) Fourierjeva vrsta funkcije \(f\) konvergira proti 
\begin{itemize}
    \item \(f(x)\), če je \(f\) zvezna v \(x\) in 
    \item \(\frac{f(x-) + f(x+)}{2}\), če \(f\) ni zvezna v \(x\).
\end{itemize}

\ 

Naj bo \(f: [0, \pi] \to \R\) funkcija. Funkcijo \(f\) s predpisom \(f(x) = f(-x), x < 0\) lahko razširimo do sode funkcije ter s predpisom \(f(x) = -f(-x)\) do lihe. Tedaj 
\[
    \FV_{\cos}(f)(x) = \FV(f_{\text{soda}}(x)) \quad \text{ter} \quad \FV_{\sin}(f)(x) = \text{FV}(f_{\text{liha}}(x))
\]

\subsection{Nasveti}
\begin{itemize}
    \item Za izračun integralov z \(\sin\) in \(\cos\) lahko uporabljamo enakost 
    \[
        \cos(nx) + \sin(nx) = e^{inx}.
    \]
    \item Če je funkcija soda, potem \(\all{n > 1} b_n = 0\); če je funkcija liha, potem \(\all{n > 0} a_n = 0\).
    \item Če želimo sešteti številsko vrsto, najprej razvijemo funkcijo v vrsto, potem vzamemo vrednost v pravi točki.
    \item Vsak polinom v \(\sin\) in \(\cos\) ima končno Fourierjevo vrsto. Dobimo jo s pomočjo trigonometrije.
\end{itemize}