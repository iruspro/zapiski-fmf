\section{Vektorska analiza}

\todo

\subsection{Integralski izreki}
\paragraph{Motivacija} \todo \ (zvezek)
\begin{izrek}[Gauss-Ostrogradsky] 
    Recimo, da     
    \begin{itemize}
        \item \(\Omega \subseteq \R^3\) omejena odprta množica z robom, sestavljenum iz končnega števila odsekoma gladkih sklenjenih ploskev, orijentiranih z zunanjo normalo glede na \(\Omega\),
        \item \(\vec{R}: \overline{\Omega} \to \R^3\) vektorsko polje, \(\vec{R} \in C^1(\overline{\Omega})\).
    \end{itemize}
    Tedaj 
    \[\iint_{\partial \Omega} \vec{R} \, d \vec{S} = \iiint_\Omega \divr \vec{R} \, d V .\]
\end{izrek}

\begin{opomba}
    \(\ds \iint_{\partial \Omega} \vec{R} \, d \vec{S} = \iint_{\partial D} \vec{R} \cdot \vec{N} dS\), kjer je \(\vec{N}\) zunanja enotska normala.
\end{opomba}

\begin{opomba}[\(n = 2\)]
    Naj bo \(D\subseteq \R^2\) omejena odprta množica z robom, sestavljenum iz končnega števila odsekoma gladkih sklenjenih krivulj, orijentiranih pozitivno glede na \(D\). Naj bo \(\vec{R}: \overline{D} \to \R^2\) vektorsko polje, \(\vec{R} \in C^1(\overline{D})\). Tedaj je 
    \[\iint_D \divr \vec{R} \, d \vec{S} = \int_{\partial D} \vec{R} \cdot \vec{n} \, dS,\]
    kjer je \(\vec{n}\) zunanja enotska normala.
\end{opomba}

\begin{izrek}[Green, Greenova formula]
    Recimo, da
    \begin{itemize}
        \item \(D \subseteq \R^2\) omejena odprta množica z odsekoma gladkim robom, sestavljenim iz končnega števila odsekoma gladkih sklenjenih krivulj, orientiranih pozitivno glede na \(D\),
        \item \(X, Y \in C^1(\overline{D})\), kjer \(\vec{R} = (X, Y)\) vektorsko polje.
    \end{itemize}
    Tedaj 
    \[\int_{\partial D} X \, dx + Y \, dy = \iint_D (Y_x - X_y) \, dxdy = \int_{\partial D} \vec{R} \, d \vec{r} \]
\end{izrek}

\begin{izrek}[Stokes]
    Recimo, da
    \begin{itemize}
        \item \(\Sigma \subseteq \R^3\) omejena, orientirana, odsekoma gladka ploskev z odsekoma gladkim robom, sestavljenim iz končnega števila odsekoma gladkih sklenjenih krivulj, orientiranih skaldno z orientacijo \(\Sigma\),
        \item \(\vec{R}: \overline{\Sigma} \to \R^3\) vektorsko polje, \(\vec{R} \in C^1(\overline{\Sigma})\).
    \end{itemize}    
    Tedaj
    \[\int_{\partial \Sigma} \vec{R} \, d \vec{r} = \iint_\Sigma \rot \vec{R} \, d \vec{S}.\]
\end{izrek}

\begin{zgled}
    \todo \ (album 03.17.25)
\end{zgled}