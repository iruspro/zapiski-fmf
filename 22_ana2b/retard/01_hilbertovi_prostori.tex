\section{Hilbertovi prostori}
\subsection{Vektorski prostori s skalarnim produktom}
Naj bo \(X\) vektorski prostor nad \(\R\) (ali nad \(\C\)).
\begin{definicija}
    \textbf{Skalarni produkt} je preslikava \(\scp{\ }{}: X \times X \to \R \text{ (oz.\ \(\C\))}\) za katero velja:
    \begin{enumerate}
        \item \(\all{x \in X} \scp{x}{x} \geq 0\);
        \item \(\all{x \in X} \scp{x}{x} = 0 \liff x = 0\);
        \item \(\all{x, y \in X} \scp{x}{y} = \overline{\scp{y}{x}}\);
        \item \(\all{x,y,z \in X} \all{\lambda, \mu \in \R \text{ (oz. \(\C\))}} \scp{\lambda x + \mu y}{z} = \lambda \scp{x}{z} + \mu \scp{y}{z}\). 
    \end{enumerate}
\end{definicija}

\begin{opomba}
    1.\,-\,2.\ je \textbf{pozitivna definitnost} skalarnega produkta, 3.\ je \textbf{poševna simetričnost} (\textbf{simetričnost} nad \(\R\)), 4.\ je linearnost v prvem faktorju.
\end{opomba}

\begin{trditev}[Cauchy-Schwartzova neenakost]
    Naj bo \(\scp{\ }{}\) skalarni produkt na \(X\). Velja: 
    \[\all{x,y \in X} |\scp{x}{y}| \leq \sqrt{\scp{x}{x}} \cdot \sqrt{\scp{y}{y}} = \norm{x} \cdot \norm{y}.\]
\end{trditev}

\begin{proof}
    Nad \(\R\): Definiramo \(t \to \scp{x + ty}{x + ty} = f(t) \geq 0\).

    Nad \(\C\): Naj bo \(x, y \in X\). Obstaja \(\alpha \in \C, \ |\alpha| = 1\), da \(\scp{x}{y} = \alpha \cdot |\scp{x}{y}|\).
\end{proof}

\begin{definicija}
    \textbf{Norma} na vektorskem prostoru \(X\) je preslikava \(\norm{\ }: X \to \R\) za katero velja:
    \begin{enumerate}
        \item \(\all{x \in X} \norm{x} \geq 0\);
        \item \(\all{x \in X} \norm{x} = 0 \liff x = 0\);
        \item \(\all{\lambda \in \R \text{ (oz. \(\C\))}} \norm{\lambda x} = |\lambda| \cdot \norm{x}\);
        \item Trikotniška neenakost: \(\all{x, y \in X} \norm{x + y} \leq \norm{x} + \norm{y}\).
    \end{enumerate}
\end{definicija}

\begin{trditev}
    Naj bo \((X, \scp{\ }{})\) vektorski prostor s skalarnim produktom. Potem je \((X, \norm{\ })\), kjer je \(\all{x \in X} \norm{x} = \sqrt{\scp{x}{x}}\), vektorski prostor z normo.
\end{trditev}

\begin{proof}
    Preverimo lastnosti. Za trikotniško neenakost uporabimo CS neenakost.
\end{proof}

\begin{trditev}
    Naj bo \((X, \norm{\ })\) vektorski prostor s normo. Potem je \((X, d)\), kjer je metrika definirana s predpisom \(\all{x, y \in X}d(x,y) = \norm{x - y}\), metrični prostor.
\end{trditev}

\begin{proof}
    Preverimo lastnosti. 
\end{proof}

\subsection{Hilbertovi prostori}
\begin{definicija}
    \textbf{Hilbertov prostor} je vektorski prostor \(X\) s skalarnim produktom \(\scp{\ }{}\), ki je v metriki, porojeni iz skalarnega produkta, poln metrični prostor.
\end{definicija}

\begin{opomba}
    \((X, \scp{\ }{}) \leadsto (X, \norm{\ }) \leadsto (X, d)\), kjer je \(\all{x, y \in X}d(x,y) = \norm{x - y}\).
\end{opomba}

\begin{opomba}
    \textbf{Banachov prostor} je vektorski prostor \(X\) z normo \(\norm{\ }\), ki je v metriki, porojeni iz norme, poln metrični prostor.
\end{opomba}

\newpage
\begin{zgled} \
\begin{enumerate}
    \item Naj bo \(X = \R^n\). Definiramo skalarni produkt. Naj bo \(x, y \in \R^n, \ x = (x_1 \ldots, x_n)\) in \(y = (y_1, \ldots, y_n)\). \textbf{Standardni skalarni produkt} je \[x \cdot y = \sum_{k=1}^{n} x_ky_k.\]
    Ta skalarni produkt nam da normo \[\norm{x} = \sqrt{\sum_{k=1}^{n} x_k^2},\]
    ki porodi metriko \[d_2(x,y) = \sqrt{\sum_{k=1}^{n} (x_k - y_k)^2}.\]
    Vemo, da je \((\R^n, d_2)\) poln metrični prostor. Torej \((\R^n, \cdot)\) Hilbertov prostor.
    \item Na \(\R^n\) lahko definiramo tudi druge norme, npr.
    \begin{itemize}
        \item \(\norm{x}_\infty = \max \set{|x_1|, \ldots, |x_n|}\);
        \item \(\norm{x}_1 = |x_1| + \ldots + |x_n|\). 
    \end{itemize}
    Te dve normi ne prideta iz skalarnega produkta, ker za njih ne velja paralelogramsko pravilo.
    \((\R^n, \norm{x}_\infty)\) in \((\R^n, \norm{x}_1)\) sta Banachova prostora.
    \item Naj bo \(X = \C^n\). Definiramo skalarni produkt. Naj bo \(z, w \in \C^n, \ z = (z_1 \ldots, z_n)\) in \(w = (w_1, \ldots, w_n)\). \textbf{Standardni skalarni produkt} je \[z \cdot w = \sum_{k=1}^{n} z_k\overline{w_k}.\]
    Ta skalarni produkt nam da normo \[\norm{z} = \sqrt{\sum_{k=1}^{n} |z_k|^2},\]
    ki porodi metriko \[d_2(z,w) = \sqrt{\sum_{k=1}^{n} |z_k - w_k|^2}.\]
    Vemo, da je \((\C^n, d_2)\) poln metrični prostor. Torej \((\C^n, \cdot)\) Hilbertov prostor.
\end{enumerate}
\end{zgled}

\subsection{Prostor \(L^2([a, b])\)}
\begin{opomba}
    Števili \(a, b\) sta lahko končni ali \(\pm\infty\).
\end{opomba}
\begin{trditev}
    Naj bo \(C([a,b])\) vektorski prostor nad \(\R\). Potem je s predpisom \[\all{f, g \in C([a,b])} \scp{f}{g} = \int_{a}^{b} f(x) g(x) \, dx\]
    definiran skalarni produkt na \(\cf{}\).
\end{trditev}

\begin{proof}
    Preverimo lastnosti.
\end{proof}

\begin{trditev}
    \((\cf{}, \scp{\ }{})\) ni Hilbertov prostor.
\end{trditev}

\begin{proof}
    Definiramo \(f_n(x) = \begin{cases}
        1; &\frac{1}{n} \leq x \leq 1 \\
        nx; &-\frac{1}{n} < x < \frac{1}{n} \\
        -1; &-1 \leq x \leq -\frac{1}{n}
    \end{cases}.\) Pokažemo, da je \((f_n)_n\) Cauchyjevo zaporedje v \(\cf{}\), ki nima limite.
\end{proof}

\begin{zgled}
    Vzemimo prostor \(((0,1), d_2)\). Z dodajanjem limitnih točk \(\set{-1, 1}\) ta prostor postane poln.
\end{zgled}

\begin{definicija}
    Naj bo \((M, d)\) metrični prostor. Pravimo, da lahko \textbf{napolnimo} prostor \(M\), če obstaja prostor \((\overline{M}, \overline{d})\), za kateri velja:
    \begin{enumerate}
        \item \((\overline{M}, \overline{d})\) je poln metrični prostor;
        \item \(M \subseteq \overline{M}\);
        \item \(\overline{d}|_{M \times M} = d\);
        \item \(M\) je gost v \(\overline{M}\), tj.\ \(\Cl M = \overline{M}\).
    \end{enumerate}
    Prostoru \(\overline{M}\) rečemo \textbf{napolnitev} prostora \(M\).
\end{definicija}

\begin{opomba}
    Ideja: \(\overline{M}\) je prostor vseh limit Cauchyjevih zaporedij v \(M\) (+ kvocient).
\end{opomba}

\begin{opomba}
    Označili smo z \(L^1(A) = \setb{f: A \to \R}{\int_{A} |f| \, dx \ \text{obstaja}, \ f \ \text{zvezna s.p.}} /_\sim\) prostor vseh absolutno integrabilnih funkcij, kjer je  \(\all{f, g \in L^1} f \sim g \liff f = g \ \text{s.p.}\)
\end{opomba}

Vpeljemo zdaj s kvadratom integrabilne funkcije: 
\begin{definicija}
    Prostor \(L^2([a,b])\) je 
    \[L^2([a, b]) = \setb{f: [a,b] \to \R}{\int_{a}^{b} f^2(x) \, dx \ \text{obstaja}, \ f \ \text{zvezna s.p.}}/_\sim,\]
    kjer je \(\all{f, g \in L^2} f \sim g \liff f = g \ \text{s.p.}\)
\end{definicija}

V tem prostoru gotovo so 
\begin{itemize}
    \item Zvezne funkcije: \(C([a,b]) \subseteq L^2([a,b])\);
    \item Odsekoma zvezni funkciji;
    \item \(f(x) = \frac{1}{\sqrt[4]{x-a}}\) itd.
\end{itemize}

\paragraph{Cilj} Želimo posplošiti prostor \((\R, \cdot)\).

Naj bo \(f, g \in L^2\), potem \(|f \cdot g| \leq \frac{|f|^2 + |g|^2}{2} \lthen f \cdot g \in L^1([a, b])\). Torej lahko definiramo 
\[\scp{f}{g} = \int_{a}^{b}f(x)g(x) \, dx.\]

\begin{trditev}
    \(L^2([a,b])\) je vektorski prostor nad \(\R\).
\end{trditev}

\begin{proof}
    Preverimo lastnosti.
\end{proof}

Torej \(L^2([a, b])\) je vektorski prostor nad \(\R\) s skalarnim produktom \[\scp{f}{g} = \int_{a}^{b}f(x)g(x) \, dx.\]

Očitno, da je \(C([a,b]) \subseteq L^2([a, b])\).

\begin{izrek}
    \(L^2([a,b])\) je Hilbertov in \(L^2([a,b])\) je napolnitev \(C([a,b])\).
\end{izrek}

\begin{opomba} Prostor \(C([a, b])\) je gost v prostoru \(L^2([a, b])\), tj.
    \[\all{f \in L^2([a,b])} \some{f_n \in C([a,b])} \lim_{n \to \infty}  f_n = f,\]
    kjer
    \[
        \lim_{n \to \infty}  f_n = f \liff \lim_{n \to \infty} \norm{f_n - f} = 0 \liff \lim_{n \to \infty} \sqrt{\int_{a}^{b}(f_n(x) - f(x))^2} \, dx = 0.
    \]
\end{opomba}

\begin{opomba}
    Nad \(\C\): \(f = u + iv, \ u, v: [a, b] \to \R\). Potem 
    \[\int_{a}^{b}f(x) \, dx = \int_{a}^{b}u(x) \, dx + i\int_{a}^{b}v(x) \, dx \]
    in 
    \[\scp{f}{g} = \int_{a}^{b}f(x) \overline{g(x)} \, dx.\]
\end{opomba}

\begin{zgled}
    Vzemimo \([0, 1]\). Definiramo \(f_n(x) = \begin{cases}
        \sqrt{n}; &0 < x \leq \frac{1}{n} \\ 0; &\text{sicer}
    \end{cases}.\)

    Čemu je enaka \(\lim_{n \to \infty} f_n(x)\) za vse \(x \in [0, 1]\) (po točkah)? Ali je \(\lim_{n \to \infty} f_n(x) = 0\) v \(L^2([0,1])\)?
\end{zgled}

\begin{zgled}
    Definiramo zaporedje \(f_n: [0, 1] \to \R\) po pravilu: začnemo z \(f_1 \equiv 1\). Nato nadaljujemo
    \[
    f_2 = \begin{cases}
            1; &x \in [0, \frac{1}{2}] \\ 0, &\text{sicer}
        \end{cases}, \ 
    f_3 = \begin{cases}
            1; &x \in [\frac{1}{2}, 1] \\ 0, &\text{sicer}
        \end{cases}, \ 
    f_4 = \begin{cases}
            1; &x \in [0, \frac{1}{3}] \\ 0, &\text{sicer}
        \end{cases}, \
    f_5 = \begin{cases}
            1; &x \in [\frac{1}{3}, \frac{2}{3}] \\ 0, &\text{sicer}
        \end{cases}
    \]
    in tako naprej. Ali obstaja limita po točkah? Ali obstaja limita v \(L^2\) smislu?
\end{zgled}

\subsection{Ortogonalnost}
\begin{definicija}
    Naj bo \((X, \scp{\ }{})\) vektorski prostor s skalarnim produktom, \(A \subseteq X\), \(A \neq \emptyset\). Naj bosta \(x, y \in X\).
    \begin{itemize}
        \item \(x\) je \textbf{pravokoten} na \(y\), če \(\scp{x}{y} = 0\), tj.\ \(x \perp y \liff \scp{x}{y} = 0\).
        \item \textbf{Ortogonalni komplement} množice \(A\) je \(A^\perp = \setb{x \in X}{\all{a \in A} x \perp a}\).
    \end{itemize}
\end{definicija}

\begin{trditev}
    \(A^\perp\) je vektorski podprostor v \(X\).
\end{trditev}

\begin{proof}
    Preverimo homogenost in linearnost.
\end{proof}

\begin{opomba}
    \(A \subseteq (A^\perp)^\perp\).
\end{opomba}

\begin{trditev}
    Naj bo \((X, \scp{\ }{})\) vektorski prostor s skalarnim produktom, \(v \in X\). Definiramo \(f: X \to \R, \ f(x) = \scp{x}{v}\). Potem \(f\) je zvezna na \(X\).
\end{trditev}

\begin{proof}
    Pokažemo, da je \(f\) Lipshitzeva.
\end{proof}

\begin{posledica}
    \(A^\perp\) je zaprt vektorski podprostor.
\end{posledica}

\begin{proof}
    Pokažemo, da je limita vsakega zaporedja v \(A^\perp\) tudi leži v \(A^\perp\).
\end{proof}

\begin{opomba}
    \(C([a, b]) \subseteq L^2([a,b])\) ni zaprt podprostor.
\end{opomba}

\begin{opomba}
    Če je \((X, \scp{\ }{})\) Hilbertov in \(A \subseteq X\) zaprt podprostor, potem \[(A^\perp)^\perp = A.\]
\end{opomba}

\begin{trditev}[Pitagorjev izrek]
    \label{trd:pitagor}
    Naj bo \((X, \scp{\ }{})\) vektorski prostor s skalarnim produktom. Naj bodo \(x_1, \ldots, x_n \in X\) taki, da \(\all{i, j \in [n]} i \neq j \lthen x_j \perp x_j\). Tedaj 
    \[\norm{x_1 + \ldots + x_n}^2 = \norm{x_1}^2 + \ldots + \norm{x_n}^2.\]
\end{trditev}

\begin{proof}
    Izračunamo normo po definiciji.
\end{proof}

\begin{definicija}
    Naj bo \((X, \scp{\ }{})\) vektorski prostor s skalarnim produktom in \(Y \leq X\) podprostor X. Naj bo \(x \in X\). \textbf{Pravokotna projekcija} vektorja \(x\) na podprostor \(Y\) (če obstaja) je tak vektor \(P_Y(x) \in Y\), da je 
    \[x - P_Y(x) \in Y^\perp.\]
\end{definicija}

\begin{trditev}
    Če je pravokotna projekcija \(x\) na \(Y\) obstaja, je enolično določena. Če obstaja, je to najboljša aproksimacija vektorja \(x\) z vektorji iz \(Y\), tj.
    \[\norm{x - P_y(x)} = \min_{w \in Y} \norm{x - w}.\]
\end{trditev}

\begin{proof}
    Enoličnost: Običajen način.

    Aproksimacija: Definicija minimuma in Pitagorjev izrek \ref{trd:pitagor}.
\end{proof}

\begin{zgled}
    Naj bosta \(Y = C([a,b])\) in \(X = L^2([a, b])\). Če si izberimo \(f \in X \setminus Y\), potem \(f\) nima najboljše aproksimacije z zveznimi funkcijami, saj, ker je \(\Cl (C([a,b])) = L^2([a, b])\), bi veljalo \(\norm{f - P_{C([a,b])}(f)} = 0\) in posledično \(f \in C([a, b])\).
\end{zgled}

\begin{opomba}
    \ 
    \begin{enumerate}
        \item \(P_Y^2 = P_Y\).
        \item \(\norm{x} \geq \norm{P_Y(x)}\), saj \(x = \underbrace{x - P_Y(x)}_{Y^\perp} + \underbrace{P_Y(x)}_Y\).
        \item Če je \(P_Y\) definiran na \(X\), potem je linearen in zvezen.
        \begin{proof}
            Definicija in enoličnost projekcije.
        \end{proof}
        \item Če je \(P_Y\) definiran na \(X\), je \(Y\) zaprt podprostor.
        \begin{proof}
            Vzamemo konvergentno zaporedje v \(Y\) in upoštevamo zveznost \(P_Y\).
        \end{proof}
        \newpage
        \item Če ima \(x\) pravokotno projekcijo na \(Y\), ima tudi pravokotno projekcijo na \(Y^\perp\).
        \begin{proof}
            Vzamemo \(x - P_Y(x)\).
        \end{proof}
    \end{enumerate}
\end{opomba}

\begin{trditev}
    \label{trd:projekcija-koncnoraz}
    Naj bo \(Y \leq X\) končno dimenzionalen podprostor z ON bazo \(\set{e_1, \ldots, e_n}\), tj.\ \(\scp{e_i}{e_j} = \delta_{ij}\). Naj bo \(x \in X\). Tedaj je \[P_Y(x) = \sum_{j=1}^{n} \scp{x}{e_j}e_j.\]
\end{trditev}

\begin{proof}
    Definicija projekcije.
\end{proof}

\begin{opomba}
    Vsak končno dimenzionalni podprostor ima pravokotno projekcijo definirano na \(X\) in tudi vsi tisti podprostori končne kodimenzije.
\end{opomba}

\subsection{Ortogonalni sistem}
\begin{definicija}
    Naj bo \((X, \scp{\ }{})\) vektorski prostor s skalarnim produktom. 
    \begin{itemize}
        \item Sistem vektorjev \((e_j)_{j=1}^\infty\) je \textbf{ortogonalen sistem (OS)}, če \[\all{i,j \in \N} i \neq j \lthen \scp{e_i}{e_j} = 0.\]
        \item Tak sistem je \textbf{ortonormiran (ONS)}, če \[\all{i,j \in \N} \scp{e_i}{e_j} = \delta_{ij}.\]
    \end{itemize}   
\end{definicija}

\begin{trditev}[Besselova neenakost]
    \label{trd:besselova-neenakost}
    Naj bo 
    \begin{itemize}
        \item \(X\) vektorski prostor s skalarnim produktom, \(x \in X\);
        \item \(\sys\) ONS.
    \end{itemize}
    Tedaj 
    \[\sum_{j=1}^{\infty}|\scp{x}{e_j}|^2 \leq \norm{x}^2.\]  
\end{trditev}

\begin{proof}
    Definiramo \(Y_n = \lin(\set{e_1, \ldots, e_n})\). Uporabimo formulo za pravokotno projekcijo na končnorazsežen prostor \ref{trd:projekcija-koncnoraz} ter Pitagorjev izrek \ref{trd:pitagor}.
\end{proof}

\begin{posledica}
    \(\lim_{j \to \infty} \scp{x}{e_j} = 0\).
\end{posledica}

\begin{opomba} \
    \begin{itemize}
        \item Absolutno vrednost potrebujemo, če gledamo prostor nad \(\C\).
        \item \((\scp{x}{e_j})_{j = 1}^\infty\) so \textbf{Fourierjevi koeficienti} \(x\) po ONS \(\sys\).
    \end{itemize}
\end{opomba}

\begin{trditev}
    \label{trd:obstoj-x}
    Naj bo 
    \begin{itemize}
        \item \(X\) vektorski prostor s skalarnim produktom;
        \item \(\sys\) ONS.
        \item \((c_j)_{j=1}^\infty\) zaporedje števil (bodisi \(\R\) bodisi \(\C\));
        \item \(\sum_{j=1}^{\infty} |c_j|^2 < \infty\);
    \end{itemize}
    Tedaj obstaja \(x \in X\), za katerega velja 
    \[\all{j \in \N} c_j = \scp{x}{e_j}.\]
    Velja tudi:
    \[x = \sum_{j = 1}^{\infty}c_je_j = \lim_{N \to \infty} \sum_{j=1}^{N}c_je_j.\]
\end{trditev}

\begin{proof}
    Trdimo, da je \(\left(\sum_{j=1}^{N}c_je_j\right)_N\) Cauchyjevo zaporedje.
\end{proof}

\begin{opomba}
    Naj bo \(X\) Hilbertov prostor ter \(\sys\) ONS. Vzemimo \(x \in X\). 
    
    Iz Besselovi neenakosti \ref{trd:besselova-neenakost} sledi, da za zaporedje \(c_j = \scp{x}{e_j}\) velja, da 
    \[\sum_{j=1}^{\infty} |c_j|^2 < \infty.\]
    Torej po trditvi \ref{trd:obstoj-x} sledi, da obstaja \(\wt{x}\), za kateri velja
    \[
    \wt{x} = \sum_{j=1}^{\infty} \scp{x}{e_j} e_j.
    \]
    Ali je \(\wt{x} = x\)?
\end{opomba}

\begin{definicija}
    Naj bo \(X\) Hilbertov prostor. ONS \(\sys\) je \textbf{kompleten} (KONS) ali \textbf{poln}, če 
    \[\all{x \in X} x = \sum_{j=1}^{\infty} \scp{x}{e_j}e_j.\]
\end{definicija}

\begin{izrek}
    \label{izr:parsevalova-enakost}
    Naj bo 
    \begin{itemize}
        \item \(X\) Hilbertov prostor;
        \item \(\sys\) ONS.
    \end{itemize}
    NTSE 
    \begin{enumerate}
        \item \(\sys\) je KONS;
        \item \(\all{x, y \in X} \scp{x}{y} = \sum_{j=1}^{\infty} \scp{x}{e_j} \scp{e_j}{y}\);
        \item Parsevalova enakost: \(\all{x \in X} \norm{x}^2 = \sum_{j=1}^{\infty} |\scp{x}{e_j}|^2\);
        \item ONS \(\sys\) ni vsebovan v nobenem strogo večjem ONS;
        \item Edini vektor, ki je pravokoten na vse vektorji \(e_j\), je vektor \(0\);
        \item Končne linearne kombinacije vektorjev \(e_j\) so goste v \(X\).
    \end{enumerate}
\end{izrek}

\begin{proof}
    Dokažemo \((1) \lthen (2) \lthen (3) \lthen (4) \lthen (5)\) in \((1) \lthen (6) \lthen (5)\).
    
    \((5) \lthen (1)\): Uporabimo trditev \ref{trd:obstoj-x} ter oglejmo razliko \(x - \wt{x}\).
\end{proof}

\begin{zgled}[Modelni Hilbertov prostor]
    Definiramo 
    \[l^2 = \setb{(a_j)_j}{a_j \in \R, \sum_{j=1}^{\infty} |a_j|^2 < \infty}.\]
    Vpeljemo skalarni produkt s predpisom 
    \[\scp{(a_j)_j}{(b_j)_j} = \sum_{j=1}^{\infty}a_jb_j \quad \text{(oz.}\, \sum_{j=1}^{\infty}a_j\overline{b_j} \ \text{nad}\ \C).\]
    Tedaj velja 
    \[\norm{(a_i)_i}_2^2 = \sum_{j=1}^{\infty}|a_j|^2.\]
    Naj bo zdaj \(X\) Hilbertov prostor, \(x \in X\) ter \(\sys\) KONS. Tedaj 
    \[(\scp{x}{e_j})_j \in l^2.\]
\end{zgled}

\newpage
\subsection{Prostor \(L^2([-\pi, \pi])\)}
\subsubsection{Fourierjeva vrsta}
Na prostoru \(L^2(-\pi, \pi)\) definiramo sistem 
\[
    \set{\frac{1}{\sqrt{2\pi}}, \frac{1}{\sqrt{\pi}} \cos x, \frac{1}{\sqrt{\pi}} \sin x, \frac{1}{\sqrt{\pi}} \cos 2x, \frac{1}{\sqrt{\pi}} \sin 2x, \ldots, \frac{1}{\sqrt{\pi}} \cos nx, \frac{1}{\sqrt{\pi}} \sin nx, \ldots}.
\]
Če delamo nad \(\C\), dobimo:
\[
    \setb{\frac{1}{\sqrt{2\pi}} e^{inx}}{n \in \N}.
\]
Trdimo, da je to ONS. Kasneje bomo tudi dokazali, da je to KONS.

\begin{opomba}
    V tem kontekstu vsako funkcijo \(f: [-\pi, \pi] \to \R\) vidimo kot zožitev periodične funkcije s periodo \(2\pi\) na interval \([\pi, \pi]\).
    Vsako tako periodično funkcijo želimo zapisati kot "`vsoto osnovnih nihanj"'.
\end{opomba}

Vpeljemo klasične Fourierjevi koeficienti
\begin{align*}
    a_n &= \frac{1}{\pi} \int_{-\pi}^{\pi} f(x) \cos (nx) \, dx, \ n = 1, 2, \ldots \\
    b_n &= \frac{1}{\pi} \int_{-\pi}^{\pi} f(x) \sin (nx) \, dx, \ n = 1, 2, \ldots
\end{align*}
Pripadajoča klasična Fourierjeva vrsta je 
\[
    f(x) = \frac{a_0}{2} + \sum_{n=1}^{\infty} a_n \cos(nx) + b_n \sin(nx).
\]

Torej za \(n = 1, 2, \ldots\) velja

\begin{align*}
    a_n &= \frac{1}{\pi} \int_{-\pi}^{\pi}f(x) \cos(nx) \, dx = \frac{1}{\sqrt{\pi}} \scp{f}{\frac{1}{\sqrt{\pi}} \cos(nx)} \\
    b_n &= \frac{1}{\pi} \int_{-\pi}^{\pi}f(x) \sin(nx) \, dx = \frac{1}{\sqrt{\pi}} \scp{f}{\frac{1}{\sqrt{\pi}} \sin(nx)}
\end{align*}
ter
\[
a_0 = \frac{1}{\pi} \int_{-\pi}^{\pi} f(x) \, dx = \frac{\sqrt{2}}{\pi} \scp{f}{\frac{1}{\sqrt{2\pi}}}.
\]

Iz Besselove neenakosti \ref{trd:besselova-neenakost} sledi
\begin{trditev}[Riemann-Lebesgueva lema]
    Naj bo \(f \in L^2(-\pi, \pi)\). Tedaj 
    \[
        \lim_{n \to \infty} a_n = 0 \quad \text{ter} \quad \lim_{n \to \infty} b_n = 0.
    \]
\end{trditev}

Če pa vemo, da imamo KONS, dobimo Parsevalovo enakost \ref{izr:parsevalova-enakost}
\begin{trditev}[Parsevalova enakost]
    Naj bo \(f \in L^2(-\pi, \pi)\). Tedaj 
    \[
        \frac{1}{\pi} \int_{-\pi}^{\pi}|f(x)|^2 \, dx = \frac{|a_0|^2}{2} + \sum_{n=1}^{\infty}(|a_n|^2+|b_n|^2).
    \]
\end{trditev}

\begin{zgled}
    \label{zgl:nič-ena}
    Definiramo funkcijo \(f: [-\pi, \pi] \to \R\) s predpisom
    \[
        f(x) = \begin{cases}
            1; &0 \leq x \leq \pi \\
            0; &-\pi < x < 0.
        \end{cases}
    \]
    Razvij funkcijo \(f\) v Fourierjevo vrsto ter s pomočjo Parsevalove enakosti določi vsoto 
    \[
        S = 1 + \frac{1}{2^2} + \frac{1}{3^2} + \frac{1}{4^2} + \ldots
    \]
\end{zgled}

\begin{lema}
    Naj bo \(f: \R \to \R\) odsekoma zvezna periodična funkcija s periodo \(2\pi\). Tedaj je 
    \[
         \all{a \in \R} \int_{a}^{a+p} f(x) \, dx = \int_{0}^{p}  f(x) \, dx.
    \]
\end{lema}

\begin{proof}
    \(\int_{a}^{a+p} f(x) \, dx = \int_{a}^{p} f(x) \, dx + \int_{p}^{a+p} f(x) \, dx\)
\end{proof}

\begin{lema}[Dirichletovo jedro]
    \[\frac{1}{2} + \sum_{k=1}^{\infty} \cos(kx) = \frac{1}{2} \frac{\sin((n+1/2)x)}{\sin(x/2)} =: D_n(x).\]
\end{lema}

\begin{proof}
    Kompleksna eksponenta.
\end{proof}

\begin{lema}
    Za Dirichletovo jedro velja
    \begin{enumerate}
        \item \(\ds \int_{-\pi}^{\pi} D_n(x) \, dx = \pi\).
        \item \(D_n(x)\) je gladka soda funkcija s periodo \(2\pi\).
        \item \(D_n(x) = \frac{1}{2} \left(\sin(nx)\frac{\cos(x/2)}{\sin(x/2)}  + \cos(nx)\right)\).
    \end{enumerate}
\end{lema}

\begin{proof}
    Račun.
\end{proof}

\begin{opomba}
    Sode/lihe funkcije. \todo{}
\end{opomba}

\begin{izrek}[Fourierjeva vrsta]
    Naj bo
    \begin{itemize}
        \item funkcija \(f\) odsekoma zvezna in odsekoma odvedljiva periodična funkcijo s periodo \(2\pi\);
        \item funkcija \(f\) na vsakem intervalu dolžine \(2\pi\) ima največ končno mnogo točk nezveznosti in v vsaki točki obstaja leva in desna limita;
        \item funkcija \(f\) ima v vsaki točki levi in desni odvod.
    \end{itemize}
    Tedaj za vsak \(x \in \R\) velja 
    \[
        \frac{f(x+) + f(x-)}{2} = \frac{a_0}{2} + \sum_{n=1}^{\infty} a_n \cos(nx) + b_n \sin(nx).
    \]
\end{izrek}

\begin{proof}
    \todo{}
\end{proof}

\begin{zgled}
    Vzemimo funkcijo iz zgleda \ref{zgl:nič-ena}. Kaj velja v točki \(x =0\) ter v točki \(x = \frac{\pi}{2}\)? Izračunaj \[
        \sum_{k=0}^{\infty}(-1)^{k}\frac{1}{2k+1}.
    \]
\end{zgled}

\begin{zgled}
    Naj bo \(f(x) = x\) na \([-\pi, \pi]\) periodična funkcija s periodo \(2\pi\). Kaj velja v točkah \(0, \pi\) in \(\frac{\pi}{2}\)? Kaj pravi Parsevalova enakost?
\end{zgled}

\begin{zgled}
    Razvij funkcijo \(f: [-\pi, \pi] \to \R, \ f(x) = \pi^2 - x^2\) v Fourierjevo vrsto. Ter seštej vsoti 
    \[
        S_1 = \sum_{n=1}^{\infty} (-1)^{n+1} \frac{1}{n^2} \quad \text{ter} \quad S_2 = \sum_{n=1}^{\infty} \frac{1}{n^2}.
    \]
\end{zgled}

\subsubsection{Cesarjeve delne vsote}
Definiramo \(n\)-to delno vsoto
\[
    S_n(x_0) = \frac{1}{\pi} \int_{-\pi}^{\pi} f(x_0 + y)D_n(y) \, dy
\]
ter \textbf{Cesarjeve delne vsote}
\[
    G_n(x) = \frac{S_0(x) + S_1(x) + \ldots + S_{n-1}(x)}{n}.
\]

Velja: \(G_n(x)\) konvergira proti funkcije \(f\). Če je \(f\) zvezna, potem konvergenca enakomerna. 

Definiramo \textbf{Fourierjevo jedro}
\[
    F_N(x) = \frac{1}{N}\sum_{n=0}^{N-1} D_n(x)
\]

\begin{trditev}[Lastnosti Fourierjeva jedra]
    \ 
    \begin{enumerate}
        \item \(F_N(x) = \frac{1}{2N} \left(\frac{\sin(NX/2)}{\sin(x/2)}\right)^2\).
        \item \(F_N\) je soda funkcija.
        \item \(\all{x \in D_f} F_N(x) \leq 0\).
        \item \(\frac{1}{\pi} \int_{-\pi}^{\pi} F_N(x) \, dx = 1\).
        \item Za vsak \(0 < a < \pi\) limita \(\lim_{N \to \infty} F_N(x) = 0\) je enakomerna na \(a \leq |x| \leq \pi\).
    \end{enumerate}
\end{trditev}

\begin{proof}
    \todo{}
\end{proof}

\begin{izrek}
    Naj bo \(f\) \(2\pi\) periodična zvezna funkcija. Tedaj Cesarjeve delne vsote konvergirajo k \(f\) enakomerno na \([-\pi, \pi]\) (oz.\ na \(\R\)).
\end{izrek}

\begin{proof}
    \todo{}
\end{proof}

\begin{izrek}
    Sistem
    \[
        \set{\frac{1}{\sqrt{2\pi}}, \frac{1}{\sqrt{\pi}} \cos x, \frac{1}{\sqrt{\pi}} \sin x, \frac{1}{\sqrt{\pi}} \cos 2x, \frac{1}{\sqrt{\pi}} \sin 2x, \ldots, \frac{1}{\sqrt{\pi}} \cos nx, \frac{1}{\sqrt{\pi}} \sin nx, \ldots}.
    \]
    je KONS v \(L^2(-\pi, \pi)\).
\end{izrek}

\begin{opomba}
    O trigonometričnih polinomih. \todo{}
\end{opomba}

\begin{izrek}[Weierstrassov izrek]
    Naj bo \(f \in C([a,b])\). Tedaj za vsak \(\epsilon > 0\) obstaja polinom \(p\), da velja 
    \[
        \norm{f - p}_\infty < \epsilon.
    \]
\end{izrek}

\begin{proof}
    \todo{}
\end{proof}


