\section{Vektorska analiza}
\begin{enumerate}
    \item Skalarno in vektorsko polje    
    \begin{itemize}
        \item \textbf{Definicija.} Skalarno polje. Vektorsko polje.
        \item \textbf{Definicija.} Pozitivno/negativno orientirana ONB.
        \item \textbf{Opomba.} Prehod med bazi.
    \end{itemize}

    \item Smerni odvod skalarnega polja
    
    Naj bo \(u: D \subseteq \R^3 \to \R\) skalarno polje.
    \begin{itemize}
        \item \textbf{Definicija.} Smerni odvod skalarnega polja \(u\).
        \item \textbf{Opomba.} Kaj meri smerni odvod? Kaj so smerni odvodi v smeri baznih vektorjev?
        \item \textbf{Opomba.} Kako izračunamo smerni odvod skalarnega polja \(u\) v točki \(p_0\), če je \(u\) diferenciabilno v \(p_0\)? Kaj to pomeni v kartezičnih koordinatih?
        \item \textbf{Trditev.} V kakšni smeri se najhitreje narašča skalarno polje? V kakšni smeri pa najhitreje pada?
        \item \textbf{Definicija.} Gradient skalarnega polja. 
        \item \textbf{Opomba.} Ali je gradient odvisen od izbire baze? Kaj smo priredili skalarnemu polju?

        \item \textbf{Definicija.} Operator nabla.
        \item \textbf{Opomba.} Kako se z operatorjem nabla izraža gradient skalarnega polja? 
        \item \textbf{Definicija.} Divergenca vektorskega polja.
        \item \textbf{Opomba.} Ali je divergenca odvisna od izbire baze?
        \item \textbf{Definicija.} Rotor vektorskega polja.
        \item \textbf{Opomba.} Odvisnost rotorja od izbire baze.
        \item \textbf{Trditev.} Rotor gradienta. Divergenca rotorja.
        \item \textbf{Opomba.} Ali je divergenca gradienta enaka nič?
        \item \textbf{Definicija.} Laplaceov operator. Harmonična funkcija.
        \item \textbf{Definicija.} Potencialno polje. Potencial. Irotacionalno (nevrtinčno) polje. Solenoidalno polje.
        \item \textbf{Opomba.} Zadosten pogoj, da je polje irotacionalno, Zadosten pogoj, da je polje solenoidalno. Kaj pa obrat?
        \item \textbf{Zgled.} Izračunaj rotor polja \(\vec{f}(x,y,z) = \left(-\frac{y}{x^2 + y^2}, \frac{x}{x^2+y^2}, 0\right)\). Ali je polje potencialno?
        \item \textbf{Definicija.} Zvezdasto območje.
        \item \textcolor{green}{(*)} \textbf{Izrek.} Kdaj je nevrtinčno polje potencialno? Kdaj je polje rotor nekega drugega polja?
        \item \textbf{Zgled.} Ali je polje \(\vec{f}(x,y,z) = (y^2z^3 + 2, 2xyz^3 + 1, 3xy^2z^2)\) potencialno? Čemu je enak \(\rot \vec{f}\)? Ali je polje \(\vec{g}(x,y,z) = (2y-1, -1, 4x - 2xy)\) solenoidalno?
        \item \textbf{Opomba.} V kakšni obliki lahko lokalno zapišemo vsako vektorsko polje?
    \end{itemize}
\end{enumerate}

\subsection{Krivuljni in ploskovni integral}
\begin{enumerate}
    \item Dolžina krivulje
    \begin{itemize}
        \item Regularna parametrizacija krivulje.
        \item \textbf{Definicija.} Dolžina krivulje.
        \item \textbf{Trditev.} Ali je definicija neodvisna od izbire regularne parametrizacije? 
        \item \textbf{Zgled.} Naravni parameter.
        \item \textbf{Zgled.} Vijačnico lahko parametriziramo s predpisom \(t \mapsto (a\cos t, a \sin t, bt)\). Določi naravno parametrizacijo vijačnice.        
    \end{itemize}

    \item Krivuljni integral skalarnega polja
    \begin{itemize}
        \item \textbf{Definicija.} Orientacija krivulje. Usklajen izbor orientacije. Orientirana krivulja.
        \item \textbf{Opomba.} Ali je vsaka krivulja orientabilna? Kaj če je krivulja odsekoma gladka? Krivulja z robom.
        \item \textbf{Definicija.} Integral skalarnega polja vzdolž krivulje.
        \item \textbf{Opomba.} Kaj je dolžina krivulje? Ali je vrednost odvisna od izbire regularne parametrizacije? Kaj je skalarno polje v fizikalnem smislu? Kaj če je krivulja odsekoma gladka?
        \item \textbf{Zgled.} Naj bo \(\Gamma = \setb{(x,y) \in \R^2}{x^2+y^2 = a^2, \ y \geq 0}\) homogena polkrožnica. Določi lego težišča \(\Gamma\).
    \end{itemize}

    \item Krivuljni integral vektorskega polja
    \begin{itemize}
        \item \textbf{Definicija.} Integral vektorskega polja vzdolž krivulje.
        \item \textbf{Opomba.} Fizikalni pomen. Ali je definicija odvisna od izbire regularne parametrizacije?
        \item \textbf{Zgled.} Naj bo \(\vec{f}(x,y,z) = (xy, z, x- z)\) ter \(\Gamma: \vec{r}(t) = (t, t, \frac{1}{2}t^2), t \in [0,1]\). Izračunaj integral \(\vec{f}\) po \(\Gamma\).
        \item \textbf{Zgled.} \todo{Delo sile teže.}
        \item \textbf{Opomba.} Zapis integrala vektorskega polja v diferencialni formi. Integral po sklenjeni krivulji. 
        \item \textbf{Trditev.} Kaj če integriramo potencialno polje?
        \item \textbf{Posledica.} Kaj če integriramo potencialno polje po sklenjeni krivulji?
        \item \textbf{Zgled.} Izračunaj integral polja \(\vec{f}(x,y,z) = \left(-\frac{y}{x^2 + y^2}, \frac{x}{x^2+y^2}, 0\right)\) po krožnice.
        \item \textcolor{red}{(*)} \textbf{Izrek.} Karakterizacija potencialnih vektorskih polj.
        \begin{proof}
            Potencial je \(u(T) = \int_{K} \vec{f} \cdot d\vec{r}\), kjer \(K\) krivulja od \(P_0\) do \(T\). Odvod \(u_x\) izračunamo po definiciji.
        \end{proof}
    \end{itemize}

    \item Površina ploskve
    \begin{itemize}
        \item Intuitivna izpeljava formule za površine ploskve.
        \item \textbf{Definicija.} Površina ploskve.
        \item \textbf{Trditev.} Ali je definicija odvisna od izbire regularne parametrizacije?
    \end{itemize}

    \item Orientacija ploskev
    
    Naj bo \(\Sigma \subseteq \R^3\) gladka ploskev.
    \begin{itemize}
        \item \textbf{Definicija.} Orientacija \(\Sigma\). Orientabilna ploskev.
        \item \textbf{Opomba.} Koliko orientacij lahko ima orientabilna povezana ploskev?
        \item \textbf{Zgled.} Določi ali je ploskev \(\Sigma\) orientabilna, če 
        \begin{itemize}
            \item \(\Sigma\) je graf funkcije;
            \item \(\Sigma\) je sfera;
            \item \(\Sigma\) je plašč valja;
            \item \(\Sigma\) je torus; je sklenjena ploskev;
            \item \(\Sigma\) je Mobiusov trak.
        \end{itemize}
        \item \textbf{Definicija.} Gladka ploskev z robom. Rob ploskve. Skladna orientacija roba.
        \item \textbf{Opomba.} Orientacija, ki je usklajena z parametrizacijo.
        \item \textbf{Definicija.} Odsekoma gladka ploskev. Orientacija odsekoma gladke ploskve.
    \end{itemize}

    \newpage
    \item Ploskovni integral skalarnega polja
    \begin{itemize}
        \item \textbf{Definicija.} Ploskovni integral skalarnega polja.
        \item \textbf{Opomba.} Kaj je površina ploskve?
        \item \textbf{Trditev.} Ali je integral odvisen od izbire regularne parametrizacije?
        \item \textbf{Opomba.} Ali je orientacija ploskve pomembna? Ali je ta integral obstaja na Mobiusovem traku?
        \item \textbf{Opomba.} Kaj je masa ploskve? Homogena ploskev.
        \item \textbf{Zgled.} Izračunaj vztrajnostni moment homogene sfere z polmerom \(\R\) okoli \(z\)-osi.
    \end{itemize}

    \item Ploskovni integral vektorskega polja
    \begin{itemize}
        \item \textbf{Definicija.} Ploskovni integral vektorskega polja. Pretok vektorskega polja skozi ploskev.
        \item \textbf{Trditev.} Ali je integral odvisen od izbire regularne parametrizacije?
        \item \textbf{Opomba.} Kaj pravi formula, če izberimo orientacijo, ki je usklajena z regularno parametrizacijo? Kaj če imamo odsekoma gladko ploskev?
        \item \textbf{Zgled.} \todo{sfera}.
        \item \textbf{Opomba.} Diferencialna 1-forma.
    \end{itemize}

    \item Integralski izreki
    \begin{itemize}
        \item \textcolor{red}{(*)} \textbf{Izrek.} Gauss-Ostrogradski.
        \begin{proof}
            Gledamo primer, ko je \(\Omega\) za vsako od treh koordinatnih ravnin leži med dvema grafoma \(C^1\) funkcij na omejeno odprto množico z ploščino.
        \end{proof}
        \item \textcolor{red}{(*)} \textbf{Izrek.} Stokesov izrek.
        \item \textcolor{red}{(*)} \textbf{Izrek.} Greenova formula.
        \item \textbf{Zgled.} \todo{Račun integralov.}
        \item \textbf{Trditev.} Dokaži, da Gaussov izrek (\(n=2\)) implicira Greenovo formulo.
        \item \textbf{Trditev.} Dokaži, da Stokesov izrek implicira Greenovo formulo.
        \item \textcolor{red}{(*)} \textbf{Trditev.} Dokaži, da Greenova formula implicira Stokesov izrek.
        \begin{proof}
            Omejimo se lahko na primer, ko je \(\Sigma\) graf nad \(D \subseteq \R^2\).
        \end{proof}
        \item \textbf{Definicija.} Divergenca, ki je neodvisna od izbire koordinatnega sistema.
        \item \textbf{Definicija.} Rotor, ki je neodvisen od izbire koordinatnega sistema.
        \item \textbf{Izrek.} Greenovi identiteti.
        \item \textbf{Opomba.} O diferencialnih formah.
    \end{itemize}
\end{enumerate}