\section{Kompleksna analiza}

\begin{enumerate}
    \item Kompleksna števila    
    \begin{itemize}
        \item Komutativni obseg \(\C\). Vložitev \(\R\) v \(\C\).
        \item Imaginarna enota \(i\). Kvadrat imaginarne enote \(i^2\).
        \item Algebraičen zapis kompleksnega števila. Realni in kompleksni del. Gaussova ravnina.
        \item Konjugiranje. Absolutna vrednost. Kaj velja za absolutno vrednost?
        \item Polarni zapis kompleksnega števila. Argument kompleksnega števila.
        \item Metrika (topologija) na \(\C\). Odprt krog v \(\C\).
        \item Zaporedja v \(\C\).
        \item Karakterizacija povezanih množic v \(\C\). Komponente za povezanost.
        \item \textbf{Definicija.} Območje.
        \item Zveznost preslikave \(f: D \subseteq \C \to \C\). Limita.
        \item Kako kompleksna funkcija definira realni? Kdaj je kompleksna funkcija \(f\) zvezna?
        \item Riemannova sfera (kompaktifikacija z eno točko).
    \end{itemize}

    \item Holomorfne funkcije
    
    Naj bo \(D \subseteq \C\) območje ter \(f: D \to \C\) kompleksna funkcija.
    \begin{itemize}
        \item \textbf{Definicija.} Kompleksni odvod funkcije \(f\) v točki \(a \in D\). Holomorfna funkcija. Cela funkcija. Množica vseh holomorfnih funkcij. 
        \item \textbf{Opomba.} Ali je kompleksni odvod močnejši od običajnega?
        \item \textbf{Posledica.} Ali je kompleksno odvedljiva funkcija v točki \(a \in D\) diferenciabilna? Ali je zvezna?
        \item \textbf{Opomba.} Ali je \(f(z) = \overline{z}\) kompleksno linearna? Ali je linearna? Ali je kompleksno odvedljiva?
        \item \textbf{Trditev.} Kakšno strukturo ima \(O(D)\)? Pravila za odvajanje.
        \item \textbf{Trditev.} Kompleksni odvod kompozicije.
    \end{itemize}

    \item Cauchy-Riemannove enačbe
    
    Naj bo \(D \subseteq \C\) območje ter \(f: D \to \C\) kompleksna funkcija.
    \begin{itemize}
        \item \textbf{Izrek.} Cauchy-Riemannove enačbe. 
        \item \textbf{Opomba.} Kako izračunamo kompleksni odvod?
        \item \textbf{Zgled.} \todo{Račun odvodov.}
        \item \textbf{Opomba.} Simboli \(\frac{\partial f}{\partial \overline{z}}\) ter \(\frac{\partial f}{\partial z}\)
        \item \textbf{Trditev.} Karakterizacija holomorfnosti \(f\). Cauchy-Riemmanova enačba.
        \item \textbf{Zgled.} \todo{Račun odvodov.}
        \item \textbf{Opomba.} Kdaj je intuitivno \(f\) holomorfna?
        \item \textbf{Trditev.} Kdaj je \(f\) holomorfna na \(D \subseteq \C\) (diferencial)?
        \item \textbf{Izrek.} Zadosten pogoj, da je \(f\) konstanta.
        \item \textbf{Izrek.} Kaj če je \(f\) holomorfna na območju \(D\) ter \(\img{f}(D) \subseteq \R\)?
        \item \textbf{Izrek.} Pišimo \(f = u + iv\). Recimo, da je \(f \in O(D)\) ter \(f \in C^2(D)\). Kaj lahko povemo o \(u\) in \(v\)?
        \item \textbf{Definicija.} Harmonična konjugiranka.
        \item \textbf{Opomba.} Kaj če imamo eno harmonično konjugiranko? V čim se razlikujeta dve harmonični konjugiranki?
        \item \textbf{Zgled.} Pokaži, da je \(u(x, y) = xy\) harmonična in določi njeno harmonično konjugiranko.  Pokaži, da je \(\log |z|\) harmonična na \(\C \setminus \set{0}\) in na \(\C \setminus \set{0}\) nima harmonične konjugiranke.
        \item \textbf{Izrek.} Zadosten pogoj za obstoj harmonične konjugiranke.
    \end{itemize}

    \item Potenčne vrste v kompleksnem
    \begin{itemize}
        \item \textbf{Definicija.} Kdaj kompleksna številska vrsta konvergira? Kdaj vrsta konvergira absolutno?
        \item \textbf{Opomba.} Kakšno strukturo ima množica konvergentnih številskih vrst? Ali pri absolutni konvergenci lahko seštevamo v poljubnem vrstnem redu?
        \item \textbf{Definicija.} Kdaj funkcijska vrsta konvergira po točkah? Kdaj konvergira enakomerno? Kdaj konvergira enakomerno po kompaktih?
        \item \textbf{Zgled.} Gledamo \(f_n(z) = z^n\) kot zaporedje oziroma 
        \[g_1(z) = 1, \ g_n(z) = z^n - z^{n-1}, \ n \geq 2\]
        kot vrsto. Ali vrsta \(\sum_{n=1}^{\infty} g_n(z)\) konvergira enakomerno na \(\triangle\)? Ali konvergira po kompaktih v \(\triangle\)? 
        \item \textbf{Izrek.} Weierstrassov kriterij.
        \item \textbf{Definicija.} Potenčna vrsta.
        \item \textbf{Izrek.} Konvergenčni polmer. Obstoj in formula.
        \item \textbf{Definicija.} Kdaj pravimo, da kompleksno funkcijo se da razviti v potenčno vrsto?
        \item \textbf{Izrek.} Kaj lahko povemo o funkciji, če jo se da razviti v potenčno vrsto v okolice točke \(a \in D\)?
        \item \textbf{Posledica.} Ali je vsota konvergentne potenčne vrste holomorfna funkcija? Kaj je njen odvod?
        \item \textbf{Posledica.} Lokalna oblika prejšnje posledice.
        \item \textbf{Zgled.} Razvoj v potenčno vrsto. Koeficienti.
    \end{itemize}

    \item Elementarne funkcije v kompleksnem
    \begin{itemize}
        \item Eksponentna funkcija.
        \item \textbf{Trditev.} Čemu je enako \(e^{z+w}\)?
        \item Funkciji sinus in kosinus. Povezava z eksponento. 
        \item Eulerjeva formula.
        \item Hiperbolični sinus in kosinus. Povezava z navadnimi.
        \item Ali ima eksponenta ničla na \(\C\)?
        \item Koliko rešitev ima enačba \(e^z=1\)? Ali je \(e^z\) periodična?
        \item Ničle funkcije sinus. Sinus vsote.
        \item Logaritemska funkcija.
        \item Korenska funkcija.
    \end{itemize}

    \item Krivuljni integral v \(\C\)
    \begin{itemize}
        \item \textbf{Definicija.} Krivuljni integral v \(\C\).
        \item \textbf{Trditev.} Trikotniška neenakost.
        \item \textbf{Trditev.} Ocena vrednosti integrala po krivulje.
        \item \textbf{Opomba.} Zapis diferencialne \(1\)-forme v kompleksne oblike. Integral po \(d\overline{z}\).
        \item \textbf{Trditev.} Osnovna formula integralskega računa v kompleksnem.
        \item \textbf{Posledica.} Naj bo \(n \in \N\). Čemu je enak integral \(\int_{\gamma} z^n \, dz\), če je \(\gamma\) sklenjena pot v \(\C\)?
        \item \textbf{Posledica.} Verzija prejšnje posledice za \(n \in \Z\).
    \end{itemize}

    \item Greenova formula v kompleksnem
    \begin{itemize}
        \item \textbf{Trditev.} Greenova formula v kompleksnem.
        \item \textbf{Posledica.} Cauchyjev izrek.
        \item \textbf{Izrek.} Cauchyjeva formula.
        \item \textbf{Opomba.} S čim je enolično določena holomorfna funkcija? Cauchyjevo jedro.
        \item \textbf{Posledica.} Lastnost povprečne vrednosti.
        \item \textbf{Opomba.} Ali je potrebna predpostavka, da je \(f \in C^1(\Omega)\)? Kaj pravi posledica?
        \item \textbf{Posledica.} Kaj če \(f \in O(\Omega) \cap C^1(\Omega)\)? Formula za odvod.
        \item \textbf{Izrek.} Morerov (Morera) izrek.
        \item \textbf{Izrek.} Goursatov (Goursat) izrek. 
        \item \textbf{Trditev.} Princip maksima.
        \item \textbf{Posledica.}  Recimo, da \(f \in C(\overline{\Omega}) \cap O(\Omega)\). Čemu je enak maksimum \(f\)?
        \item \textbf{Trditev.} O funkcijskem zaporedju holomorfnih funkcij.
    \end{itemize}

    \item Razvoj holomorfne funkcije v vrsto
    \begin{itemize}
        \item \textbf{Izrek.} Ali lahko vsako holomorfno funkcijo razvijemo v vrsto? Kje konvergira ta vrsta? Kaj je njena vsota?
        \item \textbf{Izrek.} Cauchyjeve ocene.
        \item \textbf{Izrek.} Liouvilleov (Liouville) izrek.
        \item \textbf{Posledica.} Zadosten pogoj, da je \(f\) konstanta.
        \item \textbf{Izrek.} Osnovni izrek algebre.
        \item \textbf{Posledica.} Koliko ničel ima vsak nekonstanten polinom stopnje \(n\) v \(\C\)?
        \item \textbf{Trditev.} Razcep funkcije \(f \in O(\triangle (a, r))\). Stopnja ničle.
        \item \textbf{Izrek.} Princip identičnosti.
        \item \textbf{Posledica.} Zadosten pogoj za enakost holomorfnih funkcij.
        \item \textbf{Posledica.} O izoliranih ničlah.
    \end{itemize}

    \item Izolirane singularne točke
    \begin{itemize}
        \item \textbf{Definicija.} Kdaj pravimo, da ima \(f\) v \(a\) izolirno singularno točko? Singularna točka.
        \item \textbf{Trditev.} Odpravljiva singularna točka.
        \item \textbf{Izrek.} Punktiran disk. Laurentova vrsta. Koeficienti. Regularni in glavni del vrste.
        \item \textbf{Opomba.} Ali so integrali, ki določajo koeficienti odvisni od izbire \(r\)? Kje konvergira glavni del in kje regularni?
        \item \textbf{Zgled.} \todo{Odpravljiva sungularnost.}
        \item \textbf{Definicija.} Odpravljiva singularnost. Pol stopnje \(n\). Bistvena singularnost.
        \item \textbf{Zgled.} \todo{Tipi singularnosti.}
        \item \textbf{Trditev.} Karakterizacija odpravljive singularnosti.
        \item \textbf{Trditev.} Karakterizacija pola.
        \item \textbf{Posledica.} Karakterizacija pola (limita).
        \item \textbf{Izrek.} Karakterizacija bistvene singularnosti.
        \item \textbf{Izrek.} Veliki Picardov izrek.
        \item \textbf{Zgled.} \todo{}
        \item \textbf{Posledica.} Mali Picardov izrek.
        \item \textbf{Opomba.} \todo{O točke \(\infty\) na Riemmanovi sferi.}
        \item \textbf{Zgled.} \todo{} 
    \end{itemize}

    \item Residui. Izrek o residuih.
    \begin{itemize}
        \item \textbf{Definicija.} Residuum.
        \item \textbf{Trditev.} Čemu je enak \(\Res(f, a)\), če ima funkcija \(f\) v \(a\) pol stopnje \(n\).
        \item \textbf{Zgled.} \todo{Izračun \(\Res\).}
        \item \textbf{Izrek.} Izrek o residuih.
        \item \textbf{Zgled.} \todo{Izračun integralov.}
        \item \textbf{Definicija.} Meromorfna funkcija. 
        \item \textbf{Trditev.} Karakterizacija meromorfnih funkcij.
        \item \textbf{Zgled.} Ali so racionalne funkcije meromorfne na \(\C\)?
        \item \textbf{Trditev.} Kaj če množica ničel meromorfne na \(\Omega\) funkcije ima stekališče v \(\Omega\)?
        \item \textbf{Posledica.} Kaj lahko povemo o množice polov meromorfne funkcije?
        \item \textbf{Opomba.} Razširitev meromorfne funkcije do \(f: \Omega \to \C P^1\).
        \item \textbf{Izrek.} Princip argumenta.
        \item \textbf{Opomba.} \todo{}
        \item \textbf{Izrek.} Rouchejev (Rouche) izrek.
        \item \textbf{Posledica.} Standardna oblika Rouchejeva izreka.
        \item \textbf{Zgled.} Koliko ničel ima \(f(z) = z^5 + 3z - 1\) na \(1 \leq |z| \leq 2\)?
        \item \textbf{Posledica.} Naj bo \(\overline{\triangle(a, r)} \leq \Omega\) in \(f \subseteq O(\Omega)\). Zadosten pogoj, da ima \(f\) ničlo na \(\triangle(a, r)\).
        \item \textbf{Opomba.} Ali to sledi tudi iz principa maksima?
        \item \textbf{Definicija.} Odprta preslikava.
        \item \textbf{Izrek.} Ali je vsaka nekonstantna holomorfna preslikava odprta?
        \item \textbf{Opomba.} Ali lahko odprta preslikava notranjosti definicijskega območja doseže maksimum?
        \item \textbf{Izrek.} Inverzna formula.
        \item \textbf{Trditev.} obstoj logaritma.
        \item \textbf{Opomba.} \todo{}
        \item \textbf{Posledica.} Čemu je lokalno ekvivalentna vsaka holomorfna funkcija? 
    \end{itemize}

\end{enumerate}