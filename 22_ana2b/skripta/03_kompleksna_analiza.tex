\section{Kompleksna analiza}

\begin{enumerate}
    \item Kompleksna števila    
    \begin{itemize}
        \item Komutativni obseg \(\C\). Vložitev \(\R\) v \(\C\).
        \item Imaginarna enota \(i\). Kvadrat imaginarne enote \(i^2\).
        \item Algebraičen zapis kompleksnega števila. Realni in kompleksni del. Gaussova ravnina.
        \item Konjugiranje. Absolutna vrednost. Kaj velja za absolutno vrednost?
        \item Polarni zapis kompleksnega števila.
        \item Metrika (topologija) na \(\C\). Odprt krog v \(\C\).
        \item Zaporedja v \(\C\).
        \item Karakterizacija povezanih množic v \(\C\). Komponente za povezanost.
        \item \textbf{Definicija.} Območje.
        \item Zveznost preslikave \(f: D \subseteq \C \to \C\). Limita.
        \item Kako kompleksna funkcija definira realni? Kdaj je kompleksna funkcija \(f\) zvezna?
        \item Riemannova sfera (kompaktifikacija z eno točko).
    \end{itemize}

    \item Holomorfne funkcije
    
    Naj bo \(D \subseteq \C\) območje ter \(f: D \to \C\) kompleksna funkcija.
    \begin{itemize}
        \item \textbf{Definicija.} Kompleksni odvod funkcije \(f\) v točki \(a \in D\). Holomorfna funkcija. Množica vseh holomorfnih funkcij.
        \item \textbf{Opomba.} Ali je kompleksni odvod močnejši od običajnega?
        \item \textbf{Posledica.} Ali je kompleksno odvedljiva funkcija v točki \(a \in D\) diferenciabilna? Ali je zvezna?
        \item \textbf{Opomba.} Ali je \(f(z) = \overline{z}\) kompleksno linearna? Ali je linearna? Ali je kompleksno odvedljiva?
        \item \textbf{Trditev.} Kakšno strukturo ima \(O(D)\)? Pravila za odvajanje.
        \item \textbf{Trditev.} Kompleksni odvod kompozicije.
    \end{itemize}

    \item Cauchy-Riemannove enačbe
    
    Naj bo \(D \subseteq \C\) območje ter \(f: D \to \C\) kompleksna funkcija.
    \begin{itemize}
        \item \textbf{Izrek.} Cauchy-Riemannove enačbe. 
        \item \textbf{Opomba.} Kako izračunamo kompleksni odvod?
        \item \textbf{Zgled.} \todo{Račun odvodov.}
        \item \textbf{Opomba.} Simboli \(\frac{\partial f}{\partial \overline{z}}\) ter \(\frac{\partial f}{\partial z}\)
        \item \textbf{Trditev.} Karakterizacija holomorfnosti \(f\). Cauchy-Riemmanova enačba.
        \item \textbf{Zgled.} \todo{Račun odvodov.}
        \item \textbf{Opomba.} Kdaj je intuitivno \(f\) holomorfna?
        \item \textbf{Trditev.} Kdaj je \(f\) holomorfna na \(D \subseteq \C\) (diferencial)?
        \item \textbf{Izrek.} Zadosten pogoj, da je \(f\) konstanta.
        \item \textbf{Izrek.} kaj če je \(f\) holomorfna na območju \(D\) ter \(\img{f}(D) \subseteq \R\)?
        \item \textbf{Izrek.} Pišimo \(f = u + iv\). Recimo, da je \(f \in O(D)\) ter \(f \in C^2(D)\). Kaj lahko povemo o \(u\) in \(v\)?
        \item \textbf{Definicija.} Harmonična konjugiranka.
        \item \textbf{Opomba.} Kaj če imamo eno harmonično konjugiranko? V čim se razlikujeta dve harmonični konjugiranki?
        \item \textbf{Zgled.} Pokaži, da je \(u(x, y) = xy\) harmonična in določi njeno harmonično konjugiranko.  Pokaži, da je \(\log |z|\) harmonična na \(\C \setminus \set{0}\) in na \(\C \setminus \set{0}\) nima harmonične konjugiranke.
        \item \textbf{Izrek.} Zadosten pogoj za obstoj harmonične konjugiranke.
    \end{itemize}
\end{enumerate}