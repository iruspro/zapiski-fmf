\section{Hilbertovi prostori}
\begin{enumerate}
    \item Vektorski prostor s skalarnim produktom
    
    Naj bo \(X\) vektorski prostor nad \(\R\) (ali nad \(\C\)).
    \begin{itemize}
        \item \textbf{Definicija.} Skalarni produkt.
        \item \textbf{Trditev.} Cauchy-Schwartzova neenakost.
        \item \textbf{Definicija.} Norma na vektorskem prostoru \(X\).
        \item \textbf{Trditev.} Norma, ki je dobljena iz skalarnega produkta.
        \item \textbf{Trditev.} Metrični prostor, porojeni z normo.
    \end{itemize}

    \item Hilbertovi prostori
    \begin{itemize}
        \item \textbf{Definicija.} Hilbertov prostor. Banachov prostor.
        \item \textbf{Zgled.} Standardni skalarni produkti na \(\R^n\) in \(\C^n\). Norme, ki ne pridejo iz skalarnega produkta.
    \end{itemize}

    \item Prostor \(L^2([a,b])\)
    \begin{itemize}
        \item \textbf{Trditev.} Standardni skalarni produkt na prostoru \(C([a,b])\).
        \item \textbf{Trditev.} Ali je prostor \(C([a,b])\) s standardnim skalarnim produktom Hilbertov?
        \item \textbf{Zgled.} Kako lahko napolnimo prostor \(((0,1), d_2)\)?
        \item \textbf{Definicija.} Kadar pravimo, da lahko napolnimo metrični prostor \((M, d)\)? Napolnitev prostora.
        \item \textbf{Opomba.} Kaj je ponavadi prostor \(\overline{M}\)?
        \item \textbf{Opomba.} Prostor \(L^1(A)\).
        \item \textbf{Definicija.} Prostor \(L^2([a,b])\).
        \item \textbf{Opomba.} Ali je produkt dveh \(L^2([a,b])\) funkcij \(L^1([a,b])\) funkcija? Skalarni produkt na \(L^2([a,b])\)
        \item \textbf{Trditev.} Ali je \(L^2([a,b])\) vektorski prostor nad \(\R\)?
        \item \textbf{Opomba.} Ali je \(C([a,b]) \subseteq L^2([a,b])\)? Ali je \(C([a,b])\) gost v \(L^2([a,b])\)? Kaj pomeni, da zaporedje \((f_n)_n \in L^2([a,b])\) konvergira k \(f \in L^2([a,b])\)?
        \item \textbf{Izrek.} Ali je \(L^2([a,b])\) Hilbertov prostor? Kako sta povezana prostora \(L^2([a,b])\) in \(C([a, b])\)? [brez dokaza]
        \item \textbf{Opomba.} Kako zgleda skalarni produkt nad \(\C\)?
        \item \textbf{Zgled.} Navedi primer funkcije ko limita po točkah ni enaka limite v \(L^2\) smislu. Navedi primer funkcije za katero ne obstaja limita po točkah, limita v \(L^2\) smislu pa obstaja.
    \end{itemize}

    \item Ortogonalnost
    
    Naj bo \(X\) vektorski prostor s skalarnim produktom, \(A \subseteq X\), \(A \neq \emptyset\).
    \begin{itemize}
        \item \textbf{Definicija.} Kadar sta dva vektorja pravokotna? Ortogonalni komplement množice \(A\).
        \item \textbf{Trditev.} Ali je \(A^\perp\) vektorski podprostor v \(X\)?
        \item \textbf{Opomba.} V kakšni relaciji sta \(A\) in \((A^\perp)^\perp\)?
        \item \textbf{Trditev.} Naj bo \(v \in X\). Ali je \(f: X \to \R, \ f(x) = \scp{x}{v}\) zvezna?
        \item \textbf{Posledica.} Ali je \(A^\perp\) zaprt podprostor v \(X\)?
        \item \textbf{Opomba.} Ali je \(C([a, b])\) zaprt podprostor v \(L^2([a,b])\)?
        \item \textbf{Opomba.} V kakšni relaciji sta \(A\) in \((A^\perp)^\perp\), če je \(X\) Hilbertov in \(A\) zaprt podprostor?
        \item \textbf{Trditev.} Pitagorjev izrek.
    \end{itemize}
    \newpage
    Naj bo \(X\) vektorski prostor s skalarnim produktom, \(Y \leq X\) podprostor v \(X\).
    \begin{itemize}
        \item \textcolor{red}{(*)} \textbf{Definicija.} Pravokotna projekcija vektorja \(x \in X\) na podprostor \(Y\).
        \item \textcolor{red}{(*)} \textbf{Trditev.} Kaj lahko povemo o pravokotne projekcije vektorja \(x \in X\) na \(Y\), če obstaja?
        \item \textbf{Zgled.} Ali imajo funkcije iz \(L^2([a, b]) \setminus C([a, b])\) najboljšo aproksimacijo z zveznimi funkciji?
        \item \textbf{Opomba.} Lastnosti \(P_Y\):
        \begin{itemize}
            \item Ali je \(P_Y\) idempotent?
            \item Kakšna zveza med \(\norm{x}\) in \(\norm{P_Y(x)}\)?
            \item Ali je \(P_Y: X \to Y\) linearna in zvezna?
            \item Ali je \(Y\) zaprt podprostor, če je \(P_Y\) definirana na \(X\)?
            \item Recimo, da \(P_Y(x)\) obstaja. Ali obstaja tudi \(P_{Y^\perp}(x)\)?
        \end{itemize}
        \item \textbf{Trditev.} Razvoj \(P_Y(x)\) po ONB.
    \end{itemize}

    \item Ortogonalni sistem
    
    Naj bo \(X\) vektorski prostor s skalarnim produktom.
    \begin{itemize}
        \item \textbf{Definicija.} Ortogonalni sistem (OS). Ortonormiran sistem (ONS).
        \item \textcolor{red}{(*)} \textbf{Trditev.} Besselova neenakost.
        \item \textbf{Posledica.} Čemu je enaka limita \(\lim_{j \to \infty} \scp{x}{e_j}\)?
        \item \textbf{Opomba.} Zakaj potrebujemo absolutno vrednost? Kaj so \((\scp{x}{e_j})_{j=1}^\infty\)?
        \item \textbf{Trditev.} Naj bo \(\sys\) ONS, \((c_j)_j\) tako zaporedje števil, da \(\sum_{j=1}^{\infty} |c_j|^2 < \infty\). Kaj potem?
        \item \textbf{Definicija.} Kompleten ortonormiran sistem (KONS).
        \item \textcolor{green}{(*)} \textbf{Trditev.} 6 ekvivalentnih trditev o KONS.
        \item \textbf{Zgled.} Modelni Hilbertov prostor.  
    \end{itemize}

    \item Prostor \(L^2([-\pi, \pi])\)
    \begin{itemize}
        \item ONS na prostoru \(L^2([-\pi, \pi])\). 
        \item \textbf{Opomba.} Kako lahko obravnavamo vsako funkcijo \(f: [-\pi, \pi] \to \R\) v tem kontekstu?
        \item \textcolor{red}{(*)} Klasične Fourierjevi koeficienti. Fourierjeva vrsta.
        \item \textbf{Trditev.} Riemann-Lebesgueva lema.
        \item \textbf{Trditev.} Parsevalova enakost.
        \item \textbf{Zgled.} Definiramo funkcijo \(f: [-\pi, \pi] \to \R\) s predpisom
        \[
            f(x) = \begin{cases}
                1; &0 \leq x \leq \pi \\
                0; &-\pi < x < 0.
            \end{cases}
        \]
        
        Razvij \(f\) v Fourierjevo vrsto ter izračunaj \(\sum_{n=1}^{\infty} \frac{1}{n^2}\).
        \item \textbf{Lema.} Naj bo \(f: \R \to \R\) odsekoma zvezna periodična funkcija s periodo \(p\). Čemu je enak integral \(\int_{a}^{a+p} f(x) \, dx\)?
        \item \textbf{Lema.} Dirichletovo jedro.
        \item \textbf{Lema.} 3 lastnosti Dirichletovega jedra.
        \item \textcolor{red}{(*)} \textbf{Izrek.} Fourierjeva vrsta funkcije.
        \item \textbf{Zgled.} S pomočjo vrste iz prejšnjega zgleda izračunaj \(\sum_{k=0}^{\infty}(-1)^{k}\frac{1}{2k+1}\).
        \item \textbf{Definicija.} Cesarjeve delne vsote. Fourierjevo jedro.
        \item \textbf{Trditev.} 5 lastnosti Fourierjeva jedra.
        \item \textcolor{red}{(*)}  \textbf{Izrek.} Naj bo \(f\) \(2\pi\) periodična zvezna funkcija. Kaj lahko povemo o Cesarjevih delnih vsotih?
        \item \textbf{Izrek.} Ali je prej definiran ONS na \(L^2\) KONS?
        \item \textbf{Opomba.} Trigonometrični polinomi.
        \item \textbf{Izrek.} Weierstrassov isrek.
    \end{itemize}
\end{enumerate}