% Okolja
\usepackage{amsthm}
\theoremstyle{definition}{
    \newtheorem*{definicija}{\textcolor{white}{Definicija}}
    \newtheorem*{izrek}{\textcolor{white}{Izrek}}
    \newtheorem*{trditev}{\textcolor{white}{Trditev}}
    \newtheorem*{posledica}{\textcolor{white}{Posledica}}
    \newtheorem*{lema}{\textcolor{white}{Lema}}
    \newtheorem*{aksiom}{\textcolor{white}{Aksiom}}
}

\theoremstyle{remark}{    
    \newtheorem*{opomba}{\textcolor{white}{Opomba}}
    \newtheorem*{primer}{\textcolor{white}{Primer}}
    \newtheorem*{zgled}{\textcolor{white}{Zgled}}
}

% Standardne množice
\newcommand{\N}{\mathbb{N}}
\newcommand{\Z}{\mathbb{Z}}
\newcommand{\Q}{\mathbb{Q}}
\newcommand{\R}{\mathbb{R}}
\newcommand{\C}{\mathbb{C}}
\newcommand{\F}{\mathbb{F}}
\newcommand{\HH}{\mathbb{H}}

%%% Quantifiers
\newcommand{\all}[1]{\forall #1 \,.\,}
\newcommand{\some}[1]{\exists #1 \,.\,}
\newcommand{\exactlyone}[1]{\exists! #1 \,.\,}

% Implikacija
\newcommand{\lthen}{\implies}
\newcommand{\liff}{\iff}

%%% Množice
\newcommand{\set}[1]{\left\{#1\right\}}
\newcommand{\setb}[2]{\set{#1 \ | \  #2}}

%% Preslikave
\newcommand{\img}[1]{#1_{*}}
\newcommand{\invimg}[1]{#1^{*}}
\newcommand{\lin}[1]{\mathcal{#1}}

%% Odvod
\newcommand{\podv}[2]{\frac{\partial #1}{\partial #2}}

%% Metrični prostori
\newcommand{\norm}[1]{||#1||}

\DeclareMathOperator{\grad}{grad}
\DeclareMathOperator{\id}{id}
\DeclareMathOperator{\rang}{rang}

%% New command
\newcommand{\df}[1]{\textcolor{purple}{\textbf{#1}}}
\newcommand*{\ds}{\displaystyle}

%% Algebra
% Oeratorji
\newcommand{\gen}[1]{\left\langle {#1} \right\rangle}
\DeclareMathOperator{\ch}{char}
\DeclareMathOperator{\red}{red}

\DeclareMathOperator{\Sim}{Sim}
\DeclareMathOperator{\sgn}{sgn}
\DeclareMathOperator{\GL}{GL}
\DeclareMathOperator{\SL}{SL}
\DeclareMathOperator{\End}{End}
\DeclareMathOperator{\Lin}{Lin}
\DeclareMathOperator{\Aut}{Aut}
\DeclareMathOperator{\Inn}{Inn}


%% Topologija
\newcommand{\T}{\mathcal{T}}
\newcommand{\B}{\mathcal{B}}
\newcommand{\PP}{\mathcal{P}}

\DeclareMathOperator{\Int}{Int}
\DeclareMathOperator{\Cl}{Cl}
\DeclareMathOperator{\Fr}{Fr}
\DeclareMathOperator{\pr}{pr}

\DeclareMathOperator{\Set}{Set}

%DM
\DeclareMathOperator{\ecc}{ecc}
\DeclareMathOperator{\diam}{diam}
\DeclareMathOperator{\rad}{rad}


\newcommand{\todo}[1]{\textcolor{red}{TODO: #1}}
\newcommand{\ot}[1]{\widetilde{#1}}

%Fizika
\newcommand{\vc}[3]{
    (#1, \, #2, \, #3)
}

% Analiza 2b
\newcommand{\scp}[2]{\left\langle #1, \, #2 \right\rangle }
\newcommand{\cf}[1]{C^{#1}([a,b])}
\DeclareMathOperator{\divr}{div}
\DeclareMathOperator{\rot}{rot}

%ugt
\newcommand{\op}[1]{#1^{\text{odp}}}
\newcommand{\cl}[1]{#1^{\text{zap}}}
\newcommand{\qs}[1]{{#1/_\sim}}
\newcommand{\q}[2]{#1/_{#2}}
\newcommand{\qd}{
    \begin{tikzcd}
        X \arrow[swap]{d}{q} \arrow{rd}{f} &   \\%
        X/_\sim \arrow[swap]{r}{g} & Y
    \end{tikzcd}
}
\newcommand{\sys}{(e_j)_{j=1}^\infty}

% Seminar
% Needed: \usetikzlibrary{math}
\newcommand{\drawCantor}[1]{
    \begin{tikzpicture}[xscale=6]
        \draw (0,0) -- (1,0) node[pos=0, anchor=east]{$C_0$};  % initial 𝐶₀
        \tikzmath{ % save arrays of left endpoints
          int \myN, \numInt, \i, \j, \maxMumIdx, \maxSonIdx, \sonIdx1, \sonIdx2;
          real \oldarr, \newarr, \intWidth, \rowsep, \vpos;
          \myN = #1;  % number of Cantor iterations (must be positive)
          \oldarr0 = 0;  % 𝐶ₙ₋₁
          \rowsep = 0.7;  % visual parameter
          \intWidth = 1;  % use recursive multiplication
          \maxSonIdx = 0;  % to avoid exponentiation
          \vpos = 0;  % vertical position
          for \i in {1,...,\myN} {
            \maxMumIdx = \maxSonIdx;  % max "parent" index
            \maxSonIdx = 2 * (\maxSonIdx + 1) - 1;  % #intervals in 𝐶ₙ - 1
            \intWidth = \intWidth / 3;  % length(𝐶ₙ) = (1/3)ⁿ
            \vpos = \vpos - \rowsep;
            { \node[anchor=east] at (0, \vpos cm) {$C_{\i}$}; };
            for \j in {0,...,\maxMumIdx} {
              \sonIdx1 = 2 * \j;
              \sonIdx2 = \sonIdx1 + 1;
              \newarr{\sonIdx1} = \oldarr{\j};
              \newarr{\sonIdx2} = \oldarr{\j} + 2 * \intWidth;
              {
                \draw (\newarr{\sonIdx1} cm, \vpos cm) -- ++(\intWidth cm, 0);
                \draw (\newarr{\sonIdx2} cm, \vpos cm) -- ++(\intWidth cm, 0);
              };
            };
            % oldarr ← newarr
            for \j in {0,...,\maxSonIdx} { \oldarr{\j} = \newarr{\j}; };
          };
        }
    \end{tikzpicture}
}

% Needed: \usetikzlibrary{decorations.fractals}
\newcommand{\drawKoch}{
    \begin{tikzpicture}[decoration=Koch snowflake]
        \draw (0,0) -- (4,0);
        \node[anchor=west] at (4.2,0) {$K_{0}$};
    
        \begin{scope}[yshift=-1.4cm]
            \draw decorate{ (0,0) -- (4,0) };
            \node[anchor=west] at (4.2, 0) {$K_{1}$};
        \end{scope}    
    
        \begin{scope}[yshift=-2.8cm]
            \draw decorate {
              decorate { (0,0) -- (4,0) }
            };
            \node[anchor=west] at (4.2, 0) {$K_{2}$};
        \end{scope}
    
        
        \begin{scope}[yshift=-4.2cm]
            \draw decorate {
              decorate {
                decorate { (0,0) -- (4,0) }
              }
            };
            \node[anchor=west] at (4.2, 0) {$K_{3}$};
        \end{scope}
    
        \begin{scope}[yshift=-5.6cm]
            \draw decorate {
              decorate {
                decorate {
                  decorate { (0,0) -- (4,0) }
                }
              }
            };
            \node[anchor=west] at (4.2, 0) {$K_{4}$};
        \end{scope}
    \end{tikzpicture}
}

\newcommand{\drawSmile}{
  \begin{tikzpicture}

      % Окружность (лицо)
      \draw[thick] (0, 0) circle (2);
      
      % Глаза
      \fill (0.7, 1) circle (0.2);  % правый глаз
      \fill (-0.7, 1) circle (0.2); % левый глаз
      
      % Рот (грустный)
      \draw[thick] (-1.2, -1) .. controls (-0.5, 0) and (0.5, 0) .. (1.2, -1);
    
  \end{tikzpicture}
}

\newcommand{\drawSmileFunny}{
  \begin{tikzpicture}

      % Окружность (лицо)
      \draw[thick] (0, 0) circle (2);
      
      % Глаза
      \fill (0.7, 1) circle (0.2);  % правый глаз
      \fill (-0.7, 1) circle (0.2); % левый глаз
      
      % Рот (грустный)
      \draw[thick] (-1.2, -1) .. controls (-0.5, -1.5) and (0.5, -1.5) .. (1.2, -1);
    
  \end{tikzpicture}
}

\newcommand{\drawGraf}{
  \begin{tikzpicture}
    \begin{axis}[
        axis lines=middle,
        xlabel style={at={(ticklabel* cs:1)}, anchor=west},  % ось x — подпись справа
        ylabel style={at={(ticklabel* cs:1)}, anchor=south}, % ось y — подпись сверху
        xlabel={$x$},
        ylabel={$f(x)$},
        ymin=0, ymax=4.5,
        xmin=0, xmax=5,
        samples=100,
        domain=0:5,
        xtick={2},
        xticklabels={$s_0$},  % замена 2 на s_0
        ytick={0},
        enlargelimits,
        every axis plot/.append style={thick, blue},
        axis line style={->},
      ]
  
      % "Бесконечно" до s_0
      \draw[->, thick, blue] (axis cs:0,4.5) -- (axis cs:1.98,4.5) node[above] {$\infty$};
  
      % Вертикальная пунктирная линия в s_0
      \draw[dashed, blue] (axis cs:2,0) -- (axis cs:2,4.4);
  
      % Точка в (s_0, 3)
      \node[circle, fill=blue, inner sep=1pt] at (axis cs:2,3) {};
  
      % f(x) = 0 после s_0
      \draw[<-, thick, blue] (axis cs:2.02,0) -- (axis cs:6,0);
  
    \end{axis}
  \end{tikzpicture}
}

\newcommand{\drawMesh}{
  \begin{tikzpicture}
    % Определим шаг сетки (delta)
    \def\delta{0.5}

    % Нарисуем сетку только для первой четверти
    \draw[very thin, gray] (0,0) grid (5,5);

    % Подпишем оси
    \draw[->] (-0.5,0) -- (5.5,0);
    \draw[->] (0,-0.5) -- (0,5.5);

  \end{tikzpicture}
}