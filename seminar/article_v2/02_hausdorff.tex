\section{Hausdorffova dimenzija}
Hausdorffova dimenzija je najstarejši in matematično najosnovnejši poskus formalizacije dimenzije. Njena moč izhaja iz dejstva, da temelji na zunanji meri, kar omogoča natančno in splošno definicijo.

\subsection{Konstrukcija Hausdorffove mere}
Začnemo z konstrukcijo Hausdorffove mere.

\begin{definicija}
    Naj bo \((X, d)\) metrični prostor. Zunanja mera \(\mu^*\) na \(X\) je \emph{metrična zunanja mera}, če 
    \[\mu^*(A \cup B) = \mu^*(A) + \mu^*(B),\]
    za vsaki množici \(A, B \subseteq X\), za kateri velja \(d(A, B) > 0\).
\end{definicija}

Dokaz naslednje trditve lahko najdemo v \cite[stran 349]{f-ra}.
\begin{trditev}
    \label{m-zun-mera}
    Naj bo \((X, d)\) metrični prostor. Če je \(\mu^*\) metrična zunanja mera na \(X\), potem je njena zožitev na Borelovo \(\sigma\)-algebro \(\mu = \mu^*|_{\mc{B(X)}}\) mera na merljivem prostoru \((X, \mc{B}(X))\).
\end{trditev}

Zdaj lahko definiramo Hausdorffovo zunanjo mero.

\begin{definicija}
    \label{def-haus-mera}
    Naj bo \((X, d)\) metrični prostor, \(p \geq 0\) in \(\delta > 0\). Za vsako podmnožico \(A \subseteq X\) definiramo
    \[
        H^p_\delta(A) = \inf \setb{\sum_{i=1}^{\infty} (\diam A_i)^p}
        {A_i \subseteq X,\ A \subseteq \bigcup_{i \in \N} A_i,\ \diam A_i \leq \delta},
    \]
    kjer je
    \[
        \diam A_i = \sup \setb{d(a, b)}{a, b \in A_i}
    \]
    \emph{premer množice \(A_i\)}.

    Limiti
    \[
        \mc{H}^p(A) = \lim_{\delta \to 0} H_\delta^p(A)
    \]
    pravimo \emph{\(p\)-dimenzionalna Hausdorffova zunanja mera množice \(A\)}.

    Pokritje množice \(A\) z množicami premera največ \(\delta\) imenujemo \emph{\(\delta\)-pokritje množice \(A\)}.
\end{definicija}

\begin{opomba}
    Ker se množica dovoljenih \(\delta\)-pokritij množice \(A\) z zmanjševanjem \(\delta\) oži, je funkcija \(H^p_\delta(A)\) naraščajoča. Zato limita \(\lim_{\delta \to 0} H^p_\delta(A)\) vedno obstaja (lahko je končna ali enaka \(\infty\)).
\end{opomba}

\begin{opomba} 
    Po dogovoru \(\inf \emptyset = \infty\).
\end{opomba}

\begin{opomba}
    \label{odp-zap-haus}
    V definiciji so \(A_i \subseteq X\) poljubne. Enak rezultat lahko dobimo, če se omejimo le na zaprte podmnožice, saj velja \(\diam A_i = \diam \overline{A_i}\), ali pa le na odprte podmnožice, saj lahko vsako množico \(A_i\) nadomestimo z množico \(U_i = \setb{x \in X}{d(x, A_i) < \frac{\varepsilon}{2^{i+1}}}\), ki ima premer kvečjemu \((\diam A_i) + \frac{\varepsilon}{2^i}\).
\end{opomba}

Dokažemo osnovno lastnost \(\mc{H}_p\).
\begin{trditev}
    \label{haus-m-zun}
    Naj bo \((X, d)\) metrični prostor. \(\mc{H}_p\) je metrična zunanja mera.
\end{trditev}

\begin{proof}
    Po trditvi \ref{zun-mera} sledi, da je \(H^p_\delta\) zunanja mera na \(X\). Ker je limita monotona, sledi tudi, da je \(\mc{H}^p\) zunanja mera na \(X\).

    Naj bo \(A, B \subseteq X\) množici, za kateri velja \(d(A, B) > 0\). Izberimo tak \(\delta > 0\), da velja \(\delta < d(A, B)\). Naj bo \((C_i)_{i \in \N}\) družina podmnožic \(X\), ki je \(\delta\)-pokritje množice \(A \cup B\).
    Ker je \(\diam C_i < d(A, B)\) za vsak \(i \in \N\), nobena množica \(C_i\) ne seka hkrati množice \(A\) in množice \(B\). Razdelimo vsoto \(\sum_{i=1}^{\infty} (\diam C_i)^p\) na dva dela:
    \begin{itemize}
        \item \(\displaystyle \sum_{C_i \cap B = \emptyset} (\diam C_i)^p\) in
        \item \(\displaystyle \sum_{C_i \cap A = \emptyset} (\diam C_i)^p\).
    \end{itemize}
    Po definiciji infimuma velja 
    \[\sum_{i=1}^{\infty} (\diam C_i)^p \geq \displaystyle \sum_{C_i \cap B = \emptyset} (\diam C_i)^p + \displaystyle \sum_{C_i \cap A = \emptyset} (\diam C_i)^p \geq H^p_\delta(A) + H^p_\delta(B).\]
    Ker je bilo pokritje \((C_i)_{i \in \N}\) poljubno, sledi, da \(H^p_\delta(A) + H^p_\delta(B)\) spodnja meja za množico dolžin vseh \(\delta\)-pokritij množice \(A \cup B\). Zato po definiciji infimuma velja, da
    \[H_\delta^p(A \cup B) \geq H^p_\delta(A) + H^p_\delta(B).\]
    V limiti \(\delta \to 0\) dobimo:
    \[\mc{H}^p(A \cup B) \geq \mc{H}^p(A) + \mc{H}^p(B).\]

    Po definiciji zunanje mere, ki je subaditivna, pa velja tudi:
    \[\mc{H}^p(A \cup B) \leq \mc{H}^p(A) + \mc{H}^p(B).\]
    Iz obeh neenakosti sledi enakost, torej \(\mc{H}^p\) zadošča pogoju metrike.
\end{proof}

Direktna posledica trditev \ref{m-zun-mera} in \ref{haus-m-zun} je
\begin{posledica}
    Naj bo \((X, d)\) metrični prostor. Zožitev Hausdorffove zunanje \(p\)-dimenzionalne mere na Borelovo \(\sigma\)-algebro 
    \[\mc{H}^p := \mc{H}^p|_{\mc{B}(X)}\]
    je mera na merljivem prostoru \((X,\mc{B}(X))\).
\end{posledica}

Brez dokaza navedemo še trditev, ki povezuje \(\mc{L}^n\) in \(\mc{H}^p\). Dokaz lahko najdemo v \cite[stran 351]{f-ra}.
\begin{trditev}
    \label{haus-leb}
    Naj bo \(n \in \N\). Obstaja konstanta \(c_n > 0\), da je \(c_n \mc{H}^n\) Lebesgueova mera na merljivem prostoru \((\R^n, \mc{B}(\R^n))\).
\end{trditev}

\begin{opomba}
    Konstanta \(c_n\) je volumen \(n\)-dimenzionalne krogle, tj. 
    \[c_n = \frac{\pi^{n/2}}{2^n \Gamma(\frac{n}{2} + 1)}.\]
\end{opomba}

V nadaljevanju bomo za metrični prostor \((X, d)\)  privzeli navaden evklidski prostor \((\R^n, d_2)\).

\subsection{Lastnosti Hausdorffove mere}
V tem razdelku bomo našteli in dokazali nekatere geometrijske lastnosti Hausdorffove mere, ki jih lahko prenesemo tudi na Hausdorffovo dimenzijo.

\subsubsection{Lastnosti skaliranja}
Lastnosti skaliranja so temeljne za razumevanje fraktalnih struktur, saj fraktali po svoji naravi izkazujejo samopodobnost. 

Naj bo \(P: \R^n \to \R^n\) \emph{podobnostna preslikava} s podobnostnim koeficientom \(\lambda > 0\), tj.\ preslikava, za katero velja
\[\all{x, y \in \R^n} |P(x) - P(y)| = c|x - y|.\]

Intuitivno je jasno, da če raztegnemo daljico \(\lambda\)-krat, se njena dolžina poveča \(\lambda\)-krat, torej 
\[\mc{L}^1(P(A)) = \lambda\mc{L}^1(A),\]
če povečamo stranico kvadrata za faktor \(\lambda\), se njegova ploščina poveča za faktor \(\lambda^2\), torej 
\[\mc{L}^2(P(A)) = \lambda^2\mc{L}^2(A)\]
in tako naprej za višje dimenzije.

Naravno se pojavi vprašanje: ali podobna lastnost velja tudi za Hausdorffovo mero? Odgovor nam da naslednja trditev.

\begin{trditev}
    \label{skale}
    Naj bo \(P: \R^n \to \R^n\) podobnostna preslikava s podobnostnim koeficientom \(\lambda > 0\) in \(A \subseteq \R^n\), \(p \geq 0\). Tedaj velja:
    \[\mc{H}^p(P(A)) = \lambda^p \mc{H}^p(A).\]
\end{trditev}

\begin{proof}
    Naj bo \((A_i)_{i \in \N}\) \(\delta\)-pokritje množice \(A\). Tedaj je \((P(A_i))_{i \in \N}\) \(\lambda \delta\)-pokritje množice \(P(A)\), saj velja \(\diam(P(A_i)) = \lambda \diam(A_i)\). Torej:
    \[\sum_{i=1}^{\infty} (\diam P(A_i))^p = \sum_{i=1}^{\infty}(\lambda \diam A_i)^p = \lambda^p \sum_{i=1}^{\infty}(\diam A_i)^p.\]
    Po definiciji infimuma velja 
    \[H^p_{\lambda \delta} (P(A)) \leq \lambda^p \sum_{i=1}^{\infty}(\diam A_i)^p.\]
    Ker je \((A_i)_{i \in \N}\) bilo poljubno \(\delta\)-pokritje množice \(A\), dobimo:
    \[H^p_{\lambda \delta} (P(A)) \leq \lambda^p H^p_\delta(A).\]
    V limiti \(\delta \to 0\) dobimo:
    \[\mc{H}^p (P(A)) \leq \lambda^p \mc{H}^p(A).\]    
    
    Za obratno neenakost uporabimo enak argument na preslikavi \(P^{-1}\), ki je tudi podobnostna preslikava s koeficientom \(1/\lambda\), in na množici \(P(A)\). Dobimo:
    \[
        \mc{H}^p(A) \leq (1/\lambda)^p \mc{H}^p(P(A)) \lthen \mc{H}^p(P(A)) \geq \lambda^p \mc{H}^p(A).
    \]
    Skupaj s prvo neenakostjo sledi enakost.
\end{proof}

Navedemo eno pomembno posledico

\begin{posledica}
    Naj bo \(P: \R^n \to \R^n\) izometrija, tj.\ preslikava, za katero velja
    \[\all{x, y \in \R^n} |P(x) - P(y)| = |x - y|\]
    in \(A \subseteq \R^n\), \(p \geq 0\). Tedaj velja:
    \[\mc{H}^p(P(A)) = \mc{H}^p(A).\]    
\end{posledica}

Primeri izometrij so rotacije, translacije, zrcaljenja ipd. Posledica pove, da je \(\mc{H}^p\) invariantna glede na rotacije, translacije in zrcaljenja, kar je pričakovano.

\subsubsection{Transformacijske lastnosti}
Še en zanimiv razred preslikav so \emph{H\"oldorjeve preslikave} stopnje \(\alpha > 0\), to so preslikave \(f: X \subseteq \R^m \to \R^n\), za katere velja
\[\some{c > 0} \all{x, y \in X} |f(x) - f(y)| \leq c|x-y|^\alpha.\]
V posebnem primeru, ko je \(\alpha = 1\), pravimo, da je preslikava \(f\) \emph{Lipschitzeva}.

Z istim argumentom kot prej lahko dokažemo naslednjo trditev
\begin{trditev}
    \label{mera-hold}
    Naj bo \(X \subseteq \R^m\), \(A \subseteq X\) in \(f: X \to \R^n\) Höldorjeva preslikava stopnje \(\alpha > 0\) s konstanto \(c > 0\). Potem za vsak \(p \geq 0\) velja:
    \[\mathcal{H}^{p/\alpha}(f(A)) \leq c^{p/\alpha} \mathcal{H}^p(F).\]
\end{trditev}

Direktna posledica te trditve je 
\begin{posledica}
    Naj bo \(X \subseteq \R^m\), \(A \subseteq X\) in \(f: X \to \R^n\) Lipschitzeva preslikava s konstanto \(c > 0\). Potem za vsak \(p \geq 0\) velja:
    \[\mathcal{H}^{p}(f(A)) \leq c^p \mathcal{H}^p(A).\]
\end{posledica}

Zdaj smo pripravili vsa potrebna orodja za definicijo Hausdorffove dimenzije in za preučevanje njenih lastnosti.

\subsection{Hausdorffova dimenzija}
V tem razdelku bomo definirali Hausdorffovo dimenzijo množice.

Naj bo \(A \subseteq \R^n\). Definiramo funkcijo \(\mc{H}_A: [0, \infty) \to [0, \infty]\) s predpisom 
\[\mc{H}_A(p) = \mc{H}^p(A).\]
\begin{lema}
    Naj bo \(A \subseteq \R^n\). Če je \(\mathcal{H}^{p}(A) < \infty\), potem je \(\mathcal{H}^{t}(A) = 0\) za vse \(t > p\).
\end{lema}

\begin{proof}
    Naj bo \(\varepsilon > 0\). Naj bo \(\delta > 0\) in \((A_i)_{i \in \N}\) \(\delta\)-pokritje množice \(A\). Tedaj po definiciji infimuma velja
    \[H^{p + \varepsilon}_\delta(A) \leq \sum_{i=1}^{\infty} (\diam A_i)^{p + \varepsilon} = \sum_{i=1}^{\infty} (\diam A_i)^{p} (\diam A_i)^{\varepsilon} \leq \delta^{\varepsilon} \sum_{i=1}^{\infty} (\diam A_i)^{p},\]
    torej 
    \[H^{p + \varepsilon}_\delta(A) \leq \delta^{\varepsilon} \sum_{i=1}^{\infty} (\diam A_i)^{p}.\]
    Ker je \((A_i)_{i \in \N}\) bilo poljubno \(\delta\)-pokritje množice \(A\), dobimo:
    \[H^{p + \varepsilon}_\delta(A) \leq \delta^{\varepsilon} H^{p}_\delta(A).\]
    Ker je \(\mathcal{H}^{p}(A) < \infty\), v limiti \(\delta \to 0\) dobimo:
    \[\mc{H}^{p + \varepsilon}(A) \leq 0.\]
    Ker je \(\mc{H}_A\) nenegativna funkcija, sledi
    \[\mc{H}^{p + \varepsilon}(A) = 0\]
    za vsak \(\varepsilon > 0\).
\end{proof}

Sedaj si lahko ogledamo graf funkcije \(\mc{H}_A\)
\begin{center}
    \drawGraf
\end{center}
Vidimo, da obstaja kritična točka \(p_0 \in [0, \infty)\), pri kateri vrednost funkcije \(\mc{H}^p\) ">skoči"< z \(\infty\) na \(0\). Zato definiramo

\begin{definicija}
    \emph{Hausdorffova dimenzija množice \(A \subseteq \R^n\)} je 
    \[\dim_H A = \inf \setb{p \geq 0}{\mathcal{H}^{p}(A) = 0} = \sup \setb{p \geq 0}{\mathcal{H}^{p}(A) = \infty}.\]
\end{definicija}

\begin{opomba} 
    Po dogovoru \(\sup \emptyset = 0\).
\end{opomba}

\begin{zgled}
    \label{dim-ball}
    Naj bo \(B^n = B(x, r) = \setb{y \in \R^n}{d(x, y) < r}\) odprta \(n\)-dim krogla s središčem v \(x \in \R^n\) in polmerom \(r > 0\). Po trditvi \ref{haus-leb} sledi, da velja
    \[\mc{H}^n(B^n) = \frac{1}{c_n} \mc{L}^n(B^n) = \frac{1}{c_n} \vol(B^n).\]
    Torej je 
    \[0 < \mc{H}^n(B^n) < \infty\]
    in 
    \[\dim_H B^n = n.\]
    Isti sklep velja tudi za zaprto kroglo \(\overline{B}^{\, n} \subseteq \R^n\), tj.
    \[\dim_H \overline{B}^{\, n} = n.\]
\end{zgled}

\begin{opomba}
    Obstajajo tudi druge ekvivalentne definicije Hausdorffove dimenzije, ki jih lahko najdemo v \cite{fk-fg}.
\end{opomba}

\subsection{Lastnosti Hausdorffove dimenzije}
Sedaj, ko smo definirali Hausdorffovo dimenzijo, lahko nadaljujemo z obravnavo njenih pomembnih geometričnih in topoloških lastnosti ter vpliva na strukturo množic.

Začnimo z lastnostmi, ki pokažejo, da je Hausdorffova dimenzija res naravna posplošitev običajnega pojma dimenzije.

\subsubsection{Splošne lastnosti}
Naslednja trditev je direktna posledica monotonosti Hausdorffove zunanje mere 

\begin{trditev}[Monotonost]
    \label{haus-monotonost}
    Hausdorffova dimenzija je monotona, tj.
    \[\all{A, B \subseteq \R^n} A \subseteq B \lthen \dim_H A \leq \dim_H B\]
\end{trditev}

\begin{trditev}[Števna stabilnost]
    \label{haus-stab}
    Naj bo \((A_i)_{i \in \N}\) družina podmnožic \(\R^n\). Tedaj velja 
    \[\dim_H \un{i \in \N}{A_i} = \sup \setb{\dim_H A_i}{i \in \N}.\]
\end{trditev}

\begin{proof}
    Označimo z \(s = \sup \setb{\dim_H A_i}{i \in \N}\). 
    
    Ker je \(A_i \subseteq \un{i \in \N}{A_i}\) za vsak \(i \in \N\) in je po trditvi \ref{haus-monotonost} Hausdorffova dimenzija monotona, sledi, da
    \[\dim_H F_i \leq \dim_H \un{i \in \N}{A_i}\]
    za vsak \(i \in \N\). Po definiciji supremuma velja:
    \[s \leq \dim_H \un{i \in \N}{A_i}.\]

    Naj bo \(\varepsilon > 0\). Po definiciji supremuma velja, da \(\dim_H A_i < s + \epsilon\) za vsak \(i \in \N\). Torej 
    \[\mc{H}^{s + \varepsilon}(A_i) = 0\]
    za vsak \(i \in \N\). Ker je Hausdorffova zunanja mera subaditivna, sledi, da
    \[\mc{H}^{s + \varepsilon}\left(\un{i \in \N}{A_i}\right) \leq \sum_{i=1}^{\infty} \mc{H}^{s + \varepsilon} (A_i) = 0.\]
    V limiti \(\varepsilon \to 0\) dobimo:
    \[\mc{H}^s\left(\un{i \in \N}{A_i}\right) \leq 0.\]
    Torej 
    \[\dim_H \un{i \in \N}{A_i} \leq s.\]    
    Skupaj s prvo neenakostjo sledi enakost.
\end{proof}

\begin{posledica}
    \label{dim-rn}
    Hausdorffova dimenzija \(\R^n\) je 
    \[\dim_H \R^n = n.\]
\end{posledica}

\begin{proof}
    Ker je prostor \(\R^n\) 2-števen, obstaja števno pokritje \((B^n_i)_{i \in \N}\) množice \(\R^n\) z odprtimi krogli. Po trditvi \ref{haus-stab} in zgledu \ref{dim-ball} sledi, da
    \[\dim_H \R^n \leq n.\]
    Ker pa velja \(B^n(0, 1) \subseteq \R^n\) in je po trditvi \ref{haus-monotonost} Hausdorffova dimenzija monotona, sledi tudi obratna neenakost:
    \[
        \dim_H \R^n \geq \dim_H B^n(0, 1) = n.
    \]
    Torej je \(\dim_H \R^n = n\).
\end{proof}

\begin{trditev}[Dimenzija števnih množic]
    Naj bo \(A \subseteq \R^n\). Če je \(A\) števna, potem 
    \[\dim_H A = 0.\]
\end{trditev}

\begin{proof}
    Ker je \(A\) števna množica, lahko jo zapišemo v obliki 
    \[A = \set{a_1, a_2, a_3, \ldots}.\]
    Definiramo \(A_i = \set{a_i}\) za vsak \(i \in \N\). Tedaj velja
    \[A = \un{i \in \N}{A_i}.\]
    Ker je \(0 < \mc{H}^0(A_i) = 1 < \infty\) sledi, da 
    \[\dim_H A_i = 0\]
    za vsak \(i \in \N\).
    Po trditvi \ref{haus-stab} velja
    \[\dim_H A = \dim_H \un{i \in \N}{A_i} = \sup \setb{\dim_H A_i}{i \in \N} = 0.\]
\end{proof}

\begin{zgled}
    Hausdorffova dimenzija \(\Q \subseteq \R\) je \(0\), saj je \(\Q\) števna množica. 
\end{zgled}

\begin{trditev}[Dimenzija odprtih množic]
    \label{haus-odp}
    Naj bo \(U \subseteq \R^n\). Če je \(U\) odprta in neprazna, potem 
    \[\dim_H U = n.\]
\end{trditev}

\begin{proof}
    Ker je \(U\) odprta in neprazna, vsebuje neko odprto \(n\)-dimenzionalno kroglo, po drugi strani pa je vsebovana v \(\R^n\). Trditev sledi.
\end{proof}

Ideje dokaza naslednje trditve lahko najdemo v \cite[stran 32]{fk-fg} in \cite[stran 351]{f-ra}.
\begin{trditev}[Dimenzija gladkih množic]
    Naj bo \(M \subseteq \R^n\). Če je \(M\) gladka (tj.\ \(C^\infty\)) podmnogoterost dimenzije \(m\), potem 
    \[\dim_H M = m.\]
    Posebej 
    \begin{itemize}
        \item Če je \(M\) gladka krivulja, potem \(\dim_H M = 1.\)
        \item Če je \(M\) gladka ploskev, potem \(\dim_H M = 2.\)
    \end{itemize}
\end{trditev}

\subsubsection{Transformacijske lastnosti}
Pri proučevanju Hausdorffove dimenzije se naravno zastavi vprašanje, kako se ta obnaša pri različnih transformacijah. Na primer: ali se dimenzija ohranja pri Lipschitzevih ali gladkih preslikavah? Takšna vprašanja so pomembna, saj nam transformacijske lastnosti pogosto omogočajo, da iz znane dimenzije neke množice sklepamo o dimenziji njene slike.

\begin{trditev}
    Naj bo \(X \subseteq \R^m\), \(A \subseteq X\) in \(f: X \to \R^n\) H\"oldorjeva preslikava stopnje \(\alpha > 0\) s konstanto \(c > 0\). Tedaj velja
    \[\dim_H f(A) \leq \frac{1}{\alpha} \dim_H A.\]
\end{trditev}

\begin{proof}
    Označimo z \(s:= \dim_H A\). Naj bo \(\varepsilon > 0\). Po trditvi \ref{mera-hold} velja
    \[\mc{H}^{(s +\varepsilon) / \alpha}(f(A)) \leq c^{(s +\varepsilon) / \alpha} \mc{H}^{s + \varepsilon}(A) = 0.\]
    V limiti \(\varepsilon \to 0\) dobimo
    \[\mc{H}^{s / \alpha}(f(A)) \leq 0,\]
    torej
    \[\dim_H f(A) \leq \frac{1}{\alpha} \dim_H A.\]
\end{proof}

V primeru Lipschitzevih preslikav (kjer je \(\alpha = 1\)) dobimo naslednjo pomembno posledico.

\begin{posledica}
    \label{dim-lips}
    Naj bo \(X \subseteq \R^m\), \(A \subseteq X\) in \(f: X \to \R^n\) Lipschitzeva preslikava s konstanto \(c > 0\). Tedaj velja
    \[\dim_H f(A) \leq \dim_H A.\]
\end{posledica}

Poseben primer Lipschitzevih preslikav so bi-Lipschitzeve preslikave. V tem primeru se Hausdorffova dimenzija ne spremeni.

\begin{izrek}
    Naj bo \(X \subseteq \R^m\), \(A \subseteq X\) in \(f: X \to \R^n\) bi-Lipschitzeva preslikava, tj.\ preslikava \(f: X \to \R^n\), za katero velja:
    \[\some{c_1, c_2 > 0} \all{x, y \in X} c_1|x-y| \leq |f(x) - f(y)| \leq c_2|x-y|.\]
    Tedaj velja
    \[\dim_H f(A) = \dim_H A.\]
\end{izrek}

\begin{proof}
    Ker je \(f\) injektivna, je njena zožitev \(f: X \to f(X)\) bijektivna z Lipschitzevim inverzom \(f^{-1}: f(X) \to X\) s konstanto \(1 / c_1\). Po posledici~\ref{dim-lips} imamo
    \[\dim_H f(A) \leq \dim_H A\]
    in 
    \[\dim_H A = \dim_H f^{-1}f(A) \leq \dim_H f(A).\]
    Skupaj s prvo neenakostjo sledi enakost.
\end{proof}

Izrek pove, da je Hausdorffova dimenzija invariantna glede na bi-Lipschit\-zeve preslikave. To pomeni, da so bi-Lipschitzeve transformacije ">dimenzijsko ohranjajoče"<. Če imata dve množici različno Hausdorffovo dimenzijo, potem med njima ne more obstajati bi-Lipschitzeva preslikava.

\subsubsection{Topološke lastnosti}
Hausdorffova dimenzija ima tudi zanimivo povezavo s topologijo. Naslednji rezultat kaže, da dovolj ">majhne"< množice (v smislu dimenzije) ne morejo biti povezane.

\begin{trditev}
    \label{povezanost}
    Naj bo \(A \subseteq \R^n\). Če je \(\dim_H A < 1\), potem je \(A\) popolnoma nepovezana, torej so njene komponente za povezanost enojci.
\end{trditev}

\begin{proof}
    Naj bosta \(x, y \in A, \ x \neq y\). Definiramo preslikavo \(f: A \to [0, \infty)\) s predpisom
    \[f(z) = |z-x|.\]
    Preslikava \(f\) je Lipschitzeva, saj za vse \(z, w \in A\) velja:
    \[
        |f(z) - f(w)| = ||z - x| - |w - x|| \leq |z - w|.
    \]
    Po posledici~\ref{dim-lips} sledi:
    \[
        \dim_H f(A) \leq \dim_H A < 1.
    \]
    Znano pa je (trditev \ref{haus-odp}), da vsaka neprazna odprta množica v \(\R\) ima Hausdorffovo dimenzijo \(1\), zato \(f(A)\) ne vsebuje nobene neprazne odprte podmnožice, saj je \(\dim_H\) monotona. Torej je \(\comp{f(A)}\) gosta v \(\R\).

    Ker je \(0 < f(y)\), obstaja \(r \in (0, f(y))\), da \(r \notin f(A)\). Definiramo množici:
    \[
        U := \{ z \in A \mid f(z) < r \}, \quad V := \{ z \in A \mid f(z) > r \}.
    \]
    Množici \(U\) in \(V\) sta odprti v podprostoru \(A\) (kot prasliki odprtih množic v \([0, \infty)\), disjunktni in pokrivata cel \(A\), saj \(r \notin f(A)\). Poleg tega velja \(x \in U\) (ker je \(f(x) = 0 < r\)) in \(y \in V\) (ker je \(f(y) > r\)). Torej \(x\) in \(y\) ležita v različnih komponentah za povezanost množice \(A\).

    Ker sta bili \(x\) in \(y\) poljubno izbrani različni točki, sledi, da je vsaka komponenta množice \(A\) enojec.
\end{proof}

\subsection{Primeri računanja Hausdorffove dimenzije}
V tem razdelku bomo izračunali Hausdorffovo dimenzijo Cantorjeve množice. Običajno postopamo tako, da z geometrijskim premislekom podamo spodnjo in zgornjo oceno za dimenzijo ter upamo, da se ti meji ujemata. V tem primeru lahko sklepamo, da smo našli točno dimenzijo.

\begin{trditev}
    Hausdorffova dimenzija Cantorjeve množice (slika \ref{fig:cantor-set}) je
    \[\dim_H C = \log_3 2.\]
\end{trditev}

\begin{proof}    
    Naj bo \(C_k\) \(k\)-ta generacija konstrukcije. Tedaj \(C_k\) vsebuje \(2^k\) intervalov dolžine \(3^{-k}\). Označimo \(s = \log_3 2\).

    Naj bo \(\delta > 0\). Potem obstaja \(k \in \N\), da velja \(3^{-k} \leq \delta\). Intervali iz \(C_k\) tvorijo \(\delta\)-pokritje množice \(C\), zato dobimo oceno
    \[H^s_\delta(C) \leq 2^k \cdot 3^{-ks} = 1.\]
    V limiti \(\delta \to 0\) sledi
    \[\mc{H}^s(C) \leq 1,\]
    torej 
    \[\dim_H C \leq s.\]

    Naj bo \(\delta > 0\). Naj bo \((U_i)_{i \in \N}\) poljubno \(\delta\)-pokritje množice \(C\). Po opombi~\ref{odp-zap-haus} lahko brez škode za splošnost predpostavimo, da so \(U_i\) neprazne odprte množice. Ker je \(C\) kompaktna, obstaja končno podpokritje 
    \[U_1, U_2, \ldots, U_N.\]
    Za vsak \(i \in \set{1, 2, \ldots, N}\) izberimo \(k_i \in \N\), da velja
    \[3^{-(k_i + 1)} \leq \diam U_i < 3^{-k_i}. \tag{1}\]
    Tedaj \(U_i\) seka kvečjemu en interval v generaciji \(C_{k_i}\), saj je razdalja med dvema sosednjima intervaloma v tej generaciji vsaj \(3^{-k_i}\).

    Če je \(j \geq k_i\), potem po konstrukciji Cantorjeve množice \(U_i\) lahko seka kvečjemu \(2^{j - k_i}\) intervalov generacije \(C_j\), saj se pri vsakem koraku \(C_n \to C_{n+1}\) vsak interval razdeli na dva.

    Iz (1) sledi
    \[2^{j - k_i} = 2^{k_i} \cdot 3^{-sk_i} = 2^j \cdot 3^s \cdot (3^{-(k_i+1)})^s \leq 2^j \cdot 3^s \cdot (\diam U_i)^s. \tag{2}\]
    Torej vsak \(U_i\) seka kvečjemu \(2^j \cdot 3^s \cdot (\diam U_i)^s\) intervalov generacije \(C_j\).

    Izberimo tak \(j \in \N\), da velja
    \[
    j \geq \max \set{k_1, k_2, \ldots, k_N}.
    \]
    Ker pokritje \((U_i)_{i \in \N}\) seka vsak izmed \(2^j\) intervalov v generaciji \(C_j\). S preštevanjem teh intervalov na dva načina z uporabo (2) dobimo oceno
    \[2^j \leq \sum_{i=1}^{N} 2^j \cdot 3^s \cdot (\diam U_i)^s,\]
    od koder sledi
    \[\frac{1}{3^s} \leq \sum_{i=1}^{N} (\diam U_i)^s \leq \sum_{i=1}^{\infty} (\diam U_i)^s.\]
    Ker je bilo pokritje \((U_i)_{i \in \N}\) poljubno, po definiciji infimuma sledi, da 
    \[\frac{1}{3^s} \leq H^s_\delta(C).\]
    V limiti \(\delta \to 0\) dobimo
    \[\mc{H}^s(C) \geq \frac{1}{3^s} = \frac{1}{2},\]
    torej 
    \[\dim_H C \geq s.\]

    Skupaj s prvo neenakostjo sledi enakost.
\end{proof}

Iz zgornjega računa in topoloških lastnosti Hausdorffove dimenzije sledi naslednje:
\begin{posledica}
    Cantorjeva množica \(C\) je popolnoma nepovezana.
\end{posledica}

\begin{opomba}
    Vidimo, da ima Cantorjeva množica res neničelno \(\log_3 2\)-dimenzionalno mero (oz.\ velikost). Z natančnejšo spodnjo oceno se lahko pokaže, da velja
    \[
        \mc{H}^s(C) = 1.
    \]
    Tak natančen izračun najdemo v \cite{pearse2014}.

    Če si predstavljamo, da ima Cantorjeva množica maso \(1\) kg, potem zaradi lastnosti skaliranja Hausdorffove mere (trditev \ref{skale}), če bi množico raztegnili trikrat -- torej začeli z intervalom \([0,3]\) namesto \([0,1]\) — bi se njena masa povečala za faktor \(3^{\log_3 2}\) in bi bila enaka \(2\) kg.
\end{opomba}

Vidimo, da je izračun Hausdorffove dimenzije pogosto izredno zahteven, celo za preproste množice. Najprej je treba s pomočjo geometrijskih opažanj, simetrije ali samopodobnosti uganiti pravo vrednost dimenzije, nato pa to vrednost še dokazati. Ta postopek zahteva natančno analizo in pogosto precej tehničnega znanja.
\newpage
Obstajajo tudi druge definicije dimenzije, na primer škatlasta (box-counting) dimenzija, ki jih lahko najdemo v knjigi \cite{fk-fg}. Te so pogosto enostavnejše za izračun in se uporabljajo v različnih aplikacijah, vendar imajo tudi nekatere pomanjkljivosti. Na primer, škatlasta dimenzija v splošnem ni števno stabilna.
