\section{Uvod}
Pojem dimenzije igra osrednjo vlogo v fraktalni geometriji. Intuitivno nam dimenzija množice pove, koliko prostora ta zavzema znotraj ambientnega prostora. Za običajne geometrijske objekte, kot so točke, daljice ali ploskve, se ta predstava sklada z našo intuicijo: točka ima dimenzijo \(0\), daljica \(1\), kvadrat \(2\) in kocka \(3\).

Fraktali pa pogosto kljubujejo tej klasični predstavi. Cantorjeva množica (slika \ref{fig:cantor-set}) na primer nima dolžine, vendar vsebuje neštevno mnogo točk in ima kompleksno strukturo. Zdi se, da »zavzema več kot nič, a manj kot eno dimenzijo«. Da bi takšne množice natančno opisali, potrebujemo bolj prefinjen pojem dimenzije.

Naravno se pojavi vprašanje, kako lahko definiramo dimenzijo za splošne množice? Naše izhodišče bo formalna definicija Hausdorffove dimenzije, ki jo bomo skrbno razvili in utemeljili. Nato bomo raziskali njene osnovne lastnosti in pokazali, zakaj predstavlja naravno posplošitev klasičnega pojma dimenzije.

\subsection{Cantorjeva množica}
V tem članku se bomo pogosto ukvarjali s Cantorjevo množico, ki je klasičen in hkrati izjemno pomemben primer množice s fraktalno strukturo.

\begin{figure}[ht]
    \centering
    \drawCantor{7}
    \caption{Cantorjeva množica.}
    \label{fig:cantor-set}
\end{figure}

Izgradimo Cantorjevo množico na intervalu \([0,1]\) po naslednjem postopku:
\begin{enumerate}
    \item Naj bo \(C_0 = [0,1]\).
    \item Množico \(C_0\) razdelimo na tri enake dele in odstranimo odprti srednji interval \((1/3, 2/3)\). Tako dobimo množico \(C_1 = [0, 1/3] \cup [2/3, 1]\).
    \item Recimo, da imamo množico \(C_n\), ki je unija \(2^n\) zaprtih intervalov dolžine \(3^{-n}\). Vsak izmed teh intervalov razdelimo na tri enake dele in odstranimo odprti srednji del. Tako dobimo množico \(C_{n+1}\), ki je unija \(2^{n+1}\) zaprtih intervalov dolžine \(3^{-(n+1)}\).
    \item Postopek nadaljujemo induktivno.
\end{enumerate}

\begin{definicija}
    \emph{Cantorjeva množica} je množica
    \[
        C = \bigcap_{n \in \N_0} C_n.
    \]
\end{definicija}