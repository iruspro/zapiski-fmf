\section{Uvod}
Intuitivno je dimenzija število, ki za dano množico pove, koliko prostora ta zavzema znotraj ambientnega prostora. Za običajne geometrijske objekte, kot so točke, daljice ali ploskve, se ta predstava dobro sklada z našim občutkom: točka ima dimenzijo \(0\), daljica \(1\), kvadrat \(2\) in kocka \(3\).

Vendar obstajajo množice, ki kljubujejo tem klasičnim predstavam -- imenujemo jih \emph{fraktali}. V svojem temeljnem eseju \cite{zbMATH03794378} je Benoit Mandelbrot fraktale definiral kot množice, katerih Hausdorffova dimenzija (s katero se bomo seznanili v nadaljevanju) je strogo večja od njihove topološke dimenzije, ki je vedno nenegativno \emph{celo} število. Formalno si lahko topološko dimenzijo predstavljamo tako: množica ima dimenzijo \(n \in \N_0\), če lahko v njeni notranjosti vsak del ločimo z robovi, ki so za eno dimenzijo ">nižji"< -- na primer, ravnino, ki ima topološko dimenzijo \(2\), lahko ločimo z daljicami, ki imajo topološko dimenzijo \(1\), daljico z točkami, ki imajo topološko dimenzijo \(0\), itd.

Za občutek si oglejmo enega izmed najbolj znanih primerov fraktalne množice -- Cantorjevo množico (slika \ref{fig:cantor-set}). Dobimo jo tako, da z začetnega intervala \([0,1]\) v vsakem koraku postopka odstranimo odprto srednjo tretjino vsakega preostalega intervala. S tem dobimo množico, ki nima dolžine, a vseeno vsebuje neštevno mnogo točk. Zdi se, da ">zavzema več kot nič, a manj kot eno dimenzijo"<, torej ima \emph{necelo} dimenzijo med \(0\) in \(1\) -- in ta pojav želimo matematično razumeti.

\begin{figure}[ht]
    \centering
    \drawCantor{7}
    \caption{Cantorjeva množica.}
    \label{fig:cantor-set}
\end{figure}

Naravno se pojavi vprašanje: kako definirati dimenzijo za takšne množice, ki ne sodijo v klasične kategorije?

V tej nalogi bomo odgovorili prav na to vprašanje. Naše izhodišče bo formalna definicija Hausdorffove dimenzije, ki jo bomo skrbno razvili in utemeljili. Nato bomo raziskali njene osnovne lastnosti in pokazali, zakaj gre za naravno posplošitev običajnega pojma dimenzije. Med drugim bomo izračunali dimenzijo Cantorjeve množice in za res pokazali, da ima necelo dimenzijo, ki je strogo večja od \(0\).

Večina vsebine v tem članku je povzeta po knjigi \emph{Fractal Geometry: Mathematical Foundations and Applications} avtorja Kennetha Falconerja \cite{fk-fg}, ki predstavlja temeljni vir za razumevanje Hausdorffove mere, fraktalnih dimenzij ter povezanih konceptov.