\newcommand{\comp}[1]{#1^{\text{c}}}
\newcommand{\mc}[1]{\mathcal{#1}}
\newcommand{\borel}[1]{\mathcal{B}(#1)}

%%% Množice
\newcommand{\set}[1]{\left\{#1\right\}}
\newcommand{\setb}[2]{\set{#1 \ | \  #2}}

%%% Unije in preseke
\newcommand{\un}[2]{\bigcup_{#1} #2}
\newcommand{\pr}[2]{\bigcap_{#1} #2}

%%% Logika
\newcommand{\all}[1]{\forall #1 \,.\,}
\newcommand{\some}[1]{\exists #1 \,.\,}
\newcommand{\lthen}{\implies}
\newcommand{\liff}{\iff}

%%% Other
\newcommand{\todo}[1]{\textcolor{red}{TODO: #1}}

%%% math commands
\DeclareMathOperator{\vol}{vol}
\DeclareMathOperator{\diam}{diam}
%%% custom macros
\newcommand{\drawGraf}{
  \begin{tikzpicture}
    \begin{axis}[
        axis lines=middle,
        xlabel style={at={(ticklabel* cs:1)}, anchor=north},  
        ylabel style={at={(ticklabel* cs:1)}, anchor=east}, 
        xlabel={\(p\)},
        ylabel={\(\mc{H}_A\)},
        ymin=0, ymax=4.5,
        xmin=0, xmax=5,
        samples=100,
        domain=0:5,
        xtick={2},
        xticklabels={$p_0$},
        ytick={0},
        enlargelimits,
        every axis plot/.append style={thick, blue},
        axis line style={->},
      ]

      \node[left] at (axis cs:0,4.5) {$\infty$};
  
      \draw[->, thick, blue] (axis cs:0,4.5) -- (axis cs:1.98,4.5);
  
      \draw[dashed, blue] (axis cs:2,0) -- (axis cs:2,4.4);
  
      \node[circle, fill=blue, inner sep=1.5pt] at (axis cs:2,3) {};
  
      \draw[<-, thick, blue] (axis cs:2.02,0) -- (axis cs:6,0);
  
    \end{axis}
  \end{tikzpicture}
}

\newcommand{\drawCantor}[1]{
    \begin{tikzpicture}[xscale=10]
        \draw (0,0) -- (1,0) node[pos=0, anchor=east]{};  % initial 𝐶₀
        \tikzmath{ % save arrays of left endpoints
          int \myN, \numInt, \i, \j, \maxMumIdx, \maxSonIdx, \sonIdx1, \sonIdx2;
          real \oldarr, \newarr, \intWidth, \rowsep, \vpos;
          \myN = #1;  % number of Cantor iterations (must be positive)
          \oldarr0 = 0;  % 𝐶ₙ₋₁
          \rowsep = 0.7;  % visual parameter
          \intWidth = 1;  % use recursive multiplication
          \maxSonIdx = 0;  % to avoid exponentiation
          \vpos = 0;  % vertical position
          for \i in {1,...,\myN} {
            \maxMumIdx = \maxSonIdx;  % max "parent" index
            \maxSonIdx = 2 * (\maxSonIdx + 1) - 1;  % #intervals in 𝐶ₙ - 1
            \intWidth = \intWidth / 3;  % length(𝐶ₙ) = (1/3)ⁿ
            \vpos = \vpos - \rowsep;
            % { \node[anchor=east] at (0, \vpos cm) {$C_{\i}$}; };
            for \j in {0,...,\maxMumIdx} {
              \sonIdx1 = 2 * \j;
              \sonIdx2 = \sonIdx1 + 1;
              \newarr{\sonIdx1} = \oldarr{\j};
              \newarr{\sonIdx2} = \oldarr{\j} + 2 * \intWidth;
              {
                \draw (\newarr{\sonIdx1} cm, \vpos cm) -- ++(\intWidth cm, 0);
                \draw (\newarr{\sonIdx2} cm, \vpos cm) -- ++(\intWidth cm, 0);
              };
            };
            % oldarr ← newarr
            for \j in {0,...,\maxSonIdx} { \oldarr{\j} = \newarr{\j}; };
          };
        }
    \end{tikzpicture}
}