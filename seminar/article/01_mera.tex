\section{Teorija mere}
Če želimo govoriti o fraktalni dimenziji, moramo najprej razumeti pojem mere. 
Za naš namen pa bo dovolj, da se seznanimo le z osnovnimi idejami tega področja. 
V nadaljevanju bomo obravnavali le mere na prostoru \(\R^n\), saj nas zanima vedenje množic v evklidskem prostoru. Mera je v tem kontekstu način, kako opišemo ">velikost"< množice -- ne nujno v klasičnem smislu dolžine, ploščine ali prostornine, temveč tudi za bolj zapletene in nepravilne množice, kot so fraktali.

\subsection{\(\sigma\)-algebra}
Da bi lahko govorili o meri, potrebujemo nekaj osnovnih pojmov. Eden izmed njih je \(\sigma\)-algebra.

\begin{definicija}
    Naj bo \(X\) množica. Družino podmnožic \(\mc{A}\) množice \(X\) imenujemo \emph{\(\sigma\)-algebra} na \(X\), če ima naslednje tri lastnosti:
    \begin{enumerate}
        \item \(X \in \mc{A}\);
        \item za vsako podmnožico \(S \in \mc A\) je tudi \(\comp{S} \in \mc{A}\);
        \item za vsako števno družino \((A_i)_{i \in \N}\) elementov iz \(\mc{A}\) velja, da je tudi unija \(\un{i \in \N}{A_i}\) v \(\mc{A}\).
    \end{enumerate}
    Elemente družine \(\mc{A}\) imenujemo \emph{merljive množice}. Množico \(X\), opremljeno z družino \(\mc{A}\), pa imenujemo \emph{merljiv prostor}.
\end{definicija}

\begin{opomba}
    Enostavno je videti, da za vsako \(\sigma\)-algebro \(\mc{A}\) na \(X\) velja \(\emptyset \in \mc{A}\) ter da je zaprta tudi za števne preseke.
\end{opomba}

\begin{zgled}
    Naj bo \(X\) poljubna množica.
    \begin{enumerate}
        \item Družina z dvema elementoma \(\set{\emptyset, X}\) je \(\sigma\)-algebra, vsebovana v vsaki \(\sigma\)-algebri na \(X\).
        \item Potenčna množica \(\mc{P}(X)\) je \(\sigma\)-algebra, ki vsebuje vsako \(\sigma\)-algebro na \(X\).
    \end{enumerate}
\end{zgled}

\begin{zgled}
    Družina \(\mc{A} = \set{\emptyset, \set{1,3,5}, \set{2, 4, 6}, \set{1, 2, 3, 4, 5, 6}}\) je \(\sigma\)-algebra na \(X = \set{1, 2, 3, 4, 5, 6}\).
\end{zgled}

\begin{trditev}
    \label{set-minus}
    Naj bo \(\mc{A}\) \(\sigma\)-algebra na množici \(X\). Naj bosta \(A, B \in \mc{A}\). Potem \(B \setminus A \in \mc{A}\).
\end{trditev}

\begin{proof}
    \(B \setminus A = B \cap \comp{A}\).
\end{proof}

\subsection{Borelove množice}
Kasneje bomo videli, da sta zunanja Hausdorffova mera in zunanja Lebesgueova mera, ki bosta za nas pomembni, sprva definirani na vseh podmnožicah prostora, vendar postaneta pravi meri šele, ko ju omejimo na Borelove množice. Vse množice, ki jih bomo obravnavali, bodo Borelove.

\begin{definicija}
    Naj bo \(\mc{B}\) poljubna družina podmnožic dane množice \(X\). Presek vseh tistih \(\sigma\)-algeber na \(X\), ki vsebujejo družino \(\mc{B}\) imenujemo, \emph{\(\sigma\)-algebra, generirana z \(\mc{B}\)}.
\end{definicija}

Preden nadaljujemo, se vprašajmo: zakaj je presek družine \(\sigma\)-algeber na \(X\) spet \(\sigma\)-algebra na \(X\)? Odgovor na to vprašanje nam da naslednja trditev.
\begin{trditev}
    Naj bo \((\mc{A}_\lambda)_{\lambda \in \Lambda}\) neprazna družina \(\sigma\)-algeber na množice \(X\), potem je tudi presek \(\mc{A} := \pr{\lambda \in \Lambda}{\mc{A}_\lambda}\) \(\sigma\)-algebra na \(X\).
\end{trditev}

% \begin{proof}
%     Naj bo XX poljubna množica ter (\mcAλ)λ∈Λ(\mcAλ​)λ∈Λ​ neprazna družina σσ-algeber na XX. Preveriti moramo, ali presek \mcA\mcA zadošča trem aksiomom σσ-algebre:
%     \begin{enumerate}
%     \item Ker za vsako λ∈Λλ∈Λ velja X∈\mcAλX∈\mcAλ​, sledi tudi X∈\mcAX∈\mcA.
%     \item Če je A∈\mcAA∈\mcA, potem A∈\mcAλA∈\mcAλ​ za vsak λ∈Λλ∈Λ. Ker so vse \mcAλ\mcAλ​ σσ-algebre, sledi \compA∈\mcAλ\compA∈\mcAλ​ za vsak λλ, torej \compA∈\mcA\compA∈\mcA.
%     \item Naj bo (Ai)i∈N(Ai​)i∈N​ števna družina množic iz \mcA\mcA. Potem za vsak ii in vsak λ∈Λλ∈Λ velja Ai∈\mcAλAi​∈\mcAλ​. Ker so \mcAλ\mcAλ​ σσ-algebre, velja tudi ⋃i∈NAi∈\mcAλ⋃i∈N​Ai​∈\mcAλ​ za vsak λλ, torej ⋃i∈NAi∈\mcA⋃i∈N​Ai​∈\mcA. \qedhere
%     \end{enumerate}
%     \end{proof}

\begin{proof}
    Naj bo \(X\) poljubna množica ter \((\mc{A}_\lambda)_{\lambda \in \Lambda}\) poljubna neprazna družina \(\sigma\)-algeber na \(X\).   
    Preveriti moramo tri aksiome.
    \begin{enumerate}
        \item Ker za vsak \(\lambda \in \Lambda\) velja \(X \in \mc{A}_\lambda\), sledi tudi \(X \in \mc{A}\).
        \item Če je \(A \in \mc A\) potem \(A \in \mc{A}_\lambda\) za vsak \(\lambda \in \Lambda\). Ker so \(\mc{A}_\lambda\) \(\sigma\)-algebre, sledi \(\comp{A} \in \mc{A}_\lambda\) za vsak \(\lambda \in \Lambda\), torej  \(\comp{A} \in \mc{A}\).
        \item Naj bo \((A_i)_{i \in \N}\) števna družina elementov iz \(\mc{A}\). Potem za vsak \(i \in \N\) in vsak \(\lambda \in \Lambda\) velja \(A_i \in \mc{A_\lambda}\). Ker so \(\mc{A}_\lambda\) \(\sigma\)-algebre, velja tudi \(\un{i \in \N}{A_i} \in \mc{A}_\lambda\) za vsak \(\lambda \in \Lambda\), torej \(\un{i \in \N}{A_i} \in \mc{A}\).
        \qedhere
    \end{enumerate}
\end{proof}

\begin{posledica}
    \label{min-alg}
    Naj bo \(\mc{B}\) družina podmnožic dane množice \(X\). \(\sigma\)-algebra, generirana z \(\mc{B}\) je najmanjša \(\sigma\)-algebra na \(X\), ki vsebuje \(\mc{B}\).
\end{posledica}

Zdaj lahko definiramo množice, ki jih imenujemo Borelovi.

\begin{definicija}
    Naj bo \(X\) topološki prostor in \(\mc{O}\) družina vseh odprtih podmnožic v \(X\). \(\sigma\)-algebro, generirano z \(\mc{O}\), imenujemo \emph{Borelova \(\sigma\)-algebra}, njene elemente pa \emph{Borelove množice}. Označili jo bomo z \(\borel{X}\).
\end{definicija}

\begin{opomba}
    Iz posledice \ref{min-alg} sledi, da je \(\borel{X}\) najmanjša \(\sigma\)-algebra, ki vsebuje vse odprte in vse zaprte podmnožice \(X\).
\end{opomba}

\subsection{Mera}
Zgled za mero je dolžina podmnožic v \(\R\), ploščina ravninskih likov in prostornina teles v prostoru. Ta pojem želimo posplošiti na poljubne merljive prostore.

\begin{definicija}
    \emph{Mera} na merljivem prostoru \((X, \mc{A})\) je funkcija 
    \[\mu: \mc{A} \to [0, \infty],\]
    ki zadošča naslednjima pogojema
    \begin{enumerate}
        \item \(\mu(\emptyset) = 0\) in
        \item \(\mu\left(\un{n \in \N}{A_n}\right) = \sum_{n=1}^{\infty} \mu(A_n)\) za vsako števno družino disjunktnih množic \(A_n \in \mc A\).
    \end{enumerate}
    Drugemu pogoju pravimo \emph{števna aditivnost}.
\end{definicija}

Dokažimo nekaj preprostih lastnosti mere, ki jih bomo potrebovali v nadaljevanju.
\begin{trditev}[Monotonost mere]
    \label{mera-mon}
    Vsaka mera \(\mu\) na merljivem prostoru \((X,\, \mc{A})\) je monotona, tj.
    \[\all{A, B \in \mc A} A \subseteq B \lthen \mu(A) \leq \mu(B).\]
\end{trditev}

\begin{proof}
    Po trditvi \ref{set-minus} je množica \(B \setminus A\) merljiva. Ker sta množici \(A\) in \(B \setminus A\) disjunktni in merljivi ter \(B = A \cup (B \setminus A)\), je 
    \[
    \mu(B) = \mu(A \cup (B \setminus A)) = \mu(A) + \mu(B \setminus A) \geq \mu(A).\]
\end{proof}

\begin{trditev}[Števna subaditivnost mere]
    Za vsako mero \(\mu\) na merljivem prostoru \((X,\, \mc{A})\) in poljubno števno družino \((A_i)_{i \in \N}\) elementov iz \(\mc{A}\) je 
    \[\mu\left(\un{i \in \N}{A_i}\right) \leq \sum_{i=1}^{\infty} \mu(A_i).\]
\end{trditev}

\begin{proof}
    Opazimo, da je \(A := \un{i \in \N}{A_i}\) unija zaporedja disjunktnih množic
    \[A_1,\ A_2 \setminus A_1,\ A_3 \setminus (A_1 \cup A_2), \ldots,\]
    ki so vse v \(\mc{A}\). Po števni aditivnosti in monotonosti sledi od tod
    \[\mu(A) = \mu(A_1) + \mu(A_2 \setminus A_1) + \mu(A_3 \setminus (A_1 \cup A_2)) + \cdots \leq \sum_{i=1}^{\infty} \mu(A_i).\]
\end{proof}

\begin{trditev}
    \label{mera-minus}
    Naj bo \(\mu\) mera na prostoru \((X, \mc{A})\). Naj bosta \(A, B \in \mc{A},\ A \subseteq B\) in \(\mu(A) < \infty\), potem \(\mu(B \setminus A) = \mu(B) - \mu(A)\).
\end{trditev}

\begin{proof}
    Množica \(B\) je disjunktna unija množic \(A\) in \(B \setminus A\), trditev sledi.
\end{proof}

\subsection{Zunanja mera}
Pojem zunanje mere je izredno pomembno orodje za konstruiranje mer na splošnih množicah. V nadaljevanju bomo Lebesgueovo in Hausdorffovo mero definirali prek njunih zunanjih mer.

\begin{definicija}
    \emph{Zunanja mera} na množici \(X\) je preslikava 
    \[\mu^*: \mc{P}(X) \to [0, \infty],\]
    ki zadošča naslednjim trem pogojem:
    \begin{enumerate}
        \item \(\mu^*(\emptyset) = 0\);
        \item \(\mu^*(A) \leq \mu^*(B)\), če je \(A \subseteq B\);
        \item \(\mu^*(\un{n \in \N}{A_n}) \leq \sum_{n=1}^{\infty} \mu^*(A_n)\) za vsako števno družino množic \(A_n \subseteq X\).
    \end{enumerate}
\end{definicija}

Navedemo trditev, ki nam bo koristila pri konstrukciji Hausdorffove mere. Dokaz trditve je mogoče najti v \cite[stran 20]{mb-otm}.
\begin{trditev}
    \label{zun-mera}
    Naj bo \(\mc{S}\) družina podmnožic množice \(X\), ki vsebuje \(\emptyset\) in \(X\). Naj bo \(\mu: \mc{S} \to [0, \infty]\) preslikava, za katero velja \(\mu(\emptyset) = 0\). Za vsako podmnožico \(A \subseteq X\) definiramo 
    \[\mu^*(A) = \inf \setb{\sum_{i=1}^{\infty}\mu(A_i)}{A_i \in \mc{S} \land A \subseteq \un{i \in \N}{A_i}}.\]
    Potem \(\mu^*\) je zunanja mera na \(X\).
\end{trditev}

\subsection{Lebesgueova mera}
Lebesgueova mera je posplošitev klasičnih pojmov, kot so dolžina, ploščina in prostornina, na mnogo širši razred množic. Ključna prednost Lebesgueove mere je, da omogoča merjenje zelo nepravilnih množic, ki jih ni mogoče zadovoljivo obravnavati z geometrijskimi metodami. V nadaljevanju bomo videli, da je tesno povezana s Hausdorffovo mero, ki jo bomo podrobneje obravnavali kasneje.

Natančna konstrukcija Lebesgueove mere presega okvir tega besedila in jo lahko najdemo v \cite[poglavje 1]{mb-otm}. Tukaj pa bomo predstavili osnovno idejo in rezultat, ki ga bomo uporabljali v nadaljevanju.

\begin{definicija}
    Za vsak interval \(I = (a, b) \subseteq \R\) definiramo \(l(I) := b -a\). \emph{Lebesgueova zunanja mera} je preslikava \(\mc{L}_*: \mc{P}(\R) \to [0, \infty]\) s predpisom 
    \[\mc{L}_*(A) = \inf \setb{\sum_{i=1}^{\infty} l(I_i)}{A \subseteq \un{i \in \N}{I_i}},\]
    kjer je \((I_i)_{i \in \N}\) števna družina odprtih intervalov v \(\R\).
\end{definicija}

To definicijo lahko naravno posplošimo tudi na višji dimenziji. Za vsak kvader \(K = I_1 \times \cdots \times I_n \subseteq \R^n\) z \(\vol(K) = l(I_1) \cdot \cdots \cdot l(I_n)\) definiramo njegovo prostornino.

\begin{definicija}
    Naj bo \(n \in \N\). \emph{Lebesgueova zunanja \(n\)-dimenzionalna mera} je preslikava \(\mc{L}_*^n: \mc{P}(\R^n) \to [0, \infty]\) s predpisom 
    \[\mc{L}_*^n(A) = \inf \setb{\sum_{i=1}^{\infty} \vol(K_i)}{A \subseteq \un{i \in \N}{K_i}},\]
    kjer je \((K_i)_{i \in \N}\) števna družina odprtih kvadrov v \(\R^n\).
\end{definicija}

\begin{trditev}
    Zožitev Lebesgueve zunanje \(n\)-dimenzionalne mere na Borelovo \(\sigma\)-algebro
    \[\mc{L}^n := \mc{L}_*^n|_{\mc{B}(\R^n)}\] je mera na merljivem prostoru \((\R^n,\, \mc{B}(\R^n))\).
\end{trditev}

\begin{zgled}
    Izračunajmo dolžino Cantorjeve množice \(C\). 
    
    Najprej opazimo, da je Cantorjeva množica zaprta, ker je števen presek zaprtih množic. Torej je Borelova in s tem merljiva.
    
    Oglejmo si \(\mc{L}^1([0,1] \setminus C)\). 
    Po eni strani \(([0,1] \setminus C) \subseteq [0, 1]\), zato po monotonosti mere (glej trditev~\ref{mera-mon}) sledi:
    \[\mc{L}^1([0,1] \setminus C) \leq \mc{L}^1[0,1] = 1.\] 
    Po drugi strani pa množica izrezanih odprtih intervalov (označimo jo z \(Z\)) je vsebovana v \([0,1] \setminus C\). Da pokrijemo množico \(Z\), moramo pokriti vse izrezane intervale, katerih dolžina je
    \[\frac{1}{3} + 2 \cdot \frac{2}{9} + 4 \cdot \frac{1}{81} + \cdots = \frac{1}{3}\left(1 + \frac{2}{3} + \left(\frac{2}{3}\right)^2 + \cdots \right) = 1.\]
    Ker je po definiciji \(\inf\) največja spodnja meja množice sledi, da \(\mc{L}^1(Z) \geq 1\). Ker je \(Z \subseteq ([0,1] \setminus C)\), po monotonosti sledi:
    \[1 \leq \mc{L}^1(Z) \leq \mc{L}^1([0,1] \setminus C).\]
    Torej je  \(\mc{L}^1([0,1] \setminus C) = 1\). 
    
    Po trditvi \ref{mera-minus} sledi:
    \[\mc{L}^1(C)  = \mc{L}^1[0,1] - \mc{L}^1([0,1] \setminus C) = 0.\]
\end{zgled}

Ugotovili smo, da Cantorjeva množica nima dolžine, torej ni \(1\)-dimenzionalna, vendar pa vsebuje neštevno mnogo točk, zato tudi ni \(0\)-dimenzionalna. Lahko se vprašamo: ali ji lahko pripišemo takšno smiselno dimenzijo, v kateri bo njena ">velikost"< končno, pozitivno število? In kaj nam ta dimenzija sploh pove o naravi Cantorjeve množice?