\section{Škatlasta dimenzija}
Osnovna ideja vseh definicij dimenzije temelji na principu ">meritve pri skali \(\delta > 0\)"<:
za dano pozitivno vrednost \(\delta\) množico merimo tako, da prezremo podrobnosti, manjše od \(\delta\), in nato opazujemo vedenje, ko \(\delta \to 0\).
Ali je neka definicija smiselna in uporabna, pokažeta praksa in matematična intuicija.
Pričakujemo pa tudi, da bodo za tako definirano dimenzijo večinoma veljale naravne lastnosti, omenjene v uvodu.

Škatlasta dimenzija je ena najbolj priljubljenih, saj jo lahko pogosto preprosto izračunamo ali vsaj numerično ocenimo. Kljub temu pa ima tudi svoje pomanjkljivosti, o katerih bomo govorili kasneje.

\cite{matrika-fraktalna-dimenzija} Idejo za definicijo škatlaste dimenzije dobimo iz naslednjega razmisleka.

Naj bo \(X \subseteq \R^n\). Prostor \(\R^n\) razdelimo na enakomerno mrežo kock s stranico \(\delta > 0\), in nato preštejemo število tistih kock, ki sekajo množico \(X\). Pričakujemo, da bo število teh kock, označeno z \(N_\delta(X)\), v odvisnosti od \(\delta\), odražalo ">velikost"< množice \(X\).
Na primer:
\begin{itemize}
    \item Enotski interval \(K_1 = [0,1]\) lahko pokrijemo s \(\frac{1}{\delta}\) intervali dolžine \(\delta\).
    \item Enotski kvadrat \(K_2 = [0,1]^2\) potrebujemo \(\frac{1}{\delta^2}\) kvadratov stranice \(\delta\).
    \item Enotsko kocko \(K_3 = [0,1]^3\) pokrijemo z \(\frac{1}{\delta^3}\) kockami s stranico \(\delta\).
\end{itemize}
Opazimo torej, da za množico dimenzije \(D\) približno velja:
\[
N_\delta(X) \approx \frac{1}{\delta^D}.
\]
Logaritmiranje te zveze pripelje do formule:
\[
D \approx \frac{\log N_\delta(X)}{\log(1/\delta)}.
\]

To razmerje med številom potrebnih kock in njihovo velikostjo vodi k definiciji škatlaste dimenzije:

\begin{definicija}
    Naj bo \(A \subseteq \R^n\) omejena in neprazna množica. Naj bo \(\delta > 0\). Označimo z \(N_\delta(A)\) najmanjše število množic z diametrom kvečjemu \(\delta\), ki pokrivajo množico \(A\).
    \begin{itemize}
        \item \emph{Spodnja škatlasta dimenzija množice \(A\)} je 
        \[
        \underline{\dim}_B A = \liminf_{\delta \to 0} \frac{\log N_\delta(A)}{\log(1/\delta)}.
        \]
        \item \emph{Zgornja škatlasta dimenzija množice \(A\)} je 
        \[
        \overline{\dim}_B A = \limsup_{\delta \to 0} \frac{\log N_\delta(A)}{\log(1/\delta)}.
        \]        
    \end{itemize}
    Če velja \(\underline{\dim}_B A = \overline{\dim}_B A\), potem se skupna vrednost
    \[
    \dim_B A = \lim_{\delta \to 0} \frac{\log N_\delta(A)}{\log(1/\delta)}
    \] 
    imenuje \emph{škatlasta dimenzija množice \(A\)}.
\end{definicija}

\begin{opomba} \ 
    \begin{itemize}
        \item Predpostavljamo, da je \(\delta > 0\) dovolj majhen, da je \(\log(1/\delta) > 0\).
        \item Da se izognemo težavam s \(\log \infty\) ali \(\log 0\), definiramo dimenzijo le za neprazne in omejene množice.
    \end{itemize}
\end{opomba}

\subsection{Ekvivalentne definicije škatlaste dimenzije}
V tem razdelku bomo predstavili nekatere ekvivalentne definicije škatlaste dimenzije, ki so pogosto bolj uporabne in prijazne za praktično delo.

Naj bo 
\[\mc{K} = \setb{[m_1\delta, (m_1 + 1)\delta] \times \cdots \times [m_n\delta, (m_n + 1)\delta]}{m_1, \ldots, m_n \in \Z}\]
množica kvadrov v \(\delta\)-mreži prostora \(\R^n\). 

\begin{trditev}
    Naj bo \(A \subseteq \R^n\) omejena in neprazna. Označimo z \(N_\delta'(A)\) število elementov množice \(\mc{K}\), ki sekajo \(A\). Definirajmo
    \[
    \dim_{B'} A = \lim_{\delta \to 0} \frac{\log N_\delta'(A)}{\log(1/\delta)}.
    \]
    Tedaj velja
    \[
    \dim_B A = \dim_{B'} A.
    \] 
\end{trditev}

\begin{proof}
    Naj bo \(0 < \delta < 1\). Množica kvadrov, ki sekajo množico \(A\), določa pokritje \(A\) s kockami premera največ \(\delta \sqrt{n}\), zato velja
    \[N_{\delta\sqrt{n}}(A) \leq N_\delta'(A) \lthen \log N_{\delta\sqrt{n}}(A) \leq \log N_\delta'(A).\]
    Ker je \(\log(1/\delta) > 0\), dobimo
    \[
        \lim_{\delta \to 0} \frac{\log N_{\delta\sqrt{n}}(A)}{\log(1/\delta)} \leq \lim_{\delta \to 0} \frac{\log N_\delta'(A)}{\log(1/\delta)},
    \]
    kar pomeni
    \[\dim_B A \leq \dim_{B'} A\]

    Po drugi strani, naj bo \((U_i)_{i \in \N}\) poljubno \(\delta\)-pokritje množice \(A\). Vsako množico \(U_i\) lahko pokrijemo z največ \(3^n\) kvadri iz \(\delta\)-mreže (ker je \(U_i\) premera največ \(\delta\), zadostuje razširitev po eno kocko v vsako smer). Torej velja
    \[
    N_\delta'(A) \leq 3^n N_\delta (A) \lthen \log N_\delta'(A) \leq n \log 3 + \log N_\delta (A).
    \]
    Sledi
    \[
        \lim_{\delta \to 0} \frac{\log N_\delta'(A)}{\log(1/\delta)} \leq \lim_{\delta \to 0} \frac{n \log 3 + \log N_\delta (A)}{\log(1/\delta)} = \lim_{\delta \to 0} \frac{\log N_\delta (A)}{\log(1/\delta)},
    \]
    torej
    \[\dim_B A \geq \dim_{B'} A.\]

    Skupaj s prvo neenakostjo sledi enakost.
\end{proof}

Ta verzija definicije je zelo uporabna, saj lahko narišemo mrežo in preštejemo število kock, ki sekajo množico. Zdaj je tudi jasno, zakaj se ta dimenzija imenuje škatlasta.

Podobno lahko dokažemo, da za \(N_\delta(A)\) lahko vzamemo najmanjše število poljubnih kvadrov z stranico dolžine \(\delta\), ki pokrivajo množico \(A\) ali najmanjše število zaprtih krogel z radijem \(\delta\), ki pokrivajo množico \(A\).

Ni povsem očitno, da lahko količino \(N_\delta(A)\) izrazimo tudi kot največje število paroma disjunktnih zaprtih krogel z radijem \(\delta\), katerih središča ležijo v množici \(A\). To dejstvo bomo utemeljili v naslednji trditvi.

\begin{trditev}
    Naj bo \(A \subseteq \R^n\) omejena in neprazna. Označimo z \(N_\delta'(A)\) največje število paroma disjunktnih zaprtih krogel z radijem \(\delta\), katerih središča ležijo v množici \(A\). Definirajmo
    \[
    \dim_{B'} A = \lim_{\delta \to 0} \frac{\log N_\delta'(A)}{\log(1/\delta)}.
    \]
    Tedaj velja
    \[
    \dim_B A = \dim_{B'} A.
    \] 
\end{trditev}

\begin{proof}
    Naj bo \(0 < \delta < 1\). Naj bo \(B_1, \ldots B_{N_\delta'(A)}\) družina paroma disjunktnih zaprtih krogel z radijem \(\delta\) in središči v množici \(A\). Trdimo, da množica krogel \(B_i'\), ki imajo ista središča kot \(B_i\), vendar radij \(2\delta\), tvori \(4\delta\)-pokritje množice \(A\). Namreč, če bi obstajala točka \(x \in A\), ki ni vsebovana v nobeni izmed krogel \(B_i'\), potem bi lahko okoli \(x\) postavili novo disjunktno kroglo z radijem \(\delta\) in s tem povečali družino \(B_i\), kar bi bilo v nasprotju z maksimalnostjo \(N_\delta'(A)\). Torej:
    \[
    N_{4\delta}(A) \leq N_\delta'(A).
    \]

    Po drugi strani, naj bo \((U_i)_{i \in \N}\) poljubno \(\delta\)-pokritje množice \(A\). Vsaka od disjunktnih krogel \(B_i\) mora vsebovati vsaj eno množico \(U_j\), saj pokritje pokriva središča teh krogel, in ker so krogle disjunktne, so to tudi pripadajoče množice \(U_j\). Torej velja:
    \[
    N_\delta'(A) \leq N_\delta(A).
    \]

    S tem smo dobili neenakosti:
    \[
    N_{4\delta}(A) \leq N_\delta'(A) \leq N_\delta(A).
    \]
    V limiti dobimo želeno enakost.
\end{proof}


Povzemimo
\begin{definicija}
    Naj bo \(A \subseteq \R^n\).
    \emph{Spodnja in zgornja škatlasta dimenzija množice \(A\)} podani kot 
    \[
    \underline{\dim}_B A = \liminf_{\delta \to 0} \frac{\log N_\delta(A)}{\log(1/\delta)}
    \]
    in 
    \[
    \overline{\dim}_B A = \limsup_{\delta \to 0} \frac{\log N_\delta(A)}{\log(1/\delta)}.
    \]     
    \emph{Škatlasta dimenzija množice \(A\)} obstaja, če obstaja limita
    \[
    \dim_B A = \lim_{\delta \to 0} \frac{\log N_\delta(A)}{\log(1/\delta)},
    \] 
    kjer je \(N_\delta(A)\) lahko katera koli izmed naslednjih količin:
    \begin{itemize}
        \item najmanjše število zaprtih krogel z radijem \(\delta\), ki pokrivajo množico \(A\);
        \item najmanjše število kvadrov s stranico \(\delta\), ki pokrivajo množico \(A\);
        \item število kvadrov iz \(\delta\)-mreže, ki sekajo množico \(A\);
        \item najmanjše število množic s premerom največ \(\delta\), ki pokrivajo množico \(A\);
        \item največje število paroma disjunktnih zaprtih krogel z radijem \(\delta\), katerih središča ležijo v množici \(A\).
    \end{itemize}
\end{definicija}

Ta seznam bi lahko še nadaljevali. V praksi pa izberemo tisto definicijo, ki je najbolj primerna za dano množico.

\begin{zgled}
    Škatlasta dimenzija Cantorjeve množice \(C\) je 
    \[\dim_B C = \log_3 2.\]
    Podroben izračun lahko najdemo v \cite[stran 47]{fk-fg}.
\end{zgled}

\subsection{Relacija med Hausdorffovo in škatlasto dimenzijo}
Pomembno je razumeti relacijo med Hausdorffovo in škatlasto dimenzijo.

Naj bo \(A \subseteq \R^n\) omejena in neprazna množica. Če lahko množico \(A\) pokrijemo z \(N_\delta(A)\) množicami s premerom kvečjemu \(\delta\), potem po definiciji \ref{def-haus-mera} velja
\[
    H^s_\delta(A) \leq N_\delta(A) \delta^s.
\]

Če je \(c = \mathcal{H}^s(A) = \lim_{\delta \to 0} H^s_\delta(A) > 0\), potem za dovolj majhne \(\delta\) velja
\[
    \log N_\delta(A) + s \log \delta > c \lthen s - \frac{c}{\log \delta} < \frac{\log N_\delta(A)}{\log (1 / \delta)}{}.
\]
V limiti \(\delta \to 0\) dobimo
\[
    s \leq \liminf_{\delta \to 0} \frac{\log N_\delta(A)}{\log (1 / \delta)},
\]
torej
\[
    \dim_H A \leq \underline{\dim}_B A \leq \overline{\dim}_B A
\]
za vsako množico \(A\).

V splošnem enakost ne velja, čeprav za veliko znanih primerov velja.

Vidimo torej, da lahko škatlasto dimenzijo uporabimo kot zgornjo oceno Hausdorffove dimenzije.


\subsection{Lastnosti in slabosti škatlaste dimenzije}
V tem razdelku si bomo ogledali lastnosti in slabosti škatlaste dimenzije.

\subsubsection{Lastnosti škatlaste dimenzije}
Škatlasta dimenzija ima naslednje lastnosti, ki jih lahko dokažemo podobno kot v primeru Hausdorffove dimenzije. Podrobni dokazi so na voljo v \cite[stran 47]{fk-fg}.

\begin{itemize}
    \item Če je \(M \subseteq \R^n\) gladka podmnogoterost dimenzije \(m \in \N\), potem 
    \[\dim_B M = m.\]
    \item Spodnja in zgornja škatlasta dimenzija sta monotoni.
    \item Zgornja škatlasta dimenzija je končno stabilna, tj.
    \[\overline{\dim}_B (A \cup B) = \max (\overline{\dim}_B A,\ \overline{\dim}_B  B).\]
    Ta enakost v splošnem ne velja za spodnjo škatlasto dimenzijo.
    \item Spodnja in zgornja škatlasta dimenzija sta invariantni pri bi-Lipschit\-zevih preslikavah.
\end{itemize}

Podobno kot Hausdorffova dimenzija ima tudi škatlasta dimenzija predvidljivo obnašanje pri H\"olderjevih in Lipschitzevih preslikavah.


\subsubsection{Slabosti škatlaste dimenzije}
Škatlasta dimenzija je zelo uporabna, vendar ima tudi določene pomanjkljivosti. Ena izmed njih je naslednja:

\begin{trditev}
    \label{slabost}
    Naj bo \(A \subseteq \R^n\) omejena in neprazna množica. Tedaj velja 
    \[
        \overline{\dim}_B A = \overline{\dim}_B \overline{A} \quad \text{in} \quad \underline{\dim}_B A = \underline{\dim}_B \overline{A}
    \]
\end{trditev}

\begin{proof}
    Naj bo \(B_1, \ldots, B_k\) končna družina zaprtih krogel z radijem \(\delta\), ki pokrivajo množico \(A\), tj.
    \[
        A \subseteq \bigcup_{i=1}^k B_i.
    \]
    Ker so krogle zaprte, je tudi 
    \[
        \overline{A} \subseteq \bigcup_{i=1}^k B_i,
    \]
    saj zaprtje je najmanjša zaprta množica, ki vsebuje \(A\). Zato je najmanjše število zaprtih krogel z radijem \(\delta\), ki pokrivajo \(A\), enako najmanjšemu številu zaprtih krogel z radijem \(\delta\), ki pokrivajo \(\overline{A}\). Trditev sledi.
\end{proof}

Kot posledico na prvi pogled ugledne lastnosti dobimo:
\begin{posledica}
    Škatlasta dimenzija v splošnem ni števno stabilna.
\end{posledica}

\begin{proof}
    Oglejmo si števno množico \(A = [0, 1] \cap \Q\).
    Ker je \(A\) števna, jo lahko zapišemo v obliki
    \[A = \set{a_1, a_2, \ldots}.\]
    Za vsak \(i \in \N\) definirajmo \(A_i = \{a_i\}\). Tedaj velja
    \[
        A = \bigcup_{i \in \N} A_i.
    \]
    Po eni strani je \(\dim_B A_i = 0\) za vsak \(i \in \N\). Po drugi strani pa je \(\overline{A} = [0, 1]\) in po trditvi~\ref{slabost} sledi
    \[
        \dim_B A = \dim_B \overline{A} = \dim_B [0, 1] = 1.
    \]
    Torej
    \[
        \dim_B\left( \bigcup_{i \in \N} A_i \right) = 1 \neq \sup\left\{ \dim_B A_i \mid i \in \N \right\} = 0.
    \]
    S tem pokažemo, da škatlasta dimenzija ni števno stabilna.
\end{proof}

Lahko pa se vseeno vprašamo, ali je škatlasta dimenzija morda števno stabilna na zaprtih ali kompaktnih množicah. Odgovor je v splošnem negativen, kar pokaže naslednji zgled.

\begin{zgled}
    Naj bo \(A = \setb{\frac{1}{n}}{n \in \N} \cup \set{0}\). Vemo, da je \(A\) kompaktna množica. Trdimo, da je 
    \[\dim_B A = \frac{1}{2} \neq 0.\]
    
    Naj bo \(0 < \delta < \frac{1}{2}\). Izberimo \(k \in \N\), da velja
    \[
    \frac{1}{k(k+1)} \leq \delta < \frac{1}{k(k-1)}.
    \]

    Naj bo \((U_i)_{i \in \N}\) poljubno \(\delta\)-pokritje množice \(A\). Ker je \(\diam U_i \leq \delta\), lahko vsak interval \(U_i\) pokrije največ eno točko iz množice  \(\set{1, \frac{1}{2}, \ldots, \frac{1}{k}}\), saj je razdalja med zaporednima točkama 
    \[
    \frac{1}{k-1} - \frac{1}{k} = \frac{1}{k(k-1)} > \delta.
    \]
    Torej je 
    \[
    N_\delta(A) \geq k,
    \]
    od koder sledi
    \[
    \frac{\log N_\delta(A)}{\log (1 / \delta)} \geq \frac{\log k}{\log (k(k+1))}.
    \]
    V limiti \(\delta \to 0\), oziroma \(k \to \infty\), dobimo
    \[
    \liminf_{\delta \to 0} \frac{\log N_\delta(A)}{\log (1 / \delta)} \geq \frac{1}{2}.
    \]

    Po drugi strani lahko interval \([0, \frac{1}{k}]\) pokrijemo z največ \(k+1\) intervali dolžine \(\delta\). Preostale \(k-1\) točk iz množice \(A\) lahko pokrijemo z enakim številom intervalov, torej skupaj 
    \[
    N_\delta(A) \leq 2k.
    \]
    Od tod sledi
    \[
    \frac{\log N_\delta(A)}{\log (1 / \delta)} \leq \frac{\log (2k)}{\log (k(k-1))}.
    \]
    V limiti \(\delta \to 0\), oziroma \(k \to \infty\), dobimo
    \[
    \limsup_{\delta \to 0} \frac{\log N_\delta(A)}{\log (1 / \delta)} \leq \frac{1}{2}.
    \]
    
    Skupaj s prvo neenakostjo sledi enakost.
\end{zgled}

Zdaj pa vidimo da števna stabilnost ne drži niti za kompaktne ali zaprte množice.

Kljub tej slabosti, da škatlasta dimenzija ni števno stabilna niti za zaprte ali kompaktne množice, je še vedno zelo pogosto uporabljena in uporabna dimenzija v različnih področjih matematike in aplikacij. Pogosto se tudi da pokazati, da se škatlasta dimenzija ujema z Hausdorffovo dimenzijo za nekatere množice. Zaradi svoje relativne enostavnosti izračuna in naravne interpretacije kot ">merjenje gostote"< množice pri manjših merilih je škatlasta dimenzija priljubljeno orodje pri proučevanju fraktalov in drugih kompleksnih množic.