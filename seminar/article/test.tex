\begin{zgled}
    Naj bo 
    \[
    A = \left\{ \frac{1}{n} \mid n \in \N \right\} \cup \{0\}.
    \]
    Vemo, da je \(A\) kompaktna množica. Trdimo, da je 
    \[
    \dim_B A = \frac{1}{2} \neq 0.
    \]

    Naj bo \(0 < \delta < \frac{1}{2}\). Izberimo \(k \in \N\), da velja
    \[
    \frac{1}{k(k+1)} \leq \delta < \frac{1}{k(k-1)}.
    \]

    Naj bo \((U_i)_{i \in \N}\) poljubno \(\delta\)-pokritje množice \(A\). Ker je \(\diam U_i \leq \delta\), lahko vsak interval \(U_i\) pokrije največ eno točko iz množice 
    \[
    \left\{1, \frac{1}{2}, \ldots, \frac{1}{k} \right\},
    \]
    saj je razdalja med zaporednima točkama 
    \[
    \frac{1}{k-1} - \frac{1}{k} = \frac{1}{k(k-1)} > \delta.
    \]
    Torej je 
    \[
    N_\delta(A) \geq k,
    \]
    od koder sledi
    \[
    \frac{\log N_\delta(A)}{\log (1 / \delta)} \geq \frac{\log k}{\log (k(k+1))}.
    \]
    V limiti \(\delta \to 0\), oziroma \(k \to \infty\), dobimo
    \[
    \liminf_{\delta \to 0} \frac{\log N_\delta(A)}{\log (1 / \delta)} \geq \frac{1}{2}.
    \]

    Po drugi strani lahko interval \([0, \frac{1}{k}]\) pokrijemo z največ \(k+1\) intervali dolžine \(\delta\). Preostale \(k-1\) točk iz množice \(A\) lahko pokrijemo z enakim številom intervalov, torej skupaj 
    \[
    N_\delta(A) \leq 2k.
    \]
    Od tod sledi
    \[
    \frac{\log N_\delta(A)}{\log (1 / \delta)} \leq \frac{\log (2k)}{\log (k(k-1))}.
    \]
    V limiti \(\delta \to 0\) oziroma \(k \to \infty\) dobimo
    \[
    \limsup_{\delta \to 0} \frac{\log N_\delta(A)}{\log (1 / \delta)} \leq \frac{1}{2}.
    \]

    Skupaj s prejšnjo neenakostjo sledi enakost
    \[
    \dim_B A = \frac{1}{2}.
    \]
\end{zgled}