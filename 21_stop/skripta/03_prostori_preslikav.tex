\section{Prostori preslikav}

\begin{enumerate}
    \item Topologije na prostorih preslikav
    
    Naj bosta \(X, Y\) topološka prostora
    \begin{itemize}
        \item Množica vseh preslikav iz \(A \subseteq X\) v \(U \subseteq Y\).
        \item Topologija konvergence po točkah.
        \item \textbf{Trditev.} O topologiji konvergence po točkah.
        \item \textcolor{blue}{(*)}  \textbf{Definicija.} Kompaktno-odprta topologija. Prostor zveznih funkcij, opremljen z kompaktno-odprto topologijo.
        \item \textcolor{blue}{(*)} \textbf{Trditev.} Baza \(\widehat{C}(X, Y)\), če je \(Y\) metričen prostor. Topologija enakomerne konvergence na kompaktih.
        \begin{proof} \
            \begin{enumerate}
                \item Najprej preverimo, da je to sploh baza.
                \item Kako dobimo predbazo \(\T_{co}\)? \qedhere
            \end{enumerate}
        \end{proof}
        \item \textcolor{blue}{(*)} \textbf{Trditev.} Kakšna poveza med \(Y\) in \(\widehat{C}(X, Y)\).
        \begin{proof}
            \begin{enumerate}
                \item Pokažemo, da je vložitev odprta.
                \item Pokažemo, da je slika zaprt podprostor. \qedhere
            \end{enumerate}
        \end{proof}
        \item \textcolor{blue}{(*)} \textbf{Trditev.} Povezava ločljivostih lastnosti \(Y\) in \(\widehat{C}(X, Y)\).
        \item Kaj če je domena \(X\) diskreten prostor?
    \end{itemize}

    \item Preslikave na normalnih prostorih
    \begin{itemize}
        \item Kaj so zvezne preslikave iz neskončne množice \(X\) z topologijo končnih komplementov v \(\R\)?
        \item \textcolor{red}{(*)} \textbf{Izrek.} Urisonova lema.
        \item \textcolor{green}{(*)} \textbf{Izrek.} Urisonov metrizacijski izrek.
        \item \textcolor{blue}{(*)} \textbf{Posledica.} Čemu je ekvivalentna metrizabilnost v \(2\)-števnih prostorih?
        \item \textcolor{red}{(*)} \textbf{Izrek.} Tietzejev razširitveni izrek.
    \end{itemize}

    \item Stone-Weierstrassov izrek
    \begin{itemize}
        \item \textcolor{blue}{(*)} \textbf{Izrek.} Weierstrassov izrek.
        \item \textcolor{blue}{(*)} \textbf{Izrek.} Stone-Weierstrassov izrek.
    \end{itemize}
\end{enumerate}