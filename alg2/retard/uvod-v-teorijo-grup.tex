\section{Uvod v teorijo grup}
\subsection{Osnovni pojmi teoriji grup}
\begin{definicija}
    Naj bo $S$ neprazna množica. \emph{Operacija na množice $S$} je preslikava $*: S \times S \to S, \ (a,b) \mapsto a * b$.

    Operacija $*$ je \emph{asociativna}, če $\all{a, b, c \in S} (a*b)*c = a*(b*c)$.

    Operacija $*$ je \emph{komutativna}, če $\all{a, b \in S} a* b = b* a$.
\end{definicija}

\begin{definicija}
    Neprazna množica $S$ skupaj z operacijo $*$ je \emph{polgrupa}, če je operacija $*$ asociativna.
\end{definicija}

\begin{definicija}
    Naj bo $S$ množica z operacijo $*$. Pravimo, da je $e \in S$ \emph{enota (oz. nevtralni element) za operacijo~$*$}, če~$\all{x \in S} e * x = x * e = x$.
\end{definicija}

\begin{trditev}
    Če v množici $S$ obstaja enota za operacijo $*$, potem je ena sama.
\end{trditev}

\begin{definicija}
    Polgrupa z enoto je \emph{monoid}.
\end{definicija}

\begin{definicija}
    Naj bo $S$ množica z operacijo $*$ in $e \in S$ enota. Naj bo $x \in S$.
    \begin{itemize}
        \item Element $l \in S$ je \emph{levi inverz} elementa $x$, če $l * x = e$.
        \item Element $d \in S$ je \emph{desni inverz} elementa $x$, če $x * d = e$.
        \item Element $y \in S$ je \emph{inverz} elementa $x$, če $x *y = y* x = e$.
    \end{itemize}
\end{definicija}

\begin{trditev}
    Če je $S$ monoid, $x \in S$, $l$ levi inverz $x$ ter $d$ desni inverz $x$, potem $l=d$.
\end{trditev}

\begin{definicija}
    Pravimo, da je element $x \in S$ \emph{obrnljiv}, če obstaja inverz od $x$.
\end{definicija}

\begin{definicija}
    Naj bo $S$ z operacijo $*$ monoid. Pravimo, da je $S$ \emph{grupa}, če je vsak element iz $S$ obrnljiv.

    Če je operacija $*$ komutativna, pravimo, da je $S$ \emph{Abelova grupa}.
\end{definicija}

V grupah ponavadi uporabljamo \emph{miltiplikativni zapis}: operacija: $\cdot$, enota: $1$, inverz od $x$: $x^{-1}$, potenca: $x^n$.

V Abelovih grupah uporabljamo \emph{aditivni zapis}: operacija: $+$, enota: $0$, inverz od $x$: $-x$, potenca: $nx$.

\begin{table}[h!]
    \begin{tabular}{ m{23em} | m{23em} }
        Multiplikativni zapis & Aditivni zapis (Abelova grupa) \\ \hline
        $G$ ima natanko eno enoto & $G$ ima natanko en ničeln element \\ \hline
        Vsak element iz $G$ ima natanko en inverz & Vsak element iz $G$ ima natnako en nasprotni element  \\  \hline
        $(x^{-1})^{-1} = x$ & $-(-x) = x$ \\ \hline
        $(xy)^{-1} = y^{-1}x^{-1}$ & $-(x+y) = -x - y$ \\ \hline
        $x^{m+n} = x^mx^n$ & $(m+n)x = mx + nx$ \\ \hline
        $(x^m)^n = x^{mn}$ & $n(mx) = (nm)x$ \\ \hline
        V splošnem $(xy)^n \neq x^ny^n$ & $n(x+y) = nx + ny$ \\ \hline
        $xy=xz \lthen y=z$ & \multirow{2}{20em}{$x+y = x + z \lthen y = z$ (pravila krajšanja)} \\ 
        $yx=zx \lthen y=z$ & \\ \hline
        $xy=1 \lthen yx =1$ & \\
    \end{tabular}
    \caption{Lastnosti računanja v grupah}
\end{table}

\begin{zgled}
    Nekaj primerov grup.
    \begin{enumerate}
        \item $(\Z, +), \ (\Q, +), \ (\R, +), \ (\C, +), (\Q \setminus \set{0}, \cdot)$ so Abelove grupe.
        \item Naj bo $X$ neprazna množica. Definiramo $\Sim(X) = \set{\text{vse bijektivne preslikave } f: X \to X}$. $(\Sim(X), \circ)$ je grupa, imenujemo jo \emph{simetrična grupa} množice $X$.        
        
        V posebnem primeru, ko je $X$ končna dobimo $\Sim(\set{1, 2, \ldots, n}) = S_n$. Torej običajne permutacije.
    \end{enumerate}
\end{zgled}

\begin{zgled}[Simetrije kvadrata]
    Simetrije kvadrata $K$ so izometrije $f: \R^2 \to \R^2$, da je $f(K) = K$. 
    
    Primeri simetrij: $r$ - rotacija za $90^{\circ}$ okoli središča kvadrata, $z$ - zrcaljenje čez fiksno os simetrije ter kompozicije $r$ in $z$. Iz geometrije lahko vidimo, da je $zr = r^3z$. To pomeni, da je vsak kompozitum $r$ in $z$ oblike $r^kz$.
    
    Kvadrat ima kvečjemu $8$ simetrij, ker je vsaka simetrija določena s sliko oglišča $1$ in informacijo, ali smo naredili zrcaljenje ali ne. Dobimo množico simetrij $D_{2 \cdot 4} = \set{\id, r, r^2, r^3, z, rz, r^2z, r^3z}$. $D_{2\cdot 4}$ je \emph{diedrska grupa moči $8$}.
\end{zgled}

\begin{zgled}[Diedrska grupa moči $2n$]
    Imamo naslednje simetrije pravilnega $n$-kotnika:
    \begin{itemize}
        \item $r$ - rotacija za $\frac{2 \pi}{n}$ okoli središča.
        \item $z$ - zrcaljenje čes neko fiksno os simetrije.
    \end{itemize}
    Velja: $zr = r^{n-1}z$. 
    
    Množica vseh simetrij je $D_{2n} = \set{1, r, r^2, \ldots, r^{n-1}, z, rz, r^2zn, \ldots, r^{n-1}z}$. $D_{2n}$ je \emph{diedrska grupa moči $2n$}.
\end{zgled}

\begin{zgled}[Monoid $\to$ Grupa]
    Naj bo $(S, *)$ monoid. Definiramo $S^* = \set{\text{obrnljive elementi iz } S}$, potem $S^*$ je grupa za $*$.
\end{zgled}

\begin{primer}
    Naj bo $S = (\R^{n \times n}, \cdot)$, $S^* = \setb{A \in \R^{n \times n}}{\det A \neq 0} = \GL_n(\R)$. $\GL_n(\R)$ je \emph{splošna linearna grupa $n \times n$ matrik}.
\end{primer}

\begin{zgled}[Direktni produkt grup]
    Naj bodo $G_1, G_2, \ldots, G_n$ grupe z operacijami $*_1, *_2, \ldots, *_n$. Na množice $G_1 \times G_2 \times \ldots \times G_n$ vpeljamo operacijo $(g_1, g_2, \ldots, g_n) * (h_1, h_2, \ldots, h_n) = (g_1 *_1 h_1, g_2 *_2 h_2, \ldots, g_n *_n h_n)$. Potem $(G_1 \times G_2 \times \ldots \times G_n, *)$ je grupa. 
\end{zgled}

\subsection{Ponovitev o permutacijah}
\begin{izrek}
    Vsaka permutacija je produkt disjunktnih ciklov.
\end{izrek}

\begin{definicija}
    Cikli dolžine $2$ so \emph{transpozicije}.
\end{definicija}

\begin{trditev}
    Vsaka permutacija $\pi \in S_n$ je produkt transpozicij. Teh transpozicij je vedno sodo mnogo ali vedno liho mnogo.
\end{trditev}

\begin{definicija}
    Permutacija je \emph{soda (oz. liha)}, če je produkt sodo (oz. liho) mnogo transpozicij.
\end{definicija}

\begin{definicija}
    Znak permutacije je $\sgn(\pi) = \begin{cases}
        1; &\pi \text{ je soda} \\
        -1; &\pi \text{ je liha}
    \end{cases}$.
\end{definicija}

\begin{trditev}
    $\sgn(\pi \rho) = \sgn(\pi) \cdot \sgn(\rho)$.
\end{trditev}

\subsection{Podgrupe}
\begin{definicija}
    Naj bo $G$ grupa in $H \subseteq G, \ H \neq \emptyset$. $H$ je \emph{podgrupa grupe $G$}, če je $H$ za isto operacijo tudi grupa. Oznaka $H \leq G$.
\end{definicija}

\begin{opomba} 
    Očitno o podgrupah:
    \begin{enumerate}
        \item Naj bo $G$ grupa. Vedno velja: $\set{1} \leq G$ in $G \leq G$.
        \item Če je $H \leq G$, potem (nujno!) $1 \in H$, kjer $1$ je enota v $G$.
    \end{enumerate}
\end{opomba}

\begin{opomba}
    Pri monoidih se enota ne deduje nujno, npr. $(\Z, \cdot)$ in $(\set{0}, \cdot)$.
\end{opomba}

\begin{trditev}
    Naj bo $G$ grupa, $H \subseteq G, \ H \neq \emptyset$. Naslednje trditve so ekvivalentne:
    \begin{enumerate}
        \item $H \leq G$.
        \item $\all{x,y \in H} xy^-1 \in H$.
        \item $H$ je zaprta za množenje in invertiranje.
    \end{enumerate}
\end{trditev}

\begin{proof}
    Definicija podgrupe.
\end{proof}

\begin{posledica}
    Naj bo $G$ končna grupa in $H \subseteq G, \ H \neq \emptyset$. Velja: 
    $$H \leq G \liff H \text{ je zaprta za množenje}.$$
\end{posledica}

\begin{proof}
    Ker je $G$ končna, ko potenciramo $x \in H$, ena izmed potenc zagotovo ponovi.
\end{proof}

\begin{opomba}
    V končnih grupih ni potrebno preverjati zaprtost za invertiranje.
\end{opomba}

\begin{primer}
    Primeri podrgup.
    \begin{enumerate}
        \item Vse prave podrgupe v grupi $(\Z, +)$ so oblike $n\Z, \ n \in \N$.
        \item Definiramo $\SL_n(\R) = \setb{A \in \GL_n(\R)}{\det A = 1}$. Potem $\SL_n(\R) \leq \GL_n(\R)$. $\SL_n(\R)$ imenujemo \emph{specialna linearna grupa}.
        \item Definiramo $\text{O}(n) = \setb{A \in \GL_n(\R)}{AA^T = A^TA = I}$. Potem $\text{O}(n) \leq \GL_n(\R)$.
        \item Definiramo $\text{SO}(n) = \setb{A \in \text{O}(n)}{\det A = 1}$. Potem $\text{SO}(n) \leq \text{O}(n)$. Grupo $\text{SO}(n)$ imenujemo \emph{specialne ortogonalne matrike}.
    \end{enumerate}
\end{primer}

\newpage
\begin{trditev}
    Naj bosta $H$ in $K$ podgrupi grupe $G$. Potem $H \cap K \leq G$. Enako velja za preseke poljubnih družin podgrup.
\end{trditev}

\begin{proof}
    Karakterizacija podrgupe.
\end{proof}

\begin{definicija}
    Naj bosta $H, K \leq G$. Definiramo $HK = \setb{hk}{h \in H, \ k \in K}$. Temu pravimo \emph{produkt podgrup}.
\end{definicija}

\begin{zgled}
    $H K$ ni nujno podgrupa v $G$. Vzemimo $G = S_3, \ H =\set{\id, (1\ 2)}, \ K = \set{\id, (1\ 3)}$.
\end{zgled}

\begin{trditev}
    Naj bosta $H, K \leq G$. Če velja $HK = KH$, potem je $HK \leq G$.
\end{trditev}

\begin{proof}
    Karakterizacija podrgupe in definicija produkta podgrup.
\end{proof}

\begin{opomba}
    Ni nujno, da produkt podgrup $HK$ komutativen. Torej ni nujno vsak element $hk \in HK$ se da zapisati kot~$k'h' \in KH$ za neki $k' \in K$ in $h' \in H$.
\end{opomba}

\begin{definicija}
    Naj bo $H \leq G, \ a \in G$. Definiramo množico $aHa^{-1} = \setb{aha^{-1}}{h \in H}$. Potem $aHa^{-1} \leq G$. Temu~se~reče \emph{konjungiranje podgrupe $H$ z elementom $a$}.
\end{definicija}

\begin{proof}
    Karakterizacija podrgupe.
\end{proof}

\begin{trditev}
    Naj bo $G$ grupa.
    \begin{enumerate}
        \item Definiramo $Z(G) = \setb{y \in G}{\all{x \in G} yx=xy}$. Potem $Z(G) \leq G$. Tej grupi pravimo \emph{center grupe $G$}.
        \item Naj bo $a \in G$. Definiramo $C_G(a) = \setb{y \in G}{ya = ay}$. Potem $C_G(a) \leq G$. Tej podgrupi pravimo \emph{centralizator elementa $a$ v $G$}.
    \end{enumerate}
\end{trditev}

\begin{proof}
    Karakterizacija podrgupe.
\end{proof}

\subsection{Odseki podgrup in Lagrangeev izrek}
Naj bo $G$ grupa in $H \leq G$. Definiramo relacijo na $G$ s predpsiom $\all{a, b \in G} a \sim b :\liff a^{-1}b \in H$.

\begin{trditev}
    Relacija $\sim$ je ekvivalenčna relacija na $G$.
\end{trditev}

\begin{proof}
    Preverimo refleksivnost, simetričnost in tranzitivnost.
\end{proof}

\begin{definicija}
    Naj bo $G$ grupa, $H \leq H, \ a \in G$. \emph{Ekvivalenčni razred elementa $a \in G$} je množica $[a] = \setb{b \in G}{a \sim b}$.
\end{definicija}

\begin{opomba}
    $[a] = \setb{ah}{h \in H} =: aH$.
\end{opomba}

\begin{definicija}
    Množico $aH$ imenujemo \emph{levi odsek grupe $G$ po podgrupi $H$}.
\end{definicija}

\begin{opomba}
    V grupo $G$ lahko vpeljamo tudi relacijo $\approx$ s predpisom $\all{a, b \in G} a \approx b :\liff ab^{-1} \in H$.

    To je ekvivalenčna relacija. Ekvivalentni razredi so $[a] = \setb{ha}{h \in H} =: Ha$, ki jih imenujemo \emph{desni odseki}.
\end{opomba}

\begin{definicija}
    \emph{Faktorska (oz. kvocientna) množica} glede na relacijo $\sim$ je množica $G/_\sim = \setb{aH}{a \in G} =: G/H$.
\end{definicija}

\begin{opomba}
    $G/H$ ni nujno grupa.
\end{opomba}

\begin{opomba}
    Kadar sta dva odseka enaka? $aH = bH \liff a \sim b \liff a^{-1}b \in H$.
\end{opomba}

\begin{opomba}
    Naj bo $G$ končna grupa. Potem je $G/H$ tudi končna množica.
\end{opomba}

\begin{definicija}
    Naj bo $G$ končna grupa. Moč množce $G/H$ označimo z $G:H$ (oz $[G:H]$) in jo imenujemo \emph{indeks podgrupe $H$ v grupi $G$}.
\end{definicija}

\begin{izrek}[Lagrangeev izrek]
    Če je $G$ končna grupa in $H \leq G$, potem je 
    $$|G| = |H| \cdot |G:H|.$$
\end{izrek}

\begin{proof}
    Recimo, da $|G:H| = r$. Pokažemo, da $|a_iH| = |H|$ za vse $i = 1, \ldots, r$.
\end{proof}

\begin{posledica}
    Moč vsake podgrupe končne grupe deli moč grupe.
\end{posledica}

\begin{opomba}
    Če je grupa $G$ Abelova in $H \leq G$, potem odseki pišemo kot $a + H$. Velja: $G/H = \setb{a+H}{a \in G}$.

    Vpeljamo operacijo na $G/H$: $(a+H) + (b+H) = (a + b) + H$. Ta operacija je dobro definirana, ker je $G$ Abelova.
\end{opomba}

\begin{trditev}
    $G/H$ je za to operacijo Abelova grupa.
\end{trditev}

\begin{proof}
    Enostavno preverimo aksiome.
\end{proof}

\begin{primer}
    Naj bo $G = \Z$ in $H = n\Z, \ n \in \N$. Potem $\Z/{n\Z} = \set{0 + n\Z, 1 + n\Z, \ldots, (n-1) + n\Z}$.

    Operacija $+$ na $\Z/{n\Z}$ je seštevanje po modulu $n$. Grupa $\Z_n := \Z/{n\Z}$ je \emph{grupa ostankov po modulu $n$}, $|\Z_n| = n$.
\end{primer}
\begin{posledica}
    Za vsako število $n \in \N$ obstaja vsaj ena grupa moči $n$.
\end{posledica}

\newpage
\subsection{Generatorji grup. Ciklične grupe}

\begin{definicija}
    Naj bo $G$ grupa in $X$ podmnožica v $G$. Potem označimo z $\gen{X}$ najmanjšo podgrupo v $G$, ki vsebuje množico $X$. To podgrupo imenujemo \emph{podgrupa generirana z množico $X$}.
\end{definicija}

\begin{opomba}
    $\gen{X}$ je presek vseh podgrup grupe $G$, ki vsebujejo množico $X$.
\end{opomba}

\begin{definicija}
    Naj bo $G$ grupa. 
    
    \begin{itemize}
        \item Če je $X \subseteq G$, za katero velja $G = \gen{X}$, pravimo, da je \emph{$G$ generirana z množico $X$}. Elementam množice $X$ pravimo \emph{generatorji grupe $G$}.
        
        Oznaka: Če je $X = \set{x_1, \ldots, x_n}$, pišemo $\gen{\set{x_1, \ldots, x_n}} = \gen{x_1, \ldots, x_n}$.
        \item Če je $G = \gen{x_1, \ldots, x_n}$, pravimo, da je \emph{$G$ končno generirana grupa}.
        \item Če obstaja $x \in G$, da je $G = \gen{x}$, pravimo, da je \emph{$G$ ciklična grupa}.
    \end{itemize}    
\end{definicija}

\begin{trditev}
    Naj bo $G$ grupa in $X \subseteq G$. $\gen{X} = \setb{x_{i_1}^{\pm 1} x_{i_2}^{\pm 1} \ldots x_{i_r}^{\pm 1}}{x_{i_j} \in X; \ r \in \N_0} =: S$.
\end{trditev}

\begin{proof}
    Dovolj dokazati, da je $S$ podgrupa grupe $G$.
\end{proof}

\begin{posledica}
    Naj bo $G$ grupa, $a \in G$. Potem $\gen{a} = \setb{a^n}{n \in \Z}$.
\end{posledica}

\begin{primer}
    Primeri generatorjev grup:
    \begin{itemize}
        \item $\Z = \gen{1}$. Velja tudi: $\Z = \gen{p, q}$, kjer sta $p$ in $q$ tuji.
        \item $\Z_n = \gen{1 + n\Z}$.
    \end{itemize}
\end{primer}

\begin{definicija}
    Naj bo $G$ grupa in $a \in G$. Najmanjšemu naravnemu številu $n$, za katerega velja $a^n=1$, pravimo \emph{red}~elementa $a$. Če tak $n$ ne obstaja, pravimo, da ima $a$ neskončen red.
\end{definicija}

\begin{primer}
    Primeri elementov končnega in neskončnega reda.
    \begin{itemize}
        \item Element $1 \in \Z$ ima neskončen red.
        \item Element $1 + n\Z \in \Z_n$ ima red $n$.
    \end{itemize}    
\end{primer}

\begin{trditev}
    Naj bo $G$ grupa, $a \in G$. Potem je red elementa $a$ enak $n$ natanko tedaj, ko $|\gen{a}| = n$.
\end{trditev}

\begin{proof}
    Uporabimo ustrezne definicije in izreki o celih številih.
\end{proof}

\begin{posledica}
    Naj bo $G$ končna grupa. Velja:
    \begin{enumerate}
        \item Za vsak $a \in G$ red $a$ deli $|G|$.
        \item Za vsak $a \in G$ velja, da $a^{|G|} = 1$.
        \item Če je $|G|$ praštevilo, potem je $G$ ciklična grupa. 
    \end{enumerate}
\end{posledica}

\begin{proof}
    Uporabimo ustrezne definicije in izreki.
\end{proof}