\section{Uvod v teorijo kolobarjev}
\begin{enumerate}
    \item Definicije kolobarja, obsega in polja
    \begin{itemize}
        \item \colorbox{purple!30}{\textbf{Definicija.}} Kolobar.
        \item \colorbox{blue!30}{\textbf{Trditev.}} 3 lastnosti kolobarja $K$:
        \begin{itemize}
            \item Množenje z nevtralnim elementom $0 \in K$.
            \item Množenje z nasprotnim elementom $-x \in K$. Množenje nasprotnih elementov.
            \item Množenje z $-1 \in K$.
        \end{itemize}
        \item \colorbox{yellow!30}{\emph{Primer.}} Trivialni (ničelni) kolobar.
        \item \colorbox{blue!30}{\textbf{Trditev.}} Naj bo $K$ neničeln. Kaj lahko povemo o $0$ in $1$?
        \item \colorbox{purple!30}{\textbf{Definicija.}} Komutativen kolobar.
        \item \colorbox{purple!30}{\textbf{Definicija.}} Delitelj niča. Levi delitelj niča, desni delitelj niča.
        \item \colorbox{yellow!30}{\emph{Primer.}} Poišči delitelja niča v $M_2(\R)$.
        \item \colorbox{purple!30}{\textbf{Definicija.}} Idempotent. Nilpotent.
        \item \colorbox{yellow!30}{\emph{Primer.}} Poišči idempotenti in nilpotenti v $M_2(\R)$. Kako so povezani z delitelji niča?
        \item \colorbox{yellow!30}{\emph{Opomba.}} Pravilo krajšanja v kolobarju brez deliteljev niča.
        \item \colorbox{purple!30}{\textbf{Definicija.}} Cel kolobar.
        \item \colorbox{purple!30}{\textbf{Definicija.}} Obseg. Polje.
        \item \colorbox{blue!30}{\textbf{Trditev.}} Ali lahko obrnljiv element delitelj niča? Ali v obsegu so delitelji niča?
        \item \colorbox{yellow!30}{\emph{Primer.}} Ali je $\Z$ cel kolobar? Ali je obseg?
    \end{itemize}   
    
    \item Definicija algebre
    \begin{itemize}
        \item \colorbox{purple!30}{\textbf{Definicija.}} Vektorski prostor.
        \item \colorbox{blue!30}{\textbf{Trditev.}} 4 lastnosti vektorskega prostora.
        \item \colorbox{purple!30}{\textbf{Definicija.}} Algebra nad poljem $\F$. Realna algebra. Kompleksna algebra.
        \item \colorbox{yellow!30}{\emph{Primer.}} Navedi osnovni primeri kolobarjev, obsegov, polj in algeber.
    \end{itemize}

    \item Primeri kolobarjev in algeber
    \begin{itemize}
        \item Številski kolobarji, polja in algebre.
        \item Algebra funkcij $\R^X$. Kaj dobimo, če je $X = \N$? \emph{Algebra realnih zaporedij}.
        \item \emph{Kolobar polinomov ene spremenljivke}.
        \begin{itemize}
            \item \colorbox{purple!30}{\textbf{Definicija.}} Polinom $f(X)$. Koeficienti polinoma $f(X)$. Kaj so $X^i$?
            \item \colorbox{purple!30}{\textbf{Definicija.}} Kadar sta polinoma enaka? Vsota polinomov. Produkt polinomov.
            \item \colorbox{purple!30}{\textbf{Definicija.}} Kolobar polinomov ene spremenljivke nad kolobarjem $K$. Oznaka.
            \item \colorbox{blue!30}{\textbf{Trditev.}} Karakterizacija komutativnosti kolobarja $K[X]$.
            \item \colorbox{blue!30}{\textbf{Trditev.}} Kadar $K[X]$ nima deliteljev niča?
            \item \colorbox{orange!30}{\textbf{Posledica.}} Kadar je $K[X]$ cel kolobar?
            \item Kako naravno postane $K[X]$ algebra? Ali lahko nad $K$?
        \end{itemize}
        \item \emph{Kolobar formalnih potenčnih vrst $K[[X]]$}.
        \item \emph{Kolobar polinomov več spremenljivk.}
        \begin{itemize}
            \item \colorbox{purple!30}{\textbf{Definicija.}} Monom.
            \item \colorbox{purple!30}{\textbf{Definicija.}} Kolobar polinomov dveh spremenljivk. Kolobar polinomov $n$ spremenljivk.
        \end{itemize}
        \item Naj bo $K$ cel kolobar. Kako ga povečamo do polja? \emph{Polje ulomkov celega kolobarja $K$}.
        \begin{itemize}
            \item \colorbox{purple!30}{\textbf{Definicija.}} Relacija na $K \times K \setminus \set{0}$.
            \item \colorbox{blue!30}{\textbf{Trditev.}} Ali je ekvivalenčna? Oznaka za ekvivalenčni razred elementa $(a,b) \in K \times K \setminus \set{0}$.
            \item \colorbox{purple!30}{\textbf{Definicija.}} Seštevanje in množenje na $K \times K \setminus \set{0}$.
            \item \colorbox{yellow!30}{\emph{Opomba.}} Ali sta seštevanje in množenje dobro definirani.
            \item \colorbox{blue!30}{\textbf{Trditev.}} Ali s tem postane $K \times K \setminus \set{0}$ polje?
            \item \colorbox{purple!30}{\textbf{Definicija.}} Polje $F$ ulomkov celega kolobarja $K$.
            \item \colorbox{yellow!30}{\emph{Opomba.}} Kako kolobar $K$ vložimo v polje $F$? Kaj če je $K$ že polje?
            \item \colorbox{yellow!30}{\emph{Opomba.}} Kaj je polje ulomkov kolobarja $\Z$?
            \item \colorbox{purple!30}{\textbf{Definicija.}} Polje racionalnih funkcij.
        \end{itemize}

        \item \emph{Kolobar matrik $M_n(K)$} nad kolobarjem $K$.
        \begin{itemize}
            \item Ali je komutativen? Ali ima delitelji niča?
            \item Kako dobimo algebro?
        \end{itemize}

        \newpage
        \item \emph{Kolobar endomorfizmov $\End_F (V)$} vektorskega prostora $V$ nad poljem $F$.
        \begin{itemize}
            \item \colorbox{purple!30}{\textbf{Definicija.}} Linearna preslikava. Endomorfizem vektorskega prostora $V$.
            \item \colorbox{purple!30}{\textbf{Definicija.}} Seštevanje, množenje s skalarji na $\End_F (V)$.
            \item Ali je $\End_F (V)$ algebra?
        \end{itemize}

        \item \emph{Algebra kvaternionov $\HH$}.
        \begin{itemize}
            \item Ali obstaja realna algebra lihe dimenzije več kot $1$, ki je obseg?
            \item Naj bo $\HH$ vektorski prostor z bazo $\set{1, i, j, k}$. 
            \item \colorbox{purple!30}{\textbf{Definicija.}} Množenje na $\HH$.
            \item \colorbox{blue!30}{\textbf{Trditev.}} Ali je $\HH$ algebra?
            \item \colorbox{purple!30}{\textbf{Definicija.}} Kvaternioni.
            \item \colorbox{yellow!30}{\emph{Opomba.}} Ali je $\HH$ komutativna?
            \item \colorbox{purple!30}{\textbf{Definicija.}} Konjugirani kvaternion.
            \item \colorbox{blue!30}{\textbf{Trditev.}} Ali je $\HH$ obseg?
            \item \colorbox{purple!30}{\textbf{Definicija.}} \emph{Kvaternionska grupa $Q$}.
            \item \colorbox{yellow!30}{\emph{Opomba.}} Ali so $\R, \C, \HH$ edini končnorazsežne realne algebre, ki so obsegi?
        \end{itemize}

        \item \emph{Direktni produkt kolobarjev}.
        \begin{itemize}
            \item \colorbox{purple!30}{\textbf{Definicija.}} Direktni produkt kolobarjev.
            \item \colorbox{yellow!30}{\emph{Opomba.}} Kadar direktni produkt kolobarjev ima delitelje niča? Ali je direktni produkt polj tudi polje?
            \item \colorbox{purple!30}{\textbf{Definicija.}} Direktni produkt algeber.
        \end{itemize}
    \end{itemize}

    \item Podkolobarji, podalgebre in podpolja
    \begin{itemize}
        \item \colorbox{purple!30}{\textbf{Definicija.}} Podkolobar $L$.
        \item \colorbox{yellow!30}{\emph{Primer.}} Zakaj moramo zahtevati, da $1 \in L$?
        \item \colorbox{purple!30}{\textbf{Definicija.}} Podalgebra.
        \item \colorbox{yellow!30}{\emph{Opomba.}} Kaj je podalgebra v jeziku linearne algebre in podkolobarjev?
        \item \colorbox{purple!30}{\textbf{Definicija.}} Podpolje $F$.
        \item \colorbox{yellow!30}{\emph{Primer.}} Zakaj ni zahtevamo, da $1 \in F$?
        \item \colorbox{purple!30}{\textbf{Definicija.}} Razširitev polja.
        \item \colorbox{yellow!30}{\emph{Primer.}} Navedi številski primeri ražširitev.
        \item \colorbox{blue!30}{\textbf{Trditev.}} Karakterizacija podkolobarja.
        \item \colorbox{blue!30}{\textbf{Trditev.}} Karakterizacija podalgebre.
        \item \colorbox{purple!30}{\textbf{Definicija.}} Podobseg.
        \item \colorbox{blue!30}{\textbf{Trditev.}} Karakterizacija podpolja (podobsega).
        \item \colorbox{yellow!30}{\emph{Opomba.}} Ali je presek podkolobarjev, podprostorov, podalgeber in podpolj spet ustrezna podstruktura?
        \item \colorbox{yellow!30}{\emph{Primer.}} Primeri podkolobarjev, podalgeber, podpolj.        
        \begin{itemize}
            \item Navedi primer številskega zaporedja podkolobarjev.
            \item Kako lahko opišemo odnos med $\R$ in $\C$?
            \item Ali je $K[X]$ podkolobar v kolobarju $K[[X]]$? Kaj pa v $K[X, Y]$?
            \item Kakšen odnos med celimi kolobarji in njihovimi polji ulomkov?
            \item Množica vseh diagonalnih matrik v $M_n(\R)$.
            \item Množica vseh zgoraj trikotnih matrik v $M_n(\R)$. Kaj pa množica vseh strogo zgoraj trikotnih matrik?
            \item \emph{Algebra $C(X)$} vseh zveznih funkcij.
            \item Množica $c$ konvergentnih zaporedij v algebre realnih zaporedij.
            \item 
        \end{itemize}
        \item \colorbox{yellow!30}{\emph{Primer.}} \emph{Kolobar Gaussovih celih števil $\Z[i]$}.
        \item \colorbox{yellow!30}{\emph{Primer.}} \emph{Center $Z(K)$} kolobarja $K$.
    \end{itemize}

    \newpage
    \item Kolobarji ostankov in karakteristika kolobarja.
    \begin{itemize}
        \item \colorbox{yellow!30}{\emph{Opomba.}}  Ali je vsak kolobar vsebuje kopijo celih števil?
        \item \colorbox{purple!30}{\textbf{Definicija.}} Karakteristika kolobarja $K$.
        \item \colorbox{blue!30}{\textbf{Trditev.}} 3 lastnosti kolobarja s karakteristiko $n \in \N$.
        \item \colorbox{orange!30}{\textbf{Posledica.}} Kako karakteristiko lahko ima polje?
        \item \colorbox{purple!30}{\textbf{Definicija.}} Množenje v $\Z_n$.
        \item \colorbox{blue!30}{\textbf{Trditev.}} Ali je $\Z_n$ komutativen kolobar? \emph{Kolobar $\Z_n$} ostankov po modulu $n$.
        \item \colorbox{blue!30}{\textbf{Lema.}} Kaj lahko povemo o končnem celem kolobarju?
        \item \colorbox{yellow!30}{\emph{Opomba.}} Ali predpostavka o komutativnosti odveč? Kaj lahko sklepamo?
        \item \colorbox{blue!30}{\textbf{Trditev.}} Kadar je $\Z_p$ polje?
        \item \colorbox{yellow!30}{\emph{Primer.}} Navedi osnovni primeri kolobarjev in polj s različno karakterizacijo.
        \item \colorbox{blue!30}{\textbf{Izrek.}} Vali Fermatov izrek. 
    \end{itemize}

    \item Generatorji kolobarjev, algeber in polj
    \begin{itemize}
        \item \colorbox{purple!30}{\textbf{Definicija.}} Naj bo $K$ kolobar in $X \subseteq K$. Podkolobar, generiran z $X$. 
        \item \colorbox{purple!30}{\textbf{Definicija.}} Generatorji. Končno generiran kolobar.
        \item \colorbox{yellow!30}{\emph{Opomba.}} Zapiši isti definiciji za obseg, polje in algebro.
        \item \colorbox{blue!30}{\textbf{Trditev.}} Opiši podkolobar, generiran z množico $X$.
        \item \colorbox{blue!30}{\textbf{Trditev.}} Opiši podalgebro, generirano z množico $X$.
        \item \colorbox{blue!30}{\textbf{Trditev.}} Opiši podpolje, generirano z množico $X$.
        \item \colorbox{yellow!30}{\emph{Primer.}} Primeri generatorjev.
        \begin{itemize}
            \item Kaj je podkolobar kolobarja $\C$, generiran z $1$?
            \item Kaj je podpolje kolobarja $\C$, generirano z $1$?
            \item Kaj je podkolobar kolobarja $\C$, generiran z $i$?
            \item Kaj je podpolje kolobarja $\C$, generirano z $i$?
            \item Kaj je podkolobar kolobarja $\R[X]$, generiran z $X$?
            \item S čim je generirana realna algebra $\R[X]$?
            \item S čim je generirana algebra $M_2(\R)$? Čemu je enaka $\dim M_2(\R)$.
            \item Kaj je podkolobar kolobarja $M_2(\R)$, generiran z $E_{12}$ in $E_{21}$?
        \end{itemize}
    \end{itemize}
\end{enumerate}

\newpage
\subsection*{Rezultati z vaj}
\begin{enumerate}
    \item Kolobarji, obsegi, polja
    \begin{itemize}
        \item Kako iz kolobarja brez enote lahko naredimo kolobar z enoto?
        \item \emph{Boolov kolobar.} Primer Boolova kolobarja.
    \end{itemize}

    \item Algebre
    \begin{itemize}
        \item Ali je $\Z$ lahko algebra nad kakim poljem?
        \item Naj bo $A$ končnorazsežna algebra.    
        \begin{itemize}
            \item Kaj velja za vsak $a \in A \setminus \set{0}$?
            \item Kaj če ima $a \in A$ levi ali desni inverz?
            \item Recimo, da je $A$ tudi obseg. Kaj lahko povemo o vsaki podalgebri?
        \end{itemize}
        \item Algebra kvaternionov.
        \begin{itemize}
            \item Čemu je enak $Z(\HH)$? Čemu je enak $Z(Q)$?
            \item Kaj lahko povemo o enačbi $h^2 + \alpha h + \beta = 0$ za vsak $h \in \HH$?
        \end{itemize}

        \item Kolobar $\Z_n$.
        \begin{itemize}
            \item Kadar je $k \in Z_n$ obrnljiv?
            \item Koliko je obrnljivih elementov v $\Z$? Koliko v $\Z_n$? Kaj če je $n$ praštevilo?
        \end{itemize}
    \end{itemize}
\end{enumerate}