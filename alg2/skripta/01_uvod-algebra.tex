\section{Uvod v teorijo grup}

\begin{enumerate}
    \item Osnovni pojmi teoriji grup
    \begin{itemize}
        \item \textbf{Definicija.} Binarna operacija na množice $S$. Kadar pravimo, da je operacija asociativna. Kadar pravimo, da je operacija komutativna?       
        \item \textbf{Definicija.} Polgrupa.
        \item \textbf{Definicija.} Nevtralni element.
        \item \textbf{Trditev.} Ali če v množici \(S\) obstaja enota za operacijo \(*\), potem je ena sama?
        \item \textbf{Definicija.} Monoid.
        \item \textbf{Definicija.} Levi inverz. Desni inverz. Inverz. 
        \item \textbf{Definicija.} Obrnljiv element.
        \item \textbf{Trditev.} Kaj če v monoidu ima element \(x\) levi in desni inverz?
        \item \textbf{Posledica.} Koliko inverzov lahko ima obrnljiv element v monoidu?
        \item \textbf{Posledica.} Kaj če je \(x\) obrnljiv element monoida in \(xy = 1\)?
        \item \textbf{Trditev.} Obrnljivost produkta obrnljivih elementov.
        \item \textbf{Definicija.} Grupa. Abelova grupa.
        \item \textbf{Definicija.} Multiplikativni in aditivni zapis operacije. Kdaj jih uporabljamo?
        \item \textbf{Trditev.} Računanje z potenci v grupi. Pravilo krajšanja v grupi.
        \item \textbf{Zgled.} Primeri številskih grup. Simetrična grupa množice \(X\). Grupa permutacij.
        \item \textbf{Zgled.} Grupa simetrij kvadrata. Diedrska grupa moči \(2n\).
        \item \textbf{Zgled.} Kako iz monoida dobimo grupo? Splošna linearna grupa.
        \item \textbf{Zgled.} Direktni produkt grup. 
    \end{itemize}   
    
    \item Grupa permutacij
    \begin{itemize}
        \item \textbf{Izrek.} Kako lahko zapišemo vsako permutacijo?
        \item \textbf{Definicija.} Transpozicija.
        \item \textbf{Trditev.} Kako lahko zapišemo vsako permutacijo z pomočjo transpozicij? Koliko je transpozicij v tem zapisu?
        \item \textbf{Definicija.} Soda permutacija. Liha permutacija. Znak permutacije.
        \item \textbf{Trditev.} Znak produkta permutacij.
    \end{itemize}

    \item Podgrupe
    \begin{itemize}
        \item \textbf{Definicija.} Podgrupa.
        \item \textbf{Opomba.} Kaj sta vedno podgrupi grupe \(G\)? Ali je enota vedno vsebovana v podgrupi? Ali se enota deduje pri monoidih?
        \item \textbf{Trditev.} Dve karakterizaciji podgrupe.
        \item \textbf{Posledica.} Karakterizacija podgrupe končne grupe \(G\).
        \item \textbf{Zgled.} \ 
        \begin{itemize}
            \item Kakšne so oblike vse prave podgrupe grupe \(\Z\)?
            \item Specialna linearna grupa. Grupa ortogonalnih matrik. Specialna grupa ortogonalnih matrik.
        \end{itemize}
        \item \textbf{Trditev.} Ali je presek podgrup grupe \(G\) podgrupa grupe \(G\)?
        \item \textbf{Definicija.} Produkt podgrup.
        \item \textbf{Zgled.} Ali je produkt podgrup vedno podgrupa?
        \item \textbf{Trditev.} Zadosten pogoj, da je produkt podgrup podgrupa.
        \item \textbf{Definicija.} Konjugiranje podgrupe \(H \leq G\) z elementov \(a \in G\). Ali je konjugiranje podgrupa?
        \item \textbf{Definicija.} Center grupe \(G\). Centralizator elementa \(a \in G\). Ali sta podgrupi?
    \end{itemize}

    \item Odseki podgrup in Lagrangeev izrek
    
    Naj bo \(G\) grupa in \(H \leq G\).
    \begin{itemize}
        \item Relacija \(\sim\) na \(G\). ki porodi leve odseke.
        \item \textbf{Trditev.} Ali je relacija \(\sim\) ekvivalenčna?
        \item \textbf{Definicija.} Ekvivalenčni razred elementa \(a \in G\).
        \item \textbf{Definicija.} Ekvivalenčne razredi po relaciji \(\sim\). Levi odseki \(G\) po podgrupe \(H\).
        \item \textbf{Opomba.} Z kakšno ekvivalenčno relacijo dobimo desne odseke?
        \item \textbf{Definicija.} Kvocientna množica glede na relacijo \(\sim\).
        \item \textbf{Opomba.} kaj tvorijo ekvivalenčni razredi glede na množico \(G\)?
        \item \textbf{Opomba.} Ali je \(G/H\) vedno grupa? Kadar sta dva odseka enaka? Ali je \(G/H\) končna, če je \(G\) končna?
        \item \textbf{Definicija.} Indeks podgrupe \(H\).
        \item \textbf{Izrek.} Lagrangeev izrek.
        \item \textbf{Posledica.} Ključni pomen izreka.
        \item \textbf{Opomba.} Kako lahko definiramo operacijo na \(G/H\), če je \(G\) Abelova?
        \item \textbf{Trditev.} Ali je s prej definirano operacijo \(G/H\) Abelova grupa?
        \item \textbf{Zgled.} Grupa ostankov po modulu \(n\). Ali za vsako naravno število \(n\) obstaja grupa moči \(n\)?
    \end{itemize}
\end{enumerate}

\subsection*{Rezultati vaj}
\begin{itemize}
    \item Ali je v končnem monoidu levi inverz avtomatično tudi desni inverz?
    \item Ali je element monoida obrnljiv, če obrnljiva neka njegova potenca?
\end{itemize}