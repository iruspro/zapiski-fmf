\section{Cela števila}

\begin{enumerate}
    \item Osnovni izrek o deljenju celih števil
    \begin{itemize}
        \item Načelo dobre urejenosti v $\N$.
        \item Načeli dobre urejenosti v $\Z$.
        \item \colorbox{blue!30}{\textbf{Izrek.}} Osnovni izrek o deljenju celih števil. Ostanek.
    \end{itemize}

    \item Največji skupni delitelj
    \begin{itemize}
        \item \colorbox{purple!30}{\textbf{Definicija.}} Kadar pravimo, da celo število $k \neq 0$ deli celo število $m$? Zapis.
        \item \colorbox{purple!30}{\textbf{Definicija.}} Delitelj. Število $m$ deljivo s številom $k$.
        \item \colorbox{purple!30}{\textbf{Definicija.}} Skupni delitelj. Največji skupni delitelj.
        \item \colorbox{blue!30}{\textbf{Izrek.}} Obstoj največjega skupnega delitelja. Kako lahko ga zapišemo?
        \item \colorbox{purple!30}{\textbf{Definicija.}} Tuji števili.
        \item \colorbox{orange!30}{\textbf{Posledica.}} Kadar sta števili $m$ in $n$ tuji?
    \end{itemize}

    \item Osnovni izrek aritmetike
    \begin{itemize}
        \item \colorbox{purple!30}{\textbf{Definicija.}} Praštevila.
        \item \colorbox{blue!30}{\textbf{Lema.}} Evklidova lema.
        \item \colorbox{blue!30}{\textbf{Izrek.}} Osnovni izrek aritmetike.
        \item \colorbox{blue!30}{\textbf{Izrek.}} Ali je praštevil neskončno?
    \end{itemize}
\end{enumerate}

\newpage
\section{Uvodni pojmi algebre}

\begin{enumerate}
    \item Binarne operacije
    \begin{itemize}
        \item \colorbox{purple!30}{\textbf{Definicija.}} Binarna operacija na množice $S$.        
        \item \colorbox{yellow!30}{\emph{Primer.}} Najpomembnejše operacije: seštevanje, množenje in komponiranje. Množica preslikav iz $X$ vase.
        \item \colorbox{yellow!30}{\emph{Primer.}} Navedi primeri in protiprimeri binarnih operacij.
        \item \colorbox{purple!30}{\textbf{Definicija.}} Kadar pravimo, da množica zaprta za operacijo? Notranja operacija.
        \item \colorbox{yellow!30}{\emph{Primer.}} Navedi primeri in protiprimeri množic zaprtih za operacijo. 
        \item \colorbox{purple!30}{\textbf{Definicija.}} Zunanja binarna operacija. 
        \item \colorbox{yellow!30}{\emph{Primer.}} Navedi primer zunanji operaciji.
        \item \colorbox{purple!30}{\textbf{Definicija.}} Asociativna operacija. 
        \item \colorbox{purple!30}{\textbf{Definicija.}} Kadar pravimo, da sta elementa $x$ in $y$ komutirata? Komutativna operacija.
        \item \colorbox{yellow!30}{\emph{Primer.}} Navedi primeri in protiprimeri asociativnih in komutativnih operacij.
        \item \colorbox{purple!30}{\textbf{Definicija.}} Nevtralni element.
        \item \colorbox{yellow!30}{\emph{Primer.}} Navedi primeri nevtralnih elementov za različne operacije na različnih množicah.
        \item \colorbox{blue!30}{\textbf{Trditev.}} Enoličnost nevtralnega elementa.
        \item \colorbox{purple!30}{\textbf{Definicija.}} Levi nevtralni element. Desni nevtralni element.
        \item \colorbox{yellow!30}{\emph{Opomba.}} Kako sta povazana levi in desni nevtralna elementa?
        \item \colorbox{yellow!30}{\emph{Primer.}} Ali lahko obstaja več levih nevtralnih elementov?
    \end{itemize}

    \item Polgrupe
    \begin{itemize}
        \item Kaj je algebrska struktura?
        \item \colorbox{purple!30}{\textbf{Definicija.}} Polgrupa $(S, \star )$.
        \item \colorbox{yellow!30}{\emph{Primer.}} Navedi primeri in protiprimeri polgrup.
        \item \colorbox{blue!30}{\textbf{Trditev.}} Ali lahko oklepaje v polgrupe vedno odpravimo?
        \item \colorbox{purple!30}{\textbf{Definicija.}} Potenca elementa $x \in S$.
        \item \colorbox{yellow!30}{\emph{Primer.}} Kakšne formule veljajo za potence v polgrupi?
    \end{itemize}

    \item Monoidi
    \begin{itemize}
        \item \colorbox{purple!30}{\textbf{Definicija.}} Monoid $(S, \star )$.
        \item \colorbox{yellow!30}{\emph{Primer.}} Navedi primeri in protiprimeri monoidov.
        \item \colorbox{purple!30}{\textbf{Definicija.}} Levi inverz. Desni inverz. Inverz. Obrnljiv element.
        \item \colorbox{blue!30}{\textbf{Trditev.}} Kadar lahko krajšamo v monoidu?
        \item \colorbox{yellow!30}{\emph{Primer.}} Koliko obrnljivih elementov ima vsak monoid?
        \item \colorbox{yellow!30}{\emph{Primer.}} Naštej obrnljive elemente v $(\N_0, +)$, $(\Z, \cdot)$, $(\Q, \cdot)$, $(\R, \cdot)$, $(\C, \cdot)$.
        \item \colorbox{yellow!30}{\emph{Primer.}} Naj bo $\mathcal{F}(X)$ množica vseh funkcij iz $X$ vase.
        \begin{itemize}
            \item Kadar $f \in \mathcal{F}(X)$ ima levi inverz? Kadar jih ima več?
            \item Kadar $f \in \mathcal{F}(X)$ ima desni inverz? Kadar jih ima več? 
            \item Kadar $f \in \mathcal{F}(X)$ ima inverz?
        \end{itemize}
        \item \colorbox{blue!30}{\textbf{Trditev.}} Ali so levi in desni inverzi elementa $x \in S$ sovpadata?
        \item \colorbox{orange!30}{\textbf{Posledica.}} Kaj velja, če je element $x \in S$ obrnjiv in $yx = 1$?
        \item \colorbox{orange!30}{\textbf{Posledica.}} Koliko lahko inverzov ima obrnjiv element monoida? 
        \item \colorbox{blue!30}{\textbf{Trditev.}} Ali je produkt obrnjivih elementov monoida obrnjiv? Kako dobimo inverz produkta?
        \item \colorbox{yellow!30}{\emph{Opomba.}} Kako lahko definiramo potenco obrnjivega elementa monoida za vsa cela števila? 
    \end{itemize}
\end{enumerate}

\newpage
\section{Uvod v teorijo grup}

\begin{enumerate}
    \item Grupe
    \begin{itemize}
        \item \colorbox{purple!30}{\textbf{Definicija.}} Grupa. Abelova grupa.
        \item \colorbox{yellow!30}{\emph{Opomba.}} Zapiši definicijo grupe preko aksiomov. Enota. Inverz elementa.
        \item \colorbox{yellow!30}{\emph{Opomba.}} Koliko so enot v grupi? Koliko inverzov ima vsak element? Računanje s potenci.
        \item \colorbox{yellow!30}{\emph{Opomba.}} Multiplikativni in aditivni zapis. Dogovor o aditivni grupi.
        \item \colorbox{blue!30}{\textbf{Trditev.}} Pravila krajšanja v grupi.
        \item \colorbox{purple!30}{\textbf{Definicija.}} Končna grupa. Red grupe.
        \item \colorbox{blue!30}{\textbf{Trditev.}} Kako iz monoida dobimo grupo? \df{Množica obrnljivih elementov monoida}.    
    \end{itemize}
    \item Primeri grup
    \begin{itemize}
        \item Navedi primeri in protiprimeri številskih grup za seštevanje in množenje.
        \item Kaj je trvialna grupa?
        \item Kaj je $(\mathcal{F}(X))^*$? Permutacija. \df{Simetrična grupa $\Sim(X)$} množie $S$. Ali je komutativna?
        \item \df{Grupa permutacij $S_n$} končne množice $[n]$:
        \begin{itemize}
            \item Ali je vsaka permutacija produkt disjunktnih ciklov?
            \item Ali je vsaka permutacija produkt transpozicij?
            \item Sode in lihe permutacije. Predznak permutacije. Čemu je enak predznak produkta permutacij?
            \item Čemu je enak red grupe $S_n$?
        \end{itemize}
        \item Množica vseh realnih $n \times n$ matrik $M_n(\R)$:
        \begin{itemize}
            \item Ali je Abelova grupa za seštevanje?
            \item Kaj pa za množenje? \df{Splošna linearna grupa $\GL_n(\R)$}. Ali je Abelova?
            \item Ali lahko $\R$ zamenjamo z poljubnim poljem?
        \end{itemize}
        \item Opiši simetrije kvadrata. \df{Diedrska grupa $D_{8}$}. 
        \begin{itemize}
            \item S čim je enolično določena simetrija?
            \item Ali je $D_8$ Abelova?
        \end{itemize}
        \item \df{Diedrska grupa $D_{2n}$}. Opiši elementi $D_{2n}$.
        \item \df{Diedrska grupa $D_4$} simetrij pravokotnika, ki ne kvadrat.
        \item Direktni produkt grup $G_1, G_2, \ldots, G_n$. Direktna vsota grup.
    \end{itemize}

    \item Podgrupe
    \begin{itemize}
        \item \colorbox{purple!30}{\textbf{Definicija.}} Podgrupa.
        \item \colorbox{yellow!30}{\emph{Opomba.}} Vsaj koliko podgrup ima vsaka grupa? Trvialna podgrupa. Prava podgrupa.
        \item \colorbox{yellow!30}{\emph{Opomba.}} Naj bo $H \leq G$. Ali je enota grupe $G$ vsebovana v $H$?
        \item \colorbox{blue!30}{\textbf{Trditev.}} 3 ekvivalantne trditve o podgrupe $H$ grupe $G$.
        \item \colorbox{yellow!30}{\emph{Opomba.}} Kako karakterizacije podgrupe zgledajo v aditivnem zapisu?
        \item \colorbox{orange!30}{\textbf{Posledica.}} Kadar je končna podmnožica $H$ grupe $G$ podgrupa?
        \item \colorbox{yellow!30}{\emph{Opomba.}} Kakšne oblike inverz vsakega elementa $x \in G$, če je $G$ končna grupa?        
        \item \colorbox{blue!30}{\textbf{Trditev.}} Opiši podgrupe grupe $(\Z, +)$.
        \item \colorbox{blue!30}{\textbf{Trditev.}} Ali je poljuben presek podgrup podgrupa?
        \item \colorbox{purple!30}{\textbf{Definicija.}} Produkt podgrup $H$ in $K$ grupe $G$.
        \item \colorbox{yellow!30}{\emph{Opomba.}} Ali je produkt podgrup nujno podgrupa.
        \item \colorbox{blue!30}{\textbf{Trditev.}} Zadosten pogoj, da bi bil produkt podgrup podgrupa.
        \item \colorbox{yellow!30}{\emph{Opomba.}} Kaj velja, če je $G$ Abelova?
    \end{itemize}

    \newpage
    \item Primeri podgrup
    \begin{itemize}
        \item Določi osnovne podgrupe v $(\C, \cdot)^*$. Ali so podgrupe tudi:
        \begin{itemize}
            \item $\Q^+ = \setb{x \in \Q}{x > 0}$.
            \item $\R^+ = \setb{x \in \R}{x > 0}$.
            \item $\mathbb{T} = \setb{z \in \C}{|z| = 1}$. \df{Krožna grupa $\mathbb{T}$}.
            \item $\mathbb{U}_n = \setb{z \in \C}{z^n = 1}$. \df{$n$-to koreni enote $\mathbb{U}_n$}.
        \end{itemize}
        \item \df{Alternirajoča grupa $A_n$}.
        \item Ali je diedrska grupa $D_{2n}$ podgrupa v $S_n$?
        \item Pokaži da so podgrupe grupe $\GL_n(F)$:
        \begin{itemize}
            \item $\SL_n(F) = \setb{A \in M_n(F)}{\det(A) = 1})$. \df{Specialna linearna grupa $\SL_n$}.
            \item $O_n = \setb{A \in M_n(\R)}{AA^T = I}$. \df{Ortogonalna grupa $O_n$}. \df{Specialna ortogonalna grupa $SO_n$}.
            \item $U_n = \setb{A \in M_n(\C)}{AA^H = I}$. \df{Unitarna grupa $U_n$}. \df{Specialna unitarna grupa $SU_n$}.
        \end{itemize}
        \item \colorbox{blue!30}{\textbf{Trditev.}} \df{Konjugirana podgrupa} podgrupe $H$.
        \item \colorbox{blue!30}{\textbf{Trditev.}} \df{Center $Z(G)$} grupe $G$.
        \item \colorbox{blue!30}{\textbf{Trditev.}} \df{Centralizator $C_G(a)$} elementa $a$ v $G$.
    \end{itemize}

    \item Odseki in Lagrangeev izrek
    
    Naj bo $G$ grupa in $H \leq G$. Definiramo relacijo na $G$ s predpisom $$\all{a, b \in G} a \sim b :\liff a^{-1}b \in H$$.
    \begin{itemize}
        \item \colorbox{blue!30}{\textbf{Trditev.}} Relacija $\sim$ je ekvialenčna.
        \item \colorbox{purple!30}{\textbf{Definicija.}} Ekvivalenčni razred elementa $a \in G$. Levi odsek grupe $G$ po podgrupi $H$.
        \item \colorbox{yellow!30}{\emph{Opomba.}} Kadar $aH = H$?
        \item \colorbox{yellow!30}{\emph{Opomba.}} Kako pišemo odseke, če je $G$ Abelova?
        \item \colorbox{yellow!30}{\emph{Primer.}} Kaj so odseke, če:
        \begin{itemize}
            \item $G = (\R^2, +)$, $H$ abscisna os.
            \item $G = \C^*$, $H = \mathbb{T}$.
            \item $G = S_n$, $H = A_n$.
        \end{itemize}
        \item \colorbox{yellow!30}{\emph{Opomba.}}  S kakšno relacijo dobimo desni odseki?
        \item \colorbox{yellow!30}{\emph{Opomba.}} Ali je grupa $G$ disjunktna unija odsekov?
        \item \colorbox{purple!30}{\textbf{Definicija.}} Faktorska (oz. kvocientna) množica. 
        \item \colorbox{yellow!30}{\emph{Opomba.}} Ali je $G/H$ nujno grupa?
        \item \colorbox{blue!30}{\textbf{Lema.}} Kadar sta dva odseka enaka?
        \item \colorbox{purple!30}{\textbf{Definicija.}} Indeks podgrupe $H$ v grupi $G$.
        \item \colorbox{blue!30}{\textbf{Izrek.}} Lagrangeev izrek.
        \item \colorbox{orange!30}{\textbf{Posledica.}} Kaj lahko povemo o moči vsake podgrupe končne grupe?        
    \end{itemize}

    \item Grupa ostankov
    \begin{itemize}
        \item \colorbox{yellow!30}{\emph{Opomba.}} Naj bo $G$ Abelova. Kako lahko definiramo seštevanje na $G/H$?
        \item \colorbox{blue!30}{\textbf{Trditev.}} Ali je $(G/H, +)$ Abelova?        
        \item Naj bo $n \in \N$. Kadar pravimo da sta $a, b \in \Z$ kongruentni po modulu $n$?
        \item Karakteriziruj kongruentnost z ostanki.
        \item Opiši kongruentnost kot relacijo na $\Z$.
        \item \colorbox{yellow!30}{\emph{Primer.}} \df{Grupa ostankov $\Z_n$} po modulu $n$.
        \item \colorbox{yellow!30}{\emph{Opomba.}} Ali za vsak $n \in \N$ obstaja vsaj ena grupa moči $n$?
    \end{itemize}

    \newpage
    \item Ciklične grupe
    \begin{itemize}
        \item Naj bo $G$ grupa, $a \in G$. Kaj je $\left\langle a \right\rangle $? Ali je to podgrupa grupe $G$? Ali je Abelova?
        \item \colorbox{purple!30}{\textbf{Definicija.}} Ciklična podgrupa. Ciklična grupa. Generator grupe.
        \item \colorbox{yellow!30}{\emph{Primer.}} Ali so ciklične: 
        \begin{itemize}
            \item $\Z$, $\Z_n$.
            \item $\mathbb{U}_n$.
            \item $D_4$.
        \end{itemize}
        \item \colorbox{purple!30}{\textbf{Definicija.}} Naj bo $G$ grupa. Naj bo $a \in G$. Red elementa $a$.
        \item \colorbox{yellow!30}{\emph{Primer.}} Katere elemente v grupe $G$ imajo red $1$?
        \item \colorbox{yellow!30}{\emph{Primer.}} Določi red:
        \begin{itemize}
            \item $1$ v $\Z$, $1$ v $\Z_n$.
            \item $e^{\frac{2 \pi i}{n}}$ v $\mathbb{U}_n$.
            \item Transpozicij v $S_n$.
            \item Simetrij v $D_4$.
        \end{itemize}
        \item \colorbox{blue!30}{\textbf{Trditev.}} Karakterizacija reda elementa (kadar je enak $n \in \N$)?
        \item \colorbox{orange!30}{\textbf{Posledica.}} Kadar je končna grupa ciklična?
        \item \colorbox{orange!30}{\textbf{Posledica.}} Naj bo $G$ končna grupa:
        \begin{itemize}
            \item Kako so povezani redi elementov $a \in G$ in moč $G$?
            \item Naj bo $a \in G$. Čemu je enako $a^{|G|}$?
            \item Kaj če je $|G|$ praštevilo?
        \end{itemize}        
    \end{itemize}

    \item Generatorji grup
    \begin{itemize}
        \item \colorbox{purple!30}{\textbf{Definicija.}} Podgrupa generirana z množico $X$.        
        \item \colorbox{yellow!30}{\emph{Opomba.}} Zakaj je definicija smiselna?
        \item \colorbox{blue!30}{\textbf{Trditev.}} Kako izgledajo elementi $\left\langle X \right\rangle$?   
        \item \colorbox{yellow!30}{\emph{Opomba.}} Kaj če je $G$ Abelova?
        \item \colorbox{purple!30}{\textbf{Definicija.}} Kadar pravimo, da je grupa $G$ generirana z $X$? Generatorji grupe $G$.   
        \item \colorbox{yellow!30}{\emph{Primer.}} Obravnavaj primera:
        \begin{itemize}
            \item S čim je generirana vsaka podgrupa grupe $\Z$?
            \item Naj bo $X \subseteq \Z$. Kaj je $\left\langle X \right\rangle $?
            \item S čim je generirana grupa $\Q^+$?
            \item S čim je generirana grupa $D_{2n}$?
            \item S čim je generirana grupa $S_n$?
            \item S čim je generirana grupa $A_n, \ n \geq 3$?
        \end{itemize}
        \item \colorbox{purple!30}{\textbf{Definicija.}} Končno generirana grupa.      
        \item \colorbox{yellow!30}{\emph{Primer.}} Ali je $(\Z, +)$ končno generirana?        
        \item \colorbox{yellow!30}{\emph{Primer.}} Pokaži, da $\Q$ ni končno generirana.
    \end{itemize}
\end{enumerate}