\section{Cela števila}

\begin{enumerate}
    \item Osnovni izrek o deljenju celih števil
    \begin{itemize}
        \item Načelo dobre urejenosti v $\N$.
        \item Načeli dobre urejenosti v $\Z$.
        \item \colorbox{blue!30}{\textbf{Izrek.}} Osnovni izrek o deljenju celih števil. Ostanek.
    \end{itemize}

    \item Največji skupni delitelj
    \begin{itemize}
        \item \colorbox{purple!30}{\textbf{Definicija.}} Kadar pravimo, da celo število $k \neq 0$ deli celo število $m$? Zapis.
        \item \colorbox{purple!30}{\textbf{Definicija.}} Delitelj. Število $m$ deljivo s številom $k$.
        \item \colorbox{purple!30}{\textbf{Definicija.}} Skupni delitelj. Največji skupni delitelj.
        \item \colorbox{blue!30}{\textbf{Izrek.}} Obstoj največjega skupnega delitelja. Kako lahko ga zapišemo?
        \item \colorbox{purple!30}{\textbf{Definicija.}} Tuji števili.
        \item \colorbox{orange!30}{\textbf{Posledica.}} Kadar sta števili $m$ in $n$ tuji?
    \end{itemize}

    \item Osnovni izrek aritmetike
    \begin{itemize}
        \item \colorbox{purple!30}{\textbf{Definicija.}} Praštevila.
        \item \colorbox{blue!30}{\textbf{Lema.}} Evklidova lema.
        \item \colorbox{blue!30}{\textbf{Izrek.}} Osnovni izrek aritmetike.
        \item \colorbox{blue!30}{\textbf{Izrek.}} Ali je praštevil neskončno?
    \end{itemize}
\end{enumerate}