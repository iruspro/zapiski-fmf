\section{Uvod v teorijo kolobarjev}

\begin{definicija}
    Naj bo $K$ neprazna množica z operacijama $+$ in $\cdot$. Pravimo, da je $(K, +, \cdot)$ \df{kolobar}, če
    \begin{enumerate}
        \item $(K, +)$ je Abelova grupa (enota: $0$, inverz od $a$: $-a$).
        \item $(K, \cdot)$ je monoid, tj. kolobar vedno ima enoto za $\cdot$, označimo jo z $1$, in rečemo, da je $1$ \df{enica} kolobarja $K$.
        \item Za vse $a, b, c \in K$ velja, da $a(b+c) = ab+ac$ in $(a+b)c = ac+bc$.
    \end{enumerate}
    Če je množenje komutativno, pravimo, da je $K$ \df{komutativen kolobar}.
\end{definicija}

\begin{zgled}
    Primeri kolobarjev.
    \begin{itemize}
        \item $(\Z, +, \cdot)$ je komutativen kolobar.
        \item $\Q, \R, \C$ so komutativni kolobarji.
        \item $(\R^{n \times n}, +, \cdot)$ je kolobar.
        \item Naj bo $X \subseteq \R$, $\R^X = \set{f: X \to \R}$. Definiramo $(f+g)(x) = f(x) + g(x), \ (fg)(x) = f(x)g(x)$. $\R^X$ je komutativen kolobar.
    \end{itemize}
\end{zgled}

\begin{definicija}
    Naj bo $K$ kolobar.
    \begin{itemize}
        \item $l \in K \setminus \set{0}$ je \df{levi delitelj niča}, če $\some{y \in K \setminus \set{0}. ly = 0}$.
        \item $d \in K \setminus \set{0}$ je \df{desni delitelj niča}, če $\some{y \in K \setminus \set{0}. yd = 0}$.
        \item $x \in K \setminus \set{0}$ je \df{delitelj niča}, če je levi ali desni delitelj niča. 
        \item $x \in K$ je \df{idempotent}, če $x^2 = x$.
        \item $x \in K$ je \df{nilpotent}, če $\some{n \in \N} x^n = 0$.
    \end{itemize}
\end{definicija}

\begin{zgled}
    Primeri deliteljev niča, idempotentov in nilpotentov.
    \begin{itemize}
        \item V $\R^2$ velja $\begin{bmatrix}
            1 & 0 \\ 0 & 0
        \end{bmatrix} \begin{bmatrix}
            0 & 0 \\ 0 & 1
        \end{bmatrix} = 0$.
        \item Če je $K$ poljuben kolobar, potem $1$ in $0$ sta idempotenta.
        \item V $\R^5$ matrika $\begin{bmatrix}
            0 & 1 & & & \\ 
            & 0 & 1 & & \\ 
            & & 0 & 1 & \\ 
            & & & 0 & 1 \\ 
            & & &  & 0 \\ 
        \end{bmatrix}$ je nilpotenta.
    \end{itemize}
\end{zgled}

\begin{definicija}
    \df{Cel kolobar} je komutativen kolobar brez deliteljev niča.
\end{definicija}

\begin{primer}
    $(\Z, +, \cdot)$ je cel kolobar.
\end{primer}

\begin{definicija}
    Naj bo $K$ kolobar.
    \begin{itemize}
        \item Kolobar $K$ je \df{obseg}, če je vsak neničeln element kolobarja $K$ obrnljiv, tj. $K^* = K \setminus \set{0}$. 
        \item \df{Polje} je komutativen obseg.
    \end{itemize}
\end{definicija}

\begin{primer}
    $\Q, \R, \C$ so polja.
\end{primer}

\begin{trditev}
    Obrnljiv element kolobarja $K$ ne more biti delitelj niča.
\end{trditev}

\begin{proof}
    Enostavno.
\end{proof}

\begin{definicija}
    Naj bo $A$ kolobar in $F$ polje. $A$ je \df{algebra} nad $F$, če 
    \begin{enumerate}
        \item $A$ je vektorski prostor nad $F$.
        \item $\alpha(xy) = (\alpha x)y = x(\alpha y)$.
    \end{enumerate}
\end{definicija}

\subsection{Primeri kolobarjev in algeber}
\subsubsection*{Kolobar (algebra) kvadratnih matrik}
Naj bo $K$ kolobar, $K^{n \times n} = M_n(K) = \set{n \times n \text{ matrike z elementi iz } K}$. $K^{n \times n}$ z običajnima $+$ in $\cdot$ je kolobar. Če~je~$F$~polje, potem $F^{n \times n}$ je vektorski prostor in hitro vidimo, da je $F^{n \times n}$ algebra nad $F$.

Bolj splošno: Naj bo $V$ vektorski prostor nad $F$. Vzemimo množico $\End V$. Potem $\End V$ je algebra nad $F$ (rečemo tudi $F$-algebra).

\subsubsection*{Algebra realnih funkcij}
Naj bo $X \subseteq \R^n, \ X \neq \emptyset$. Gledamo funkcije $\R^X$. Na $\R^X$ lahko definiramo $+$, $\cdot$ in množenje s skalarjem iz $\R$ po točkah. $\R^X$ je algebra nad $\R$.

\subsubsection*{Polinomi}
Naj bo $K$ kolobar. \df{Polinom} s koeficienti iz $K$ je formalna vrsta oblike
$$p(x) = \sum_{i \geq 0} a_iX^i = a_0 + a_1X + a_2 X^2 + \ldots + a_k X^k, \ a_i \in K, \ k \geq 0.$$
Manj baročno:
$$(a_0, a_1, \ldots, a_k, 0, 0, \ldots).$$
Torej polinom je končno zaporedje elementov iz $K$.

Naj bo $K[X]$ je množica vseh polinomov s koeficienti iz $K$. V $K[X]$ definiramo seštevanje in množenje:
\begin{itemize}
    \item $\sum_{i \geq 0} a_iX^i + \sum_{i \geq 0} b_iX^i := \sum_{i \geq 0} (a_i + b_i)X^i$.
    \item $\sum_{i \geq 0} a_iX^i \cdot \sum_{i \geq 0} b_iX^i := \sum_{i \geq 0} c_iX^i$, kjer $c_i = \sum_{j \geq 0}^{i} a_{i - j} b_j$.
\end{itemize}
S temi operacijami $K[X]$ postane kolobar.
\begin{opomba}
    Če je $K$ polje, v $K[X]$ lahko vpeljamo množenje s skalarjem:
    \begin{itemize}
        \item $\alpha (\sum_{i \geq 0} a_iX^i) = \sum_{i \geq 0} (\alpha a_i)X^i$
    \end{itemize} 
    Potem $K[X]$ postane algebra nad $K$.
\end{opomba}
\emph{Možni pospološitvi $K[X]$:}
\begin{itemize}
    \item Polinomi več spremenljivk: $K[X_1, \ldots, X_n] = K[X_1, \ldots, X_n][X_n]$.
    \item Če se ne omejimo na končne formalne vsote, dobimo \df{kolobar formalnih potenčnih vrst $K[[X]]$}.
\end{itemize}     

\begin{trditev}
    Velja:
    \begin{itemize}
        \item Če je $K$ komutativen kolobar, je tudi $K[X]$ komutativen.
        \item $K$ je brez deliteljev nična natanko tedaj, ko $K[X]$ brez deliteljev niča.
        \item $K$ je cel kolobar natanko tedaj, ko $K[X]$ cel.
    \end{itemize}
\end{trditev}

\begin{proof}
    Enostavno.
\end{proof}

\subsubsection*{Polje ulomkov celega kolobarja}    
Naj bo $K$ cel kolobar. Gledamo množico $P = \setb{(a,b)}{a \in K; b \in K \setminus \set{0}}$. Na $P$ vpeljamo relacijo:
$$(a, b) \sim (a', b') \liff ab' = a'b.$$

\begin{trditev}
    Relacija $\sim$ je ekvivalenčna.
\end{trditev}

\begin{proof}
    Kot v $\Q$.
\end{proof}

Definiramo $F = P/_\sim$. Ekvivalenčni razred para $(a,b)$ označimo z $\frac{a}{b}$. Definiramo seštevanje in množenje na $F$:
\begin{itemize}
    \item $\frac{a}{b} + \frac{a'}{b'} := \frac{ab' + a'b}{bb'}$.
    \item $\frac{a}{b} \cdot \frac{a'}{b'} = \frac{aa'}{b'b}$.
\end{itemize}
Preveriti moramo, da sta seštevanje in množenje na $F$ res dobro definirani.
\begin{trditev}
    Množica $F$ s tema operacijama je polje. Pravimo mu \df{polje ulomkov kolobarja $K$}.
\end{trditev}

\begin{primer}
    $K = \Z$, potem $F = \Q$.
\end{primer}

\begin{opomba}
    Za ulomki oblike $\frac{a}{1}, \ a \in K$ velja:
    \begin{itemize}
        \item $\frac{a}{1} + \frac{b}{1} = \frac{a+b}{1}$.
        \item $\frac{a}{1} \cdot \frac{b}{1} = \frac{ab}{1}$
    \end{itemize}
    Zato lahko $\frac{a}{1}$ identificiramo z $a$. Torej kolobar $K$ je \df{vložen} v $F$.
\end{opomba}

\subsubsection*{Algebre, ki so obsege}
Gledamo algebre nad $\R$:
\begin{itemize}
    \item $\R$ je algebra nad $\R$, $\R$ polje.
    \item $\C$ je dvorazsežna algebra nad $\R$, $\C$ polje.
\end{itemize}

\begin{trditev}
    Naj bo $A$ algebra nad $\R$. Če je $\dim A$ liho število večje od $1$, potem $A$ ni obseg.
\end{trditev}

\begin{proof}
    Izberimo $a \in A \setminus \Lin \set{1}$ in definiramo endomorfizem $\lin{A}: A \to A, \ \lin{A}x = ax$. Poiščemo s pomočjo karakterisitčnega polinoma delitelji niča.
\end{proof}

\subsubsection*{Algebra kvaternionov}
\begin{primer}
    Vzemimo realni vektorski prostor dimenzije $4$. Naj bo njegova baza $\set{1, i, j, k}$. Označimo ta prostor s $\HH$.
    
    Elementi $\HH$ so oblike $\lambda_1 \cdot 1 + \lambda_2 \cdot i + \lambda_3 \cdot j + \lambda_4 \cdot k$. Zaradi zveze med množenjem in množenjem s skalarji v algebri, dovolj, da definiramo množenje le na baznih vektorjih:
    \begin{itemize}
        \item $1$ je enota za množenje.
        \item Elementi $i, j, k$ med sabo množimo po naslednji shemi:
        $$\xymatrix{
            & i \ar@{->}[rd] &  \\
           k \ar@{->}[ru] &  & j \ar@{->}[ll]
           }$$
        Torej ko gremo v smeri urinega kazalca, dobimo naslednji element ($ij = k$), ki gremo v nasprotni smeri dobimo nasprotni element naslednjega elementa ($kj = -i$).
    \end{itemize}
    Elementi množice $\HH$ imenujemo \df{kvaternione}.

    Naj bo $z = \lambda_1 \cdot 1 + \lambda_2 \cdot i + \lambda_3 \cdot j + \lambda_4 \cdot k$. Element $\overline{z} = \lambda_1 \cdot 1 - \lambda_2 \cdot i - \lambda_3 \cdot j - \lambda_4 \cdot k$ je \df{konjugirani kvaternion}.    
\end{primer}

\begin{trditev}
    $\HH$ je obseg.
\end{trditev}

\begin{proof}
    Dovolj dokazati, da je vsak neničelni element obrnljiv.
\end{proof}

\begin{trditev}
    $\HH$ je algebra.
\end{trditev}

\begin{proof}
    Preverimo usklajenost množenja in množenja s skalarjem.
\end{proof}

Pravimo, da je $\HH$ \df{kvaternionska algebra}. 

Grupa za množenje $(\set{\pm 1, \pm i, \pm j, \pm k}, \cdot)$ je \df{kvaternionska grupa}. Označimo jo z $Q$.
