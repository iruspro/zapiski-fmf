\section{Kombinatorika}

\begin{enumerate}
    \item Osnovna načela kombinatorike
    \begin{itemize}
        \item \colorbox{blue!30}{\textbf{Trditev.}} Načelo produkta.
        \item \colorbox{blue!30}{\textbf{Trditev.}} Posplošeno načelo produkta.
        \item \colorbox{blue!30}{\textbf{Trditev.}} Načelo vsote.
        \item \colorbox{blue!30}{\textbf{Trditev.}} Posplošeno načelo vsote.
        \item \colorbox{blue!30}{\textbf{Trditev.}} Načelo enakosti.
        \item \colorbox{yellow!30}{\emph{Primer.}} Določi moč množice \(\mathcal{P}(A)\), kjer \(|A| = n\).
        \item \colorbox{blue!30}{\textbf{Trditev.}} Načelo dvojnega preštevanja.
        \item \colorbox{yellow!30}{\emph{Primer.}} Eulerjeva funkcija \(\phi\). Določi $\sum_{d | n} \phi(d)$.
        \item \colorbox{blue!30}{\textbf{Trditev.}} Dirichletovo načelo.
        \item \colorbox{yellow!30}{\emph{Opomba.}}  Kombinatorična interpretacija Dirichletovega načela.
        \item \colorbox{yellow!30}{\emph{Primer.}} Naj bo $X \subset [100], \ |X|=10$. Pokaži, da $X$ vsebuje dve disjunktni podmnožici z isto vsoto.
    \end{itemize}

    Izpitna vprašanja:
    \begin{itemize}
        \item Katera so osnovna načela kombinatoričnega preštevanja? Kako Dirichletovo načelo izrazimo v jeziku funkcij? Kako z enim izmed osnovnih načel dokažemo formulo $\sum_{d | n} \phi(d) = n$?
    \end{itemize}

    \item Število preslikav
    \begin{itemize}
        \item \colorbox{purple!30}{\textbf{Definicija.}} Množica vseh preslikav iz \(A\) v \(B\).
        \item \colorbox{purple!30}{\textbf{Definicija.}} Padajoča potenca. Naraščajoča potenca. \(n\)-fakulteta.
        \item \colorbox{blue!30}{\textbf{Trditev.}}  Koliko je preslikav iz \(n\)-množici v \(k\)-množico? Koliko je injektivnih? Koliko je bijektivnih?
    \end{itemize}

    \item Binomski koeficienti in binomski izrek
    \begin{itemize}
        \item \colorbox{purple!30}{\textbf{Definicija.}} Binomski koeficienti.
        \item \colorbox{blue!30}{\textbf{Trditev.}} Binomska števila.
        \item \colorbox{yellow!30}{\emph{Opomba.}} Čemu je enako \(\binom{0}{0}\) in \(\binom{n}{k}\) za \(0 \leq k \leq n\)?
        \item \colorbox{purple!30}{\textbf{Definicija.}} Množica vseh \(k\)-podmnožic množice \(N\).
        \item \colorbox{blue!30}{\textbf{Trditev.}} Moč \(\binom{N}{k}\).
        \item \colorbox{blue!30}{\textbf{Trditev.}} Rekurzivna formula za binomska števila.
        \item \colorbox{purple!30}{\textbf{Definicija.}} Paskalov trikotnik.
        \item \colorbox{blue!30}{\textbf{Izrek.}} Binomski izrek. 
        \item \colorbox{yellow!30}{\emph{Opomba.}} Kaj sta \(a\) in \(b\) v binomskem izreku?
    \end{itemize}

    Izpitna vprašanja:
    \begin{itemize}
        \item Koliko je vseh preslikav med končnima množicama, koliko je vseh injektivnih preslikav, bijektivnih preslikav in surjektivnih preslikav? Zapišite binomski izrek.
    \end{itemize}

    \item Izbori
    
    Naj bo \(N\) \(n\)-množica. Opazujemo izbori \(k\) elementov.
    \begin{itemize}
        \item Koliko je urejenih izborov z ponavljanjem in brez?
        \item Koliko je neurejenih izborov brez ponavljanja?
        \item \colorbox{blue!30}{\textbf{Trditev.}} Koliko je neurejenih izborov s ponavljanjem?
    \end{itemize}
    
    Izpitna vprašanja:
    \begin{itemize}
        \item Koliko je urejenih in neurejenih izborov z in brez ponavljanja? Utemeljite formulo za neurejene izbore s ponavljanjem.
    \end{itemize}

    \item Multimnožice
    \begin{itemize}
        \item \colorbox{purple!30}{\textbf{Definicija.}} Permutacija množice \(A\). Inverzija. Soda permutacija. Liha permutacija.
        \item \colorbox{blue!30}{\textbf{Trditev.}} Določi \(|S_n|\).
        \item \colorbox{purple!30}{\textbf{Definicija.}} Multimnožica. Kratnost elementa. Moč multimnožice.
        \item \colorbox{yellow!30}{\emph{Opomba.}} Kako formalno podamo multimnožico?
        \item \colorbox{yellow!30}{\emph{Opomba.}} Kakšna zvezna med multimnožico in neurejenim izborom z ponavljanjem?
        \item \colorbox{purple!30}{\textbf{Definicija.}} Permutacija multimnožice.
        \item \colorbox{blue!30}{\textbf{Trditev.}} Število permutacij multimnožice.
        \item \colorbox{purple!30}{\textbf{Definicija.}} Multinomski koeficient.
        \item \colorbox{blue!30}{\textbf{Trditev.}} Multinomski izrek.
    \end{itemize}

    Izpitna vprašanja:
    \begin{itemize}
        \item Kaj je permutacija multimnožice? Definirajte multinomske koeficiente in zapišite multinomski izrek.
    \end{itemize}

    \item Kompozicije naravnega števila
    \begin{itemize}
        \item \colorbox{purple!30}{\textbf{Definicija.}} Kompozicija naravnega števila. Dolžina kompozicije. Velikost kompozicije. Členi kompoziciji.
        \item \colorbox{blue!30}{\textbf{Trditev.}} Število kompozicij števila \(n \in \N\). Število kompozicij števila \(n \in \N\) dolžine \(k\).
        \item \colorbox{purple!30}{\textbf{Definicija.}}  Šibka kompozicija naravnega števila.
        \item \colorbox{blue!30}{\textbf{Trditev.}} Število šibkih kompozicij števila \(n \in \N\) dolžine \(k\).
    \end{itemize}

    Izpitna vprašanja:
    \begin{itemize}
        \item Kaj je kompozicija naravnega števila? Koliko je vseh kompozicij števila \(n\) in koliko jih ima \(k\) členov? Kaj je šibka kompozicija naravnega števila in koliko je takih kompozicij števila \(n\) s \(k\) členi?
    \end{itemize}

    \item Razčlenitve naravnega števila
    \begin{itemize}
        \item \colorbox{purple!30}{\textbf{Definicija.}} Razčlenitev naravnega števila. Dolžina razčlenitve. Velikost razčlenitve. Členi razčlenitve.
        \item \colorbox{purple!30}{\textbf{Definicija.}} Ferrersov diagram razčlenitve. Konjugirana razčlenitev.
        \item \colorbox{blue!30}{\textbf{Trditev.}} Število razčlenitev števila \(n \in \N\). Število razčlenitev števila \(n \in \N\) s \(k\) členi. Število razčlenitev števila \(n \in \N\) z največ \(k\) členi. 
    \end{itemize}

    Izpitna vprašanja:
    \begin{itemize}
        \item Kaj je razčlenitev naravnega števila? Koliko je vseh razčlenitev števila \(n\) s \(k\) členi in koliko s kvečjemu \(k\) členi?
    \end{itemize}

    \item Stirlingova števila I.\ vrste
    \begin{itemize}
        \item \colorbox{purple!30}{\textbf{Definicija.}} Stirlingovo število I.\ vrste. \(C(n,0), \ n > 0\) in \(C(0,0)\)
        \item \colorbox{yellow!30}{\emph{Primer.}} Izračunaj \(C(n,n), \ C(n, 1)\) in \(C(4, 2)\). 
        \item \colorbox{blue!30}{\textbf{Trditev.}} Osnovna rekurzivna zveza za Stirlingova števila I.\ vrste.
        \item Stirlingova matrika I.\ vrste.
        \item \colorbox{blue!30}{\textbf{Trditev.}} Dokaži, da \(x^{\overline{n}} = \sum_{k=0}^{\infty}C(n,k)x^k\).
    \end{itemize}

    Izpitna vprašanja:
    \begin{itemize}
        \item Kako so definirana Stirlingova števila prve vrste in kako jih izračunamo? Zapišite začetni del Stirlingove matrike prve vrste.
    \end{itemize}

    \item Stirlingova števila II.\ vrste in Bellova števila
    \begin{itemize}
        \item \colorbox{purple!30}{\textbf{Definicija.}} Razdelitev množice.
        \item \colorbox{purple!30}{\textbf{Definicija.}} Stirlingovo število II.\ vrste. \(S(n,0), \ n > 0\) in \(S(0,0)\)
        \item \colorbox{yellow!30}{\emph{Primer.}} Izračunaj \(S(n,n), \ S(n, 1)\) in \(S(n, 2)\). 
        \item \colorbox{blue!30}{\textbf{Trditev.}} Osnovna rekurzivna zveza za Stirlingova števila II.\ vrste.
        \item Stirlingova matrika II.\ vrste.
        \item \colorbox{blue!30}{\textbf{Trditev.}} Dokaži, da \(x^n=\sum_{k=0}^{\infty} S(n,k) x^{\underline{k}}\).
        \item \colorbox{blue!30}{\textbf{Trditev.}} Število surjekcij iz \(n\)-množice v \(k\)-množico.
        \item \colorbox{purple!30}{\textbf{Definicija.}} Bellovo število.
        \item \colorbox{blue!30}{\textbf{Trditev.}} Osnovna rekurzivna zveza za Bellova števila.
    \end{itemize}

    Izpitna vprašanja:
    \begin{itemize}
        \item Kako so definirana Stirlingova števila druge vrste in kako jih izračunamo? Zapišite začetni del Stirlingove matrike druge vrste. Kakšna je povezava med temi števili in ekvivalenčnimi relacijami?
    \end{itemize}

    \item Lahova števila
    \begin{itemize}
        \item \colorbox{purple!30}{\textbf{Definicija.}} Lahovo število.
        \item \colorbox{blue!30}{\textbf{Trditev.}} Osnovna rekurzivna zveza za Lahova števila.
        \item \colorbox{blue!30}{\textbf{Trditev.}} Eksplicitna formula za Lahova števila.
    \end{itemize}

    Izpitna vprašanja:
    \begin{itemize}
        \item Kako so definirana Lahova števila in kako jih izračunamo - kako rekurzivno in kako eksplicitno?
    \end{itemize}

    \item Dvajnastera pot
    \begin{itemize}
        \item \colorbox{blue!30}{\textbf{Izrek.}} Dvajnastera pot.
    \end{itemize}

    Izpitna vprašanja:
    \begin{itemize}
        \item Kaj je dvanajstera pot? Zapišite in napolnite ustrezno tabelo.
    \end{itemize}

    \item Načelo vključitev in izključitev
    \begin{itemize}
        \item \colorbox{yellow!30}{\emph{Primer.}} Koliko so števil v \([30]\) ni tujih s \(30\)? Koliko so tujih?
        \item \colorbox{blue!30}{\textbf{Izrek.}} Načelo vključitev in izključitev.
        \item \colorbox{yellow!30}{\emph{Primer.}} Na koliko načinov lahko razporedimo $n$ označenih predmetov v $k$ označenih predalov, če je vsaj en predal prazen?
        \item \colorbox{orange!30}{\textbf{Posledica.}} Naj bo \(X\) \(N\)-množica in so \(A_1, \ldots, A_n \subseteq X\). Koliko je elementov množice \(X\), ki niso v nobeni izmed množic \(A_1, \ldots, A_n\)?
        \item \colorbox{purple!30}{\textbf{Definicija.}} Premestitev množice.
        \item \colorbox{yellow!30}{\emph{Primer.}} Izračunaj število premestitev množice \([n]\).
        \item \colorbox{blue!30}{\textbf{Izrek.}} Naj bo $n = p_1^{e_1} \ldots p_r^{e^r}$ razcep $n \in \N$ na prafaktorji. Čemu je enako \(\phi(n)\)?
    \end{itemize}

    Izpitna vprašanja:
    \begin{itemize}
        \item Formulirajte in dokažite načelo vključitev in izključitev.
    \end{itemize}

    \item Rekurzivne enačbe
    \begin{itemize}
        \item \colorbox{yellow!30}{\emph{Primer.}} Na koliko načinov lahko prehodimo \(n\) stopnic, če vsakič prehodimo \(1\) ali \(2\)?
        \item \colorbox{yellow!30}{\emph{Primer.}} Koliko je dvojiških dreves s korenom z $n$ vozlišč?
        \item \colorbox{blue!30}{\textbf{Izrek.}} Splošna rešitev 2-člene rekurzije. Karakteristična enačba.
        \item \colorbox{purple!30}{\textbf{Definicija.}} \(d\)-člena linearna rekurzija s konstantnimi koeficienti. Homogena rekurzija.
        \item \colorbox{blue!30}{\textbf{Izrek.}} Splošna rešitev \(d\)-člene homogene linearne rekurzije s konstantnimi koeficienti.
        \item Kako rešemo nehomogeno rekurzijo?
    \end{itemize}

    Izpitna vprašanja:
    \begin{itemize}
        \item Pojasnite pojem linearne rekurzivne enačbe s konstantnimi koeficienti. Kako lahko zapišemo splošno rešitev dvočlene rekurzije? Kako formulo dokažemo?
        \item Kakšna je splošna rešitev \(d\)-člene linearne rekurzivne enačbe s konstantnimi koeficienti? Opišite korake dokaza te formule.
    \end{itemize}

    \item Formalne potenčne vrste (Rodovne funkcije)
    \begin{itemize}
        \item \colorbox{purple!30}{\textbf{Definicija.}} Formalna potenčna vrsta zaporedja \((a_n)_n\).
        \item \colorbox{purple!30}{\textbf{Definicija.}} Seštevanje potenčnih vrst, množenje potenčne vrste s skalarji, množenje potenčnih vrst.
        \item \colorbox{yellow!30}{\emph{Opomba.}} Kakšno strukturo ima množica formalnih potenčnih vrst?
        \item \colorbox{purple!30}{\textbf{Definicija.}} Obrnljiva formalna potenčna vrsta.
        \item \colorbox{blue!30}{\textbf{Trditev.}} Karakterizacija obrnljivosti.
        \item \colorbox{purple!30}{\textbf{Definicija.}} Rodovna funkcija.
        \item \colorbox{yellow!30}{\emph{Primer.}} Določi rodovno funkcijo Fibonaccijeva zaporedja.
        \item \colorbox{purple!30}{\textbf{Definicija.}} Odvod formalne potenčne vrste.
        \item \colorbox{blue!30}{\textbf{Trditev.}} Odvod produkta.
        \item \colorbox{yellow!30}{\emph{Primer.}} Naj bo $a_0 = 2, a_1 = 3, a_n = 2a_{n-1}-a_{n-2}, n \geq 2$. Določi splošno formulo za $a_n$.
        \item Koraki splošnega reševanja nekega kombinatoričnega problema.
    \end{itemize}

    Izpitna vprašanja:
    \begin{itemize}
        \item Kaj je formalna potenčna vrsta in kaj je rodovna funkcija? Katere formalne potenčne vrste so obrnljive? Kakšen je splošen recept za reševanje rekurzivnih enačb s pomočjo rodovnih funkcij?
    \end{itemize}

    \item Catalanova števila
    \begin{itemize}
        \item \colorbox{purple!30}{\textbf{Definicija.}} Catalanova števila.
        \item \colorbox{blue!30}{\textbf{Trditev.}} Rekurzivna zveza za Catalanova števila.
        \item \colorbox{blue!30}{\textbf{Trditev.}} Eksplicitna formula za Catalanova števila.
        \item \colorbox{yellow!30}{\emph{Primer.}} Kaj lahko preštejemo s Catalonovi števili?
    \end{itemize}

    Izpitna vprašanja:
    \begin{itemize}
        \item Kaj so Catalanova števila? Naštejte nekaj primerov kombinatoričnih objektov, ki jih preštejejo Catalanova števila.
    \end{itemize}
\end{enumerate}


