\section{Kombinatorika}

\begin{enumerate}
    \item Osnovna načela kombinatorike
    \begin{itemize}
        \item \colorbox{blue!30}{\textbf{Trditev.}} Načelo produkta.
        \item \colorbox{blue!30}{\textbf{Trditev.}} Posplošeno načelo produkta.
        \item \colorbox{blue!30}{\textbf{Trditev.}} Načelo vsote.
        \item \colorbox{blue!30}{\textbf{Trditev.}} Posplošeno načelo vsote.
        \item \colorbox{blue!30}{\textbf{Trditev.}} Načelo enakosti.
        \item \colorbox{yellow!30}{\emph{Primer.}} Določi moč množice \(\mathcal{P}(A)\), kjer \(|A| = n\).
        \item \colorbox{blue!30}{\textbf{Trditev.}} Načelo dvojnega preštevanja.
        \item \colorbox{yellow!30}{\emph{Primer.}} Eulerjeva funkcija \(\phi\). Določi $\sum_{d | n} \phi(d)$.
        \item \colorbox{blue!30}{\textbf{Trditev.}} Dirichletovo načelo.
        \item \colorbox{yellow!30}{\emph{Opomba.}}  Kombinatorična interpretacija Dirichletovega načela.
        \item \colorbox{yellow!30}{\emph{Primer.}} Naj bo $X \subset [100], \ |X|=10$. Pokaži, da $X$ vsebuje dve disjunktni podmnožici z isto vsoto.
    \end{itemize}

    Izpitna vprašanja:
    \begin{itemize}
        \item Katera so osnovna načela kombinatoričnega preštevanja? Kako Dirichletovo načelo izrazimo v jeziku funkcij? Kako z enim izmed osnovnih načel dokažemo formulo $\sum_{d | n} \phi(d) = n$?
    \end{itemize}

    \item Število preslikav
    \begin{itemize}
        \item \colorbox{purple!30}{\textbf{Definicija.}} Množica vseh preslikav iz \(A\) v \(B\).
        \item \colorbox{purple!30}{\textbf{Definicija.}} Padajoča potenca. Naraščajoča potenca. \(n\)-fakulteta.
        \item \colorbox{blue!30}{\textbf{Trditev.}}  Koliko je preslikav iz \(n\)-množici v \(k\)-množico? Koliko je injektivnih? Koliko je bijektivnih?
    \end{itemize}

    \item Binomski koeficienti in binomski izrek
    \begin{itemize}
        \item \colorbox{purple!30}{\textbf{Definicija.}} Binomski koeficienti.
        \item \colorbox{blue!30}{\textbf{Trditev.}} Binomska števila.
        \item \colorbox{yellow!30}{\emph{Opomba.}} Čemu je enako \(\binom{0}{0}\) in \(\binom{n}{k}\) za \(0 \leq k \leq n\)?
        \item \colorbox{purple!30}{\textbf{Definicija.}} Množica vseh \(k\)-podmnožic množice \(N\).
        \item \colorbox{blue!30}{\textbf{Trditev.}} Moč \(\binom{N}{k}\).
        \item \colorbox{blue!30}{\textbf{Trditev.}} Rekurzivna formula za binomska števila.
        \item \colorbox{purple!30}{\textbf{Definicija.}} Paskalov trikotnik.
        \item \colorbox{blue!30}{\textbf{Izrek.}} Binomski izrek. 
        \item \colorbox{yellow!30}{\emph{Opomba.}} Kaj sta \(a\) in \(b\) v binomskem izreku?
    \end{itemize}

    Izpitna vprašanja:
    \begin{itemize}
        \item Koliko je vseh preslikav med končnima množicama, koliko je vseh injektivnih preslikav, bijektivnih preslikav in surjektivnih preslikav? Zapišite binomski izrek.
    \end{itemize}
\end{enumerate}
