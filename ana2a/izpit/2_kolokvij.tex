\section{Ekstremi}
Kako poiščemo ekstreme funkcije $f$ na množici $K$?
\begin{enumerate}
    \item Kandidati v $\Int K$ so stacionarne točke.
    \item Ekstremi na robu $\partial K$:
    \begin{enumerate}
        \item Parametriziramo rob.
        \item Tvorimo Lagrangeevo funkcijo $L = f(x) - \sum_{i=1}^{n}\lambda_i g_i$, kjer $g_i$ so pogoji, in iščemo njene stac.\ točke.
    \end{enumerate}
\end{enumerate}

\section{Integrali s parametri}
\paragraph{Običajen integral s parametri.}
Naj bo \(F(y) = \int_{u}^{v} f(x,y) \, dx, \ u,v \in I, \ y \in D\).
\begin{center}
    \begin{tabular}{ c | c | c | c }
     & Zveznost & Odvod & Vrstni red integriranja \\ \hline
    Zveznost \(f(x,y)\) na \(I \times I \times D\) & + & + & + \\  [0.5ex]
    Zveznost \(\podv{f}{y}(x,y)\) na \(I \times I \times D\) &  & + &   
    \end{tabular}
\end{center}

Velja: \(F'(y) = \int_{\alpha(y)}^{\beta(y)} \podv{f}{y}(x,y) \, dx + \beta'(y)f(\beta(y), y) - \alpha'(y)f(\alpha(y), y)\)

\paragraph{Posplošeni integral s parametri.} Naj bo \(F(y) = \int_{a}^{\infty} f(x,y) \, dx, \ a \in \R, \ y \in D\).
\begin{center}
    \begin{tabular}{ c | c | c | c }
     & Zveznost & Odvod & Vrstni red integriranja \\ \hline 
    Zveznost \(f(x,y)\) na \([a, \infty] \times D\) & + & + & + \\ [0.5ex]
    L.E.K.\ \( y \mapsto \int_{a}^{\infty} f(x,y)\) na \(D\) & + &  & + \\ [0.5ex]
    Zveznost \(\podv{f}{y}(x,y)\) na \([a, \infty] \times D\) &  & + &  \\ [0.5ex]
    L.E.K.\ \( y \mapsto \int_{a}^{\infty} \podv{f}{y}(x,y)\) na \(D\) &  & + & \\ [0.5ex]
    Obstoj \( y \mapsto \int_{a}^{\infty} f(x,y)\) v \(y_0 \in D\) & & + &
    \end{tabular}
\end{center}

\begin{definicija}
    Integral \(F(y) = \int_{a}^{\infty} f(x,y) \, dx\) \df{konvergira enakomerno} na \(D\), če 
    $$\all{\epsilon > 0} \some{b_0 \geq a} \all{b \geq b_0} \all{y \in D} \left|\int_{b}^{\infty} f(x,y) \, dx \right| < \epsilon.$$
\end{definicija}

\subsection{Funkciji gama in beta}
\begin{definicija}
    $\Gamma (s) = \int_{0}^{\infty} x^{s-1}e^{-x} \, dx$ je \df{Eulerjeva funkcija gama}. Velja: $D_\Gamma = (0, \infty)$.
\end{definicija}

\begin{trditev}
    Lastnosti Eulerjeve funkcije gama:
    \begin{itemize}
        \item \(\Gamma(s+1) = s \Gamma(s)\). Če je \(n \in \N\), potem \(\Gamma(n) = (n-1)!\).
        \item \(\Gamma^{(k)} = \Gamma (s) = \int_{0}^{\infty} x^{s-1} \ln^k x \, e^{-x} \, dx\), \(\Gamma \in C^\infty ((0, \infty))\).
    \end{itemize}
\end{trditev}

\begin{definicija}
    \( B(p,q) = \int_{0}^{1}x^{p-1}(1-x)^{q-1} \, dx\) je \df{Eulerjeva funkcija beta}. Velja: \(D_B = (0, \infty) \times (0, \infty)\).
\end{definicija}

\begin{trditev}
    \(\ds \frac{1}{2} B(\frac{\alpha + 1}{2}, \frac{\beta + 1}{2}) = \int_{0}^{\frac{\pi}{2}} \sin^\alpha t \cos^\beta t \, dt\) za \(\alpha, \beta > -1\).
\end{trditev}

\begin{trditev}
    \(\ds B(p,q) = \int_{0}^{\infty} \frac{t^{p-1}}{(1+t)^{p+q}} \, dt\).
\end{trditev}

\begin{posledica}
    \(\ds B(p, 1-p) = \int_{0}^{\infty} \frac{t^{p-1}}{1+t} \, dt\) za \(0<p<1\).
\end{posledica}

\begin{opomba}
    Za \(p \in (0,1)\) velja: \[B(p, 1-p) = \frac{\pi}{\sin (p \pi)}.\]
\end{opomba}

\begin{izrek}[Stirlingova formula]
    \[\lim_{s \to \infty} \frac{\Gamma(s+1)}{s^s \, e^{-s} \, \sqrt{2 \pi s}} = 1.\]
\end{izrek}

\newpage
\section{Riemannov integral}
\begin{izrek}[Uvedba novih spremenljivk]
    Naj bo \(D \subseteq \R^n\). Našo spremenljivko \(x = (x_1, \ldots, x_n) \in D\) podamo kot funkcijo novih, tj. \((x_1, \ldots, x_n) = F(u_1, \ldots, u_n)\), kjer \(F: \Delta \to D\) difeomorfizem.  Potem \[\int_D f(x_1, \ldots, x_n)\, dx_1 \ldots dx_n = \int_\Delta f(F(u_1, \ldots, u_n)) |\det JF(u_1, \ldots, u_n)| \, du_1, \ldots, du_n.\]
\end{izrek}

\paragraph{Vpeljava novih spremenljivk}
\begin{center}
    \begin{tabular}{ c | c | c | c | c }
        & Polarne & Valjne & Sferične & Torusne (\(0 < a < R\)) \\ \hline 
        \(x\) & \(r \cos \phi\) & \(r \cos \phi\) &  \(r \cos \phi \cos \psi\) & \((R + r \cos \psi) \cos \phi\)\\ 
        \(y\) & \(r \sin \phi\) & \(r \sin \phi\) &  \(r \sin \phi \cos \psi\) & \((R + r \cos \psi) \sin \phi \) \\
        \(z\) & & \(z\) & \(r \sin \psi\) & \(r \sin \psi\) \\
        \(\det JF\) & \(r\) & \(r\) & \(r^2 \cos \psi\) & \(r(R+r \cos \psi)\) \\
        Omejitve & \(\phi \in [0, 2 \pi]\) & \(\phi \in [0, 2 \pi]\) & \(\phi \in [0, 2 \pi]\) in \(\psi \in [-\frac{\pi}{2}, \frac{\pi}{2}]\) & \(\phi, \psi \in [0, 2 \pi]\) in \(r \in [0, a]\)
    \end{tabular}
\end{center}

\paragraph{Fizikalne količine}
\begin{enumerate}
    \item Vztrajnostni moment okoli \(z\)-osi:
    \begin{itemize}
        \item Nehomogeno telo: \(J_z = \int_D (x^2+y^2) \rho(x,y,z) \, dV\).
        \item Homogeno telo: \(J_z = \frac{m(D)}{V(D)} \int_D (x^2+y^2) \, dV\), ker \(\rho_0 = \frac{m(D)}{V(D)}\).
    \end{itemize}
    \item Težišče (\(x\)-koordinata):
    \begin{itemize}
        \item Nehomogeno telo: \(x_T(D) = \frac{1}{m(D)} \int_D x \, \rho(x,y,z) \, dV \), kjer \(m(D) = \int_D \rho(x,y,z) dV\).
        \item Homogeno telo: \(x_T(D) = \frac{1}{V(D)} \int_D x \, dV \)
    \end{itemize}
\end{enumerate}

\section{Splošno}
\subsection{Ideji in nasveti}
\begin{itemize}
    \item Odvod lihe funkcije je soda funkcija. Odvod sode funkcije je liha funkcija.
    \item \(\int_{0}^{2 \pi} \cos^{2k+1} \psi \, d \psi = 0\) za \(k = 0, 1, \ldots\)
    \item Za izračun gravitacijske sile sprojecirajmo vse sile na os rezultante.
    \item 3D sliko lahko si predstavljamo s pomočjo nivojnic.
    \item Opazimo simetrije, npr. rotacijsko: \(f(z, x^2+y^2)\) itd.
\end{itemize}

\subsection{Prostornine}
\begin{itemize}
    \item Tetraedr: \(V = \frac{1}{3}S_{o} h\).
    \item Valj: \(V = \pi r^2 h, \ S = 2\pi rh+2\pi r^2\).
    \item Sfera: \(V = \frac{4}{3} \pi r^3, \ S = 4 \pi r^2\).
\end{itemize}

\subsection{Teylor}
\textbf{Taylorjeva formula:} $\displaystyle f(a+h) = f(a) + (D_hf)(a) + \frac{1}{2!}(D_h^2f)(a) + \ldots + \frac{1}{k!} (D_h^kf)(a) + R_k$,

kjer je $D_h = h_1D_1 + h_2D_2 + \ldots + h_nD_n$ \df{odvod v smeri $h$} in $R_k = \frac{1}{(k+1)!} (D_h^{k+1}f)(a + \theta h)$ \df{ostanek}.
\begin{center}
    \begin{tabular}{ l r }
     \(\ds e^x = \sum_{n=0}^{\infty} \frac{x^n}{n!} \ (R=\infty)\) & \(\ds \qquad \sin x = \sum_{n=0}^{\infty}(-1)^n \frac{x^{2n+1}}{(2n+1)!} \ (R=\infty)\) \\
     \(\ds \cos x = \sum_{n=0}^{\infty} (-1)^n \frac{x^{2n}}{(2n)!} \ (R=\infty)\) & \(\qquad \ds \ln (1+x) = \sum_{n=1}^{\infty}(-1)^{n-1}\frac{x^n}{n} \ (R=1)\) \\
     \(\ds (1+x)^\alpha = \sum_{n=0}^{\infty}\binom{\alpha}{n}x^n \ (R=1)\)
    \end{tabular}
\end{center}

\subsection{Hiperbolične funkcije}
\(\displaystyle \sinh x = \frac{e^x - e^{-x}}{2}, \ \cosh x = \frac{e^x + e^{-x}}{2}, \ \tanh x = \frac{e^x - e^{-x}}{e^x + e^{-x}}, \ \cosh^2 x - \sinh^2 x = 1\).

\subsection{Trigonometrija}
\begin{center}
    \begin{tabular}{ l r }
     \(\sin x + \sin y = 2 \sin \frac{x+y}{2} \cos \frac{x-y}{2}\) & \(\qquad \sin x \cos y = \frac{1}{2}\left(\sin(x+y) + \sin (x-y)\right)\) \\ [0.5ex]
     \(\cos x + \cos y = 2 \cos \frac{x+y}{2} \cos \frac{x-y}{2}\) & \(\qquad \cos x \cos y = \frac{1}{2}\left(\cos(x+y) + \cos (x-y)\right)\) \\ [0.5ex]
     \(\cos x - \cos y = -2 \sin \frac{x+y}{2} \sin \frac{x-y}{2}\) & \(\qquad \sin x \sin y = \frac{1}{2}\left(\cos(x-y) - \cos (x+y)\right)\)
    \end{tabular}
\end{center}

\subsection{Nedoločeni integral}

\subsubsection{Integracija racionalnih funkcij} \

\textbf{Metoda nastavka.} Integriramo \(R(x) = p(x) + \frac{r(x)}{q(x)}\).
    \begin{center}
        \begin{tabular}{ l c r }
        \(\ds \frac{1}{(x-a)^k} \leadsto A \ln |x-a|,\) & \(\ds \qquad \frac{1}{(x^2+bx+c)^l} \leadsto B \ln |x^2+bx+c| + C \arctan \frac{2x+b}{\sqrt{4c-b^2}},\) &
        \(\ds \qquad \frac{\widetilde{r}(x)}{\widetilde{q}(x)}\),
        \end{tabular}
    \end{center}
    kjer polinom $\widetilde{q}$ dobimo iz polinoma $q$ z znižanjem potence vsakega faktorja za ena, polinom $\widetilde{r}$ pa ima stopnji za eno nižjo kot $\widetilde{q}$.
    Število neznak je enako stopnje polinoma $q$.

\subsubsection{Integracija korenckih funkcij}
\begin{itemize}
    \item Integrale oblike $\displaystyle \int R(x, \sqrt[n]{\frac{ax+b}{cx+d}}) \, dx$ integriramo z uvedbo nove spremenljivke $t = \sqrt[n]{\frac{ax+b}{cx+d}}$. Tako dobimo integral racionalne funkcije v spremenljivke $t$.
    \item Integrale oblike $\displaystyle \int \frac{p(x)}{\sqrt{ax^2+bx+c}} \, dx$ računamo z postopkom:
    \begin{enumerate}
        \item Če je $p$ konstanta, z zapisom v temenski obliki integral prevedemo:
        \begin{center}
            \begin{tabular}{ l r }
            \(\ds \int \frac{dx}{\sqrt{a^2-x^2}} = \arcsin \left(\frac{x}{a}\right) + C\) & \(\ds \qquad \int \frac{dx}{\sqrt{x^2 - a^2}} = \ln |x + \sqrt{x^2-a^2}| + C\) \\
            \(\ds \int \frac{dx}{\sqrt{x^2 + a^2}} = \ln|x+\sqrt{x^2+a^2}| + C\)
            \end{tabular}
        \end{center}
        \item Če je $p$ poljuben, pa uporabimo nastavek:
        $$\displaystyle \int \frac{p(x)}{\sqrt{ax^2+bx+c}} \, dx = \widetilde{p}(x) \sqrt{ax^2+bx+c} + \int \frac{A \, dx}{\sqrt{ax^2+bx+c}},$$
        kjer $\widetilde{p}$ ima stopnjo 1 manj kot $p$ in je $A$ konstanta.
    \end{enumerate}
    \item Integrale oblike $\displaystyle \int R \, (x, \sqrt{ax^2+bx+c}) \, dx$ vedno lahko z univerzalno substitucijo prevedemo na integral racionalne funkcije:
    \begin{center}
        \begin{tabular}{ l r }
        \(\ds a>0: \ \sqrt{ax^2+bx+c} = \sqrt{a}(u-x)\) & \(\ds \qquad a<0: \ \sqrt{ax^2+bx+c} = \sqrt{-a}(x-x_1)u\),
        \end{tabular}
    \end{center}
    kjer je $x_1$ ničla kvadratne funkcije. Ta metoda v principu vedno deluje, ni pa nujno najbolj optimalna.
    \item Pri integralih oblike $\displaystyle \int \frac{dx}{(x+\alpha)^k \sqrt{ax^2+bx+c}}$ se splača uvesti novo spremenljivko $t = \frac{1}{x+\alpha}$.
\end{itemize}


\subsubsection{Integracija trigonometričnih funkcij}
\begin{itemize}
    \item Integrale oblike $\displaystyle \int_0^\frac{\pi}{2} \sin^px \cos^q x \, dx$, kjer sta $p,q > -1$, računamo s beto funkcijo.
    \item Integrale oblike $\displaystyle \int R \, (\sin x, \cos x) \, dx$ lahko z univerzalno trigonometrično substitucijo $\displaystyle t = \tan \left(\frac{x}{2}\right)$ prevedemo na integral racionalne funkcije spremenljivke $t$. Pri tem:
    $$\bullet \ dx = \frac{2dt}{1+t^2} \qquad \bullet \ \cos x = \frac{1-t^2}{1+t^2} \qquad \bullet \ \sin x = \frac{2t}{1+t^2}$$
    Pri uporabe metode, če se da, poskusimo na začetku z uporabo adicijskih izrekov potence čim bolj znižati, da dobimo bolj enostavno racionalno funkcijo.
\end{itemize}

\subsubsection*{Dodatek}
\begin{center}
    \begin{tabular}{ l r }
     \(\ds e^{ax} \sin (bx) \, dx = \frac{e^{ax}}{a^2+b^2}(a  \sin (bx) - b \cos (bx))\) &
    \(\ds \qquad e^{ax} \cos (bx) \, dx = \frac{e^{ax}}{a^2+b^2}(a  \cos (bx) + b \sin (bx)) \) \\ [2ex]
    \(\ds \frac{dx}{x^2-a^2} = \frac{1}{2a} \ln \left| \frac{x-a}{x+a} \right|\) & \(\ds \int \frac{dx}{a^2+b^2 x^2} = \frac{1}{ab} \arctan \left(\frac{bx}{a}\right) + C\)
    \end{tabular}
\end{center}