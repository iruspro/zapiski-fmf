\section{Integral s parametri}
Naj bo $f:[a,b]_x \times [c,d]_y \to \R$ funkcija. Gledamo funkcijo $\ds F(y) = \int_{a}^{b}f(x,y) \, dx$, kjer $y \in [c,d]$ je \df{parameter}.

Zanima nas v kakšni so povezavi lastnosti funkcije $f$ in funkcije $F$.

\begin{enumerate}
    \item Integral s parametri
    \begin{itemize}
        \item \colorbox{purple!30}{\textbf{Definicija.}} Lokalno kompaktna podmnožica.
        \item \colorbox{blue!30}{\textbf{Trditev.}} Zadostni pogoj za zveznost funkcije $F(u,v,y) = \int_{u}^{v}f(x,y) \, dx$.
        \item \colorbox{orange!30}{\textbf{Posledica.}} Zadostni pogoj za zveznost funkcije $F(y) = \int_{a}^{b}f(x,y) \, dx$.
    \end{itemize}
    \item Odvajanje integrala s parametri
    \begin{itemize}
        \item \colorbox{blue!30}{\textbf{Trditev.}} Zadostni pogoj, da smemo zamenjati vrstni red odvajanja in integriranja v \(\frac{d}{dy} \int_{a}^{b}f(x,y) \, dx\). \\ Kaj lahko povemo o funckiji $F(y) = \int_{a}^{b}f(x,y) \, dx$?
        \item \colorbox{orange!30}{\textbf{Posledica.}} Čemu je enako \(\frac{d}{dy} \int_{\alpha(y)}^{\beta(y)}f(x,y) \, dx\)? Pri kakšnih zadostnih pogojih?
        \item \colorbox{orange!30}{\textbf{Posledica.}} Naj bo \(y \in D^\text{odp} \subseteq \R^n\). Zadostni pogoj, da smemo zamenjati vrstni red odvajanja in integriranja v \(\podv{}{y_j} \int_{a}^{b}f(x,y) \, dx\) za vse \(j \in \set{1, \ldots, n}\).  Kaj lahko povemo o funckiji $F(y) = \int_{a}^{b}f(x,y) \, dx$?
    \end{itemize}

    \item Integral integrala s parametri
    \begin{itemize}
        \item \colorbox{blue!30}{\textbf{Trditev.}} Zadostni pogoji, da smemo zamenjati vrstni red integriranja v \(\int_{c}^{d} \left(\int_{a}^{b} f(x,y) \,dx \right) \,dy\).
    \end{itemize}

    \item Posplošeni integral s parametri
    
    Naj bo $Y$ neka množica, $a \in \R$, $f: [a, \infty)_x \times Y_y \to \R$ funkcija. Standardni predpostavki:
    \begin{itemize}
        \item[] Funkcija $f$ za vsak $y \in Y$ zvezna, tj. $x \mapsto f(x,y)$ zvezna na $[a, \infty)$ za vsak \(y \in Y\).
        \item[] Za vsak \(y \in Y\) obstaja integral \( \ds F(y)= \int_{a}^{\infty} f(x,y) \,dx \)
    \end{itemize}
    \begin{itemize}
        \item \colorbox{purple!30}{\textbf{Definicija.}} Kadar \( F(y) = \int_{a}^{\infty} f(x,y) \, dx\) konvergira enakomerno na \(Y\)?
        \item \colorbox{blue!30}{\textbf{Trditev.}} Zadostni pogoji za zveznost funkcije $F(y) = \int_{a}^{\infty} f(x,y) \, dx$.
        \item \colorbox{purple!30}{\textbf{Definicija.}} Kadar \( F(y) = \int_{a}^{\infty} f(x,y) \, dx\) konvergira lokalno enakomerno na \(Y\)?
        \item \colorbox{blue!30}{\textbf{Trditev.}} Test enakomerne konvergence.
        \item \colorbox{blue!30}{\textbf{Trditev.}} Zadostni pogoji, da smemo zamenjati vrstni red integriranja v \(\int_{a}^{\infty}\left(\int_{c}^{d} f(x,y) \, dy\right) \, dx\).
        \item \colorbox{blue!30}{\textbf{Trditev.}} Zadostni pogoji, da smemo zamenjati vrstni red integriranja v \(\int_{a}^{\infty}\left(\int_{c}^{\infty} f(x,y) \, dy\right) \, dx\), če je $f$ nenegativna.
        \item \colorbox{blue!30}{\textbf{Trditev.}} Zadostni pogoji, da smemo zamenjati vrstni red integriranja v \(\int_{a}^{\infty}\left(\int_{c}^{\infty} |f(x,y)| \, dy\right) \, dx\).
        \item \colorbox{blue!30}{\textbf{Trditev.}} Zadostni pogoji, da smemo zamenjati vrstni red odvajanja in integriranja v \(\frac{d}{dy} \int_{a}^{\infty}f(x,y) \, dx\).
        \\ Kaj lahko povemo o funckiji $F(y) = \int_{a}^{\infty}f(x,y) \, dx$?
        \item \colorbox{orange!30}{\textbf{Posledica.}} Naj bo \(y \in D^\text{odp} \subseteq \R^n\). Zadostni pogoj, da smemo zamenjati vrstni red odvajanja in integriranja v \(\podv{}{y_j} \int_{a}^{\infty}f(x,y) \, dx\) za vse \(j \in \set{1, \ldots, n}\).  Kaj lahko povemo o funckiji $F(y) = \int_{a}^{\infty}f(x,y) \, dx$?
    \end{itemize}
    \item Eulerjeva funkcija gama
    \begin{itemize}
        \item \colorbox{purple!30}{\textbf{Definicija.}} Eulerjeva funkcija gama.
        \item \colorbox{blue!30}{\textbf{Trditev.}} Lastnosti Eulerjeve funkcije gama.
        \item \colorbox{blue!30}{\textbf{Izrek.}} S čim je enolično določena Eulerjeva funkcija gama?
        \item \colorbox{blue!30}{\textbf{Izrek.}} Stirlingova forumla.
        \item \colorbox{orange!30}{\textbf{Posledica.}} Čemu je enako \(\lim_{n \to \infty} \frac{n!}{n^n \, e^{-n} \, \sqrt{2 \pi n}}\)? Kaj to pomeni?
    \end{itemize}

    \item Eulerjeva funkcija beta
    \begin{itemize}
        \item \colorbox{purple!30}{\textbf{Definicija.}} Eulerjeva funkcija beta.
        \item \colorbox{blue!30}{\textbf{Trditev.}} Lastnosti Eulerjeve funkcije beta.
        \item \colorbox{blue!30}{\textbf{Trditev.}} Kaj če v \(B(p,q)\) vpeljamo \(x = \sin^2 t\)?
        \item \colorbox{blue!30}{\textbf{Trditev.}} Kaj če v \(B(p,q)\) vpeljamo \(t = \frac{t}{1+t}\)?
        \item \colorbox{orange!30}{\textbf{Posledica.}} Čemu je enako \(B(p, 1-p)\)?
        \item \colorbox{orange!30}{\textbf{Posledica.}} Čemu je enako \(B(\frac{1}{2}, \frac{1}{2})\)?
        \item \colorbox{blue!30}{\textbf{Izrek.}} Osnovna povezava med \(B\) in \(\Gamma\).
        \item \colorbox{orange!30}{\textbf{Posledica.}} Čemu je enako \(\Gamma(\frac{1}{2})\)?
    \end{itemize}
\end{enumerate}

