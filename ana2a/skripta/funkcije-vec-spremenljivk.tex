\section{Funckcije več spremenljivk}
\subsection{Prostor $\R^n$}
\begin{enumerate}
    \item Prostor $\R^n$
    \begin{itemize}
        \item \colorbox{purple!30}{\textbf{Definicija.}} Prostor $\R^n$. Seštevanje in množenje s skalarjem na $\R^n$. Ali je $\R^n$ vektorksi prostor?
        \item \colorbox{purple!30}{\textbf{Definicija.}} Skalarni produkt na $\R^n$. Norma vektorja na $\R^n$. Metrika na $\R^n$.
        \item \colorbox{purple!30}{\textbf{Definicija.}} Zaprt kvader. Odprt kvader.
        \item \colorbox{yellow!30}{\emph{Opomba.}} Ali imata prostori $(\R^n, d_2)$ in $(\R^n, d_\infty)$ isto topologijo?
        \item \colorbox{blue!30}{\textbf{Izrek.}} Karakterizacija kompaktnosti množic v $\R^n$. \todo{*}
    \end{itemize}

    \item Zaporedja v $\R^n$
    \begin{itemize}
        \item \colorbox{purple!30}{\textbf{Definicija.}} Zaporedje v $\R^n$.
        \item \colorbox{yellow!30}{\emph{Opomba.}} Koliko realnih zaporedij porodi zaporedje v $\R^n$?
        \item \colorbox{blue!30}{\textbf{Trditev.}} Karakterizacija konvergence zaporedij v $\R^n$ (porojene podzaporedja).
    \end{itemize}
\end{enumerate}

\subsection{Zveznost preslikav iz $\R^n$ v $\R^m$}
\begin{enumerate}
    \item Zveznost preslikav
    
    Naj bo $D \subseteq \R^n$, $f: D \to \R^m$ preslikava. 
    \begin{itemize}
        \item \colorbox{purple!30}{\textbf{Definicija.}} Kadar je $f$ zvezna v točki $a \in D$? Kadar je $f$ zvezna na~$D$?
        \item \colorbox{blue!30}{\textbf{Trditev.}} Karakterizacija zveznosti $f$ v točki $a \in D$ z zaporedji.
        \item \colorbox{purple!30}{\textbf{Definicija.}} Kadar je $f$ enakomerno zvezna na $D$?
        \item \colorbox{blue!30}{\textbf{Trditev.}} Kaj lahko povemo o zvezni preslikavi na kompaktu? \todo{*}
        \item \colorbox{blue!30}{\textbf{Trditev.}} Kaj lahko povemo o slike zvezne preslikave na kompaktu?
        \item \colorbox{purple!30}{\textbf{Definicija.}} Kadar je $f$ $C$-Lipschitzova?
        \item \colorbox{blue!30}{\textbf{Trditev.}} V kakšni zvezi so $C$-Lipschitzovost, enakomerna zveznost in zveznost?
        \item \colorbox{blue!30}{\textbf{Trditev.}} Kaj lahko povemo o vsote, razlike, produktu in kvocientu zveznih v točki $a \in D$ funkcij?
        \item \colorbox{blue!30}{\textbf{Trditev.}} Kaj lahko povemo o kompozitume zveznih preslikav?
        \item \colorbox{yellow!30}{\emph{Zgled.}} Ali je projekcija zvezna na $\R^n$? Kaj pa polinomi in racionalne funkcije?
        \item \colorbox{purple!30}{\textbf{Definicija.}} Funkcija $n$-spremenljivk.
        \item \colorbox{yellow!30}{\emph{Opomba.}} Ali je vsaka zožitev zvezne funkcije zvezna funkcija?
        \item \colorbox{blue!30}{\textbf{Trditev.}} Ali je zvezna v točki $a \in D$ funkcija zvezna v točki $a \in D$ kot funkcija posameznih spremenljivk?
        \item \colorbox{yellow!30}{\emph{Zgled.}} Navedi primer funkcije, ki je zvezna kot funkcija posameznih spremenljivk, vendar ni zvezna.
        \item \colorbox{yellow!30}{\emph{Zgled.}} Navedi primer funkcije, ki ima zvezne zožitve na premice, vendar ni zvezna.
    \end{itemize}

    \item Zveznost preslikav iz $\R^n$ v $\R^m$
    
    Naj bo $D \subseteq \R^n$ in $F: D \to \R^m$ preslikava. Naj bo $x \in D$, potem $F(x) = (y_1, \ldots, y_m) \in \R^m$.

    Lahko pišemo $F(x) = (f_1(x), \ldots, f_m(x))$. Torej $F$ določa $m$ funkcij $n$-spremenljivk.
    \begin{itemize}
        \item \colorbox{blue!30}{\textbf{Trditev.}} Karakterizacija zveznosti $F$ v točki $a \in D$ z koordinatnimi funkciji.
        \item \colorbox{yellow!30}{\emph{Zgled.}} Kaj pomeni, da so linearne preslikave omejene? \todo{*}
        \item \colorbox{blue!30}{\textbf{Trditev.}} Čemu je ekvivalentna omejenost linearnih preslikav?
        \item \colorbox{blue!30}{\textbf{Trditev.}} Ali so linearne preslikave zvezne? Matrična norma.
        \item \colorbox{purple!30}{\textbf{Definicija.}} Afina preslikava.
    \end{itemize}
\end{enumerate}

\newpage
\subsection{Parcialni odvodi in difrenciabilnost}
\begin{enumerate}
    \item Parcialni odvodi
    
    Naj bo $D \subseteq \R^n$, $a \in D$ notranja, $f: D \to \R$ funkcija.
    
    \begin{itemize}
        \item \colorbox{purple!30}{\textbf{Definicija.}} Kadar je $f$ parcialno odvedljiva po spremenljivke $x_j$ v točki $a \in D$? Kaj je parcialni odvod?
        \item \colorbox{yellow!30}{\emph{Opomba.}} Kaj lahko povemo o parcialne odvedljivosti elementarnih funkcij?
    \end{itemize}

    \item Diferenciabilnost
    
    Naj bo $D \subseteq \R^n$, $a \in D$ notranja, $f: D \to \R$ funkcija.
    
    \begin{itemize}
        \item \colorbox{purple!30}{\textbf{Definicija.}} Kadar je $f$ diferenciabilna v točki $a \in D$? Diferencial $f$ v točki $a \in D$.
        \item \colorbox{yellow!30}{\emph{Opomba.}} Ali je diferencial, če obstaja, enolično določen?
        \item \colorbox{yellow!30}{\emph{Opomba.}} Kaj je diferencial v smislu aproksimacije funkcije? Zapis diferenciala v matrične oblike.
        \item \colorbox{blue!30}{\textbf{Trditev.}} (Potrebni pogoji za diferenciabilnost). Zveza med diferencialom, parcialnimi odvodi in zveznostjo.
        \item \colorbox{yellow!30}{\emph{Opomba.}} Kako lahko izrazimo diferencial z parcialnimi odvodi? Gradient funkcije. Operator nabla.
        \item \colorbox{yellow!30}{\emph{Zgled.}} Ali je $f(x, y) = \begin{cases}
            \frac{2xy}{x^2+y^2}; &(x, y) \neq (0,0) \\ 0; &(x, y) = (0,0)
        \end{cases}$ diferenciabilna?
        \item \colorbox{yellow!30}{\emph{Zgled.}} Ali je $f(x, y) = \begin{cases}
            \frac{2x^2y}{x^2+y^2}; &(x, y) \neq (0,0) \\ 0; &(x, y) = (0,0)
        \end{cases}$ zvezna, parcialno odvedljiva, diferenciabilna?
        \item \colorbox{blue!30}{\textbf{Izrek.}} Zadosten pogoj za diferenciabilnost $f$ v točki $a \in D$. \todo{*}
    \end{itemize}

    \item Višji parcialni odvodi
    
    Naj bo $f: D^\text{odp} \subseteq \R^n \to \R$ funkcija. Denimo, da je $f$ parcialno odvedljiva po vseh spremenljivkah na $D$. Parcialni odvodi so tudi funkcije $n$-spremenljivk in morda so tudi te parcialno odvedljive po vseh oz. nekaterih spremenljivkah.
    
    \begin{itemize}
        \item \colorbox{blue!30}{\textbf{Trditev.}} Zadostni pogoj za enakost mešanih odvodov. \todo{*}
        \item \colorbox{purple!30}{\textbf{Definicija.}} Kadar je $f$ razreda $C^k$ na $D$?
        \item \colorbox{purple!30}{\textbf{Definicija.}}  Množica $k$-krat zvezno odvedljivih funkcij. Množica gladkih funkcij. Množica zveznih funkcij.
        \item \colorbox{yellow!30}{\emph{Opomba.}} Kakšno strukturo ima množica $C^k(D)$ z operacijama $+$, $\circ$ in množenja s skalarji?
    \end{itemize}

    \item Diferenciabilnost preslikav
    
    Naj bo $D \subseteq \R^n$, $a \in D$ notranja, $F: D \to \R^m$ preslikava.
    \begin{itemize}
        \item \colorbox{purple!30}{\textbf{Definicija.}} Kadar je $F$ diferenciabilna v točki $a \in D$? Diferencial $F$ v točki $a \in D$.
        \item \colorbox{yellow!30}{\emph{Opomba.}} Ali je diferencial, če obstaja, enolično določen?
        \item \colorbox{yellow!30}{\emph{Zgled.}} Ali sta diferenciabilni $\mc{A}: \R^n \to \R^m$ linearna, $F(x) = \mc{A}x$ in $F: 
        \R^{n \times n} \to \R^{n \times n}, \ F(X) = X^2$?
        \item \colorbox{blue!30}{\textbf{Izrek.}} Karakterizacija diferenciabilnosti $F$ v točki $a \in D$ s koordinatnimi funkciji. 
        \item \colorbox{yellow!30}{\emph{Opomba.}} Kako se izraža diferencial $F$ v točki $a \in D$ z koordinatnimi funkciji? Jacobijeva matrika.
        \item \colorbox{orange!30}{\textbf{Posledica.}} Zadosten pogoj za diferenciabilnost $F$ v točki $a$.
        \item \colorbox{purple!30}{\textbf{Definicija.}} Kadar je $F$ razreda $C^k$ na $D$?
        \item \colorbox{blue!30}{\textbf{Izrek.}} Verižno pravilo. \todo{*}
        \item \colorbox{yellow!30}{\emph{Opomba.}} Kako se izraža diferencial kompozituma funkcij z Jacobijevimi matriki?
        \item \colorbox{orange!30}{\textbf{Posledica.}} Verižno pravilo za funkcijo $n$-spremenljivk. Odvod funkcije po \(i\)-te spremenljivke.
    \end{itemize}
\end{enumerate}

\newpage
\subsection{Izrek o implicitni funkciji}
\begin{enumerate}
    \item Izrek o inverzni preslikavi
    
    Naj bo $D, \Omega \subseteq \R^n$ odprti, $\Phi: D \to \Omega$ preslikava razreda $C^1(D)$. Kakšne so zadostni pogoji za (lokalno) obrnljivost preslikave $\Phi$?
    
    \begin{itemize}
        \item \colorbox{purple!30}{\textbf{Definicija.}} Kadar rečemo, da je $\Phi$ $C^k$-difeomorfizem?
        \item \colorbox{yellow!30}{\emph{Zgled.}} Ali je $f: \R \to \R, \ f(x) = x^3$ difeomorfizem?        
        \item \colorbox{blue!30}{\textbf{Trditev.}} Potreben pogoj, da je $\Phi$ difeomorfizem.
        \item \colorbox{blue!30}{\textbf{Posledica.}} Kako izračunamo diferencial inverzne preslikave?

        \item \colorbox{yellow!30}{\emph{Zgled.}} Ali velja obrat? $\Phi: \R^2 \to \R^2, \ \Phi(x,y) = (e^x \cos y, e^x \sin y)$.
        \item \colorbox{blue!30}{\textbf{Lema.}} Lagrangeev izrek za funkcijo več spremenljivk.
        \item \colorbox{orange!30}{\textbf{Posledica.}} Kaj če so vsi parcialno odvodi omejeni (ocena razlike)?
        \item \colorbox{orange!30}{\textbf{Posledica.}} Kaj če so vsi parcialno odvodi omejeni in je \(f\) preslikava (ocena razlike)?
        \item \colorbox{blue!30}{\textbf{Lema.}} Pomožna trditev za dokaz izreka o inverzni preslikavi.
        \item \textbf{Opomba.} Poenostavitev.
        \item \colorbox{blue!30}{\textbf{Izrek.}} Izrek o inverzni preslikavi. Lokalni difeomorfizem.
        \item \colorbox{orange!30}{\textbf{Posledica.}} Kaj če je $F$ razreda $C^k(D)$?
        \item \colorbox{purple!30}{\textbf{Definicija.}} Kadar rečemo, da je $\Phi$ lokalni $C^k$-difeomorfizem?
        \item \colorbox{yellow!30}{\emph{Opomba.}} Kaj pravi izrek, če je $n = 1$?
        \item \colorbox{yellow!30}{\emph{Zgled.}}  Naj bo $F: \R^{n \times n} \to \R^{n \times n}, \ F(X) = X^2$. Ali je $F$ v okolici točke $I \in \R^{n \times n}$ lokalni difeomorfizem? Kaj to pomeni?
    \end{itemize}
    

    \item Osnovna verzija izreka o implicitni preslikavi
    
    Naj bo $D \subseteq \R^2$ odprta, $(a, b) \in D$, $f: D \to \R$ funkcija razreda $C^1(D)$.
    \begin{itemize}
        \item \colorbox{blue!30}{\textbf{Izrek.}} Osnovna verzija izreka o implicitni funkciji.
        \item \colorbox{orange!30}{\textbf{Posledica.}} Kaj če je $f$ razreda $C^k(D)$?
        \item \colorbox{yellow!30}{\emph{Zgled.}} Kaj če pogoji niso izpolnjeni:
        \begin{enumerate}
            \item $f(x,y) = (x-y)^2, \ f(x,y) = 0$ v okolici točke $(0,0)$.
            \item $f(x,y) = y^3-x, \ f(x,y) = 0$ v okolici točke $(0,0)$.
            \item $f(x,y) = y^2-x^2-x^4, \ f(x,y) = 0$ v okolici točke $(0,0)$.
            \item $f(x,y) = y^2+x^2+x^4,  \ f(x,y) = 0$ v okolici točke $(0,0)$.
        \end{enumerate}
    \end{itemize}

    \item Izrek o implicitni funkciji
    
    Imamo $n+m$ spremenljivk $(x,y)$, kjer $x = (x_1, \ldots, x_n), \ y = (y_1, \ldots, y_m)$ in $m$ enačb. Pričakujemo, da bomo lahko $m$ spremenljivk izrazili kot funkcijo $n$ ostalih, tj. najdemo presliavo $\Phi: D \subseteq \R^n \to \R^m$, da velja $y = \Phi(x)$.

    Naj bo $D \subseteq \R^n_x \times \R^m_y$ odprta, $F = (f_1, \ldots, f_m): D \to \R^m$ preslikava razreda $C^1(D)$.

    \begin{itemize}
        \item \colorbox{purple!30}{\textbf{Definicija.}} Parcialni diferencial na prvo spremenljivko. Parcialni diferencial na drugo spremenljivko.
        \item \colorbox{yellow!30}{\emph{Opomba.}} Kako se izraža parcialna difernicala z matriko? Kako se izraža diferencial $F$ z parcialnima diferenicalama?  
        \item \colorbox{yellow!30}{\emph{Opomba.}} Kako ta diferenical deluje na vektorju $\begin{bmatrix}
            h \\ k
        \end{bmatrix}, \ h \in \R^n, \ k \in \R^m$?
        \item \colorbox{blue!30}{\textbf{Izrek.}} Izrek o implicitni funkciji.
        \item \colorbox{orange!30}{\textbf{Posledica.}} Kaj če je $F$ razreda $C^k(D)$?
        \item \colorbox{yellow!30}{\emph{Zgled.}} Naj bo $F(x,y) = x^2+y^2-1$ in naj rešujemo enačbo $F(x,y) = 0$ v okolici točke $(0,1)$. \\ S pomočjo dokaza izreka o implicitni preslikavi določi $y = \varphi(x)$. 
        \item \colorbox{yellow!30}{\emph{Zgled.}} Naj bosta $f(x,y,z)= y +xy+xz^2$ in $g(x,y,z) = z + zy+x^2$. Dokaži, da sistem enačb $f(x,y,z) = 0$ in $g(x,y,z) = 0$ v okolici točke $(0,0,0)$ enolično določa $C^\infty$ funkciji $y = y(x)$ in $z = z(x)$ in razvij jih v Taylorjevo vrsto do členov reda $2$.
    \end{itemize}

    \newpage
    \item Rang preslikave
    
    Naj bo $D \subseteq \R^n$ odprta, $a \in D$ in $F: D \to \R^m$ preslikava razreda $C^1$.
    \begin{itemize}
        \item \colorbox{yellow!30}{\emph{Zgled.}} Naj bo $F: \R^3 \to \R$ funkcija. Recimo, da rešujemo enačbo $F(x,y,z) = 0$ in vemo, da $F(a,b,c) = 0$. Kakšen je zadosten pogoj za to, da bi lahko vsaj eno spremenljivko izrazili kot funkcijo ostalih?
        \item \colorbox{yellow!30}{\emph{Zgled.}} Naj bosta $F: \R^3 \to \R$ in $G: \R^3 \to \R$ funkciji. Recimo, da rešujemo sistem enačb $F(x,y,z) = 0$ in $G(x,y,z) = 0$ in vemo, da  $F(a,b,c) = 0$ in  $G(a,b,c) = 0$. Kakšen je zadosten pogoj za to, da bi lahko vsaj dve spremenljivke izrazili kot funkcijo tretje?
        \item \colorbox{purple!30}{\textbf{Definicija.}} Rang $F$ v točki $a \in D$. Rang $F$. Kadar rečemo, da je $F$ v točki $a \in D$ maksimalnega ranga?
        \item \colorbox{yellow!30}{\emph{Opomba.}} Ali je maksimalnost ranga lokalno stabilna?
        \item \colorbox{yellow!30}{\emph{Primer.}} Obrnljiva matrika in permutacija koordinat. \textcolor{red}{TODO}
        \item \colorbox{orange!30}{\textbf{Posledica IIF.}} Čemu je ekvivalentna enačba $F(x) = 0$, če je $m < n$?
        \item \colorbox{yellow!30}{\emph{Primer.}} Naj bo $\mc{A}: \R^n \to \R^m$ linearna, $m \leq n$, $\rang \mc{A} = m$. Kakšno dimenzijo ima prostor rešitev enačbe~$\mc{A}x=b$?
        \item \colorbox{orange!30}{\textbf{Posledica.}} Kaj lahko povemo o $F$ v točki $a \in D$, če je $\rang_a F = m$, če $m \leq n$?
    \end{itemize}
\end{enumerate}

\newpage
\subsection{Podmnogoterosti v $\R^n$}
Podmnogoterosti je posplošitev pojmov "`krivulja"' in "`ploskev"'.
\begin{enumerate}
    \item Podmnogoterosti
    \begin{itemize}
        \item \colorbox{purple!30}{\textbf{Definicija.}} Gladka podmnogoterost. Lokalne definicijske funkcije.
        \item \colorbox{yellow!30}{\emph{Opomba.}} Kaj je podmnogoterost, če je njena kodimenzija enaka $0$?
        \item \colorbox{yellow!30}{\emph{Zgled.}} Gledamo v $\R^3$. Naj bo $\mc{F}_1, \mc{F}_2, \mc{F}_3: \R^3 \to \R$ linearni. Kaj dobimo, če vzamemo za definiciske funkcije eno, dve ali tri funkcije izmed $\mc{F}_1, \mc{F}_2, \mc{F}_3$? Kadar govorimo o krivuljah in kadar o ploskvah?
        \item \colorbox{yellow!30}{\emph{Zgled.}} Ugotovi, ali je podmnogoterost:
        \begin{itemize}
            \item $M = \setb{(x,y,z) \in \R^3}{x^2+y^2+z^2 = 1}$.
            \item $M = \setb{(x,y,z) \in \R^3}{x^2+y^2+z^2 = 1, \ x+y+z = 0}$.
            \item $M = (\set{0} \times \R) \cup (\R \times \set{0})$.
        \end{itemize}
        \item \colorbox{yellow!30}{\emph{Opomba.}} Ali je rob kvadrata z stranico $2$ in središčem v $(0,0)$ gladka podmnogoterost v $\R^2$?
        \item \colorbox{yellow!30}{\emph{Zgled.}} Ugotovi, ali je podmnogoterost:
        \begin{itemize}
            \item $\GL_n(\R) = \setb{A \in \R^{n \times n}}{\det A \neq 0}$.
            \item $\SL_n(\R) = \setb{A \in \R^{n \times n}}{\det A = 1}$.
        \end{itemize}
        \item \colorbox{yellow!30}{\emph{Opomba.}} Kadar rečemo, da je podmnogoterost podana implicitno?
        \item \colorbox{blue!30}{\textbf{Trditev.}} Karakterizacija podmnogoterosti (ali je lokalno graf?)
        \item \colorbox{yellow!30}{\emph{Opomba.}} Kadar rečemo, da je podmnogoterost podana eksplicitno?
        \item \colorbox{yellow!30}{\emph{Zgled.}} Ali je $M = \setb{(x, x^2)}{x \in \R}$ podmnogoterost?
    \end{itemize}

    \item Parametrično padajanje mnogoterosti
    \begin{itemize}
        \item \colorbox{yellow!30}{\emph{Zgled.}} Ali je parametrizacija $\varphi \mapsto (a \cos \varphi, a \sin \varphi), \ a > 0, \ \varphi \in [0, 2 \pi)$ določa podmnogoterost?
        \item \colorbox{blue!30}{\textbf{Trditev.}}
    \end{itemize}
\end{enumerate}

\newpage
\subsection{Ekstremi funkcij več spremenljivk}
\begin{enumerate}
    \item Ekstremi funkcij več spremenljivk
    
    Naj bo $D \subseteq \R^n, \ a \in D$, $f: D \to \R$ funkcija.
    
    \begin{itemize}
        \item \colorbox{purple!30}{\textbf{Definicija.}} Lokalni maksimum/minimum. Strogi lokalni maksimum/minimum. Maksimum/minimum (globalni). Lokalni ekstrem. Globalni ekstrem.
        \item \colorbox{yellow!30}{\emph{Opomba.}} Kaj ima zvezna funkcija na kompaktu?
        \item \colorbox{purple!30}{\textbf{Definicija.}} Stacionarna (oz. kritična) točka $a \in D^\text{odp}$ diferenciabline funkcije $f$.
        \item \colorbox{blue!30}{\textbf{Trditev.}} Kaj če ima diferenciablina funkcija $f$ v točki $a \in D^\text{odp}$ lokalni ekstrem?
        \item \colorbox{yellow!30}{\emph{Zgled.}} Poišči minimum in maksimum $f(x,y) = x^2 - xy + y^2 -3x +4$ na $K = \setb{(x, y) \in \R^2}{x^2+y^2 \leq 3}$.
    \end{itemize}    

    \item Potrebni in zadostni pogoji na 2. odvodi, da je kritična točka lokalni ekstrem
    
    Naj bo $D \subseteq \R^n$ odprta, $f: D \to R$ razreda $C^2$ na $D$.
    \begin{itemize}
        \item \colorbox{purple!30}{\textbf{Definicija.}} Hessejeva matrika $Hf$ 2.\ odvodov. Hessejeva forma.
        \item \colorbox{yellow!30}{\emph{Opomba.}} Kaj lahko povemo o Hessejeve matrike?
        \item \colorbox{purple!30}{\textbf{Definicija.}} Pozitivno (semi)definitna $Hf$. Negativno (semi)definitna $Hf$.
        \item \colorbox{yellow!30}{\emph{Opomba.}} Karakterizacija pozivne/negativne (semi)definitnosti s lastnimi vrednosti $Hf$.
        \item \colorbox{blue!30}{\textbf{Trditev.}} (Potrebni pogoji). Kaj velja, če ima $f$ v točki $a \in D$ lokalni maksimum/minimum?
        \item \colorbox{blue!30}{\textbf{Trditev.}} (Zadostni pogoji.) Kadar je stacionarna točka $a \in D$ funkcije $f$ lokalni minimum/maksimum? Kadar nič od tega?
        \item \colorbox{yellow!30}{\emph{Zgled.}} Določi $(Hf_i)(0,0)$ za $f_1(x,y) = \frac{1}{2}(x^2+y^2)$, $f_2(x,y) = \frac{1}{2}(-x^2-y^2)$, $f_3(x,y) = \frac{1}{2}(x^2-y^2)$.
        \item \colorbox{orange!30}{\textbf{Posledica.}} Kako zgledajo zadostni pogoji za primer $n = 2$?
        \item \colorbox{yellow!30}{\emph{Zgled.}} Naj bo $f(x,y,z) = x^2+y^2+z^2 + 2xyz$. Klasificiraj vse stacionarne točke funkcije $f$.
    \end{itemize}

    \item Vezani ekstremi
    
    \begin{itemize}
        \item \colorbox{blue!30}{\textbf{Izrek.}} Obstoj Lagrangeevih multiplikatorjev.
        \item \colorbox{yellow!30}{\emph{Opomba.}} Lagrangeeva metoda za iskanja vezanih ekstremov.
        \item \colorbox{yellow!30}{\emph{Zgled.}} Določi stacionarne točke $f(x,y,z)=z$ na $M = \setb{(x,y,z) \in \R^3}{x^2+y^2+z^2=1; \ x+ y + z = 0}$.
        \item \colorbox{yellow!30}{\emph{Zgled.}} Določi stacionarne točke  $f(x,y,z)= x^2 - xy +y^2 - 3x +4$ na robu $x^2 + y^2 = 9$. 
    \end{itemize}
\end{enumerate}