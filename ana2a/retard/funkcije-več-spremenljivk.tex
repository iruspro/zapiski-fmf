\section{Funkcije več spremenljivk}
\subsection{Prostor $\RR^n$}
\subsubsection{Uvodni pojmi}
$\RR^n$ je vektorski prostor nad $\RR$. 
Če je $x, y \in \RR^n$, potem $x = (x_1, \ldots, x_n)$, $y = (y_1, \ldots, y_n)$. 

Naj bo $\alpha \in \RR$. Operaciji $x+y$ in $\alpha x$ sta definirani po komponentah.

\begin{definicija}
    \emph{Standardna baza} prostora $\RR^n$ je množica $\set{e_j; \ j = 1, \ldots, n}$, kjer $\displaystyle e_j = (0, \ldots, 0, 1_j , 0, \ldots, 0)$.
\end{definicija}

\begin{opomba}
    V prostorah $\RR, \ \RR^2, \ \RR^3$ ponavadi koordinate točk označimo z $x, y, z$.
\end{opomba}

\begin{definicija}
    Standardna baza prostora $\RR^3$ je množica $\set{\vec{i}, \vec{j}, \vec{k}}$.
\end{definicija}

\begin{definicija}
    \emph{Standardni skalarni produkt vektrojev $x, y \in \RR^n$} je $x \cdot y = \sum_{j=1}^{n} x_jy_j$.

    \emph{Norma vektorja $x \in \RR^n$} je  $||x||_2= \sqrt{x \cdot x} = \sqrt{x_1^2+\ldots+x_n^2}$.

    \emph{Razdalja med vektorjama $x, y \in \RR^n$} je $d_2(x,y) = ||x-y||_2 = \sqrt{(x_1-y_1)^2 + \ldots + (x_n - y_n)^2}$.
\end{definicija}

\begin{opomba}
    Standardno normo $|| \cdot ||$ bomo pisali kot $| \cdot |$.
\end{opomba}

\begin{definicija}
    \emph{Odprta krogla s središčem v $a \in \RR^n$ in polmerom $r>0$} je množica $K(a, r) = \set{x \in \RR^n; \ ||x-a||<r}.$

    \emph{Zaprta krogla s središčem v $a \in \RR^n$ in polmerom $r>0$} je množica $K(a, r) = \set{x \in \RR^n; \ ||x-a|| \leq r}$.

    \emph{Sfera} je množica $S(a,r) = \set{x \in \RR^n; \ ||x-a|| = r}$
\end{definicija}

Metrični prostor $(\RR^n, d_2)$ porodi topologijo na $\RR^n$. Oznaki: $D^\text{odp}$ je odprta množica, $Z$ je zaprta množica.

\begin{opomba}
    Metrična prostora $(\RR^n, d_2)$ in $(\RR^n, d_\infty)$ imata isto topologijo.
\end{opomba}

\begin{izrek}
    Naj bo $K \subset \RR^n$, potem

    $$K \text{ je kompaktna} \liff K \text{ je zaprta in omejena}.$$
\end{izrek}

\begin{definicija}
    Naj bo $a, b \in \RR^n, \ a = (a_1, \ldots, a_n), \ b= (b_1, \ldots, b_n)$. Definiramo:
    \begin{align*}
        a \leq b \liff \all{j = 1, \ldots, n} a_j \leq b_j. \\
        a < b \liff \all{j = 1, \ldots, n} a_j < b_j.
    \end{align*}
\end{definicija}

\begin{definicija}
    Naj bo $a, b \in \RR^n, \ a< b$. \emph{Odprti kvader $(a,b)$} je množica $(a,b) = \set{x \in \RR^n; \ a<x<b}$.

    Naj bo $a, b \in \RR^n, \ a \leq b$. \emph{Zaprti kvader $[a,b]$} je množica $[a,b] = \set{x \in \RR^n; \ a \leq x \leq b}$.
\end{definicija}

\begin{opomba}
    Dolžine stranic kvadra $[a,b]$ je $b_j-a_j$. Volumen kvadra $[a,b]$ je $\Pi_{j=1}^n(b_j-a_j)$. 
    
    Če so vse strani kvadra enaki, potem kvader je \emph{kocka}.
\end{opomba}

\subsubsection{Zaporedja v $\RR^n$}
\begin{definicija}
    \emph{Zaporedje v $\RR^n$} je preslikava $a: \NN \to \RR^n$. Namesto $a(m)$ pišimo $a_m$, $a_m = (a_1^m, \ldots, a_n^m)$.
\end{definicija}

\begin{opomba}
    Zaporedje v $\RR^n$ porodi $n$ zaporedij v $\RR$.
\end{opomba}

\begin{trditev}
    Naj bo $(a_m)_m$ zaporedje v $\RR^n$, $a_m = (a_1^m, \ldots, a_n^m)$. Velja:
    $$\text{Zaporedje } (a_m)_m \text{ konvergia } \liff \text{ konvergira zaporedja } (a_1^m)_m, \ldots, (a_n^m)_m.$$
    V primeru konvergence velja:
    $$\lim_{m \to \infty} a_m = (\lim_{m \to \infty} a_1^m, \ldots, \lim_{m \to \infty} a_n^m).$$
\end{trditev}

\begin{proof}
    Definicija limite.
\end{proof}

\subsection{Zveznost preslikav iz $\RR^n$ v $\RR^m$}
\subsubsection{Zveznost preslikav iz $\RR^n$ v $\RR$}
\begin{opomba}
    Če je $m=1$, potem preslikave rečemo \emph{funkcija}.
\end{opomba}

\begin{definicija}
    Naj bo $f: D \subseteq \RR^n \to \RR^m$ preslikava. Naj bo $a \in D$. \emph{Preslikava $f$ je zvezna v $a$}, če 
    $$\all{\epsilon > 0} \some{\delta >0} \all{x \in D} d(x, a) \lthen d(f(x), f(a)).$$
\end{definicija}

\begin{definicija}
    Naj bo $f: D \subseteq \RR^n \to \RR^m$ preslikava. Preslikava $f$ je \emph{zvezna na $D$}, če je zvezna v vsaki točki $a \in D$.
\end{definicija}

\begin{definicija}
    Naj bo $f: D \subseteq \RR^n \to \RR^m$ preslikava. Preslikava $f$ je \emph{enakomerno zvezna na $D$}, če
    $$\all{\epsilon > 0} \some{\delta > 0} \all{x, x' \in D} d(x, x') < \delta \lthen d(f(x), f(x')) < \epsilon.$$
\end{definicija}

\begin{opomba}
    Velja karakterizacija zveznosti v točki z zaporedji.
\end{opomba}

\begin{opomba}
    Zvezna preslikava na kompaktne množice je enakomerno zvezna.
\end{opomba}

\begin{definicija}
    Preslikava $f: D \to X'$ je \emph{$C$-lipschitzova}, če
    $$\all{x, x' \in D} d'(f(x), f(x')) \leq Cd(x, x').$$
\end{definicija}

\begin{trditev}
    Za preslikavo $f: D \to X'$ velja:

    $$f \text{ je $C$-lipschitzova } \lthen f \text{ je enakomerno zvezna } \lthen f \text{ je zvezna}.$$
\end{trditev}

\begin{trditev}
    Naj bosta $f, g: D \subset \RR^n \to \RR$ zvezni funkciji v $a \in D$. Naj bo $\lambda \in \RR$. Tedaj so v $a$ zvezni tudi funkcije:
    $$f+g, \ f-g, \ \lambda f, \ fg.$$
\end{trditev}

\begin{proof}
    Z zaporedji kot pri analizi 1.
\end{proof}

\begin{zgled}
    Nekaj primerov zveznih preslikav.
    \begin{itemize}
        \item Preslikava $\Pi_j(x_1, \ldots, x_n) = x_j$ je zvezna na $\RR^n$ za vsak $j = 1, \ldots, n$.
        \item Vsi polinomi v $n$-spremenljivkah so zvezne funkcije na $\RR^n$.
        \item Vse racionalne funkcije so zvezne povsod, razen tam, kjer je imenovalec enak $0$.
    \end{itemize}
\end{zgled}

\begin{definicija}
    Preslikava $f: D \subset \RR^n \to \RR$ je \emph{funkcija $n$-spremenljivk.}
\end{definicija}

\begin{opomba}
    Naj bo $(M, d)$ metrični prostor in $N \subset M$. Naj bo $f: M \to \RR$ zvezna funkcija na $M$. Potem $f|_N$ je tudi zvezna funkcija na $N$.
\end{opomba}

\begin{trditev}
    Naj bosta $D \subseteq \RR^n$ in $D_j = \Pi_j(D)$. Naj bo $a \in D, \ a =(a_1, \ldots, a_n)$ in $f: D \to \RR$ zvezna v $a$. Tedaj za vsak $j = 1, \ldots, n$ funkcija $\varphi_j: D_j \to \RR, \ \varphi_j(t) = f(a_1, \ldots, a_{j-1}, t, a_{j+1}, \ldots, a_n)$ zvezna v $a_j$.
\end{trditev}

\begin{proof}
    Definicija zveznosti v točki.
\end{proof}

\begin{opomba}
    Če je funkcija več spremenljivk zvezna v neki točki $a \in \RR^n$, je zvezna tudi kot funkcija posameznih spremenljivk.
\end{opomba}

\begin{zgled}
    Naj bo $f(x,y) = \begin{cases}
        \frac{2xy}{x^2+y^2}; &(x,y) \neq (0,0) \\
        0; &(x,y) = (0,0)
    \end{cases}$. Ali je $f$ zvezna kot funkcija vsake spremenljivke posebej? Ali je $f$ zvezna na $\RR^2$?
\end{zgled}

\begin{zgled}
    Naj bo $f(x,y) = \begin{cases}
        \frac{2x^2y}{x^4+y^2}; &(x,y) \neq (0,0) \\
        0; &(x,y) = (0,0)
    \end{cases}$. Ali je $f$ zvezna kot funkcija vsake spremenljivke posebej? Ali je zvezna na vsaki premici? Ali je $f$ zvezna na $\RR^2$?
\end{zgled}

\begin{opomba}
    Zgleda pokažeta, da obrat v prejšnji trditvi ne velja.
\end{opomba}

\newpage
\subsubsection{Zveznost preslikav iz $\RR^n$ v $\RR^m$}
Naj bo $D \subseteq \RR^n$ in $F: D \to \RR^m$ preslikava. Naj bo $x \in D$, potem $F(x) \in \RR^m, \ F(x) = y = (y_1, \ldots, y_m)$.

Lahko pišemo $F(x) = (f_1(x), \ldots, f_m(x))$. Torej $F$ določa $m$ funkcij $n$-spremenljivk.

\begin{trditev}
    Naj bo $a \in D \subseteq \RR^n$. Naj bo $F=(f_1, \ldots, f_m): D \to \RR^m$ preslikava. Velja:
    $$\text{Preslikava } F \text{ je zvezna v } a \liff f_1,\ldots, f_m \text{ so zvezne v } a.$$
\end{trditev}

\begin{proof}
    Definicija zveznosti v točki.
\end{proof}

\begin{opomba}
    Linearne preslikave so zvezne, saj so vse koordinatne funkcije linearne (polinomi 1. stopnje).
\end{opomba}

\begin{zgled}[Omejenost linearnih preslikav]
    Naj bo $\lin{A}: \RR^n \to \RR^m$ linearna preslikava, potem $$\some{ M \in \RR \, . \, M \geq 0} \all{x \in \RR^n \, . \, x \neq 0 }  \frac{|\lin{A}x|}{|x|} \leq M.$$
    Lahko zapišemo $\sup \frac{|\lin{A}x|}{|x|} = \sup_{|x|=1}|\lin{A}x| = ||A||$. Dobimo eno izmed norm na matrikah.

    Trdimo: Naj bo $\lin{A}: \RR^n \to \RR^m$ linearna preslikava. Tedaj je $\lin{A}$ zvezna na $\RR^n$. Zveznost linearnih preslikav je ekvivalentna zveznosti v točki $0$. Vse skupaj je ekvivalentno omejenosti linearnih preslikav.
\end{zgled}

\begin{proof}
    Definicija zveznosti in omejenosti.
\end{proof}

\begin{definicija}
    Naj bo $\lin{A}: \RR^n \to \RR^m$ linearna preslikava. Preslikavo $\lin{A}: \RR^n \to \RR^m, \ x \mapsto \lin{A}x + b, \ b \in \RR^m$ imenujemo \emph{afina preslikava}.
\end{definicija}

\subsection{Parcialni odvodi in diferenciabilnost}
\subsubsection{Parcialni odvod}
\begin{definicija}
    Naj bo $f: D \subset \RR^n \to \RR$ funkcija. Naj bo $a = (a_1, \ldots, a_n) \in D$ notranja točka. Funkcija $f$ je \emph{parcialno odvedljiva} po spremenljivki $x_j$ v točki $a$, če obstaja limita 
    $$\lim_{h \to 0} \frac{f(a_1, \ldots, a_{j-1}, a_j+h, a_{j+1}, \ldots, a_n) - f(a_1, \ldots, a_n)}{h},$$
    oz. če je funkcija 
    $$x_j \mapsto f(a_1, \ldots, a_{j-1}, x_j, a_{j+1}, \ldots, a_n)$$
    odvedliva v točki $a_j$.

    Če je ta limita obstaja, je to \emph{parcialni odvod} funkcije $f$ po spremenljivki $x_j$ v točki $a$.
    Oznaki: $\frac{\partial f}{\partial x_j}(a), \ f_{x_j}(a), \ (D_jf)(a)$.
\end{definicija}

\begin{opomba}
    Vse elementarne funkcije so parcialno odvedljive po vseh spremenljivkah tam, kjer so definirane.
\end{opomba}

\begin{zgled}
    Naj bo $f(x, y, z) = e^{x+2y} + \cos(xz^2)$. Potem $f_x(x, y, z) = \podv{f}{x}(x, y, z) = e^{x+2y} - z^2 \sin(xz^2)$.
\end{zgled}

\subsubsection{Diferenciabilnost}
\begin{definicija}
    Naj bo $f: D \subset \RR^n \to \RR$ funkcija. Naj bo $a = (a_1, \ldots, a_n) \in D$ notranja točka. Funkcija $f$ je \emph{diferenciabilna} v točki $a$, če obstaja tak linearen funkcional $\lin{L}: \RR^n \to \RR$, da velja:
    $$f(a+h) = f(a) + \lin{L}(h) + o(h),$$
    kjer $\lim_{h \to 0} \frac{|o(h)|}{|h|} = 0$.
\end{definicija}

\begin{opomba}
    Če je tak $\lin{L}$ obstaja, je enolično določen.
\end{opomba}

\begin{proof}
    Pokažemo, da iz $\lin{L}(h) = (\lin{L}_1 - \lin{L}_2)(h) = (o_2 - o_1)(h) = o(h)$ sledi, da je $L = 0$. Zato uporabimo vektor $v = tv_0$, kjer $|v_0| = 1, \ |v| = t$ na linearnem funkcionalu $\lin{L}$.
\end{proof}

\begin{definicija}
    Če je $f$ diferenciabilna v $a$ je $\lin{L}$ natanko določen in ga imenujemo \emph{diferencial} funkcije $f$ v točki $a$. Oznaka: $\lin{L} = df_a$. Linearen funkcional $\lin{L}$ imenujemo tudi \emph{odvod} funkcije $f$ v točki $a$. Oznaka: $(Df)(a)$.
\end{definicija}

\begin{opomba}
    Recimo, da je funkcija $f$ diferenciabilna v točki $a$. Preslikava $h \mapsto f(a) + (df_a)(h)$ je najboljša afina aproksimacija funkcije $h \to f(a+h)$.
\end{opomba}

\begin{trditev}
    Naj bo $f: D \subset \RR^n \to \RR$ diferenciabilna v notranji točki $a \in D$. Tedaj je $f$ v točki $a$ parcialno odvedljiva po vseh spremenljivkah. Poleg tega je zvezna v točki $a$. Pri tem za $h = (h_1, \ldots, h_n)$ velja:
    $$(df_a)(h) = \podv{f}{x_1}(a) \cdot h_1 + \ldots + \podv{f}{x_n}(a) \cdot h_n = f_{x_1}(a) \cdot h_1 + \ldots + f_{x_n}(a) \cdot h_n$$
\end{trditev}

\begin{opomba}
    Naj bo $\lin{L}: \RR^n \to \RR$ linearen funkcional, $x \in \RR^n$, potem $\lin{L}(x) = l_1x_1+\ldots+l_nx_n = \begin{bmatrix}
        l_1 & \ldots & l_n
    \end{bmatrix} \cdot \begin{bmatrix}
        x_1 \\ \vdots \\ x_n
    \end{bmatrix}$, kjer~$\begin{bmatrix}
        l_1 & \ldots & l_n
    \end{bmatrix}$ matrika linearnega funkcionala glede na standardne baze.    
\end{opomba}

\begin{proof}
    Zveznost pokažemo z limito. Za parcialno odvedljivost poglejmo kaj se dogaja za $h = (h_1, 0, \ldots, 0)$.
\end{proof}

\begin{opomba}
    Trditev pove, da je $(df_a)(h) = \begin{bmatrix}
        \podv{f}{x_1}(a) & \ldots & \podv{f}{x_n}(a) 
    \end{bmatrix} \cdot \begin{bmatrix}
        h_1 \\ \vdots \\ h_n
    \end{bmatrix} = (\podv{f}{x_1}(a), \ldots, \podv{f}{x_n}(a)) \cdot (h_1, \ldots, h_n)$.

    Zapis: $(\vec{\nabla} f) (a) = (\grad f)(a) = (\podv{f}{x_1}(a), \ldots, \podv{f}{x_n}(a))$.

    Vektor $(\grad f)(a)$ imenujemo \emph{gradient funkcije} $f$ v točki $a$. Operator $\vec{\nabla} = (\frac{\partial}{\partial x_1}, \ldots, \frac{\partial}{\partial x_n})$ je \emph{operator Nabla}.
\end{opomba}

\begin{zgled}
    Naj bo $f(x,y) = \begin{cases}
        \frac{2xy}{x^2+y^2}; &(x,y) \neq (0,0) \\
        0; &(x,y) = (0,0)
    \end{cases}$. Ali je $f$ diferenciabilna?
\end{zgled}

\begin{zgled}
    Naj bo $f(x,y) = \begin{cases}
        \frac{2x^2y}{x^2+y^2}; &(x,y) \neq (0,0) \\
        0; &(x,y) = (0,0)
    \end{cases}$. Ali je $f$ zvezna? Ali je $f$ parcialno odvedljiva? Ali je $f$ diferenciabilna?
\end{zgled}

\begin{opomba}
    Zgleda pokažeta, da obrat v prejšnji trditvi ne velja
\end{opomba}

\begin{izrek}
    Naj bo $f: D \subseteq \RR^n \to \RR$ funkcija in naj bo $a \in D$ notranja točka. Denimo, da je $f$ parcialno odvedljiva po vseh spremenljivkah v točki $a$ in so parcialni odvodi zvezni v točki $a$. Tedaj je $f$ diferenciabilna v točki $a$.
\end{izrek}

\begin{proof}
    Za $n=2$. Definicija diferenciabilnosti + 2-krat Lagrangeev izrek.
\end{proof}

\subsubsection{Višji parcialni odvodi}
Naj bo $f: D^{\text{odp}} \subseteq \RR^n \to \RR$ funkcija. Denimo, da je $f$ parcialno odvedljiva po vseh spremenljivkah na $D$: $f_{x_1}, \ldots, f_{x_n}$. To so tudi funkcije $n$-spremenljivk in morda so tudi te parcialno odvedljive po vseh oz. nekatareih spremenljivkah.

\begin{trditev}
    Naj bo funkcija $f$ definirana v okolici $a \in \RR^n$. Naj bosta $i, j \in \set{1, 2, \ldots, n}$. Denimo, da na tej okolici obstajata $\podv{f}{x_i}, \ \podv{f}{x_j}$ in tudi druga odvoda $\frac{\partial}{\partial x_j} (\podv{f}{x_i}), \ \frac{\partial}{\partial x_i} (\podv{f}{x_j})$. Če sta $\frac{\partial}{\partial x_j} (\podv{f}{x_i}), \ \frac{\partial}{\partial x_i} (\podv{f}{x_j})$ zvezna v $a$, potem sta enaka v točki $a$: 
    $$\frac{\partial}{\partial x_j} \left(\podv{f}{x_i}\right)(a) = \frac{\partial}{\partial x_i} \left(\podv{f}{x_j}\right)(a).$$
\end{trditev}

\begin{proof}
    Dovolj za $n=2$. 
    
    Definiramo $J = f(a+h, b+k) - f(a+h, b) - f(a, b+k) + f(a,b)$ in $\varphi(x) = f(x, b+k) - f(x, b), \ \psi(y) = f(a+h, y) - f(a, y)$. Zapišemo $J$ s pomočjo funkcij $\varphi, \ \psi$ ter uporabimo 2-krat Lagrangeev izrek in upoštevamo zveznost.
\end{proof}

\subsubsection{Diferenciabilnost preslikav}
\begin{definicija}
    Naj bo $F: D \subseteq \RR^n \to \RR^m$ preslikava, $a \in D$ notranja točka. Preslikava $F$ je \emph{diferenciabilna} v točki~$a$, če obstaja taka linearna preslikava $\lin{L}: \RR^n \to \RR^m$, da velja:
    $$F(a+h) = F(a) + \lin{L}(h) + o(h),$$
    kjer je $\lim_{h \to 0} \frac{|o(h)|_m}{|h|_n}$. 

    Preslikavo $\lin{L}$ imenujemo \emph{diferencial} $F$ v točki $a$. Oznaka: $dF_a$. Imenujemo ga tudi \emph{odvod} $F$ v točki $a$. Oznaka: $(DF)(d)$.
\end{definicija}

\begin{opomba}
    Kot pri funkcijah, če je tak $\lin{L}$ obstaja, je enolično določen.
\end{opomba}

\begin{zgled}
    Obravnavaj diferenciabilnost preslikav:
    \begin{itemize}
        \item $\lin{A}: \RR^n \to \RR^m$ linearna, $F(x) = \lin{A}x$.
        \item $F: \RR^{n \times n} \to \RR^{n \times n}, \ F(X) = X^2$. 
        
        Namig: $|A| = \sqrt{a_{11}^2 + a_{12}^2 + \ldots + a_{nn}^2}$. S pomočjo neenakosti CSB pokažimo, da $|H^2| \leq |H|^2$.
    \end{itemize}
\end{zgled}

\newpage
\begin{izrek}
    Naj bo $a \in D$ notranja točka. Naj bo $F = (f_1, \ldots, f_m): D \to \RR^m$ preslikava. Velja: 
    $$\text{Preslikava } F \text{ je diferenciabilna v } a \in D \liff \text{so } f_1, \ldots, f_m \text{ diferenciabilne v } a.$$
    Tedaj
    $$(DF)(a) = \begin{bmatrix}
        \podv{f_1}{x_1}(a) & \ldots & \podv{f_1}{x_n}(a) \\
        \vdots & & \vdots \\
        \podv{f_m}{x_1}(a) & \ldots & \podv{f_m}{x_n}(a) 
    \end{bmatrix}$$
\end{izrek}

\begin{proof}
    $(\lthen)$ Zapišemo enakost $F(a+h) = F(a) + dF_a(h) + o(h)$ po komponentah.

    $(\Leftarrow)$ Definicija diferenciabilnosti.
\end{proof}

\begin{posledica}
    Naj bo $a \in D$ notranja točka. Naj bo $F = (f_1, \ldots, f_m): D \to \RR^m$ preslikava. Velja:
    
    Če so vse funkcije $f_1, \ldots, f_m$ v točki $a$ parcialno odvedlivi po vseh spremenljivkah in so ti vsi odvodi zvezni v točki $a$, potem je $F$ diferenciabilna v točki $a$.
\end{posledica}

\begin{zgled}
    Naj bo $F(x,y,z) = (x^2+2y+e^z, xy+z^2), \ f: \RR^3 \to \RR^2$. Določi $(DF)(1,0,1)$.
\end{zgled}

\begin{opomba}
    Preslikava $F: D^{\text{odp}} \subseteq \RR^n \to \RR^m$ je razreda $C^k(D)$ oz. je $k$-krat zvezno odvedljiva, če so $f_1, \ldots, f_m \in C^k(D)$.
\end{opomba}

\begin{izrek}[Verižno pravilo]
    Naj bo $a \in D \subseteq \RR^n$ notranja točka. Naj bo $b \in \Omega \subseteq \RR^m$ notranja točka. Naj bo $F: D \to \Omega$ diferenciabilna v točki $a$ in velja $F(a) = b$. Naj bo $G: \Omega \to \RR^k$ diferenciabilna v točki $b$. Tedaj $G \circ F$ diferenciabilna v točki $a$ in velja:
    $$D(G \circ F)(a) = (DG)(b) \cdot (DF)(a) = (DG)(F(a)) \cdot (DF)(a).$$
    Označimo $F(x_1, \ldots, x_n) = (f_1(x_1, \ldots, x_n), \ldots, f_m(x_1, \ldots, x_n))$ in $G(y_1, \ldots, y_m) = (g_1(y_1, \ldots, y_m), \ldots, g_k(y_1, \ldots, y_m))$. Potem 
    $$D(G \circ F)(a) = \begin{bmatrix}
        \podv{g_1}{y_1} & \ldots & \podv{g_1}{y_m} \\
        \vdots & & \vdots \\
        \podv{g_k}{y_1} & \ldots & \podv{g_k}{y_m}
    \end{bmatrix}(b) \cdot \begin{bmatrix}
        \podv{f_1}{x_1} & \ldots & \podv{f_1}{x_n} \\
        \vdots & & \vdots \\
        \podv{f_m}{x_1} & \ldots & \podv{f_m}{x_n}
    \end{bmatrix}(a)$$
\end{izrek}

\begin{proof}
    Definicija diferenciabilnosti.
\end{proof}

\begin{posledica}[$k=1$, $G = g$ funkcija]
    Naj bo $\Phi(x_1, \ldots, x_n) = g(f_1(x_1, \ldots, x_n), \ldots, f_m(x_1, \ldots, x_n))$. Potem 
    $$\podv{\Phi}{x_j}(a) = \podv{g}{y_1}(b) \cdot \podv{f_1}{x_j}(a) + \podv{g}{y_2}(b) \cdot \podv{f_2}{x_j}(a) + \ldots + \podv{g}{y_m}(b) \cdot \podv{f_m}{x_j}(a)$$
\end{posledica}

\begin{zgled}
    Naj bo $F(x,y) = (x^2 + y, xy), \ g(u, v) = uv + v^2$. Naj bo $\Phi = g \circ F$. Izračunaj $(D\Phi)(x,y)$ na dva načina.
\end{zgled}

\subsection{Izrek o implicitni funkciji}
Radi bi poiskali zadostni pogoji na funkcijo $f(x, y)$, da bi enačba $f(x, y) = 0$ lokalno v okolici točki $(a,b)$, za katero velja $f(a,b) = 0$, predstavljala graf funkcije $y = \varphi(x)$.

\begin{izrek}[Osnovna verzija izreka o implicitni funkciji]
    Naj bo $D^{\text{odp}} \subseteq \RR^2$. Naj bo $(a, b) \in D$. Naj bo $f \in C^1(D)$ in naj velja:
    \begin{enumerate}
        \item $f(a,b) = 0$.
        \item $f_y(a,b) \neq 0$.
    \end{enumerate}
    Potem obstajata $\delta > 0$ in $\epsilon > 0$, da velja: $I \times J \subseteq D$, kjer je $I = (a - \delta, a + \delta), \ J = (b-\epsilon, b+\epsilon)$ in enolično določena~$C^1$ funkcija $\varphi: I \to J$, za katero velja:
    \begin{enumerate}
        \item $\varphi(a) = b$.
        \item $\all{(x,y) \in I \times J} f(x, y) = 0 \liff y = \varphi (x)$ (rešitve enačbe $f(x,y) = 0$ so natanko graf funkcije $\varphi$).
        \item $\varphi'(x) = -\frac{f_x(x, \varphi(x))}{f_y(x, \varphi(x))}$ za vsak $x \in I$.
    \end{enumerate}
\end{izrek}