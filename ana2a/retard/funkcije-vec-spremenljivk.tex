\section{Funkcije več spremenljivk}
\subsection{Prostor $\R^n$}
\subsubsection{Prostor $\R^n$}
\begin{definicija}
    \emph{Prostor $\R^n$} je kartezični produkt
    $\underbrace{\R \times \dots \times \R}_n$. Na njem definiramo
    seštevanje in množenje s skalarjem po komponentah. S tema
    operacijama je $(\R,+,\cdot)$ vektorski prostor nad $\R$. Posebej
    definiramo še skalarni produkt
    \[
    x \cdot y = \sum_{i=1}^n x_iy_i,
    \]
    ki nam da normo $\norm{x}=\sqrt{x \cdot x}$ in metriko
    $d(x,y) = \norm{x-y}$. $(\R^n,d)$ je tako metrični prostor.
    \end{definicija}

\begin{definicija}
    Naj bosta $a,b \in \R^n$ vektorja, za katera je $a_i \leq b_i$ za
    vse $i \in \set{1,\dots,n}$. \emph{Zaprt kvader}\index{Kvader}, ki
    ga določata $a$ in $b$, je množica
    \[
    [a,b] = \setb{x\in\R^n}{\forall i \in \set{1,\dots,n} \colon
    a_i \leq x_i \leq b_i}.
    \]
    Podobno definiramo \emph{odprt kvader} kot
    \[
    (a,b) = \setb{x\in\R^n}{\forall i \in \set{1,\dots,n} \colon
    a_i < x_i < b_i}.
    \]
    \end{definicija}
    
    \begin{opomba}
    Odprte množice v normah $\norm{x}_\infty$ in $\norm{x}_2$ so iste.
    \end{opomba}
    
    \begin{izrek}
    Množica $K\subseteq\R^n$ je kompaktna natanko tedaj, ko je zaprta
    in omejena.
    \end{izrek}

\subsubsection{Zaporedja v $\R^n$}
\begin{definicija}
    \emph{Zaporedje v $\R^n$} je preslikava $a: \N \to \R^n$. Namesto $a(m)$ pišemo $a_m$, kjer $a_m = (a_1^m, \ldots, a_n^m)$.
\end{definicija}

\begin{opomba}
    Zaporedje v $\R^n$ porodi $n$ zaporedij v $\R$.
\end{opomba}

\begin{trditev}
    Naj bo $(a_m)_m$ zaporedje v $\R^n$, $a_m = (a_1^m, \ldots, a_n^m)$. Velja:
    $$\text{Zaporedje } (a_m)_m \text{ konvergia } \liff \text{ konvergira zaporedja } (a_1^m)_m, \ldots, (a_n^m)_m.$$
    V primeru konvergence velja:
    $$\lim_{m \to \infty} a_m = (\lim_{m \to \infty} a_1^m, \ldots, \lim_{m \to \infty} a_n^m).$$
\end{trditev}

\begin{proof}
    Definicija limite.
\end{proof}

\newpage
\subsection{Zveznost preslikav iz $\R^n$ v $\R^m$}
\subsubsection{Zveznost preslikav iz $\R^n$ v $\R$}

\begin{opomba}
    Če je $m=1$, potem preslikave rečemo \emph{funkcija}.
\end{opomba}

\begin{definicija}
    Naj bo $f: D \subseteq \R^n \to \R^m$ preslikava. Naj bo $a \in D$. \emph{Preslikava $f$ je zvezna v točki $a$}, če 
    $$\all{\epsilon > 0} \some{\delta >0} \all{x \in D} \norm{x-a} \lthen \norm{f(x)- f(a)}.$$
    Preslikava $f$ je \emph{zvezna na $D$}, če je zvezna v vsaki točki $a \in D$.
\end{definicija}

\begin{trditev}
    Naj bo $f: D \subseteq \R^n \to \R^m$ preslikava. Naj bo $a \in D$. Preslikava $f$ je zvezna v točki $a$ natanko tedaj, ko za vsako zaporedje $(x_n)_n, \ x_n \in D$, ki konvergira proti $a$, zaporedje $(f(x_n))_n, \ f(x_n) \in \R^m$ konvergira proti $f(a)$.
\end{trditev}

\begin{definicija}
    Naj bo $f: D \subseteq \R^n \to \R^m$ preslikava. Preslikava $f$ je \emph{enakomerno zvezna na $D$}, če
    $$\all{\epsilon > 0} \some{\delta > 0} \all{x, x' \in D} \norm{x - x'} < \delta \lthen \norm{f(x) - f(x')} < \epsilon.$$
\end{definicija}

\begin{trditev}
    Zvezna preslikava na kompaktne množice je enakomerno zvezna.
\end{trditev}

\begin{trditev}
    Naj bo $f: K^\text{komp} \subseteq \R^n \to \R^m$ zvezna preslikava. Potem je $\img{f}(K)$ kompaktna.
\end{trditev}

\begin{definicija}
    Preslikava $f: D \subseteq \R^n \to \R^m$ je \emph{$C$-lipschitzova}, če
    $$\some{C \in \R} \all{x, x' \in D} \norm{f(x) - f(x')} \leq C \norm{x - x'}.$$
\end{definicija}

\begin{trditev}
    Za preslikavo $f: D \to X'$ velja:

    $$f \text{ je $C$-lipschitzova } \lthen f \text{ je enakomerno zvezna } \lthen f \text{ je zvezna}.$$
\end{trditev}

\begin{trditev}
    Naj bosta $f, g: D \subset \R^n \to \R$ zvezni funkciji v $a \in D$. Naj bo $\lambda \in \R$. Tedaj so v $a$ zvezni tudi funkcije:
    $$f+g, \ f-g, \ \lambda f, \ fg.$$
    Če za vsak $x \in D, \ g(x) \neq 0$ , tedaj so v $a$ zvezna tudi funkcija: 
    $$\frac{f}{g}.$$
\end{trditev}

\begin{trditev}
    Kompozitum zveznih preslikav je zvezna preslikava.
\end{trditev}

\begin{proof}
    Z zaporedji kot pri analizi 1.
\end{proof}

\begin{zgled}
    Nekaj primerov zveznih preslikav.
    \begin{itemize}
        \item Preslikava $\Pi_j(x_1, \ldots, x_n) = x_j$ je zvezna na $\R^n$ za vsak $j = 1, \ldots, n$.
        \item Vse polinomi v $n$-spremenljivkah so zvezne funkcije na $\R^n$.
        \item Vse racionalne funkcije so zvezne povsod, razen tam, kjer je imenovalec enak $0$.
    \end{itemize}
\end{zgled}

\begin{definicija}
    Preslikava $f: D \subset \R^n \to \R$ je \emph{funkcija $n$-spremenljivk.}
\end{definicija}

\begin{opomba}
    Naj bo $(M, d)$ metrični prostor in $N \subset M$. Naj bo $f: M \to \R$ zvezna funkcija na $M$. Potem $f|_N$ je tudi zvezna funkcija na $N$.
\end{opomba}

\begin{trditev}
    Naj bosta $D \subseteq \R^n$ in $D_j = \Pi_j(D)$. Naj bo $a \in D, \ a =(a_1, \ldots, a_n)$ in $f: D \to \R$ zvezna v $a$. Tedaj za vsak $j = 1, \ldots, n$ funkcija $\varphi_j: D_j \to \R, \ \varphi_j(t) = f(a_1, \ldots, a_{j-1}, t, a_{j+1}, \ldots, a_n)$ zvezna v $a_j$.
\end{trditev}

\begin{proof}
    Definicija zveznosti v točki.
\end{proof}

\begin{opomba}
    Če je funkcija več spremenljivk zvezna v neki točki $a \in \R^n$, je zvezna tudi kot funkcija posameznih spremenljivk.
\end{opomba}

\begin{zgled}
    Naj bo $f(x,y) = \begin{cases}
        \frac{2xy}{x^2+y^2}; &(x,y) \neq (0,0) \\
        0; &(x,y) = (0,0)
    \end{cases}$. Ali je $f$ zvezna kot funkcija vsake spremenljivke posebej? Ali je $f$ zvezna na $\R^2$?
\end{zgled}

\begin{zgled}
    Naj bo $f(x,y) = \begin{cases}
        \frac{2x^2y}{x^4+y^2}; &(x,y) \neq (0,0) \\
        0; &(x,y) = (0,0)
    \end{cases}$. Ali je $f$ zvezna kot funkcija vsake spremenljivke posebej? Ali je zvezna na vsaki premici? Ali je $f$ zvezna na $\R^2$?
\end{zgled}

\begin{opomba}
    Zgleda pokažeta, da obrat v prejšnji trditvi ne velja.
\end{opomba}

\newpage
\subsubsection{Zveznost preslikav iz $\R^n$ v $\R^m$}
Naj bo $D \subseteq \R^n$ in $F: D \to \R^m$ preslikava. Naj bo $x \in D$, potem $F(x) \in \R^m, \ F(x) = y = (y_1, \ldots, y_m)$.

Lahko pišemo $F(x) = (f_1(x), \ldots, f_m(x))$. Torej $F$ določa $m$ funkcij $n$-spremenljivk.

\begin{trditev}
    Naj bo $a \in D \subseteq \R^n$. Naj bo $F=(f_1, \ldots, f_m): D \to \R^m$ preslikava. Velja:
    $$\text{Preslikava } F \text{ je zvezna v } a \liff f_1,\ldots, f_m \text{ so zvezne v } a.$$
\end{trditev}

\begin{proof}
    Definicija zveznosti v točki.
\end{proof}

\begin{zgled}[Omejenost linearnih preslikav]
    Naj bo $\lin{A}: \R^n \to \R^m$ linearna preslikava, potem $$\some{ M \in \R \, . \, M \geq 0} \all{x \in \R^n \, . \, x \neq 0 }  \frac{\norm{\lin{A}x}}{\norm{x}} \leq M \ (\text{oz. } \norm{\lin{A}x} \leq M \norm{x} ).$$
\end{zgled}

\begin{trditev}
    Linearne preslikave so zvezne
\end{trditev}

\begin{proof}
    Vse koordinatne funkcije linearne (polinomi 1. stopnje).
\end{proof}

\begin{trditev}
    Naj bo $\lin{A}: \R^n \to \R^m$ linearna preslikava. Velja:
    $$\text{Preslikava } \lin{A} \text{ je zvezna} \liff \text{Preslikava } \lin{A} \text{ je zvezna v točki } 0 \liff \text{Preslikava } \lin{A} \text{ je omejena}.$$
\end{trditev}

\begin{proof}
    Definicija zveznosti in omejenosti.
\end{proof}

\begin{definicija}
    Naj bo $\lin{A}: \R^n \to \R^m$ linearna preslikava. Preslikavo $\lin{A}: \R^n \to \R^m, \ x \mapsto \lin{A}x + b, \ b \in \R^m$ imenujemo \emph{afina preslikava}.
\end{definicija}

\newpage
\subsection{Parcialni odvodi in diferenciabilnost}
\subsubsection{Parcialni odvod}
\begin{definicija}
    Naj bo $f: D \subset \R^n \to \R$ funkcija. Naj bo $a = (a_1, \ldots, a_n) \in D$ notranja točka. Funkcija $f$ je \emph{parcialno odvedljiva po spremenljivki $x_j$ v točki $a$}, če obstaja limita 
    $$\lim_{h \to 0} \frac{f(a_1, \ldots, a_{j-1}, a_j+h, a_{j+1}, \ldots, a_n) - f(a_1, \ldots, a_n)}{h},$$
    oz. če je funkcija 
    $$x_j \mapsto f(a_1, \ldots, a_{j-1}, x_j, a_{j+1}, \ldots, a_n)$$
    odvedliva v točki $a_j$.

    Če je ta limita obstaja, je to \emph{parcialni odvod} funkcije $f$ po spremenljivki $x_j$ v točki $a$.
    Oznaki: $\frac{\partial f}{\partial x_j}(a), \ f_{x_j}(a), \ (D_jf)(a)$.
\end{definicija}

\begin{opomba}
    Vse elementarne funkcije so parcialno odvedljive po vseh spremenljivkah tam, kjer so definirane.
\end{opomba}

\begin{zgled}
    Naj bo $f(x, y, z) = e^{x+2y} + \cos(xz^2)$. Izračunaj $f_x(x,y,z), \ f_y(x,y,z), \ f_z(x,y,z)$.
\end{zgled}

\subsubsection{Diferenciabilnost}
\begin{definicija}
    Naj bo $f: D \subset \R^n \to \R$ funkcija. Naj bo $a = (a_1, \ldots, a_n) \in D$ notranja točka. Funkcija $f$ je \emph{diferenciabilna v točki $a$}, če obstaja tak linearen funkcional $\lin{L}: \R^n \to \R$, da velja:
    $$f(a+h) = f(a) + \lin{L}(h) + o(h),$$
    kjer $\lim_{h \to 0} \frac{\norm{o(h)}}{\norm{h}} = 0.$
\end{definicija}

\begin{opomba}
    Če je tak $\lin{L}$ obstaja, je enolično določen.
\end{opomba}

\begin{proof}
    Pokažemo, da iz $\lin{L}(h) = (\lin{L}_1 - \lin{L}_2)(h) = (o_2 - o_1)(h) = o(h)$ sledi, da je $L = 0$.
\end{proof}

\begin{definicija}
    Če je $f$ diferenciabilna v $a$ je $\lin{L}$ natanko določen in ga imenujemo \emph{diferencial} funkcije $f$ v točki $a$. Oznaka: $\lin{L} = df_a$. Linearen funkcional $\lin{L}$ imenujemo tudi \emph{odvod} funkcije $f$ v točki $a$. Oznaka: $(Df)(a)$.
\end{definicija}

\begin{opomba}
    Recimo, da je funkcija $f$ diferenciabilna v točki $a$. Preslikava $h \mapsto f(a) + (df_a)(h)$ je najboljša afina aproksimacija funkcije $h \to f(a+h)$.
\end{opomba}

\begin{trditev}
    Naj bo $f: D \subset \R^n \to \R$ diferenciabilna v notranji točki $a \in D$. Tedaj je $f$ v točki $a$ parcialno odvedljiva po vseh spremenljivkah. Poleg tega je zvezna v točki $a$. Pri tem za $h = (h_1, \ldots, h_n)$ velja:
    $$(df_a)(h) = \podv{f}{x_1}(a) \cdot h_1 + \ldots + \podv{f}{x_n}(a) \cdot h_n = f_{x_1}(a) \cdot h_1 + \ldots + f_{x_n}(a) \cdot h_n$$
\end{trditev}

\begin{opomba}
    Naj bo $\lin{L}: \R^n \to \R$ linearen funkcional, $x \in \R^n$, potem $\lin{L}(x) = l_1x_1+\ldots+l_nx_n = \begin{bmatrix}
        l_1 & \ldots & l_n
    \end{bmatrix} \cdot \begin{bmatrix}
        x_1 \\ \vdots \\ x_n
    \end{bmatrix}$, kjer~$\begin{bmatrix}
        l_1 & \ldots & l_n
    \end{bmatrix}$ matrika linearnega funkcionala glede na standardne baze.    
\end{opomba}

\begin{proof}
    Zveznost pokažemo z limito. Za parcialno odvedljivost poglejmo kaj se dogaja za $h = (h_1, 0, \ldots, 0)$.
\end{proof}

\begin{opomba}
    Trditev pove, da je $df_a = \begin{bmatrix}
        \podv{f}{x_1}(a) & \ldots & \podv{f}{x_n}(a) 
    \end{bmatrix} = (\podv{f}{x_1}(a), \ldots, \podv{f}{x_n}(a))$.

    Zapis: $(\vec{\nabla} f) (a) = (\grad f)(a) = (\podv{f}{x_1}(a), \ldots, \podv{f}{x_n}(a))$.

    Vektor $(\grad f)(a)$ imenujemo \emph{gradient funkcije} $f$ v točki $a$. Operator $\vec{\nabla} = (\frac{\partial}{\partial x_1}, \ldots, \frac{\partial}{\partial x_n})$ je \emph{operator Nabla}.
\end{opomba}

\begin{zgled}
    Naj bo $f(x,y) = \begin{cases}
        \frac{2xy}{x^2+y^2}; &(x,y) \neq (0,0) \\
        0; &(x,y) = (0,0)
    \end{cases}$. Ali je $f$ diferenciabilna?
\end{zgled}

\begin{zgled}
    Naj bo $f(x,y) = \begin{cases}
        \frac{2x^2y}{x^2+y^2}; &(x,y) \neq (0,0) \\
        0; &(x,y) = (0,0)
    \end{cases}$. Ali je $f$ zvezna? Ali je $f$ parcialno odvedljiva? Ali je $f$ diferenciabilna?
\end{zgled}

\begin{opomba}
    Zgleda pokažeta, da obrat v prejšnji trditvi ne velja
\end{opomba}

\begin{izrek}
    Naj bo $f: D \subseteq \R^n \to \R$ funkcija in naj bo $a \in D$ notranja točka. Denimo, da je $f$ parcialno odvedljiva po vseh spremenljivkah v točki $a$ in so parcialni odvodi zvezni v točki $a$. Tedaj je $f$ diferenciabilna v točki $a$.
\end{izrek}

\begin{proof}
    Za $n=2$. Definicija diferenciabilnosti + 2-krat Lagrangeev izrek.
\end{proof}

\subsubsection{Višji parcialni odvodi}
Naj bo $f: D^{\text{odp}} \subseteq \R^n \to \R$ funkcija. Denimo, da je $f$ parcialno odvedljiva po vseh spremenljivkah na $D$: $f_{x_1}, \ldots, f_{x_n}$. To so tudi funkcije $n$-spremenljivk in morda so tudi te parcialno odvedljive po vseh oz. nekatareih spremenljivkah.

\begin{trditev}
    Naj bo funkcija $f$ definirana v okolici $a \in \R^n$. Naj bosta $i, j \in \set{1, 2, \ldots, n}$. Denimo, da na tej okolici obstajata $\podv{f}{x_i}, \ \podv{f}{x_j}$ in tudi druga odvoda $\frac{\partial}{\partial x_j} (\podv{f}{x_i}), \ \frac{\partial}{\partial x_i} (\podv{f}{x_j})$. Če sta $\frac{\partial}{\partial x_j} (\podv{f}{x_i}), \ \frac{\partial}{\partial x_i} (\podv{f}{x_j})$ zvezna v $a$, potem sta enaka v točki $a$: 
    $$\frac{\partial}{\partial x_j} \left(\podv{f}{x_i}\right)(a) = \frac{\partial}{\partial x_i} \left(\podv{f}{x_j}\right)(a).$$
\end{trditev}

\begin{proof}
    Dovolj za $n=2$. 
    
    Definiramo $J = f(a+h, b+k) - f(a+h, b) - f(a, b+k) + f(a,b)$ in $\varphi(x) = f(x, b+k) - f(x, b), \ \psi(y) = f(a+h, y) - f(a, y)$. Zapišemo $J$ s pomočjo funkcij $\varphi, \ \psi$ ter uporabimo 2-krat Lagrangeev izrek in upoštevamo zveznost.
\end{proof}

\begin{opomba}
     Pravimo, da parcialni odvodi komutirajo in pišemo $\frac{\partial^2 f}{\partial x_i \partial x_j}$.
\end{opomba}

\begin{definicija}
    Naj bo $D^\text{odp} \subseteq \R^n$. \emph{Množico vseh $k$-krat zvezno parcialno odvedljivih funkcij} označimo z~$C^k(D)$. \emph{Množica gladkih funkcij} je $C^\infty(D) = \bigcap_{k=1}^\infty C^k(D)$. \emph{Množica zveznih funkcij na $D$} je $C(D)$.
\end{definicija}

\begin{opomba}
    Funkcija $f \in C^k(D)$, če obstajajo vse parcialni odvodi funkcije $f$ do reda $k$ in so vse ti parcialni odvodi zvezni na $D$. 
\end{opomba}

\begin{opomba}
    Množica $C^k(D)$ z operacijama seštevanja, množenja s skalarji in komponiranja preslikav je algebra nad $\R$.
\end{opomba}

\subsubsection{Diferenciabilnost preslikav}

\begin{definicija}
    Naj bo $F: D \subseteq \R^n \to \R^m$ preslikava, $a \in D$ notranja točka. Preslikava $F$ je \emph{diferenciabilna} v točki~$a$, če obstaja taka linearna preslikava $\lin{L}: \R^n \to \R^m$, da velja:
    $$F(a+h) = F(a) + \lin{L}(h) + o(h),$$
    kjer je $\lim_{h \to 0} \frac{|o(h)|_m}{|h|_n}$. 

    Preslikavo $\lin{L}$ imenujemo \emph{diferencial} $F$ v točki $a$. Oznaka: $dF_a$. Imenujemo ga tudi \emph{odvod} $F$ v točki $a$. Oznaka: $(DF)(a)$.
\end{definicija}

\begin{opomba}
    Kot pri funkcijah, če je tak $\lin{L}$ obstaja, je enolično določen.
\end{opomba}

\begin{zgled}
    Obravnavaj diferenciabilnost preslikav:
    \begin{itemize}
        \item $\lin{A}: \R^n \to \R^m$ linearna, $F(x) = \lin{A}x$.
        \item $F: \R^{n \times n} \to \R^{n \times n}, \ F(X) = X^2$.         
        Namig: S pomočjo neenakosti CSB pokažimo, da $|H^2| \leq |H|^2$.
    \end{itemize}
\end{zgled}

\begin{izrek}
    Naj bo $a \in D$ notranja točka. Naj bo $F = (f_1, \ldots, f_m): D \to \R^m$ preslikava. Velja: 
    $$\text{Preslikava } F \text{ je diferenciabilna v } a \in D \liff \text{so } f_1, \ldots, f_m \text{ diferenciabilne v } a.$$
    Tedaj
    $$(DF)(a) = \begin{bmatrix}
        \podv{f_1}{x_1}(a) & \ldots & \podv{f_1}{x_n}(a) \\
        \vdots & & \vdots \\
        \podv{f_m}{x_1}(a) & \ldots & \podv{f_m}{x_n}(a) 
    \end{bmatrix}$$
    Matrika linearne preslikave $(DF)(a)$, ki je zapisana v standardnih bazah, se imenuje \emph{Jacobijeva matrika}.
\end{izrek}

\begin{proof}
    $(\lthen)$ Zapišemo enakost $F(a+h) = F(a) + dF_a(h) + o(h)$ po komponentah.

    $(\Leftarrow)$ Definicija diferenciabilnosti.
\end{proof}

\begin{posledica}
    Naj bo $a \in D$ notranja točka. Naj bo $F = (f_1, \ldots, f_m): D \to \R^m$ preslikava. Velja:
    
    Če so vse funkcije $f_1, \ldots, f_m$ v točki $a$ parcialno odvedlivi po vseh spremenljivkah in so ti vse odvodi zvezni v točki $a$, potem je $F$ diferenciabilna v točki $a$.
\end{posledica}

\begin{zgled}
    Naj bo $F(x,y,z) = (x^2+2y+e^z, xy+z^2), \ f: \R^3 \to \R^2$. Določi $(DF)(1,0,1)$.
\end{zgled}

\begin{definicija}
    Preslikava $F: D^{\text{odp}} \subseteq \R^n \to \R^m$ je razreda $C^k(D)$, če so $f_1, \ldots, f_m \in C^k(D)$.
\end{definicija}

\begin{izrek}[Verižno pravilo]
    Naj bo $a \in D \subseteq \R^n$ notranja točka. Naj bo $b \in \Omega \subseteq \R^m$ notranja točka. Naj bo $F: D \to \Omega$ diferenciabilna v točki $a$ in velja $F(a) = b$. Naj bo $G: \Omega \to \R^k$ diferenciabilna v točki $b$. Tedaj $G \circ F$ diferenciabilna v točki $a$ in velja:
    $$D(G \circ F)(a) = (DG)(b) \cdot (DF)(a) = (DG)(F(a)) \cdot (DF)(a).$$
    Označimo $F(x_1, \ldots, x_n) = (f_1(x_1, \ldots, x_n), \ldots, f_m(x_1, \ldots, x_n))$ in $G(y_1, \ldots, y_m) = (g_1(y_1, \ldots, y_m), \ldots, g_k(y_1, \ldots, y_m))$. Potem 
    $$D(G \circ F)(a) = \begin{bmatrix}
        \podv{g_1}{y_1} & \ldots & \podv{g_1}{y_m} \\
        \vdots & & \vdots \\
        \podv{g_k}{y_1} & \ldots & \podv{g_k}{y_m}
    \end{bmatrix}(b) \cdot \begin{bmatrix}
        \podv{f_1}{x_1} & \ldots & \podv{f_1}{x_n} \\
        \vdots & & \vdots \\
        \podv{f_m}{x_1} & \ldots & \podv{f_m}{x_n}
    \end{bmatrix}(a)$$
\end{izrek}

\begin{proof}
    Definicija diferenciabilnosti.
\end{proof}

\begin{posledica}[$k=1$, $G = g$ funkcija]
    Naj bo $\Phi(x_1, \ldots, x_n) = g(f_1(x_1, \ldots, x_n), \ldots, f_m(x_1, \ldots, x_n))$. Potem 
    $$\podv{\Phi}{x_j}(a) = \podv{g}{y_1}(b) \cdot \podv{f_1}{x_j}(a) + \podv{g}{y_2}(b) \cdot \podv{f_2}{x_j}(a) + \ldots + \podv{g}{y_m}(b) \cdot \podv{f_m}{x_j}(a)$$
\end{posledica}

\begin{zgled}
    Naj bo $F(x,y) = (x^2 + y, xy), \ g(u, v) = uv + v^2$. Naj bo $\Phi = g \circ F$. Izračunaj $(D\Phi)(x,y)$ na dva načina.
\end{zgled}

\newpage
\subsection{Izrek o implicitni preslikavi}
\subsubsection{Osnovna verzija izreka o implicitni preslikavi}
Radi bi poiskali zadostni pogoji na funkcijo $f(x, y)$, da bi enačba $f(x, y) = 0$ lokalno v okolici točki $(a,b)$, za katero velja $f(a,b) = 0$, predstavljala graf funkcije $y = \varphi(x)$.

\begin{izrek}[Osnovna verzija izreka o implicitni preslikavi]
    Naj bo $D \subseteq \R^2$ odprta,  $(a, b) \in D$,  $f: D^\text{odp} \to \R$ funkcija razreda $ C^1(D)$ in naj velja:
    \begin{enumerate}
        \item $f(a,b) = 0$.
        \item $f_y(a,b) \neq 0$.
    \end{enumerate}
    Potem obstajata $\delta > 0$ in $\epsilon > 0$, da velja: $I \times J \subseteq D$, kjer je $I = (a - \delta, a + \delta), \ J = (b-\epsilon, b+\epsilon)$ in enolično določena funkcija $\varphi: I \to J$ razreda $C^1$, za katero velja:
    \begin{enumerate}
        \item $\varphi(a) = b$.
        \item $\all{(x,y) \in I \times J} f(x, y) = 0 \liff y = \varphi (x)$ (rešitve enačbe $f(x,y) = 0$ so natanko graf funkcije $\varphi$).
        \item $\varphi'(x) = -\frac{f_x(x, \varphi(x))}{f_y(x, \varphi(x))}$ za vsak $x \in I$.
    \end{enumerate}
\end{izrek}

\begin{proof}
    Funkcijo $\varphi$ konstruiramo s pomočjo izreka o bisekciji z upoštevanjem stroge monotonosti funkciji $y \mapsto f(x, y)$.

    Zveznost ($\overline{I} \times \overline{J}$ je kompaktna), odvedljivost in zveznost odvoda pokažemo z pomočjo izraza $f(x + \Delta x, y + \Delta y) - f(x,y) = 0$ in Lagrangeeva izreka, kjer $x + \Delta x \in (a - \delta, a + \delta), \ y= \varphi (x), \ y + \Delta y = \varphi(x + \Delta x)$.
\end{proof}

\begin{posledica}
    Če je funkcija $f$ razreda $C^k$, potem je tudi funkcija $\varphi$ razreda $C^k$.
\end{posledica}

\begin{zgled}
    Kaj če pogoji niso izpolnjeni?
    \begin{enumerate}
        \item $f(x,y) = (x-y)^2, \ f(x,y) = 0$ v okolici točke $(0,0)$ (pogoji ni potrebni).
        \item $f(x,y) = y^3 - x, \ f(x,y) = 0$ v okolici točke $(0,0)$ (odvedljivost $\varphi$).
        \item $f(x,y) = y^2-x^2-x^4, \ f(x,y) = 0$ v okolici točke $(0,0)$ (enoličnost $\varphi$).
        \item $f(x,y) = y^2+x^2+x^4, \ f(x,y) = 0$ v okolici točke $(0,0)$ (množica rešitev).
    \end{enumerate}
\end{zgled}

\subsubsection{Izrek o inverzni preslikavi}
Naj bo $\Phi: D^\text{odp} \subseteq \R^m \to \R^m$ preslikava, $\Phi \in C^1(D)$. Kakšne so zadostni pogoji za (lokalno) obrnljivost preslikave $\Phi$?
\begin{definicija}
    Naj bosta $D, \, \Omega \subseteq \R^m$ odprti. Preslikava $\Phi: D  \to \Omega$ je \emph{$C^1$-difeomorfizem}, če
    \begin{enumerate}
        \item $\Phi$ je bijekcija, 
        \item $\Phi \in C^1(D)$,
        \item $\Phi^{-1} \in C^1(\Omega)$.
    \end{enumerate} 
    Podobno definiramo \emph{$C^k$-difeomorfizem} za $k \in \N \cup \set{\infty}$.
\end{definicija}

\begin{zgled}
    Ali je $f(x) = x^3, \ f: \R \to \R$ difeomorfizem?
\end{zgled}

\begin{trditev}
    Naj bosta $D, \, \Omega \subseteq \R^m$ odprti. Naj bo $\Phi: D  \to \Omega$ $C^1$-difeomorfizem. Tedaj je $\det(D\Phi) \neq 0$ na $D$. 
\end{trditev}

\begin{proof}
    Pogledamo $\Phi^{-1} \circ \Phi = \id_D$ (verižno pravilo).
\end{proof}

\begin{posledica}
    $(D\Phi^{-1})(y) = (D\Phi)^{-1}(x)$, kjer $y = \Phi(x)$.
\end{posledica}

\begin{zgled}
    Ali velja obrat trditve? 
    Naj bo $\Phi(x,y) = (e^x \cos y, e^x \sin y), \ \Phi: \R^2 \to \R^2$. Ali je $\Phi$ difeomorfizem?
\end{zgled}

\begin{lema}[Lagrangeev izrek za funkcijo več spremenljivk]
    Naj bo $D \subseteq \R^n$ odprta množica, točki $a, b \in D$ taki, da za vsak $t \in [0,1]$ daljica $(1-t)a + tb \in D$, $f: D \to \R$ funkcija razreda $C^1$.     
    Tedaj obstaja taka točka $\xi$ iz daljice med~$a$~in~$b$, da je $f(b)-f(a) = (Df)(\xi)(b-a)$.
\end{lema}

\begin{proof}
    Lagrangeev izrek za funkcijo $\varphi(t) = f((1-t)a + tb)$.
\end{proof}

\begin{lema}
    Predpostavki kot prej. Naj obstaja tak $M \in \R$, da za vsak $j = 1, \ldots, n$ in vsak $x \in D$ velja: $\left|\podv{f}{x_j}(x)\right| \leq M$. Tedaj~$|f(b)-f(a)| \leq M \sqrt{n}|b-a|$.
\end{lema}

\begin{proof}
    Uporabimo prejšnjo trditev.
\end{proof}

\begin{lema}
    Naj bo $D \subseteq \R^n, \ a,b \in D$ kot prej. Naj bo $F: D \subseteq \R^n \to \R^m, \ F = (f_1, \ldots, f_m)$ preslikava razreda~$C^1$. Naj obstaja tak $M \in \R$, da za vsak $j = 1, \ldots, n$, vsak $i = 1, \ldots, m$ in vsak $x \in D$ velja: $\left|\podv{f_i}{x_j}(x)\right| \leq M$. 
    
    Tedaj~$\norm{F(b) - F(a)}~\leq~M~\sqrt{mn}~\norm{b - a}$.
\end{lema}

\begin{izrek}[Izrek o inverzni preslikavi]
    Naj bo $D \subseteq \R^m$ odprta, $F: D \to \R^m$ preslikava razreda $C^1$,  $a \in D$ in $b = F(a)$.
    Če je $\det(DF)(a) \neq 0$, potem obstajata okolici $a \in U \subseteq \R^m$ in $b \in V \subseteq \R^m$, da je $F: U \to V$ $C^1$-difeomorfizem.    
\end{izrek}

\begin{definicija}
    Če je $F: D \to \Omega$ preslikava med odprtimi množicami v $\R^m$ in je $\det(DF)(x) \neq 0$ za vse $x \in D$, pravimo, da je $F$ \emph{lokalni difeomorfizem}.
\end{definicija}

\begin{proof}
    Dovolj, da izrek dokažemo za primer, ko $a=b=0, \ (DF)(0) = I$.

    \textcolor{red}{TODO}
\end{proof}

\begin{posledica}
    Če je $\Phi$ razreda $C^k$ za $k \in \N \cup \set{\infty}$, je $\Phi$ lokalni $C^k$ difeomorfizem.
\end{posledica}

\begin{opomba}
    Če je $m=1$, potem $f: I \subseteq \R \to \R$. Naj bo $a \in I, \ f \in C^1(I), \ f'(a) \neq 0$. Potem $f'(x) \neq 0$ v okolici $a$, torej~$f$ ima lokalni $C^1$ inverz.
\end{opomba}

\begin{zgled}
    Naj bo $F: \R^{n \times n} \to \R^{n \times n}, \ F(X) = X^2$. Ali je $F$ v okolici točke $I \in \R^{n \times n}$ lokalni difeomorfizem? Kaj to pomeni?
\end{zgled}

\newpage
\subsubsection{Izrek o implicitni preslikavi}
Imamo $n+m$ spremenljivk: $(x, y)$, kjer $x = (x_1, \ldots, x_n), \ y = (y_1, \ldots, y_m)$ in $m$ enačb. Pričakujemo, da bomo lahko $m$~spremenljivk izrazili kot funkcijo ostalih, tj. najdemo preslikavo $\Phi: D \subseteq \R^n \to \R^m$, da velja $y = \Phi(x)$.


\begin{primer}[Linearen primer]
    Naj bosta $\lin{A}: \R^n \to \R^m, \ \lin{B}: \R^m \to \R^m$ linearni, $b \in \R^m$. Naj rešujemo enačbo $Ax + By = b$. Kdaj lahko za vsak $b \in \R^m$ iz te  enačbe $y$ razrišemo kot funkcijo $x$? 

    Če je $n=0$, potem rešujemo enačbo $By = b$. Kdaj lahko to enačbo enolično rešimo za vsak $b \in \R^m$?
\end{primer}

Naj bo $F: D^\text{odp} \subseteq \R^n_x \times \R^m_y \to \R^m, \ F = (f_1, \ldots, f_m)$ preslikava razreda $C_1$. 

Za vsak $y \in \R^m$ naj bo $\podv{F}{x}$ diferencial preslikave $x \mapsto F(x,y)$. Imenujemo ga \emph{parcialni diferenical na prvo spremenljivko}. 

Za vsak $x \in \R^n$ naj bo $\podv{F}{y}$ diferencial preslikave $y \mapsto F(x,y)$. Imenujemo ga \emph{parcialni diferenical na drugo spremenljivko}. 

Velja: 
$\podv{F}{x}(x,y) = \begin{bmatrix}
    \podv{f_1}{x_1}(x,y) & \ldots & \podv{f_1}{x_n}(x,y) \\
    \vdots & & \vdots \\
    \podv{f_m}{x_1}(x,y) & \ldots & \podv{f_m}{x_n}(x,y) 
\end{bmatrix} \text{ in } \podv{F}{y}(x,y) = \begin{bmatrix}
    \podv{f_1}{y_1}(x,y) & \ldots & \podv{f_1}{y_m}(x,y) \\
    \vdots & & \vdots \\
    \podv{f_m}{x_1}(x,y) & \ldots & \podv{f_m}{y_m}(x,y) 
\end{bmatrix}$.

Diferencial preslikave $F$ je potem enak $(DF)(x,y) = \begin{bmatrix}
    \podv{F}{x}(x,y) & \podv{F}{y}(x,y)
\end{bmatrix}$ (bločni zapis).

\begin{opomba}
    Za vektor $\begin{bmatrix}
        h \\ k
    \end{bmatrix}$, kjer je $h \in \R^n, \ k \in \R^m$ velja: $(DF)(x,y)  \begin{bmatrix}
        h \\ k
    \end{bmatrix} = \podv{F}{x}(x,y)  h + \podv{F}{y}(x,y)  k \in \R^m$.
\end{opomba}

\begin{izrek}[Izrek o implicitni preslikavi]
    Naj bo $D\subseteq \R_x^n \times \R_y^m$ odprta množica, $(a, b) \in D$, $F: D \to \R^m$ preslikava razreda $C^1$. Naj velja:
    \begin{enumerate}
        \item $F(a, b) = 0$,
        \item $\det(\podv{F}{y}(a,b)) \neq 0$.
    \end{enumerate}
    Tedaj obstaja okolica $U \subseteq \R^n$ točke $a$ in okolica $V \subseteq \R^m$ točke $b$ in taka enolično določena preslikava $\varphi: U \to V$ razreda~$C^1$, da velja:
    \begin{enumerate}
        \item $\phi(a) = b$.
        \item $\all{(x,y) \in U \times V} F(x,y) = 0 \liff y = \varphi(x)$ (rešitve te enačbe je isto kot graf $\varphi$ znotraj $U \times V$).
        \item $(D \varphi)(x) = - \left(\podv{F}{y}(x,y)\right)^{-1} \podv{F}{x}(x,y), \ y = \varphi(x)$ za vsak $x \in U$.
    \end{enumerate}
\end{izrek}

\begin{proof}
    Uporabimo izrek o inverzni preslikavi. 
    
    Definiramo preslikavo $\Phi: D \subseteq \R^n \times \R^m \to  \R^n \times \R^m, \ \Phi(x,y) = (x, F(x,y))$. Kandidata za preslikavo $\varphi$ najdemo v oblike inverza $\Phi^{-1}$, nato enostavno preverimo lastnosti.
\end{proof}

\begin{posledica}
    Če je preslikava $F$ razreda $C^k$, je tudi preslikava $\varphi$ razreda $C^k$.
\end{posledica}

\begin{zgled}
    Naj bo $x,y \in \R, \ F(x, y) = x^2 + y^2 -1$. S pomočjo izreka o implicitni preslikavi pokaži, da v okolici točke $(0,1)$ rešitve enačbe $F(x,y) = 0$ graf neke preslikave $\varphi$. Določi tudi preslikavo $\varphi$.
\end{zgled}

\begin{zgled}
    Naj bo $F(x,y,z) = (y + xy+xz^2, z+zy+x^2), \ F = (f, g)$ in naj rešujemo enačbo $F(x,y,z) = 0$. Preveri zahteve izreka v okolici točke $(0,0,0)$ in zapiši spremenljivki $y$ in $z$ kot funkciji spremenljivke $x$. Določi tudi prvi in drugi odvod funkcij $f$ in $g$ po spremenljivke $x$. Kaj je rezultat?
\end{zgled}

\begin{zgled}
    Naj bo $F: \R^3 \to \R$ in naj rešuejmo enačbo $F(x,y,z) = 0$. Recimo, da $F(a,b,c) = 0$. Kakšna povezava med zadostnimi pogajami in rangom $(DF)(a,b,c)$? Kaj če gledamo preslikavo $F: \R^3 \to \R^2$?
\end{zgled}

\begin{definicija}
    Naj bo $D^\text{odp} \in \R^n$ in $F: D \to \R^m$ preslikava razreda $C^1$, $a \in D$. 
    \begin{enumerate}
        \item \emph{Rang preslikave $F$ v točki $a$} je~$\rang_aF := \rang (DF)(a)$.
        \item Če je $\rang_aF$ konstanten na $D$, je $F$ tega ranga na $D$, tj. $\rang F = \rang_aF$.
        \item Preslikava $F$ je \emph{maksimalnega ranga v točki $a$}, če je $\rang_aF = \min \set{m, n}$.
    \end{enumerate}      
\end{definicija}

\begin{opomba}
    Ta pogoj je lokalno stabilen, tj. če je $\rang_aF = \min \set{n, m}$, potem obstaja okolica od $a$, kjer $\rang F$~maksimalen.
\end{opomba}

\newpage
\begin{posledica}
    Naj bo $F: D^\text{odp} \subseteq \R^n \to \R^m$ preslikava razreda $C^k, \ k \in \N$ in naj velja $m < n$. Naj bo $a \in D, \ F(a) = 0$ in $F$ maksimalnega ranga v točki $a$. Tedaj obstajajo indeksi $i_1 < i_2 < \ldots < i_{n-m}$, $j_1 < j_2 < \ldots <j_m$, $i_k \neq j_l$ za vse $k$ in $l$ in take funkcije $\varphi_1, \ldots, \varphi_m$ razreda $C^k$ definirane v okolici točke $(a_{i_1}, \ldots, a_{i_{n-m}})$, da je v neki okolici~$U$ točke $a$ enačba $F(x) = 0$ ekvivalentna sistemu enačb: 
    \begin{align*}
        x_{j_1} = \varphi_1(x_{i_1}, \ldots, x_{i_{n-m}}) \\
        \vdots \\
        x_{j_m} = \varphi_m(x_{i_1}, \ldots, x_{i_{n-m}})
    \end{align*}
    Ekvivalentno: Obstaja permutacija $\sigma \in S_n$, da v okolici točke $a$ velja:
    $$F(x) = 0 \liff (x_{\sigma(1)}, \ldots, x_{\sigma(n)}) = (x_{\sigma(1)}, \ldots, x_{\sigma(n-m)}, \varphi(x_\sigma')), \text{ kjer } \varphi = (\varphi_1, \ldots, \varphi_m).$$
\end{posledica}

\begin{proof}
    \textcolor{red}{TODO}
\end{proof}

\begin{primer}
    Naj bo $\lin{A}: \R^n \to \R^m$ linearna, $m \leq n$, $\rang \lin{A} = m$ ($\lin{A}$ je surjektivna). Rešujemo enačno $\lin{A}x=b$. Prostor rešitev je $n-m$ dimenzialen.
\end{primer}

\begin{posledica}
    Naj bo $F: D^\text{odp} \subseteq \R^n \to \R^m$ preslikava razreda $C^1$, $m \leq n$, $a \in D$ in naj velja $\rang_aF=m$. Tedaj obstaja okolica $V$ točke $F(a) = b$ in okolica $U$ točke $a$, da je $F: U \to V$ surjektivna.
\end{posledica}

\begin{proof}
    \textcolor{red}{TODO}
\end{proof}

\subsection{Podmnogoterosti v $\R^n$}
Podmnogoterost je posplošitev pojmov "`krivulja"' in "`ploskev"'.

\begin{definicija}
    Naj bo $M \subseteq \R^{n+m}, \ M \neq \emptyset$. Množica $M$ je \emph{gladka (vsaj razreda $C^1$) podmnogoterost dimenzije $n$ in kodimenzije~$m$ prostora $\R^{n+m}$}, če 
    za vsako točko $a \in M$ obstaja okolica $U$ v $\R^{n+m}$ in take $C^1$ funkcije $F_1, \ldots, F_m: U \to \R$, da velja:
    \begin{enumerate}
        \item $M \cap U = \setb{x \in U}{F_1(x) = \ldots = F_m(x) = 0} = F^*(\set{0})$.
        \item $\rang(F_1, \ldots, F_m) = m$ na $U$.
    \end{enumerate}
\end{definicija}

\begin{opomba}
    Funkcije $F_1, \ldots, F_m$ se imenujejo \emph{lokalne definicijske funkcije za $M \cap U$}.
\end{opomba}

\textcolor{red}{TODO}

\newpage
\subsection{Taylorjeva formula}
Naj bo $f: D^{\text{odp}} \subseteq \R^n \to \R$ funkcija, $a \in D$. Funkcijo $f$ bi radi v okolici točke $a$ aproksimirali s polinomi.

\begin{izrek}
    Recimo, da velja
    \begin{enumerate}
        \item Množica $D \subseteq \R^n$ odprta, $a \in D$.
        \item $f: D^{\text{odp}} \to \R$ funkcija razreda $C^{k+1}(D)$.
        \item Vektor $h \in \R^n$ tak, da daljica med $a$ in $a+h$ leži v $D$.
    \end{enumerate}
    Tedaj obstaja tak $\theta \in (0,1)$, da je 
    $$f(a+h) = f(a) + (D_hf)(a) + \frac{1}{2!}(D_h^2f)(a) + \ldots + \frac{1}{k!} (D_h^kf)(a) + R_k \ \textcolor{red}{(*)},$$
    kjer je $D_h = h_1D_1 + h_2D_2 + \ldots + h_nD_n$ \df{odvod v smeri $h$} in $R_k = \frac{1}{(k+1)!} (D_h^{k+1}f)(a + \theta h)$ \df{ostanek}.

    Izraz \textcolor{red}{(*)} je \df{Taylorjeva formula} za funkcijo več spremenljivk.
\end{izrek}

\begin{proof}
    \textcolor{red}{TODO}
\end{proof}

\begin{opomba}
    Pokaži, da velja
    \begin{enumerate}
        \item $(D_hf)(a) = \sum_{j=1}^{n}h_j \podv{f}{x_j}(a)$.
        \item $(D_h^2f)(a) = \sum_{k=1}^{n} \sum_{j=1}^{n}h_kh_j \frac{\partial^2f}{\partial x_k \partial x_j}(a)$. 
    \end{enumerate}
\end{opomba}

\begin{primer}
    Pokaži, da za $n=2$ velja $D_{(h,k)}^m = \sum_{j=0}^{m} \binom{m}{j} h^j k^{m-j} \frac{\partial^m}{\partial x^j \partial y^{m-j}}$.
\end{primer}

\begin{opomba}
    $h \mapsto f(a) + (D_hf)(a) + \frac{1}{2!}(D_h^2f)(a) + \ldots + \frac{1}{k!} (D_h^kf)(a) $ je polinom stopnje največ $k$ v spremenljivkah $h_1, h_2, \ldots, h_n$.
\end{opomba}

\begin{opomba}
    Če je funkcija $f$ razreda $C^\infty(D)$ lahko tvorimo \df{Taylorjevo vrsto}:
    $$\sum_{j=1}^{\infty} \frac{1}{j!} (D_h^j f)(a).$$
    \begin{itemize}
        \item Vrsta sigurno konvergira za $h=0$.
        \item Tudi, če vrsta konvergira za nek $h \neq 0$, ne konvergira nujno k $f(a+h)$.
    \end{itemize}
\end{opomba}

\begin{definicija}
    Če Taylorjeva vrsta konvergira k $f(a+h)$ za vse vse $\norm{h} \leq \delta$ za nek $\delta > 0$, tj. 
    $$f(a+h) = \sum_{j=1}^{\infty} \frac{1}{j!} (D_h^j f)(a),$$
    potem rečemo, da je \df{funkcija $f$ v okolici točke $a$ (realno) analitična}.
\end{definicija}

\begin{zgled}
    Razvij funkcijo $f(x,y) = e^{xy}$ v Taylorjevo vrsto v okolici točke $(0,0)$.
\end{zgled}

\begin{posledica}
    Recimo, da velja
    \begin{enumerate}
        \item Podmnožica $D \subseteq \R^n$ odprta, $a \in D$.
        \item $f: D^{\text{odp}} \to \R$ funkcija razreda $C^{k+1}(D)$.
        \item Vektor $h \in \R^n$ tak, da daljica med $a$ in $a+h$ leži v $D$.
    \end{enumerate}
    Potem je 
    \begin{enumerate}
        \item $R_k = o(\norm{h}^k)$ za $h \to 0$.
        \item $R_k = O(\norm{h}^{k+1})$ za $h \to 0$.
    \end{enumerate}
\end{posledica}

\begin{opomba}
    Velja:
    \begin{enumerate}
        \item $R_k = o(\norm{h}^k) \liff \lim_{h \to 0} \frac{|R_k|}{\norm{h}^k} = 0$ (izraz je majhen).
        \item $R_k = O(\norm{h}^{k+1}) \liff \some{M \in \R} \frac{|R_k|}{\norm{h}^{k+1}} \leq M, \text{ ko gre } h \text{ proti } 0$ (velikostni red).
    \end{enumerate}
\end{opomba}

\begin{proof}
    \textcolor{red}{TODO}
\end{proof}

\begin{opomba}
    Naj bo $f: \R^2 \to \R$ funkcija razreda $C^\infty$ v okolici točke $(0,0)$, $h = (x, y)$. 
    
    Pokaži, da za koeficient $a_{nm}$ pred $x^ny^m$ velja: $(\frac{\partial^{n+m}}{\partial x^n \partial y^m} f)(0,0) = a_{nm} n!m!$. 
\end{opomba}

\newpage
\subsection{Ekstremi funkcij več spremenljivk}
\begin{definicija}
    Naj bo $f: D \subseteq \R^n \to \R$ funkcija, $a \in D$.
    \begin{enumerate}
        \item Funkcija $f$ ima v točki $a$ \df{lokalni maksimum}, če 
        $$\some{r>0} \all{x \in D \cap K(a,r)} f(a) \geq f(x).$$ 
        Funkcija $f$ ima v točki $a$ \df{strogi lokalni maksimum}, če
        $$\some{r>0} \all{x \in D \cap K(a,r)} f(a) > f(x).$$ 
        \item Funkcija $f$ ima v točki $a$ \df{(globalni) maksimum na $D$}, če 
        $$\all{x \in D} f(a) \geq f(x).$$
        \item Podobno definiramo: \df{lokalni minimum}, \df{(globalni) minimum}.
        \item \df{Lokalni ekstrem} (oz. \df{globalni ekstrem}) je skupno ime za lokalni (oz. globalni) minumum in maksimum.
    \end{enumerate}
\end{definicija}

\begin{opomba}
    Če je $K^\text{komp} \subseteq \R^n$ in $f: K \to \R$ zvezna funkcija, potem ima $f$ na $K$ maksimum in minimum.
\end{opomba}

\begin{definicija}
    Naj bo $D^\text{odp} \subseteq \R^n$ in $f: D \to \R$ funkcija razreda $C^1$ (dovolj, da diferenciabilna). 
    
    Rečemo, da je točka $a \in D$ \df{stacionarna (oz. kritična) točka  funkcije $f$}, če $$(Df)(a) = 0, \text{ tj. } \podv{f}{x_1}(a) = \podv{f}{x_2}(a) = \ldots = \podv{f}{x_n}(a) = 0.$$
\end{definicija}

\begin{trditev}
    Recimo, da velja
    \begin{enumerate}
        \item Podmnožica $D \subseteq \R^n$ odprta, $a \in D$.
        \item $f: D \to \R$ funkcija razreda $C^1$.
    \end{enumerate}
    Tedaj, če ima funkcija $f$ v točki $a$ lokalni ekstrem, je $a$ kritična točka za $f$.
\end{trditev}

\begin{proof}
    \textcolor{red}{TODO}
\end{proof}

\begin{zgled}
    Naj bo $K = \setb{(x, y) \in \R^2}{x^2+y^2 \leq 3}$, $f(x, y) = x^2 - xy + y^2 - 3x +4$. 
    Poišči minimum in maksimum funkcije~$f$.
\end{zgled}

\subsubsection{Potrebni in zadostni pogoji na 2. odvodi, da je kritična točka lokalni ekstrem}
Naj bo $D \subseteq \R^n$ odprta, $f: D \to R$ funkcija razreda $C^2$. Definiramo \df{Hessejevo matriko} 2. odvodov:
$$(Hf) (x) = \begin{bmatrix}
    \frac{\partial^2 f}{\partial x_1^2} & \frac{\partial^2 f}{\partial x_1 \partial x_2} & \ldots & \frac{\partial^2 f}{\partial x_1 \partial x_n} \\
    \frac{\partial^2 f}{\partial x_2 \partial x_1} & \frac{\partial^2 f}{\partial x_2^2} & \ldots & \frac{\partial^2 f}{\partial x_2 \partial x_n} \\
    \vdots & & & \\
    \frac{\partial^2 f}{\partial x_n \partial x_1} & \frac{\partial^2 f}{\partial x_n \partial x_2} & \vdots & \frac{\partial^2 f}{\partial x_n^2}
\end{bmatrix}.$$

\begin{opomba}
    Če je $f \in C^2(D)$, potem mešani odvodi so enaki, tj. $(Hf)^T = Hf$. Torej Hessejeva matrika je simetrična, torej ima v vsaki točki realne lastne vrednosti. 
\end{opomba}

$\left\langle (Hf)h, h\right\rangle$ je \df{Hessejeva forma} (kvadratna forma, ki pripada matrike $(Hf)(a)$).

\begin{definicija}
    Hessejeva matrika $Hf$ je
    \begin{itemize}
        \item \df{pozitivno semidefinitna} (pišemo $Hf \geq 0$), če $\all{v \in D} \left\langle (Hf)v, v\right\rangle \geq 0 \liff \text{vse lastne vrednosti so nenagitvne}$;
        \item \df{pozitivno definitna} (pišemo $Hf > 0$), če $\all{v \in D} v \neq 0 \lthen \left\langle (Hf)v, v\right\rangle > 0 \liff \text{vse lastne vrednosti so pozitivne}$;
        \item \df{negativno semidefinitna} (pišemo $Hf \leq 0$), če $\all{v \in D}\left\langle (Hf)v, v\right\rangle \leq 0 \liff \text{vse lastne vrednosti so nepozitivne}$;
        \item \df{negativno definitna} (pišemo $Hf < 0$), če $\all{v \in D} v \neq 0 \lthen \left\langle (Hf)v, v\right\rangle < 0 \liff \text{vse lastne vrednosti so negativne}$.
    \end{itemize}
\end{definicija}

\begin{trditev}[Potrebni pogoji]
    Recimo, da velja
    \begin{enumerate}
        \item Podmnožica $D \subseteq \R^n$ odprta, $a \in D$.
        \item Funkcija $f: D \to \R$ razreda $C^2$.
    \end{enumerate}
    Tedaj
    \begin{itemize}
        \item Če ima $f$ v točki $a$ lokalni maksimum, potem
        \begin{enumerate}
            \item $(Df)(a) = 0$,
            \item $Hf(a) \leq 0$.
        \end{enumerate}
        \item Če ima $f$ v točki $a$ lokalni minimum, potem
        \begin{enumerate}
            \item $(Df)(a) = 0$,
            \item $(Hf)(a) \geq 0$.
        \end{enumerate}
    \end{itemize}
\end{trditev}

\begin{proof}
    \textcolor{red}{TODO}
\end{proof}

\newpage
\begin{izrek}[Zadostni pogoji]
    Recimo, da velja
    \begin{enumerate}
        \item Podmnožica $D \subseteq \R^n$ odprta, $a \in D$.
        \item Funkcija $f: D \to \R$ razreda $C^2$.
        \item $a \in D$ stacionarna točka funkcije $f$.
    \end{enumerate}
    Tedaj
    \begin{itemize}
        \item Če je $(Hf)(a) > 0$, potem ima funkcija $f$ v točki $a$ (strogi) lokalni minimum.
        \item Če je $(Hf)(a) < 0$, potem ima funkcija $f$ v točki $a$ (strogi) lokalni maksimum.
        \item Če ima $(Hf)(a)$ tako pozitivne, kot negativne lastne vrednosti, potem funkcija $f$ v točki $a$ nima lokalnega ekstrema.
    \end{itemize}
\end{izrek}

\begin{zgled}
    Določi $(Hf_i)(0,0)$ za $f_1(x,y) = \frac{1}{2}(x^2+y^2)$, $f_2(x,y) = \frac{1}{2}(-x^2-y^2)$, $f_3(x,y) = \frac{1}{2}(x^2-y^2)$.
\end{zgled}

\begin{posledica}[Zadostni pogoji, $n=2$]
    Recimo, da velja
    \begin{enumerate}
        \item Podmnožica $D \subseteq \R^2$ odprta, $(a, b) \in D$.
        \item Funkcija $f: D \to \R$ razreda $C^2$.
        \item $(a, b) \in D$ stacionarna točka funkcije $f$.    
    \end{enumerate}
    Tedaj
    \begin{itemize}
        \item Če je $f_{xx}f_{yy} - f_{xy}^2(a,b) > 0$, potem ima funkcija $f$ v točki $(a, b)$.
        \begin{itemize}
            \item Če je $f_{xx}(a,b)>0$, potem ima funkcija $f$ v točki $(a,b)$ lokalni minimum.
            \item Če je $f_{xx}(a,b)<0$, potem ima funkcija $f$ v točki $(a,b)$ lokalni maksimum.
        \end{itemize}
        \item Če je $f_{xx}f_{yy} - f_{xy}^2(a,b) < 0$, potem funkcija $f$ v točki $(a, b)$ nima lokalnega ekstrema.
    \end{itemize}
\end{posledica}

\begin{proof}
    \textcolor{red}{TODO}
\end{proof}

\begin{zgled}
    Naj bo $f(x,y,z) = x^2+y^2+z^2 + 2xyz$. Klasificiraj vse stacionarne točke funkcije $f$.
\end{zgled}

\subsection{Vezani ekstremi}
\begin{izrek}
    Recimo, da velja
    \begin{enumerate}
        \item Podmnožica $D \subseteq \R^n$ odprta.
        \item Funkciji $f, g_1, \ldots, g_m$ razreda $C^1(D)$, $m < n$.
        \item Preslikava $G = (g_1, \ldots, g_m): D \to \R^m$ maksimalnega ranga.
        \item $M = G^{-1}(\set{0}) \neq \emptyset$, tj. $M = \setb{x \in D}{\underbrace{g_1(x) = 0, \ldots, g_m(x) = 0}}$ podmnogoterost v $D$.
        \item Funkcija $f: M \to \R$ ima v točki $a \in M$ lokalni ekstrem (kot funkcija iz $M$ v $\R$).        
    \end{enumerate}
    Tedaj obstajajo take realne konstante $\lambda_1, \ldots, \lambda_m$, da je 
    $$(Df)(a) = \lambda_1 (Dg_1)(a) + \ldots + \lambda_m (Dg_m)(a) = \sum_{j=1}^{m}\lambda_j(Dg_j)(a).$$ 
\end{izrek}

\begin{proof}
    \textcolor{red}{TODO}
\end{proof}

\begin{opomba}
    Lagrangeeva metoda za iskanja vezanih ekstremov
    \begin{enumerate}
        \item Tvorimo funkcijo $F(\underbrace{x_1, \ldots, x_n}_x, \lambda_1, \ldots, \lambda_m) = f(x) - \sum_{j=1}^{m} \lambda_j g_j(x)$.
        \item Iščemo stacionarne točke $F$:
        \begin{itemize}
            \item $D_xF = (Df)(x) - \sum_{j=1}^{m}\lambda_j(Dg_j)(x) = 0$ ($n$ enačb).
            \item $D_{\lambda_j}F = -g_j(x) = 0$ za $j = 1, \ldots, m$ ($m$ enačb).
        \end{itemize}
        Konstante $\lambda_1, \ldots, \lambda_m$ so \df{Lagrangeevi multiplikatorji}.
    \end{enumerate}    
\end{opomba}

\begin{zgled}
    Določi stacionarne točke funkcije $f(x,y,z)=z$ na $M = \setb{(x,y,z) \in \R^3}{x^2+y^2+z^2=1; \ x+ y + z = 0}$.
\end{zgled}

\begin{zgled}
    Določi stacionarne točke funkcije $f(x,y,z)= x^2 - xy +y^2 - 3x +4$ na robu $x^2 + y^2 = 9$.
\end{zgled}