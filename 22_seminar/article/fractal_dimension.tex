\documentclass[a4paper,12pt]{article}

\usepackage[slovene]{babel}
\usepackage{amsfonts, amssymb, amsmath, amsthm}
\usepackage[utf8]{inputenc}
\usepackage[T1]{fontenc}
\usepackage{lmodern}
\usepackage{graphicx}
\usepackage{xcolor}

% bibliography
\usepackage[
    backend=biber,
    sorting=nty,
]{biblatex}
\addbibresource{sample.bib}

% graphics
\usepackage{tikz}
\usetikzlibrary{math}
\usetikzlibrary{decorations.fractals}
\usetikzlibrary{lindenmayersystems}
\usepackage{graphicx}
\usepackage{pgfplots}
\pgfplotsset{compat=1.18}

% Standardne množice
\newcommand{\NN}{\mathbb{N}}
\newcommand{\ZZ}{\mathbb{Z}}
\newcommand{\QQ}{\mathbb{Q}}
\newcommand{\RR}{\mathbb{R}}
\newcommand{\CC}{\mathbb{C}}
\newcommand{\FF}{\mathbb{F}}

\newcommand{\Aa}{\mathcal{A}}
\newcommand{\Bb}{\mathcal{B}}
\newcommand{\Ff}{\mathcal{F}}

\newcommand{\Fmn}{\FF^{m \times n}}
\newcommand{\Fnn}{\FF^{n \times n}}
\newcommand{\Fxy}[2]{\FF^{{#1} \times {#2}}}

%%% Množice
\newcommand{\set}[1]{\left\{#1\right\}}

%%% Zaporedja
\newcommand{\zaporedje}[2]{({#1}_{#2})_{#2}}

%%% Vsote in vrste
\newcommand{\vsota}[2]{{#1}_1+{#1}_2+\ldots+{#1}_{#2}}
\newcommand{\vrsta}[1]{\sum_{n=1}^{\infty} {#1}_n}

%%% Seznami
\renewcommand{\labelenumi}{\arabic{enumi}.}
\renewcommand{\labelenumii}{\arabic{enumi}.\arabic{enumii}}

\newcommand{\x}[1]{x_1, \ldots, x_{#1}}
\newcommand{\vecs}[2]{{#1}_1, \ldots, {#1}_{#2}}

\DeclareMathOperator{\Lin}{Lin}
\DeclareMathOperator{\End}{End}
\DeclareMathOperator{\im}{im}
\DeclareMathOperator{\id}{id}
\DeclareMathOperator{\rang}{rang}


\def\N{\mathbb{N}} % mnozica naravnih stevil
\def\Z{\mathbb{Z}} % mnozica celih stevil
\def\Q{\mathbb{Q}} % mnozica racionalnih stevil
\def\R{\mathbb{R}} % mnozica realnih stevil
\def\C{\mathbb{C}} % mnozica kompleksnih stevil
\newcommand{\geslo}[2]{\noindent\textbf{#1} \quad \hangindent=1cm #2\\[-1pc]}

\def\qed{$\hfill\Box$}   % konec dokaza
\def\qedm{\qquad\Box}   % konec dokaza v matematičnem načinu
\newtheorem{izrek}{Izrek}[section]
\newtheorem{trditev}{Trditev}[section]
\newtheorem{posledica}{Posledica}[section]
\newtheorem{lema}{Lema}[section]
\newtheorem{opomba}{Opomba}[section]
\newtheorem{definicija}{Definicija}[section]
\newtheorem{zgled}{Zgled}[section]

\title{Fraktalne dimenzije \\ 
\Large Seminar}
\author{Ruslan Urazbakhtin \\
Fakulteta za matematiko in fiziko \\
Oddelek za matematiko}
\date{1.\ julij 2025}

\begin{document}


%%%%%%%%%%%%%%%%%%%%%%%%%%%%%%%%%%%%%%%%%%%%%%%%%%%%%%%%%%%%%%%%%%%%%


\maketitle


%%%%%%%%%%%%%%%%%%%%%%%%%%%%%%%%%%%%%%%%%%%%%%%%%%%%%%%%%%%%%%%%%%%%%
\section{Uvod}
Pojem dimenzije igra osrednjo vlogo v fraktalni geometriji. Intuitivno nam dimenzija množice pove, koliko prostora ta zavzema znotraj ambientnega prostora. Za običajne geometrijske objekte, kot so točke, daljice ali ploskve, se ta predstava sklada z našo intuicijo: točka ima dimenzijo \(0\), daljica \(1\), kvadrat \(2\) in kocka \(3\).

Fraktali pa pogosto kljubujejo tej klasični predstavi. Cantorjeva množica (slika \ref{fig:cantor-set}) na primer nima dolžine, vendar vsebuje neštevno mnogo točk in ima kompleksno strukturo. Zdi se, da »zavzema več kot nič, a manj kot eno dimenzijo«. Da bi takšne množice natančno opisali, potrebujemo bolj prefinjen pojem dimenzije.

Naravno se pojavi vprašanje, kako lahko definiramo dimenzijo za splošne množice? Naše izhodišče bo formalna definicija Hausdorffove dimenzije, ki jo bomo skrbno razvili in utemeljili. Nato bomo raziskali njene osnovne lastnosti in pokazali, zakaj predstavlja naravno posplošitev klasičnega pojma dimenzije.

\subsection{Cantorjeva množica}
V tem članku se bomo pogosto ukvarjali s Cantorjevo množico, ki je klasičen in hkrati izjemno pomemben primer množice s fraktalno strukturo.

\begin{figure}[ht]
    \centering
    \drawCantor{7}
    \caption{Cantorjeva množica.}
    \label{fig:cantor-set}
\end{figure}

Izgradimo Cantorjevo množico na intervalu \([0,1]\) po naslednjem postopku:
\begin{enumerate}
    \item Naj bo \(C_0 = [0,1]\).
    \item Množico \(C_0\) razdelimo na tri enake dele in odstranimo odprti srednji interval \((1/3, 2/3)\). Tako dobimo množico \(C_1 = [0, 1/3] \cup [2/3, 1]\).
    \item Recimo, da imamo množico \(C_n\), ki je unija \(2^n\) zaprtih intervalov dolžine \(3^{-n}\). Vsak izmed teh intervalov razdelimo na tri enake dele in odstranimo odprti srednji del. Tako dobimo množico \(C_{n+1}\), ki je unija \(2^{n+1}\) zaprtih intervalov dolžine \(3^{-(n+1)}\).
    \item Postopek nadaljujemo induktivno.
\end{enumerate}

\begin{definicija}
    \emph{Cantorjeva množica} je množica
    \[
        C = \bigcap_{n \in \N_0} C_n.
    \]
\end{definicija}

\section{Teorija mere}
Če želimo govoriti o fraktalni dimenziji, moramo najprej razumeti pojem mere. 
Za naš namen pa bo dovolj, da se seznanimo le z osnovnimi idejami tega področja. 

Zgledi za mero so dolžina podmnožic v \(\R\), ploščina ravninskih likov in prostornina teles v prostoru. Ta pojem želimo posplošiti na poljubne merljive prostore.

\begin{definicija}
    Naj bo \(X\) množica. Družino podmnožic \(\mc{A}\) množice \(X\) imenujemo \emph{\(\sigma\)-algebra na \(X\)}, če ima naslednje tri lastnosti:
    \begin{enumerate}
        \item \(X \in \mc{A}\);
        \item za vsako podmnožico \(S \in \mc A\) je tudi \(\comp{S} \in \mc{A}\);
        \item za vsako števno družino \((A_i)_{i \in \N}\) elementov iz \(\mc{A}\) velja, da je tudi njihova unija \(\un{i \in \N}{A_i}\) element \(\mc{A}\).
    \end{enumerate}
    Elemente družine \(\mc{A}\) imenujemo \emph{merljive množice}. Množico \(X\), opremljeno z družino \(\mc{A}\), pa imenujemo \emph{merljiv prostor}.
\end{definicija}

\begin{opomba}
    Enostavno je videti, da za vsako \(\sigma\)-algebro \(\mc{A}\) na \(X\) velja \(\emptyset \in \mc{A}\) ter da je zaprta tudi za števne preseke.
\end{opomba}

Izkaže se, da je presek družine \(\sigma\)-algeber na množici \(X\) spet \(\sigma\)-algebra na \(X\). Zato lahko definiramo:

\begin{definicija}
    Naj bo \(X\) topološki prostor in \(\mc{O}\) družina vseh odprtih podmnožic v \(X\). Presek vseh \(\sigma\)-algeber, ki vsebujejo \(\mc{O}\), imenujemo \emph{Borelova \(\sigma\)-algebra}, njene elemente pa \emph{Borelove množice}. Označili jo bomo z \(\borel{X}\).
\end{definicija}

\begin{opomba}
    Borelova \(\sigma\)-algebra je najmanjša \(\sigma\)-algebra, ki vsebuje vse odprte in vse zaprte podmnožice \(X\).
\end{opomba}

Zdaj lahko definiramo mero

\begin{definicija}
    \emph{Mera} na merljivem prostoru \((X, \mc{A})\) je funkcija 
    \[\mu: \mc{A} \to [0, \infty],\]
    ki zadošča naslednjima pogojema
    \begin{enumerate}
        \item \(\mu(\emptyset) = 0\) in
        \item \(\mu\left(\un{n \in \N}{A_n}\right) = \sum_{n=1}^{\infty} \mu(A_n)\) za vsako števno družino disjunktnih množic \(A_n \in \mc A\).
    \end{enumerate}
    Drugemu pogoju pravimo \emph{števna aditivnost}.
\end{definicija}

Izredno pomembno orodje za konstruiranje mer na splošnih množicah je pojem zunanje mere. V nadaljevanju bomo tako Lebesgueovo kot Hausdorffovo mero definirali s pomočjo njunih zunanjih mer.

\begin{definicija}
    \emph{Zunanja mera} na množici \(X\) je preslikava 
    \[\mu^*: \mc{P}(X) \to [0, \infty],\]
    ki zadošča naslednjim trem pogojem:
    \begin{enumerate}
        \item \(\mu^*(\emptyset) = 0\);
        \item \(\mu^*(A) \leq \mu^*(B)\), če je \(A \subseteq B\);
        \item \(\mu^*(\un{n \in \N}{A_n}) \leq \sum_{n=1}^{\infty} \mu^*(A_n)\) za vsako števno družino množic \(A_n \subseteq X\).
    \end{enumerate}

    Tretjemu pogoju pravimo \emph{števna subaditivnost}.
\end{definicija}

Navedemo trditev, ki nam bo koristila pri konstrukciji Hausdorffove mere. Dokaz trditve je mogoče najti v \cite[stran 20]{mb-otm}.
\begin{trditev}
    \label{zun-mera}
    Naj bo \(\mc{S}\) družina podmnožic množice \(X\), ki vsebuje \(\emptyset\) in \(X\). Naj bo \(\mu: \mc{S} \to [0, \infty]\) preslikava, za katero velja \(\mu(\emptyset) = 0\). Za vsako podmnožico \(A \subseteq X\) definiramo 
    \[\mu^*(A) = \inf \setb{\sum_{i=1}^{\infty}\mu(A_i)}{A_i \in \mc{S} \land A \subseteq \un{i \in \N}{A_i}}.\]
    Potem \(\mu^*\) je zunanja mera na \(X\).
\end{trditev}

\subsection{Lebesgueova mera}
Lebesgueova mera je posplošitev klasičnih pojmov, kot so dolžina, ploščina in prostornina, na mnogo širši razred množic. 

Natančna konstrukcija Lebesgueove mere presega okvir tega besedila in jo lahko najdemo v \cite[poglavje 1]{mb-otm}. Tukaj pa bomo predstavili osnovno idejo in rezultat, ki ga bomo uporabljali v nadaljevanju.

\begin{definicija}
    Za vsak interval \(I = (a, b) \subseteq \R\) definiramo \(l(I) := b -a\). \emph{Lebesgueova zunanja mera} je preslikava \(\mc{L}_*: \mc{P}(\R) \to [0, \infty]\) s predpisom 
    \[\mc{L}_*(A) = \inf \setb{\sum_{i=1}^{\infty} l(I_i)}{A \subseteq \un{i \in \N}{I_i}},\]
    kjer je \((I_i)_{i \in \N}\) števna družina odprtih intervalov v \(\R\).
\end{definicija}

To definicijo lahko naravno posplošimo tudi na višje dimenzije. Za vsak kvader \(K = I_1 \times \cdots \times I_n \subseteq \R^n\) z \(\vol(K) = l(I_1) \cdot \cdots \cdot l(I_n)\) definiramo njegovo prostornino.

\begin{definicija}
    Naj bo \(n \in \N\). \emph{Lebesgueova zunanja \(n\)-dimenzionalna mera} je preslikava \(\mc{L}_*^n: \mc{P}(\R^n) \to [0, \infty]\) s predpisom 
    \[\mc{L}_*^n(A) = \inf \setb{\sum_{i=1}^{\infty} \vol(K_i)}{A \subseteq \un{i \in \N}{K_i}},\]
    kjer je \((K_i)_{i \in \N}\) števna družina odprtih kvadrov v \(\R^n\).
\end{definicija}

\begin{trditev}
    Zožitev Lebesgueve zunanje \(n\)-dimenzionalne mere na Borelovo \(\sigma\)-algebro
    \[\mc{L}^n := \mc{L}_*^n|_{\mc{B}(\R^n)}\] je mera na merljivem prostoru \((\R^n,\, \mc{B}(\R^n))\).
\end{trditev}

\begin{zgled}
    Cantorjeva množica je merljiva, saj je zaprta. Zato lahko govorimo o njeni dolžini.

    Izkaže se, da je dolžina Cantorjeve množice enaka nič, torej \(\mc{L}^1(C) = 0\).
\end{zgled}

Cantorjeva množica nima dolžine, vendar pa vsebuje neštevno mnogo točk. Lahko se vprašamo: ali ji lahko pripišemo takšno smiselno dimenzijo, v kateri bo njena ">velikost"< končno, pozitivno število? In kaj nam ta dimenzija sploh pove o naravi Cantorjeve množice? S tem vprašanjem se bomo ukvarjali v nadaljevanju.

\section{Hausdorffova dimenzija}
Hausdorffova dimenzija je najstarejši in matematično najosnovnejši poskus formalizacije dimenzije, ki velja tudi za fraktalne množice. Njena moč izhaja iz dejstva, da temelji na zunanji meri, kar omogoča natančno in splošno definicijo, ki velja za katerokoli podmnožico evklidskega prostora.

Čeprav je Hausdorffova dimenzija pogosto težko izračunljiva ali celo numerično nedostopna, ostaja ključno orodje pri razumevanju in klasifikaciji fraktalnih struktur. V nadaljevanju bomo spoznali njen formalni zapis ter nekaj osnovnih primerov in geometričnih lastnosti. Pokazali bomo tudi, da Hausdorffova dimenzija ustreza našemu intuitivnemu pojmu ">dobre"< dimenzije, ki smo jo iskali na začetku.

\subsection{Konstrukcija Hausdorffove mere}
Začnemo z konstrukcijo Hausdorffove mere, ki nam bo omogočila natančno definicijo Hausdorffove dimenzije in pripomogla k boljšemu razumevanju geometrije takšnih množic.

Najprej definiramo osnovna pojma.

\begin{definicija}
    Naj bo \((X, d)\) metrični prostor. Zunanja mera \(\mu^*\) na \(X\) je \emph{metrična zunanja mera}, če 
    \[\mu^*(A \cup B) = \mu^*(A) + \mu^*(B),\]
    za vsaki množici \(A, B \subseteq X\), za kateri velja \(d(A, B) > 0\).
\end{definicija}

Dokaz naslednje trditve lahko najdemo v \cite[stran 349]{f-ra}.
\begin{trditev}
    \label{m-zun-mera}
    Naj bo \((X, d)\) metrični prostor. Če je \(\mu^*\) metrična zunanja mera na \(X\), potem je njena zožitev na Borelovo \(\sigma\)-algebro \(\mu = \mu^*|_{\mc{B(X)}}\) mera na merljivem prostoru \((X, \mc{B}(X))\).
\end{trditev}

\begin{opomba}
    \(d(A, B) := \inf\setb{d(a, b)}{a \in A,\ b \in B}\).
\end{opomba}



Zdaj lahko definiramo Hausdorffovo zunanjo mero.
\begin{definicija}
    \label{def-haus-mera}
    Naj bo \((X, d)\) metrični prostor, \(p \geq 0\) in \(\delta > 0\). Za vsako podmnožico \(A \subseteq X\) definiramo 
    \[H^p_\delta(A) = \inf \setb{\sum_{i=1}^{\infty}(\diam A_i)^p}{A_i \subseteq X \land A \subseteq \un{i \in \N}{A_i} \land \diam A_i \leq \delta}.\]
    Ker se množica možnih pokritij množice \(A\) z zmanjševanjem \(\delta\) zmanjšuje, je funkcija \(H^p_\delta(A)\) naraščajoča glede na \(\delta\). Zato obstaja limita
    \[\mc{H}^p = \lim_{\delta \to 0} H_\delta^p(A)\]
    (končna ali neskončna), ki ji pravimo \emph{\(p\)-dimenzionalna Hausdorffova zunanja mera množice \(A\)}.

    Pokritje množice \(A\) z množicami premera največ \(\delta\) imenujemo \emph{\(\delta\)-pokritje množice \(A\)}.
\end{definicija}

\begin{opomba} \ 
    \begin{itemize}
        \item \(\diam A = \sup \setb{d(x, y)}{x, y \in A}\)
        \item \(\inf \emptyset = \infty\)
    \end{itemize}
\end{opomba}

\begin{opomba}
    \label{odp-zap-haus}
    V definiciji so \(A_i \subseteq X\) poljubne. Enak rezultat lahko dobimo, če se omejimo le na zaprte podmnožice, saj velja \(\diam A_i = \diam \overline{A_i}\), ali pa le na odprte podmnožice, saj lahko vsako množico \(A_i\) nadomestimo z množico \(U_i = \setb{x \in X}{d(x, A_i) < \frac{\varepsilon}{2^{i+1}}}\), ki ima premer kvečjemu \((\diam A_i) + \frac{\varepsilon}{2^i}\).

    Tukaj je \(d(x, A_i) = \inf \setb{d(x, a)}{a \in A_i}\).
\end{opomba}

Dokažemo osnovno lastnost \(\mc{H}_p\).
\begin{trditev}
    \label{haus-m-zun}
    Naj bo \((X, d)\) metrični prostor. \(\mc{H}_p\) je metrična zunanja mera.
\end{trditev}

\begin{proof}
    Po trditvi \ref{zun-mera} sledi, da je \(H^p_\delta\) zunanja mera na \(X\). Ker je limita monotona, sledi tudi, da je \(\mc{H}^p\) zunanja mera na \(X\).

    Naj bo \(A, B \subseteq X\) množici, za kateri velja \(d(A, B) > 0\). Izberimo tak \(\delta > 0\), da velja \(\delta < d(A, B)\). Naj bo \((C_i)_{i \in \N}\) družina podmnožic \(X\), ki je \(\delta\)-pokritje množice \(A \cup B\).
    Ker je \(\diam C_i < d(A, B)\) za vsak \(i \in \N\), nobena množica \(C_i\) ne seka hkrati množice \(A\) in množice \(B\). Razdelimo vsoto \(\sum_{i=1}^{\infty} (\diam C_i)^p\) na dva dela:
    \begin{itemize}
        \item \(\displaystyle \sum_{C_i \cap B = \emptyset} (\diam C_i)^p\) in
        \item \(\displaystyle \sum_{C_i \cap A = \emptyset} (\diam C_i)^p\).
    \end{itemize}
    Po definiciji infimuma velja 
    \[\sum_{i=1}^{\infty} (\diam C_i)^p \geq H^p_\delta(A) + H^p_\delta(B).\]
    Ker je bilo pokritje \((C_i)_{i \in \N}\) poljubno, sledi, da \(H^p_\delta(A) + H^p_\delta(B)\) spodnja meja za množico dolžin vseh \(\delta\)-pokritij množice \(A \cup B\). Zato po definiciji infimuma velja, da
    \[H_\delta^p(A \cup B) \geq H^p_\delta(A) + H^p_\delta(B).\]
    V limiti \(\delta \to 0\) dobimo:
    \[\mc{H}^p(A \cup B) \geq \mc{H}^p(A) + \mc{H}^p(B).\]

    Po definiciji zunanje mere, ki je subaditivna, pa velja tudi:
    \[H^p(A \cup B) \leq H^p(A) + H^p(B).\]
    Iz obeh neenakosti sledi enakost, torej \(\mc{H}^p\) zadošča pogoju metrike.
\end{proof}

Direktna posledica trditev \ref{m-zun-mera} in \ref{haus-m-zun} je
\begin{posledica}
    Naj bo \((X, d)\) metrični prostor. Zožitev Hausdorffove zunanje \(p\)-dimenzionalne mere na Borelovo \(\sigma\)-algebro 
    \[\mc{H}^p := \mc{H}^p|_{\mc{B}(X)}\]
    je mera na merljivem prostoru \((X,\mc{B}(X))\).
\end{posledica}

Brez dokaza navedemo še trditev, ki povezuje \(\mc{L}^n\) in \(\mc{H}^p\). Dokaz lahko najdemo v \cite[stran 351]{f-ra}.
\begin{trditev}
    \label{haus-leb}
    Naj bo \(n \in \N\). Obstaja konstanta \(c_n > 0\), da je \(c_n \mc{H}^n\) Lebesgueova mera na merljivem prostoru \((\R^n, \mc{B}(\R^n))\).
\end{trditev}

\begin{opomba}
    Konstanta \(c_n\) je volumen \(n\)-dimenzionalne krogle, tj. 
    \[c_n = \frac{\pi^{n/2}}{2^n \Gamma(\frac{n}{2} + 1)}.\]
\end{opomba}

V nadaljevanju bomo za metrični prostor \((X, d)\)  privzeli navaden evklidski prostor \((\R^n, d_2)\).

\subsection{Lastnosti Hausdorffove mere}
V tem razdelku bomo našteli in dokazali nekatere geometrijske lastnosti Hausdorffove mere, ki jih lahko prenesemo tudi na Hausdorffovo dimenzijo.

\subsubsection{Lastnosti skaliranja}
Lastnosti skaliranja so temeljne za razumevanje fraktalnih struktur, saj fraktali po svoji naravi izkazujejo samopodobnost. 

Naj bo \(P: \R^n \to \R^n\) \emph{podobnostna preslikava} s podobnostnim koeficientom \(\lambda > 0\), tj.\ preslikava, za katero velja
\[\all{x, y \in \R^n} |P(x) - P(y)| = c|x - y|.\]

\begin{opomba}
    Opazimo, da je vsaka podobnostna preslikava injektivna.
\end{opomba}

Intuitivno je jasno, da če raztegnemo daljico \(\lambda\)-krat, se njena dolžina poveča \(\lambda\)-krat, torej 
\[\mc{L}^1(P(A)) = \lambda\mc{L}^1(A),\]
če povečamo stranico kvadrata za faktor \(\lambda\), se njegova ploščina poveča za faktor \(\lambda^2\), torej 
\[\mc{L}^2(P(A)) = \lambda^2\mc{L}^2(A)\]
in tako naprej za višje dimenzije.

Naravno se pojavi vprašanje: ali podobna lastnost velja tudi za Hausdorffovo mero? Odgovor nam da naslednja trditev.

\begin{trditev}
    \label{skale}
    Naj bo \(P: \R^n \to \R^n\) podobnostna preslikava s podobnostnim koeficientom \(\lambda > 0\) in \(A \subseteq \R^n\), \(p \geq 0\). Tedaj velja:
    \[\mc{H}^p(P(A)) = \lambda^p \mc{H}^p(A).\]
\end{trditev}

\begin{proof}
    Naj bo \((A_i)_{i \in \N}\) \(\delta\)-pokritje množice \(A\). Tedaj je \((P(A_i))_{i \in \N}\) \(\lambda \delta\)-pokritje množice \(P(A)\), saj velja \(\diam(P(A_i)) = \lambda \diam(A_i)\). Torej:
    \[\sum_{i=1}^{\infty} (\diam P(A_i))^p = \sum_{i=1}^{\infty}(\lambda \diam A_i)^p = \lambda^p \sum_{i=1}^{\infty}(\diam A_i)^p.\]
    Po definiciji infimuma velja 
    \[H^p_{\lambda \delta} (P(A)) \leq \lambda^p \sum_{i=1}^{\infty}(\diam A_i)^p.\]
    Ker je \((A_i)_{i \in \N}\) bilo poljubno \(\delta\)-pokritje množice \(A\), dobimo:
    \[H^p_{\lambda \delta} (P(A)) \leq \lambda^p H^p_\delta(A).\]
    V limiti \(\delta \to 0\) dobimo:
    \[\mc{H}^p (P(A)) \leq \lambda^p \mc{H}^p(A).\]    
    
    Za obratno neenakost uporabimo enak argument na preslikavi \(P^{-1}\), ki je tudi podobnostna preslikava s koeficientom \(1/\lambda\), in na množici \(P(A)\). Dobimo:
    \[
        \mc{H}^p(A) \leq (1/\lambda)^p \mc{H}^p(P(A)) \lthen \mc{H}^p(P(A)) \geq \lambda^p \mc{H}^p(A).
    \]
    Skupaj s prvo neenakostjo sledi enakost.
\end{proof}

Navedemo eno pomembno posledico

\begin{posledica}
    Naj bo \(P: \R^n \to \R^n\) izometrija, tj.\ preslikava, za katero velja
    \[\all{x, y \in \R^n} |P(x) - P(y)| = |x - y|\]
    in \(A \subseteq \R^n\), \(p \geq 0\). Tedaj velja:
    \[\mc{H}^p(P(A)) = \mc{H}^p(A).\]    
\end{posledica}

Primeri izometrij so rotacije, translacije, zrcaljenja ipd. Posledica pove, da je \(\mc{H}^p\) invariantna glede na rotacije, translacije in zrcaljenja, kar je pričakovano.

\subsubsection{Transformacijske lastnosti}
Še en zanimiv razred preslikav so \emph{H\"oldorjeve preslikave} stopnje \(\alpha > 0\), t.j.\ preslikave \(f: X \subseteq \R^m \to \R^n\), za katere velja
\[\some{c > 0} \all{x, y \in X} |f(x) - f(y)| \leq c|x-y|^\alpha.\]
V posebnem primeru, ko je \(\alpha = 1\), pravimo, da je preslikava \(f\) \emph{Lipschitzeva}.

Z istim argumentom kot prej lahko dokažemo naslednjo trditev
\begin{trditev}
    \label{mera-hold}
    Naj bo \(X \subseteq \R^m\), \(A \subseteq X\) in \(f: X \to \R^n\) Höldorjeva preslikava stopnje \(\alpha > 0\) s konstanto \(c > 0\). Potem za vsak \(p \geq 0\) velja:
    \[\mathcal{H}^{p/\alpha}(f(A)) \leq c^{p/\alpha} \mathcal{H}^p(F).\]
\end{trditev}

Direktna posledica te trditve je 
\begin{posledica}
    Naj bo \(X \subseteq \R^m\), \(A \subseteq X\) in \(f: X \to \R^n\) Lipschitzeva preslikava s konstanto \(c > 0\). Potem za vsak \(p \geq 0\) velja:
    \[\mathcal{H}^{p}(f(A)) \leq c^p \mathcal{H}^p(A).\]
\end{posledica}

Zdaj smo pripravili vsa potrebna orodja za definicijo Hausdorffove dimenzije in za preučevanje njenih lastnosti.

\subsection{Hausdorffova dimenzija}
V tem razdelku bomo definirali Hausdorffovo dimenzijo množice.

Naj bo \(A \subseteq \R^n\). Definiramo funkcijo \(\mc{H}_A: [0, \infty) \to [0, \infty]\) s predpisom 
\[\mc{H}_A(p) = \mc{H}^p(A).\]
\begin{lema}
    Naj bo \(A \subseteq \R^n\). Če je \(\mathcal{H}^{p}(A) < \infty\), potem je \(\mathcal{H}^{t}(A) = 0\) za vse \(t > p\).
\end{lema}

\begin{proof}
    Naj bo \(\varepsilon > 0\). Naj bo \(\delta > 0\) in \((A_i)_{i \in \N}\) \(\delta\)-pokritje množice \(A\). Tedaj po definiciji infimuma velja
    \[H^{p + \varepsilon}_\delta(A) \leq \sum_{i=1}^{\infty} (\diam A_i)^{p + \varepsilon} = \sum_{i=1}^{\infty} (\diam A_i)^{p} (\diam A_i)^{\varepsilon} \leq \delta^{\varepsilon} \sum_{i=1}^{\infty} (\diam A_i)^{p},\]
    torej 
    \[H^{p + \varepsilon}_\delta(A) \leq \delta^{\varepsilon} \sum_{i=1}^{\infty} (\diam A_i)^{p}.\]
    Ker je \((A_i)_{i \in \N}\) bilo poljubno \(\delta\)-pokritje množice \(A\), dobimo:
    \[H^{p + \varepsilon}_\delta(A) \leq \delta^{\varepsilon} H^{p}_\delta(A).\]
    Ker je \(\mathcal{H}^{p}(A) < \infty\), v limiti \(\delta \to 0\) dobimo:
    \[\mc{H}^{p + \varepsilon}(A) \leq 0.\]
    Ker je \(\mc{H}_A\) nenegativna funkcija, sledi
    \[\mc{H}^{p + \varepsilon}(A) = 0\]
    za vsak \(\varepsilon > 0\).
\end{proof}

Sedaj si lahko ogledamo graf funkcije \(\mc{H}_A\)
\begin{center}
    \drawGraf
\end{center}
Vidimo, da obstaja kritična točka \(p_0 \in [0, \infty)\), pri kateri vrednost funkcije \(\mc{H}^p\) ">skoči"< z \(\infty\) na \(0\). Zato definiramo

\begin{definicija}
    \emph{Hausdorffova dimenzija množice \(A \subseteq \R^n\)} je 
    \[\dim_H A = \inf \setb{p \geq 0}{\mathcal{H}^{p}(A) = 0} = \sup \setb{p \geq 0}{\mathcal{H}^{p}(A) = \infty}.\]
\end{definicija}

\begin{opomba} \ 
    \begin{itemize}
        \item Po dogovoru je \(\sup \emptyset = 0\).
        \item Hausdorffova dimenzija je definirana za poljubno množico \(A \subseteq \R^n\), saj je opredeljena s pomočjo Hausdorffove zunanje mere.
    \end{itemize}    
\end{opomba}

\begin{opomba}
    Naj bo \(A \subseteq \R^n\). Potem velja:
        \[
            \mathcal{H}^{p}(A) = \begin{cases}
                \infty; &0 \leq p < \dim_H A \\ 
                0; &p > \dim_H A,
            \end{cases}
        \]        
        pri \(p = \dim_H A\) pa je \(\mc{H}^p(A)\) lahko katerakoli vrednost iz intervala \([0, \infty]\).
\end{opomba}

\begin{zgled}
    \label{dim-ball}
    Naj bo \(B^n = B(x, r) = \setb{y \in \R^n}{d(x, y) < r}\) odprta \(n\)-dim krogla s središčem v \(x \in \R^n\) in polmerom \(r > 0\). Po trditvi \ref{haus-leb} sledi, da velja
    \[\mc{H}^n(B^n) = \frac{1}{c_n} \mc{L}^n(B^n) = \frac{1}{c_n} \vol(B^n).\]
    Torej je 
    \[0 < \mc{H}^n(B^n) < \infty\]
    in 
    \[\dim_H B^n = n.\]
    Isti sklep velja tudi za zaprto kroglo \(\overline{B}^{\, n} \subseteq \R^n\), tj.
    \[\dim_H \overline{B}^{\, n} = n.\]
\end{zgled}

\begin{opomba}
    Obstajajo tudi druge ekvivalentne definicije Hausdorffove dimenzije, ki jih lahko najdemo v \cite{fk-fg}.
\end{opomba}

\subsection{Lastnosti Hausdorffove dimenzije}
Sedaj, ko smo definirali Hausdorffovo dimenzijo, lahko nadaljujemo z obravnavo njenih pomembnih geometričnih in topoloških lastnosti ter vpliva na strukturo množic.

\subsubsection{Splošne lastnosti}
Naslednja trditev je direktna posledica monotonosti Hausdorffove zunanje mere 

\begin{trditev}[Monotonost]
    \label{haus-monotonost}
    Hausdorffova dimenzija je monotona, tj.
    \[\all{A, B \subseteq \R^n} A \subseteq B \lthen \dim_H A \leq \dim_H B\]
\end{trditev}

\begin{trditev}[Števna stabilnost]
    \label{haus-stab}
    Naj bo \((A_i)_{i \in \N}\) družina podmnožic \(\R^n\). Tedaj velja 
    \[\dim_H \un{i \in \N}{A_i} = \sup \setb{\dim_H A_i}{i \in \N}.\]
\end{trditev}

\begin{proof}
    Označimo z \(s = \sup \setb{\dim_H A_i}{i \in \N}\). 
    
    Ker je \(A_i \subseteq \un{i \in \N}{A_i}\) za vsak \(i \in \N\) in je po trditvi \ref{haus-monotonost} Hausdorffova dimenzija monotona, sledi, da
    \[\dim_H F_i \leq \dim_H \un{i \in \N}{A_i}\]
    za vsak \(i \in \N\). Po definiciji supremuma velja:
    \[s \leq \dim_H \un{i \in \N}{A_i}.\]

    Naj bo \(\varepsilon > 0\). Po definiciji supremuma velja, da \(\dim_H A_i < s + \epsilon\) za vsak \(i \in \N\). Torej 
    \[\mc{H}^{s + \varepsilon}(A_i) = 0\]
    za vsak \(i \in \N\). Ker je Hausdorffova zunanja mera subaditivna, sledi, da
    \[\mc{H}^{s + \varepsilon}\left(\un{i \in \N}{A_i}\right) \leq \sum_{i=1}^{\infty} \mc{H}^{s + \varepsilon} (A_i) = 0.\]
    V limiti \(\varepsilon \to 0\) dobimo:
    \[\mc{H}^s\left(\un{i \in \N}{A_i}\right) \leq 0.\]
    Torej 
    \[\dim_H \un{i \in \N}{A_i} \leq s.\]    
    Skupaj s prvo neenakostjo sledi enakost.
\end{proof}

\begin{posledica}
    \label{dim-rn}
    Hausdorffova dimenzija \(\R^n\) je 
    \[\dim_H \R^n = n.\]
\end{posledica}

\begin{proof}
    Ker je prostor \(\R^n\) 2-števen, obstaja števno pokritje \((B^n_i)_{i \in \N}\) množice \(\R^n\) z odprtimi krogli. Po trditvi \ref{haus-stab} in zgledu \ref{dim-ball} sledi, da
    \[\dim_H \R^n \leq n.\]
    Ker pa velja \(B^n(0, 1) \subseteq \R^n\) in je po trditvi \ref{haus-monotonost} Hausdorffova dimenzija monotona, sledi tudi obratna neenakost:
    \[
        \dim_H \R^n \geq \dim_H B^n(0, 1) = n.
    \]
    Torej je \(\dim_H \R^n = n\).
\end{proof}

\begin{trditev}[Dimenzija števnih množic]
    Naj bo \(A \subseteq \R^n\). Če je \(A\) števna, potem 
    \[\dim_H A = 0.\]
\end{trditev}

\begin{proof}
    Ker je \(A\) števna množica, lahko jo zapišemo v obliki 
    \[A = \set{a_1, a_2, a_3, \ldots}.\]
    Definiramo \(A_i = \set{a_i}\) za vsak \(i \in \N\). Tedaj velja
    \[A = \un{i \in \N}{A_i}.\]
    Ker je \(0 < \mc{H}^0(A_i) = 1 < \infty\) sledi, da 
    \[\dim_H A_i = 0\]
    za vsak \(i \in \N\).
    Po trditvi \ref{haus-stab} velja
    \[\dim_H A = \dim_H \un{i \in \N}{A_i} = \sup \setb{\dim_H A_i}{i \in \N} = 0.\]
\end{proof}

\begin{zgled}
    Hausdorffova dimenzija \(\Q \subseteq \R\) je \(0\), saj je \(\Q\) števna množica. 
\end{zgled}

\begin{trditev}[Dimenzija odprtih množic]
    \label{haus-odp}
    Naj bo \(U \subseteq \R^n\). Če je \(U\) odprta in neprazna, potem 
    \[\dim_H U = n.\]
\end{trditev}

\begin{proof}
    Ker je \(U\) odprta in neprazna, vsebuje neko odprto \(n\)-dimenzionalno kroglo, po drugi strani pa je vsebovana v \(\R^n\). Trditev sledi.
\end{proof}

Ideje dokaza naslednje trditve lahko najdemo v \cite[stran 32]{fk-fg} in \cite[stran 351]{f-ra}.
\begin{trditev}[Dimenzija gladkih množic]
    Naj bo \(M \subseteq \R^n\). Če je \(M\) gladka (tj.\ \(C^\infty\)) podmnogoterost dimenzije \(m\), potem 
    \[\dim_H M = m.\]
    Posebej 
    \begin{itemize}
        \item Če je \(M\) gladka krivulja, potem \(\dim_H M = 1.\)
        \item Če je \(M\) gladka ploskev, potem \(\dim_H M = 2.\)
    \end{itemize}
\end{trditev}

\subsubsection{Transformacijske lastnosti}
Pri proučevanju Hausdorffove dimenzije se naravno zastavi vprašanje, kako se ta obnaša pri različnih transformacijah. Na primer: ali se dimenzija ohranja pri Lipschitzevih ali gladkih preslikavah? Takšna vprašanja so pomembna, saj nam transformacijske lastnosti pogosto omogočajo, da iz znane dimenzije neke množice sklepamo o dimenziji njene slike.

\begin{trditev}
    Naj bo \(X \subseteq \R^m\), \(A \subseteq X\) in \(f: X \to \R^n\) H\"oldorjeva preslikava stopnje \(\alpha > 0\) s konstanto \(c > 0\). Tedaj velja
    \[\dim_H f(A) \leq \frac{1}{\alpha} \dim_H A.\]
\end{trditev}

\begin{proof}
    Označimo z \(s:= \dim_H A\). Naj bo \(\varepsilon > 0\). Po trditvi \ref{mera-hold} velja
    \[\mc{H}^{(s +\varepsilon) / \alpha}(f(A)) \leq c^{(s +\varepsilon) / \alpha} \mc{H}^{s + \varepsilon}(A) = 0.\]
    V limiti \(\varepsilon \to 0\) dobimo
    \[\mc{H}^{s / \alpha}(f(A)) \leq 0,\]
    torej
    \[\dim_H f(A) \leq \frac{1}{\alpha} \dim_H A.\]
\end{proof}

V primeru Lipschitzevih preslikav (kjer je \(\alpha = 1\)) dobimo naslednjo pomembno posledico.

\begin{posledica}
    \label{dim-lips}
    Naj bo \(X \subseteq \R^m\), \(A \subseteq X\) in \(f: X \to \R^n\) Lipschitzeva preslikava s konstanto \(c > 0\). Tedaj velja
    \[\dim_H f(A) \leq \dim_H A.\]
\end{posledica}

Poseben primer Lipschitzevih preslikav so bi-Lipschitzeve preslikave. V tem primeru se Hausdorffova dimenzija ne spremeni -- to je povsem pričakovano, saj bi-Lipschitzeva preslikava ">niti ne sesuje"< niti ">neskončno ne raztegne"< razdalj: ohrani geometrijske razdalje ">v razumnem obsegu"<.

\begin{izrek}
    Naj bo \(X \subseteq \R^m\), \(A \subseteq X\) in \(f: X \to \R^n\) bi-Lipschitzeva preslikava, tj.\ preslikava \(f: X \to \R^n\), za katero velja:
    \[\some{c_1, c_2 > 0} \all{x, y \in X} c_1|x-y| \leq |f(x) - f(y)| \leq c_2|x-y|.\]
    Tedaj velja
    \[\dim_H f(A) = \dim_H A.\]
\end{izrek}

\begin{proof}
    Ker je \(f\) injektivna, je njena zožitev \(f: X \to f(X)\) bijektivna z Lipschitzevim inverzom \(f^{-1}: f(X) \to X\) s konstanto \(1 / c_1\). Po posledici~\ref{dim-lips} imamo
    \[\dim_H f(A) \leq \dim_H A\]
    in 
    \[\dim_H A = \dim_H f^{-1}f(A) \leq \dim_H f(A).\]
    Skupaj s prvo neenakostjo sledi enakost.
\end{proof}

Izrek pove, da je Hausdorffova dimenzija invariantna glede na bi-Lipschit\-zeve preslikave. To pomeni, da so bi-Lipschitzeve transformacije ">dimenzijsko ohranjajoče"<. Če imata dve množici različno Hausdorffovo dimenzijo, potem med njima ne more obstajati bi-Lipschitzeva preslikava.

\begin{opomba}
    Podoben princip poznamo tudi iz topologije: če imata prostora različni topološki invarianti (npr.\ različno število luknj), potem med njima ne more obstajati homeomorfizem. Hausdorffova dimenzija je torej v tem smislu geometrična in varna pred ">zmerno deformacijo"<.
\end{opomba}

\subsubsection{Topološke lastnosti}
Hausdorffova dimenzija ima tudi zanimivo povezavo s topologijo. Naslednji rezultat kaže, da dovolj ">majhne"< množice (v smislu dimenzije) ne morejo biti povezane.

\begin{trditev}
    \label{povezanost}
    Naj bo \(A \subseteq \R^n\). Če je \(\dim_H A < 1\), potem je \(A\) popolnoma nepovezana, torej so njene komponente za povezanost enojci.
\end{trditev}

\begin{proof}
    Naj bosta \(x, y \in A, \ x \neq y\). Definiramo preslikavo \(f: A \to [0, \infty)\) s predpisom
    \[f(z) = |z-x|.\]
    Preslikava \(f\) je Lipschitzeva, saj za vse \(z, w \in A\) velja:
    \[
        |f(z) - f(w)| = ||z - x| - |w - x|| \leq |z - w|.
    \]
    Po posledici~\ref{dim-lips} sledi:
    \[
        \dim_H f(A) \leq \dim_H A < 1.
    \]
    Znano pa je (trditev \ref{haus-odp}), da vsaka neprazna odprta množica v \(\R\) ima Hausdorffovo dimenzijo \(1\), zato \(f(A)\) ne vsebuje nobene neprazne odprte podmnožice, saj je \(\dim_H\) monotona. Torej je \(\comp{f(A)}\) gosta v \(\R\).

    Ker je \(0 < f(y)\), obstaja \(r \in (0, f(y))\), da \(r \notin f(A)\). Definiramo množici:
    \[
        U := \{ z \in A \mid f(z) < r \}, \quad V := \{ z \in A \mid f(z) > r \}.
    \]
    Množici \(U\) in \(V\) sta odprti v podprostoru \(A\) (kot prasliki odprtih množic v \([0, \infty)\), disjunktni in pokrivata cel \(A\), saj \(r \notin f(A)\). Poleg tega velja \(x \in U\) (ker je \(f(x) = 0 < r\)) in \(y \in V\) (ker je \(f(y) > r\)). Torej \(x\) in \(y\) ležita v različnih komponentah za povezanost množice \(A\).

    Ker sta bili \(x\) in \(y\) poljubno izbrani različni točki, sledi, da je vsaka komponenta množice \(A\) enojec.
\end{proof}

Ta rezultat pokaže, kako lahko Hausdorffova dimenzija vpliva na topološke lastnosti množic. Če je dimenzija strogo manjša od \(1\), je množica tako »tanka«, da je ni mogoče povezati niti z najkrajšo zvezno potjo. 

\subsection{Primeri računanja Hausdorffove dimenzije}
V tem razdelku bomo obravnavali primere izračuna Hausdorffove dimenzije za bolj zapletene množice. Običajno postopamo tako, da z geometrijskim premislekom podamo spodnjo in zgornjo oceno za dimenzijo ter upamo, da se ti meji ujemata. V tem primeru lahko sklepamo, da smo našli točno dimenzijo.

% \subsubsection{Cantorjev prah}
% Izračunamo dimenzijo Cantorjevega praha (slika \ref{fig:cantor-dust}). Cantorjev prah je fraktal, ki ga dobimo na naslednji način:

% \begin{enumerate}
%     \item Naj bo \(E_0\) kvadrat s stranico dolžine \(1\).
%     \item Kvadrat \(E_0\) razdelimo na \(16\) enakih manjših kvadratov tako, da vsako stranico razdelimo na 4 enake dele.
%     \item V vsaki vrstici izberemo en kvadrat (po določenem pravilu), ostale kvadrate pa odstranimo. Tako dobimo množico \(E_1\), ki je unija 4 kvadratov z dolžino stranice \(4^{-1}\).
%     \item Recimo, da imamo množico \(E_n\), ki jo sestavlja \(4^n\) kvadratov z dolžino stranice \(4^{-n}\). Za generacijo \(E_{n+1}\) vsak kvadrat iz \(E_n\) razdelimo na 16 manjših kvadratov in iz vsake vrstice izberemo po en kvadrat po istem pravilu kot prej, ostale odstranimo.
%     \item Postopek nadaljujemo induktivno.
% \end{enumerate}

% \begin{figure}[ht]
%     \centering
%     \includegraphics[width=\textwidth]{img/cantor_dust.pdf}
%     \caption{Cantorjev prah (Cantor dust).}
%     \label{fig:cantor-dust}
% \end{figure}

% \begin{definicija}
%     \emph{Cantorjev prah} je množica
%     \[
%     E = \bigcap_{i \in \N_0} E_i.
%     \]
% \end{definicija}

% \begin{trditev}
%     Hausdorffova dimenzija Cantorjeva praha je
%     \[\dim_H E = 1.\]
% \end{trditev}

% \begin{proof}
%     Naj bo \(E_k\) \(k\)-ta generacija konstrukcije. Tedaj \(E_k\) vsebuje \(4^k\) kvadratov s stranico \(4^{-k}\) in premerom največ \(4^{-k} \sqrt{2}\).

%     Naj bo \(\delta > 0\). Potem obstaja \(k \in \N\), da velja \(4^{-k} \sqrt{2} \leq \delta\). Kvadrati iz \(E_k\) tvorijo \(\delta\)-pokritje množice \(E\), zato dobimo oceno
%     \[H^1_\delta \leq 4^k 4^{-k}\sqrt{2} = \sqrt{2}.\]
%     V limiti \(\delta \to 0\) sledi
%     \[\mc{H}^1(E) \leq \sqrt{2},\]
%     torej 
%     \[\dim_H E \leq 1.\]

%     Naj bo zdaj \(\pi_x: E \to [0,1]\) projekcija na \(x\)-os, podana s predpisom
%     \[
%         \pi_x(x, y) = x.
%     \]
%     Po konstrukciji množice \(E\) je \(\pi_x(E) = [0,1]\), torej je preslikava surjektivna. Poleg tega je \(\pi_x\) Lipschitzeva, saj
%     \begin{multline*}
%         |(x_1, y_1) - (x_2, y_2)| = |(x_1 - x_2, y_1 - y_2)| = \sqrt{(x_1 - x_2)^2 + (y_1 - y_2)^2} \geq \\ \sqrt{(x_1 - x_2)^2} = |x_1 - x_2| = |\pi_x(x_1, y_1) - \pi_x(x_2, y_2)|.
%     \end{multline*}
%     Po posledici \ref{dim-lips} sledi
%     \[1 = \dim_H \pi_x (E) \leq \dim_H E.\]
%     Skupaj s prvo neenakostjo sledi enakost.
% \end{proof}

\subsubsection{Cantorjeva množica}
Izračunamo dimenzijo Cantorjeve množice (slika \ref{fig:cantor-set}).

\begin{trditev}
    Hausdorffova dimenzija Cantorjeve množice je
    \[\dim_H C = \log_3 2.\]
\end{trditev}

\begin{proof}    
    Naj bo \(C_k\) \(k\)-ta generacija konstrukcije. Tedaj \(C_k\) vsebuje \(2^k\) intervalov dolžine \(3^{-k}\). Označimo \(s = \log_3 2\).

    Naj bo \(\delta > 0\). Potem obstaja \(k \in \N\), da velja \(3^{-k} \leq \delta\). Intervali iz \(C_k\) tvorijo \(\delta\)-pokritje množice \(C\), zato dobimo oceno
    \[H^s_\delta(C) \leq 2^k \cdot 3^{-ks} = 1.\]
    V limiti \(\delta \to 0\) sledi
    \[\mc{H}^s(C) \leq 1,\]
    torej 
    \[\dim_H C \leq s.\]

    Naj bo \(\delta > 0\). Naj bo \((U_i)_{i \in \N}\) poljubno \(\delta\)-pokritje množice \(C\). Po opombi~\ref{odp-zap-haus} lahko brez škode za splošnost predpostavimo, da so \(U_i\) odprte množice. Ker je \(C\) kompaktna (tj.\ omejena in zaprta), obstaja končno podpokritje 
    \[U_1, U_2, \ldots, U_N.\]
    Za vsak \(i \in \set{1, 2, \ldots, N}\) izberimo \(k \in \N\), da velja
    \[3^{-(k + 1)} \leq \diam U_i < 3^{-k}. \tag{1}\]
    Tedaj \(U_i\) seka kvečjemu en interval v generaciji \(C_k\), saj je razdalja med dvema sosednjima intervaloma v tej generaciji vsaj \(3^{-k}\).

    Če je \(j \geq k\), potem po konstrukciji Cantorjeve množice \(U_i\) lahko seka kvečjemu \(2^{j - k}\) intervalov generacije \(C_j\), saj se pri vsakem koraku \(C_n \to C_{n+1}\) vsak interval razdeli na dva.

    Iz (1) sledi
    \[2^{j - k} = 2^k \cdot 3^{-sk} = 2^j \cdot 3^s \cdot (3^{-(k+1)})^s \leq 2^j \cdot 3^s \cdot (\diam U_i)^s. \tag{2}\]
    Torej vsak \(U_i\) seka kvečjemu \(2^j \cdot 3^s \cdot (\diam U_i)^s\) intervalov generacije \(C_j\).

    Če izberemo dovolj velik \(j \in \N\), da za vsak \(i \in \set{1, 2, \ldots, N}\) velja
    \[3^{-(j+1)} \leq \diam U_i,\]
    potem pokritje \((U_i)_{i \in \N}\) seka vsak izmed \(2^j\) intervalov v generaciji \(C_j\). S preštevanjem teh intervalov in uporabo (2) dobimo oceno
    \[2^j \leq \sum_{i=1}^{N} 2^j \cdot 3^s \cdot (\diam U_i)^s,\]
    od koder sledi
    \[\frac{1}{3^s} \leq \sum_{i=1}^{N} (\diam U_i)^s \leq \sum_{i=1}^{\infty} (\diam U_i)^s.\]
    Ker je bilo pokritje \((U_i)_{i \in \N}\) poljubno, po definiciji infimuma sledi, da 
    \[\frac{1}{3^s} \leq H^s_\delta(C).\]
    V limiti \(\delta \to 0\) dobimo
    \[\mc{H}^s(C) \geq \frac{1}{3^s} = \frac{1}{2},\]
    torej 
    \[\dim_H C \geq s.\]

    Skupaj s prvo neenakostjo sledi enakost.
\end{proof}

Iz zgornjega računa in topoloških lastnosti Hausdorffove dimenzije sledi naslednje:
\begin{posledica}
    Cantorjeva množica \(C\) je popolnoma nepovezana.
\end{posledica}

\begin{opomba}
    Iz zgornjega računa vidimo, da ima Cantorjeva množica res neničelno \(\log_3 2\)-dimenzionalno mero (oz.\ velikost). Z natančnejšo spodnjo oceno se lahko pokaže, da velja
    \[
        \mc{H}^s(C) = 1.
    \]
    Tak natančen izračun najdemo v \cite{pearse2014}.

    Če si predstavljamo, da ima Cantorjeva množica maso 1 kg, potem zaradi lastnosti skaliranja Hausdorffove mere (trditev \ref{skale}), če bi množico raztegnili dvakrat -- torej začeli z intervalom \([0,2]\) namesto \([0,1]\) — bi se njena masa povečala za faktor \(2^{\log_3 2}\) in bi bila približno enaka \(1,59\) kg.
\end{opomba}

Vidimo, da je izračun Hausdorffove dimenzije pogosto izredno zahteven, celo za preproste množice. Najprej je treba s pomočjo geometrijskih opažanj, simetrije ali samopodobnosti uganiti pravo vrednost dimenzije, nato pa to vrednost še dokazati. Ta postopek zahteva natančno analizo in pogosto precej tehničnega znanja.

Zato se naravno pojavi vprašanje: katere alternativne definicije dimenzije so nam na voljo?


% \section{Škatlasta dimenzija}
Osnovna ideja vseh definicij dimenzije temelji na principu ">meritve pri skali \(\delta > 0\)"<:
za dano pozitivno vrednost \(\delta\) množico merimo tako, da prezremo podrobnosti, manjše od \(\delta\), in nato opazujemo vedenje, ko \(\delta \to 0\).
Ali je neka definicija smiselna in uporabna, pokažeta praksa in matematična intuicija.
Pričakujemo pa tudi, da bodo za tako definirano dimenzijo večinoma veljale naravne lastnosti, omenjene v uvodu.

Škatlasta dimenzija je ena najbolj priljubljenih, saj jo lahko pogosto preprosto izračunamo ali vsaj numerično ocenimo. Kljub temu pa ima tudi svoje pomanjkljivosti, o katerih bomo govorili kasneje.

\cite{matrika-fraktalna-dimenzija} Idejo za definicijo škatlaste dimenzije dobimo iz naslednjega razmisleka.

Naj bo \(X \subseteq \R^n\). Prostor \(\R^n\) razdelimo na enakomerno mrežo kock s stranico \(\delta > 0\), in nato preštejemo število tistih kock, ki sekajo množico \(X\). Pričakujemo, da bo število teh kock, označeno z \(N_\delta(X)\), v odvisnosti od \(\delta\), odražalo ">velikost"< množice \(X\).
Na primer:
\begin{itemize}
    \item Enotski interval \(K_1 = [0,1]\) lahko pokrijemo s \(\frac{1}{\delta}\) intervali dolžine \(\delta\).
    \item Enotski kvadrat \(K_2 = [0,1]^2\) potrebujemo \(\frac{1}{\delta^2}\) kvadratov stranice \(\delta\).
    \item Enotsko kocko \(K_3 = [0,1]^3\) pokrijemo z \(\frac{1}{\delta^3}\) kockami s stranico \(\delta\).
\end{itemize}
Opazimo torej, da za množico dimenzije \(D\) približno velja:
\[
N_\delta(X) \approx \frac{1}{\delta^D}.
\]
Logaritmiranje te zveze pripelje do formule:
\[
D \approx \frac{\log N_\delta(X)}{\log(1/\delta)}.
\]

To razmerje med številom potrebnih kock in njihovo velikostjo vodi k definiciji škatlaste dimenzije:

\begin{definicija}
    Naj bo \(A \subseteq \R^n\) omejena in neprazna množica. Naj bo \(\delta > 0\). Označimo z \(N_\delta(A)\) najmanjše število množic z diametrom kvečjemu \(\delta\), ki pokrivajo množico \(A\).
    \begin{itemize}
        \item \emph{Spodnja škatlasta dimenzija množice \(A\)} je 
        \[
        \underline{\dim}_B A = \liminf_{\delta \to 0} \frac{\log N_\delta(A)}{\log(1/\delta)}.
        \]
        \item \emph{Zgornja škatlasta dimenzija množice \(A\)} je 
        \[
        \overline{\dim}_B A = \limsup_{\delta \to 0} \frac{\log N_\delta(A)}{\log(1/\delta)}.
        \]        
    \end{itemize}
    Če velja \(\underline{\dim}_B A = \overline{\dim}_B A\), potem se skupna vrednost
    \[
    \dim_B A = \lim_{\delta \to 0} \frac{\log N_\delta(A)}{\log(1/\delta)}
    \] 
    imenuje \emph{škatlasta dimenzija množice \(A\)}.
\end{definicija}

\begin{opomba} \ 
    \begin{itemize}
        \item Predpostavljamo, da je \(\delta > 0\) dovolj majhen, da je \(\log(1/\delta) > 0\).
        \item Da se izognemo težavam s \(\log \infty\) ali \(\log 0\), definiramo dimenzijo le za neprazne in omejene množice.
    \end{itemize}
\end{opomba}

\subsection{Ekvivalentne definicije škatlaste dimenzije}
V tem razdelku bomo predstavili nekatere ekvivalentne definicije škatlaste dimenzije, ki so pogosto bolj uporabne in prijazne za praktično delo.

Naj bo 
\[\mc{K} = \setb{[m_1\delta, (m_1 + 1)\delta] \times \cdots \times [m_n\delta, (m_n + 1)\delta]}{m_1, \ldots, m_n \in \Z}\]
množica kvadrov v \(\delta\)-mreži prostora \(\R^n\). 

\begin{trditev}
    Naj bo \(A \subseteq \R^n\) omejena in neprazna. Označimo z \(N_\delta'(A)\) število elementov množice \(\mc{K}\), ki sekajo \(A\). Definirajmo
    \[
    \dim_{B'} A = \lim_{\delta \to 0} \frac{\log N_\delta'(A)}{\log(1/\delta)}.
    \]
    Tedaj velja
    \[
    \dim_B A = \dim_{B'} A.
    \] 
\end{trditev}

\begin{proof}
    Naj bo \(0 < \delta < 1\). Množica kvadrov, ki sekajo množico \(A\), določa pokritje \(A\) s kockami premera največ \(\delta \sqrt{n}\), zato velja
    \[N_{\delta\sqrt{n}}(A) \leq N_\delta'(A) \lthen \log N_{\delta\sqrt{n}}(A) \leq \log N_\delta'(A).\]
    Ker je \(\log(1/\delta) > 0\), dobimo
    \[
        \lim_{\delta \to 0} \frac{\log N_{\delta\sqrt{n}}(A)}{\log(1/\delta)} \leq \lim_{\delta \to 0} \frac{\log N_\delta'(A)}{\log(1/\delta)},
    \]
    kar pomeni
    \[\dim_B A \leq \dim_{B'} A\]

    Po drugi strani, naj bo \((U_i)_{i \in \N}\) poljubno \(\delta\)-pokritje množice \(A\). Vsako množico \(U_i\) lahko pokrijemo z največ \(3^n\) kvadri iz \(\delta\)-mreže (ker je \(U_i\) premera največ \(\delta\), zadostuje razširitev po eno kocko v vsako smer). Torej velja
    \[
    N_\delta'(A) \leq 3^n N_\delta (A) \lthen \log N_\delta'(A) \leq n \log 3 + \log N_\delta (A).
    \]
    Sledi
    \[
        \lim_{\delta \to 0} \frac{\log N_\delta'(A)}{\log(1/\delta)} \leq \lim_{\delta \to 0} \frac{n \log 3 + \log N_\delta (A)}{\log(1/\delta)} = \lim_{\delta \to 0} \frac{\log N_\delta (A)}{\log(1/\delta)},
    \]
    torej
    \[\dim_B A \geq \dim_{B'} A.\]

    Skupaj s prvo neenakostjo sledi enakost.
\end{proof}

Ta verzija definicije je zelo uporabna, saj lahko narišemo mrežo in preštejemo število kock, ki sekajo množico. Zdaj je tudi jasno, zakaj se ta dimenzija imenuje škatlasta.

Podobno lahko dokažemo, da za \(N_\delta(A)\) lahko vzamemo najmanjše število poljubnih kvadrov z stranico dolžine \(\delta\), ki pokrivajo množico \(A\) ali najmanjše število zaprtih krogel z radijem \(\delta\), ki pokrivajo množico \(A\).

Ni povsem očitno, da lahko količino \(N_\delta(A)\) izrazimo tudi kot največje število paroma disjunktnih zaprtih krogel z radijem \(\delta\), katerih središča ležijo v množici \(A\). To dejstvo bomo utemeljili v naslednji trditvi.

\begin{trditev}
    Naj bo \(A \subseteq \R^n\) omejena in neprazna. Označimo z \(N_\delta'(A)\) največje število paroma disjunktnih zaprtih krogel z radijem \(\delta\), katerih središča ležijo v množici \(A\). Definirajmo
    \[
    \dim_{B'} A = \lim_{\delta \to 0} \frac{\log N_\delta'(A)}{\log(1/\delta)}.
    \]
    Tedaj velja
    \[
    \dim_B A = \dim_{B'} A.
    \] 
\end{trditev}

\begin{proof}
    Naj bo \(0 < \delta < 1\). Naj bo \(B_1, \ldots B_{N_\delta'(A)}\) družina paroma disjunktnih zaprtih krogel z radijem \(\delta\) in središči v množici \(A\). Trdimo, da množica krogel \(B_i'\), ki imajo ista središča kot \(B_i\), vendar radij \(2\delta\), tvori \(4\delta\)-pokritje množice \(A\). Namreč, če bi obstajala točka \(x \in A\), ki ni vsebovana v nobeni izmed krogel \(B_i'\), potem bi lahko okoli \(x\) postavili novo disjunktno kroglo z radijem \(\delta\) in s tem povečali družino \(B_i\), kar bi bilo v nasprotju z maksimalnostjo \(N_\delta'(A)\). Torej:
    \[
    N_{4\delta}(A) \leq N_\delta'(A).
    \]

    Po drugi strani, naj bo \((U_i)_{i \in \N}\) poljubno \(\delta\)-pokritje množice \(A\). Vsaka od disjunktnih krogel \(B_i\) mora vsebovati vsaj eno množico \(U_j\), saj pokritje pokriva središča teh krogel, in ker so krogle disjunktne, so to tudi pripadajoče množice \(U_j\). Torej velja:
    \[
    N_\delta'(A) \leq N_\delta(A).
    \]

    S tem smo dobili neenakosti:
    \[
    N_{4\delta}(A) \leq N_\delta'(A) \leq N_\delta(A).
    \]
    V limiti dobimo želeno enakost.
\end{proof}


Povzemimo
\begin{definicija}
    Naj bo \(A \subseteq \R^n\).
    \emph{Spodnja in zgornja škatlasta dimenzija množice \(A\)} podani kot 
    \[
    \underline{\dim}_B A = \liminf_{\delta \to 0} \frac{\log N_\delta(A)}{\log(1/\delta)}
    \]
    in 
    \[
    \overline{\dim}_B A = \limsup_{\delta \to 0} \frac{\log N_\delta(A)}{\log(1/\delta)}.
    \]     
    \emph{Škatlasta dimenzija množice \(A\)} obstaja, če obstaja limita
    \[
    \dim_B A = \lim_{\delta \to 0} \frac{\log N_\delta(A)}{\log(1/\delta)},
    \] 
    kjer je \(N_\delta(A)\) lahko katera koli izmed naslednjih količin:
    \begin{itemize}
        \item najmanjše število zaprtih krogel z radijem \(\delta\), ki pokrivajo množico \(A\);
        \item najmanjše število kvadrov s stranico \(\delta\), ki pokrivajo množico \(A\);
        \item število kvadrov iz \(\delta\)-mreže, ki sekajo množico \(A\);
        \item najmanjše število množic s premerom največ \(\delta\), ki pokrivajo množico \(A\);
        \item največje število paroma disjunktnih zaprtih krogel z radijem \(\delta\), katerih središča ležijo v množici \(A\).
    \end{itemize}
\end{definicija}

Ta seznam bi lahko še nadaljevali. V praksi pa izberemo tisto definicijo, ki je najbolj primerna za dano množico.

\begin{zgled}
    Škatlasta dimenzija Cantorjeve množice \(C\) je 
    \[\dim_B C = \log_3 2.\]
    Podroben izračun lahko najdemo v \cite[stran 47]{fk-fg}.
\end{zgled}

\subsection{Relacija med Hausdorffovo in škatlasto dimenzijo}
Pomembno je razumeti relacijo med Hausdorffovo in škatlasto dimenzijo.

Naj bo \(A \subseteq \R^n\) omejena in neprazna množica. Če lahko množico \(A\) pokrijemo z \(N_\delta(A)\) množicami s premerom kvečjemu \(\delta\), potem po definiciji \ref{def-haus-mera} velja
\[
    H^s_\delta(A) \leq N_\delta(A) \delta^s.
\]

Če je \(c = \mathcal{H}^s(A) = \lim_{\delta \to 0} H^s_\delta(A) > 0\), potem za dovolj majhne \(\delta\) velja
\[
    \log N_\delta(A) + s \log \delta > c \lthen s - \frac{c}{\log \delta} < \frac{\log N_\delta(A)}{\log (1 / \delta)}{}.
\]
V limiti \(\delta \to 0\) dobimo
\[
    s \leq \liminf_{\delta \to 0} \frac{\log N_\delta(A)}{\log (1 / \delta)},
\]
torej
\[
    \dim_H A \leq \underline{\dim}_B A \leq \overline{\dim}_B A
\]
za vsako množico \(A\).

V splošnem enakost ne velja, čeprav za veliko znanih primerov velja.

Vidimo torej, da lahko škatlasto dimenzijo uporabimo kot zgornjo oceno Hausdorffove dimenzije.


\subsection{Lastnosti in slabosti škatlaste dimenzije}
V tem razdelku si bomo ogledali lastnosti in slabosti škatlaste dimenzije.

\subsubsection{Lastnosti škatlaste dimenzije}
Škatlasta dimenzija ima naslednje lastnosti, ki jih lahko dokažemo podobno kot v primeru Hausdorffove dimenzije. Podrobni dokazi so na voljo v \cite[stran 47]{fk-fg}.

\begin{itemize}
    \item Če je \(M \subseteq \R^n\) gladka podmnogoterost dimenzije \(m \in \N\), potem 
    \[\dim_B M = m.\]
    \item Spodnja in zgornja škatlasta dimenzija sta monotoni.
    \item Zgornja škatlasta dimenzija je končno stabilna, tj.
    \[\overline{\dim}_B (A \cup B) = \max (\overline{\dim}_B A,\ \overline{\dim}_B  B).\]
    Ta enakost v splošnem ne velja za spodnjo škatlasto dimenzijo.
    \item Spodnja in zgornja škatlasta dimenzija sta invariantni pri bi-Lipschit\-zevih preslikavah.
\end{itemize}

Podobno kot Hausdorffova dimenzija ima tudi škatlasta dimenzija predvidljivo obnašanje pri H\"olderjevih in Lipschitzevih preslikavah.


\subsubsection{Slabosti škatlaste dimenzije}
Škatlasta dimenzija je zelo uporabna, vendar ima tudi določene pomanjkljivosti. Ena izmed njih je naslednja:

\begin{trditev}
    \label{slabost}
    Naj bo \(A \subseteq \R^n\) omejena in neprazna množica. Tedaj velja 
    \[
        \overline{\dim}_B A = \overline{\dim}_B \overline{A} \quad \text{in} \quad \underline{\dim}_B A = \underline{\dim}_B \overline{A}
    \]
\end{trditev}

\begin{proof}
    Naj bo \(B_1, \ldots, B_k\) končna družina zaprtih krogel z radijem \(\delta\), ki pokrivajo množico \(A\), tj.
    \[
        A \subseteq \bigcup_{i=1}^k B_i.
    \]
    Ker so krogle zaprte, je tudi 
    \[
        \overline{A} \subseteq \bigcup_{i=1}^k B_i,
    \]
    saj zaprtje je najmanjša zaprta množica, ki vsebuje \(A\). Zato je najmanjše število zaprtih krogel z radijem \(\delta\), ki pokrivajo \(A\), enako najmanjšemu številu zaprtih krogel z radijem \(\delta\), ki pokrivajo \(\overline{A}\). Trditev sledi.
\end{proof}

Kot posledico na prvi pogled ugledne lastnosti dobimo:
\begin{posledica}
    Škatlasta dimenzija v splošnem ni števno stabilna.
\end{posledica}

\begin{proof}
    Oglejmo si števno množico \(A = [0, 1] \cap \Q\).
    Ker je \(A\) števna, jo lahko zapišemo v obliki
    \[A = \set{a_1, a_2, \ldots}.\]
    Za vsak \(i \in \N\) definirajmo \(A_i = \{a_i\}\). Tedaj velja
    \[
        A = \bigcup_{i \in \N} A_i.
    \]
    Po eni strani je \(\dim_B A_i = 0\) za vsak \(i \in \N\). Po drugi strani pa je \(\overline{A} = [0, 1]\) in po trditvi~\ref{slabost} sledi
    \[
        \dim_B A = \dim_B \overline{A} = \dim_B [0, 1] = 1.
    \]
    Torej
    \[
        \dim_B\left( \bigcup_{i \in \N} A_i \right) = 1 \neq \sup\left\{ \dim_B A_i \mid i \in \N \right\} = 0.
    \]
    S tem pokažemo, da škatlasta dimenzija ni števno stabilna.
\end{proof}

Lahko pa se vseeno vprašamo, ali je škatlasta dimenzija morda števno stabilna na zaprtih ali kompaktnih množicah. Odgovor je v splošnem negativen, kar pokaže naslednji zgled.

\begin{zgled}
    Naj bo \(A = \setb{\frac{1}{n}}{n \in \N} \cup \set{0}\). Vemo, da je \(A\) kompaktna množica. Trdimo, da je 
    \[\dim_B A = \frac{1}{2} \neq 0.\]
    
    Naj bo \(0 < \delta < \frac{1}{2}\). Izberimo \(k \in \N\), da velja
    \[
    \frac{1}{k(k+1)} \leq \delta < \frac{1}{k(k-1)}.
    \]

    Naj bo \((U_i)_{i \in \N}\) poljubno \(\delta\)-pokritje množice \(A\). Ker je \(\diam U_i \leq \delta\), lahko vsak interval \(U_i\) pokrije največ eno točko iz množice  \(\set{1, \frac{1}{2}, \ldots, \frac{1}{k}}\), saj je razdalja med zaporednima točkama 
    \[
    \frac{1}{k-1} - \frac{1}{k} = \frac{1}{k(k-1)} > \delta.
    \]
    Torej je 
    \[
    N_\delta(A) \geq k,
    \]
    od koder sledi
    \[
    \frac{\log N_\delta(A)}{\log (1 / \delta)} \geq \frac{\log k}{\log (k(k+1))}.
    \]
    V limiti \(\delta \to 0\), oziroma \(k \to \infty\), dobimo
    \[
    \liminf_{\delta \to 0} \frac{\log N_\delta(A)}{\log (1 / \delta)} \geq \frac{1}{2}.
    \]

    Po drugi strani lahko interval \([0, \frac{1}{k}]\) pokrijemo z največ \(k+1\) intervali dolžine \(\delta\). Preostale \(k-1\) točk iz množice \(A\) lahko pokrijemo z enakim številom intervalov, torej skupaj 
    \[
    N_\delta(A) \leq 2k.
    \]
    Od tod sledi
    \[
    \frac{\log N_\delta(A)}{\log (1 / \delta)} \leq \frac{\log (2k)}{\log (k(k-1))}.
    \]
    V limiti \(\delta \to 0\), oziroma \(k \to \infty\), dobimo
    \[
    \limsup_{\delta \to 0} \frac{\log N_\delta(A)}{\log (1 / \delta)} \leq \frac{1}{2}.
    \]
    
    Skupaj s prvo neenakostjo sledi enakost.
\end{zgled}

Zdaj pa vidimo da števna stabilnost ne drži niti za kompaktne ali zaprte množice.

Kljub tej slabosti, da škatlasta dimenzija ni števno stabilna niti za zaprte ali kompaktne množice, je še vedno zelo pogosto uporabljena in uporabna dimenzija v različnih področjih matematike in aplikacij. Pogosto se tudi da pokazati, da se škatlasta dimenzija ujema z Hausdorffovo dimenzijo za nekatere množice. Zaradi svoje relativne enostavnosti izračuna in naravne interpretacije kot ">merjenje gostote"< množice pri manjših merilih je škatlasta dimenzija priljubljeno orodje pri proučevanju fraktalov in drugih kompleksnih množic.

\section*{Angleško-slovenski slovar strokovnih izrazov}

% \geslo{box-counting dimension}{škatlasta dimenzija}

% \geslo{Cantor dust}{Cantorjev prah}

\geslo{countable stability}{števna stabilnost}

\geslo{dimension}{dimenzija}

\geslo{Hausdorff dimension}{Hausdorffova dimenzija}

\geslo{manifold}{mnogoterost}

\geslo{measure}{mera}

\geslo{metric space}{metrični prostor}

\geslo{motion}{translacija}

\geslo{outer measure}{zunanja mera}

\geslo{rotation}{rotacija}

\geslo{similarity}{podobnost}

\geslo{totally disconnected set}{popolnoma nepovezana množica}




\printbibliography

\end{document}

\section*{Angleško-slovenski slovar strokovnih izrazov}