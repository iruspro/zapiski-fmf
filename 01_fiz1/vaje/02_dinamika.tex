\section{Dinamika}
\subsection{Sile}
\paragraph{Newtonovi zakoni}
\begin{enumerate}
    \item \(\sum \vec{F} = 0 \lthen \vec{v} = \text{const}\)
    \item \(\sum \vec{F} = m \vec{a}\) 
    \item \(\vec{F}_{12} = -\vec{F}_{21}\) 
\end{enumerate}

\paragraph{Sila trenja}
\begin{itemize}
    \item \(F_\text{tr} \leq k_\text{tr} \cdot F_N\), kjer je \(F_N\) normalna sila
\end{itemize}

\paragraph{Sila vzmeti}
\begin{itemize}
    \item \(F_\text{vz} = kx\), kjer je \(k\) koeficient vzmeti in je \(x\) raztezek
\end{itemize}

\paragraph{Težišče}
\begin{itemize}
    \item \df{Težišče} je \(\vec{r}_T = \frac{1}{M} \sum m_j \vec{r}_j\), kjer je \(M = \sum m_j\) \df{skupna masa}
    \item \textbf{II.\ Newtonov zakon za težišče:} \(\sum \vec{F}_\text{zun} = M \vec{a}_T\)
\end{itemize}

\paragraph{Splošni nasveti}
\begin{itemize}
    \item Zapišemo vse sile, ki delujejo v našem sistemu. Sistem lahko izberimo poljubno
    \item Ponavadi \(\vec{F_g}\) razbijemo na statično in dinamično komponento
    \item Sile vrvi na škripec delujejo vzdolž vrvi:
\end{itemize}

\paragraph{Neinercialni sistemi} \ 

Naj bo \(K_1\) ne pospešen (inercialni) sistem. Zapišemo II.\ Newtonov zakon v različnih neinercialnih (pospešenih) sistemih.
\begin{itemize}
    \item \textbf{Linearno pospešen sistem \(K_2\) z pospeškom \(\vec{a_0}\)} 
   
   \begin{itemize}
    \item \textbf{II.\ Newtonov zakon:} \(\boxed{\vec{F}_1 + \vec{F}_\text{sist} = m \vec{a}_2}\), kjer \(\vec{F}_\text{sist} = - m \vec{a}_0\)
     \begin{itemize}
         \item \(\vec{F}_1\) je rezultanta vseh sil na telo v sistemu \(K_1\)
         \item \(\vec{a}_2 = \vec{a}_1 - \vec{a_0}\) je pospešek telesa v sistemu \(K_2\)
     \end{itemize}
   \end{itemize}
    \item \textbf{Sistem \(K_2\) se vrsti okoli fiksne osi s kotno hitrostjo \(\omega = \omega(t)\)}
    \begin{itemize}
        \item \textbf{II.\ Newtonov zakon:} \(\boxed{\vec{F}_1 - m \vec{\alpha} \times \vec{r} - 2m \vec{\omega} \times \vec{v}_2 - m \vec{\omega} \times (\vec{\omega} \times \vec{r}) = m \vec{a}_2}\)
        \begin{itemize}
            \item \(-m \vec{\alpha} \times \vec{r}\) je \df{tangentna sila} (pospešuje vrtenje)
            \item \(-2m \vec{\omega} \times \vec{v}_2\) je \df{Coriolisova sila}
            \item \(-m \vec{\omega} \times (\vec{\omega} \times \vec{r})\) je \df{centrifugalna sila} (lahko jo ne upoštevamo pri delu z gravitacijo)
            \item \(\vec{v_2}\) je hitrost telesa v sistemu \(K_2\), \(\vec{a_2}\) je pospešek telesa v sistemu \(K_2\)
        \end{itemize}
    \end{itemize}
\end{itemize}

\newpage
\subsection{Energija}
Ko čas gre iz igre (nas ne zanima kdaj se nekaj zgodilo) se lahko ukvarjamo z energijo.
\paragraph{Konetična energija točkastega delca} \ 

\[\vec{F} = m \vec{a} = m \frac{d \vec{v}}{dt} \ / \cdot d \vec{s} \lthen \int_{1}^{2} \vec{F} \cdot d \vec{s} = \Delta(W_\text{k}), \ \textcolor{red}{(*)}\]
kjer \(W_\text{k} = \frac{mv^2}{2}\) \df{kinetična energija} točkastega delca, \([W_\text{k}] = \text{J} = \text{Nm}\).
\begin{itemize}
    \item \(\ds \int_{1}^{2} \vec{F} \cdot d \vec{s} = A\) je \df{delo} sile \(\vec{F}\), kjer \(d \vec{s}\) je premik \textbf{prijemališča} sile
    \item \(\textcolor{red}{(*)}\) je izrek o mehanske (kinetične energije)
\end{itemize}

\paragraph{Sistem točkastih teles: } \(\ds \int_{1}^{2} \vec{F}_\text{zun} \cdot d \vec{s}_T = \Delta(W_\text{k, T})\), kjer \(W_\text{k, T}= \frac{1}{2} mv^2_T\) \df{kinetična energija težišča}
\begin{itemize}
    \item \(\ds \widetilde{A}_\text{zun} = \int_{1}^{2} \vec{F}_\text{zun} \cdot d \vec{s}_T \) je \df{psevdodelo} rezultante zunanjih sil
\end{itemize}

\paragraph{Potencialna in prožnostna energija} \ 

Eksplicitno izračunamo delo silo teže in delo sile vzmeti, dobimo:
\[A_{\text{F}_g} = - mgh \quad \text{in} \quad A_\text{vz} = \frac{1}{2}k s^2\]
Potem lahko zapišemo \textbf{izrek o mehanske energije} v oblike
\[\boxed{\widetilde{A}_\text{zun} = \Delta(W) = W_\text{konec} - W_\text{začetek}, \ W = W_\text{k} + W_\text{p} + W_\text{pr}}\]
kjer je \(W_\text{p} = mgh\) \df{potencialna energija} in \(W_\text{pr} = \frac{1}{2} k s^2\) \df{prožnostna energija} ter \(\widetilde{A}_\text{zun}\) psevdodelo vseh zunanjih sil razen sile teže in sil vzmeti. V posebnem primeru, ko ni zunanjih sil: \(\widetilde{A}_\text{zun} = 0\), tj. energija se ohranja.

\paragraph{Moč} \ 

Včasih je pomembno, kako hitro opravimo neko delo.

\begin{itemize}
    \item \df{Moč} \(P\) je \(P = \frac{dA}{dt}\), \([P] = \frac{\text{J}}{\text{s}} = \text{Watt}\)
\end{itemize}

\subsection{Gibalna količina}
\paragraph{Točkasto telo}
\[\vec{F} = m \vec{a} = m \frac{d \vec{v}}{dt} \ / \cdot d \vec{t} \lthen \int \vec{F} \cdot d \vec{t} = m(v_\text{konec} - v_\text{začetek}) \lthen \int_{1}^{2}  \vec{F} dt = \Delta \vec{G}, \ \textcolor{red}{(*)}\]
\begin{itemize}
    \item \(\textcolor{red}{(*)}\) je \textbf{izrek o gibalne količine}
    \item \(\vec{G} = m \vec{v}\) je \df{gibalna količina} za točkasto telo
    \item \(\ds \int_{1}^{2} \vec{F} dt\) je \df{sunek sile}
\end{itemize}

\paragraph{Sistem točkastih teles: } \(\ds \int_{1}^{2} \vec{F}_\text{z} dt = \Delta \vec{G}_T\)

\begin{itemize}
    \item Če \(\ds \int_{1}^{2} \vec{F} dt = 0\) ali \(\ds \int_{1}^{2} \vec{F}_\text{z} dt = 0\), potem gibalna količina se ohranja
\end{itemize}

\paragraph{Trki} \ 

\begin{enumerate}
    \item \textbf{Neelastični (neprožni) trk:} telesa se zlepijo in po trku gibljejo skupaj
    \begin{itemize}
        \item Gibalna količina se ohranja
        \item \(W_\text{k}\) se NE ohranja \(\leadsto\) stvari se segrejejo 
    \end{itemize}
    \item \textbf{Elastični trk:} telesa se odbijejo
    \begin{itemize}
        \item Gibalna količina se ohranja
        \item \(W_\text{k}\) se ohranja
        \item \(\boxed{v_1 = - \frac{1 - \mu}{1 + \mu} v, \ v_2 = \frac{2 \mu}{1 + \mu} v}\), kjer \(\mu = \frac{m}{M}\)
    \end{itemize}
\end{enumerate}

\paragraph{Sila curka}
\begin{itemize}
    \item \(\vec{F}_\text{c} = \phi_m \Delta v\), kjer je \(\phi_m = \frac{\Delta m}{\Delta t}\) \df{masni tok}
    \begin{itemize}
        \item \(\phi_m = \frac{dm}{dt} = \phi_V \rho\), kjer je \(\phi_V = \frac{dV}{dt} = \frac{Svdt}{dt} = Sv\) \df{prostorninski tok}
        \item Zapišemo izrek o gibalne količine (sunek sile je enak spremembe gibalne količine)
    \end{itemize}
\end{itemize}

\paragraph{Raketa} 
\begin{itemize}
    \item Za sistem si izberimo raketo + majhni drobec goriva. Gibalna količina se ohranja. Dobimo enačbo:
    \begin{itemize}
        \item \(u dm_g = mdv\), kjer \(u\) hitrost izpušnih plinov glede na raketo in \(m\) trenutna masa rakete in goriva
        \begin{itemize}
            \item Definiramo: \(dm = m - (m + dm_g) \lthen dm = -dm_g\), dobimo: \(dm = -dm_g\)
        \end{itemize}
    \end{itemize}
\end{itemize}


\paragraph{Splošni nasveti}
\begin{itemize}
    \item Izberimo si sistem, za kateri znamo zapisati želene količine
    \item Poglejmo tik do in po trku
    \item Lahko zapišemo gibalno količino za celoten sistem ali za vsako telo posebej
\end{itemize}

\subsection{Statika}
\begin{itemize}
    \item Uporaba III.\ Newtonovega zakona
    \item \df{Navor} je \(\vec{M} = \vec{r} \times \vec{F}\), kjer je \(\vec{r}\) vektor od osi vrtenja do telesa.
\end{itemize}

\subsection{Vztrajnostni moment}
\begin{itemize}
    \item \df{Vztrajnostni moment} okoli fiksne osi je \(\ds J = \int_{\text{po telesu}} r^2 \, dm\) (oz.\ diskretna vsota)
    \item \textbf{Newtonov zakon za vrtenje okoli fiksne osi}: \[F = ma \lthen F \cdot r = ma \cdot r \lthen M = m \alpha r \cdot r \lthen M = mR^2 \alpha \lthen M = J\alpha\]
    \item \textbf{Šteinerjev izrek:} Vztrajnostni moment telesa pri vrtenju okoli fiksne osi \(\xi\) je \[J_\xi = J_T + ma^2,\]
    kjer je \(J_T\) vztrajnostni moment telesa pri vrtenju okoli težišča in \(a\) pravokotna razdalja do osi vrtenja.
\end{itemize}

\paragraph{Osnovne vztrajnostni momenti} \ 

\begin{table}[h!]
    \centering
    \begin{tabular}{|l|c|}
    \hline
    \textbf{Telo} & \textbf{Vztrajnostni moment \(J\)} \\
    \hline
    Točkasta masa \(m\) & $J = mr^2$  \\
    \hline
    Obroč s polmerom \(r\) & \(J = mr^2\) \\
    \hline
    Palica dolžine \(l\) okrog težišča & \(J = \frac{1}{12}ml^2\) \\
    \hline
    Palica dolžine \(l\) okrog krajišča & \(J = \frac{1}{3}ml^2\) \\
    \hline
    Okrogla plošča s polmerom \(r\) & \(J = \frac{1}{2}mr^2\) \\
    \hline
    Valj s polmerom \(r\) & \(J = \frac{1}{2}mr^2\) \\
    \hline
    Stožec z višino \(h\) in polmerom \(r\) & \(J = \frac{3}{10}mr^2\) \\
    \hline
    Stožec z višino \(h\) in polmerom \(r\) & \(J = \frac{3}{10}mr^2\) \\
    \hline
    Krogla s polmerom \(r\) okrog simetrijske osi & \(J = \frac{2}{5}mr^2\) \\
    \hline
    \end{tabular}
\end{table}

\paragraph{Drsenje/Kotaljenje} \ 
\begin{itemize}
    \item Pogoj, da ni drsanja: \(v_t = \omega r\), tj.\ spodnja točka miruje.
    \begin{itemize}
        \item Če telo drsi: \(F_\text{tr} = k_\text{tr} N\) in \(a \neq \alpha r\)
        \item Če telo kotali: \(F_\text{tr} < k_\text{tr} N\) in \(a = \alpha r\)
    \end{itemize} 
\end{itemize}

\paragraph{Splošni nasveti}
\begin{itemize}
    \item Lahko prištejemo in odštejemo isto silo, in pogledamo kaj vpliva na težišče in kaj vpliva na vrtenje.
\end{itemize}

\newpage
\subsection{Kinetična energija vrtenja}
\begin{itemize}
    \item Kinetična energija vrtenja:
    \[W_\text{k} = \int \frac{v^2dm}{2} = \int \frac{r^2 \omega^2 dm}{2} = \frac{\omega^2}{2} \int r^2 dm = \frac{1}{2} J \omega^2\]
    \item Kinetična energija kotaljenja: \(W_\text{k} = \frac{1}{2}mv^2_T + \frac{1}{2}J_T \omega^2\)
\end{itemize}

\subsection{Vrtilna količina}
\begin{itemize}
    \item \textbf{Izrek o vrtilni količine} (pri vrtenju okoli fiksne osi): 
    \[\int M \, dt = \Delta \Gamma = Jw_\text{k} - Jw_\text{z}\]
\end{itemize}

\paragraph{Splošni nasveti}
\begin{itemize}
    \item Izrek velja za vrtenje okoli fiksne osi, če jih imamo več, zapišemo izrek za vsako telo posebej.
\end{itemize}

\subsection{Gravitacija}
\paragraph{Par točkastih teles}
\[F_g = G \frac{m_1m_2}{r^2},\]
kjer \(G = 6,67 \cdot 10^{-11} \frac{\text{Nm}^2}{\text{kg}^2}\)
\begin{itemize}
    \item Gravitacijski pospešek na Zemlje: \(g(h) = G \frac{M}{(R+h)^2}\)
\end{itemize}
\paragraph{Potencialna energija} Par točkastih mas:
\[A = \int_{r}^{\infty} F_g \, dr =  \int_{r}^{\infty}G \frac{m_1m_2}{r^2} \, dr = G\frac{m_1m_2}{r} = W_p(\infty) - W_p(r).\]
Definiramo \(W_p(\infty) = 0\) sledi, da \(W_p(r) = - G\frac{m_1m_2}{r}\)
\paragraph{Sateliti} \ 
\begin{itemize}
    \item Pogoj, da satelit ne pade na Zemlju: \(F_g = F_\text{cf}\)
    \item Ubežna hitrost:
    \begin{itemize}
        \item Začetek: Kinetična energija + Gravitacijska potencialna energija
        \item Konec (smo v \(\infty\)): \(W_p(\infty) = 0\), \(W_k(\infty) = 0\)
    \end{itemize}
\end{itemize}

\paragraph{Približevanje} \ 
\begin{itemize}
    \item Energija je konstantna.
    \item Vrtilna količina v najbližjih točkah je enaka (se ohranja): \(\vec{\Gamma} = m \vec{r} \times \vec{v}\), če \(M \gg m\), sicer reducirana masa \(\frac{1}{\mu} = \frac{1}{m} + \frac{1}{M}\)
\end{itemize}
