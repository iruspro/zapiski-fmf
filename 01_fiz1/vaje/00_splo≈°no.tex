\section*{Splošno}
\begin{itemize}
    \item \textbf{Kosinusni izrek.} \(c^2 = a^2 + b^2 - 2ab \cos \alpha\), kjer je \(\alpha\) kot med stranicama \(a\) in \(b\)
    \item \textbf{Vektorski produkt.} \(\begin{bmatrix}
        a_1 \\ a_2 \\ a_ 3
    \end{bmatrix} \times \begin{bmatrix}
        b_1 \\ b_2 \\ b_3
    \end{bmatrix} = \begin{bmatrix}
        a_2b_3 - a_3b_2 \\ a_3b_1 - a_1b_3 \\ a_1b_2 - a_2b_1
    \end{bmatrix} \), \(|\vec{a} \times \vec{b}| = ab \sin \alpha \), \(\vec{a} \times (\vec{b} \times \vec{c}) = (\vec{a} \cdot \vec{c}) \vec{b} - (\vec{a} \cdot \vec{b}) \vec{c}\)
    \item \textbf{Radiani - Stopinji.} \(1 \text{ rd} = 1 \text{ deg} \cdot \frac{\pi}{180^\circ}\)
    \item \textbf{Celzij - Kelvin.} \(x^{\ \circ} \text{C} = x + 273 \text{ K}\)
    \item \textbf{Litri - dm.} \(1 \ \text{L} = 1 \ \text{dm}^3\), \(1000 \  \text{L} = 1 \ \text{m}^3\)
\end{itemize}

\paragraph{Osnovne konstante} \ 

\begin{table}[h!]
    \centering
    % Mehanika
    \begin{tabular}{|l|c|l|}
    \hline
    \textbf{Velikost} & \textbf{Oznaka} & \textbf{Vrednost} \\
    \hline
    Hitrost svetlobe v vakuumu & $c$ & $3 \times 10^8 \ \text{m/s}$ \\
    \hline
    Hitrost zvoka v zraku (pri 20°C) & $v_{\text{zvok}}$ & $340 \ \text{m/s}$ \\
    \hline    
    Gravitacijski pospešek & $g$ & $9{,}8 \ \text{m/s}^2$ \\
    \hline
    Gravitacijska konstanta & $G$ & \(6{,}67 \cdot 10^{-11} \ \text{Nm}^2 / \text{kg}^2\) \\
    \hline
    Radij Zemlje & $R$ & \(6400 \ \text{km}\) \\
    \hline
    Masa Zemlje & $M$ & \(6 \cdot 10^{24} \ \text{kg}\) \\
    \hline

    % Termodinamika
    Gostota vode (pri 4°C) & $\rho_{\text{voda}}$ & $1000 \ \text{kg/m}^3$ \\
    \hline
    Gostota zraka (pri 20°C in 1 atm) & $\rho_{\text{zrak}}$ & $1{,}204 \ \text{kg/m}^3$ \\
    \hline
    Avogadrovo število & $N_A$ & \(6,02 \cdot 10^{26} \ 1 \ \text{kmol}^{-1}\) \\
    \hline
    Boltzmannova konstanta & $k_B$ & \(1,38 \cdot10^{-23} \ \text{J} / \text{K}\) \\
    \hline
    Univerzalna plinska konstanta & $R$ & \(8300 \ \text{J} / (\text{kmol} \cdot \text{K})\) \\
    \hline
    Kilomolska masa zraka & $\mu_\text{zrak}$ & \(29 \ \text{kg} / \text{kmol}\) \\
    \hline

    \end{tabular}
    \caption{Osnovne fizikalne konstante v mehaniki in sorodnih področjih}
\end{table}

\paragraph{Splošni nasveti}
\begin{itemize}    
    \item Če se da, izognemo se kvadratnih enačb
\end{itemize}